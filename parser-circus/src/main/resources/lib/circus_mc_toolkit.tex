%%%%%%%%%%%%%%%%%%%%%%%%%%%%%%%%%%%%%%%%%%%%%%%%%%%%%%%%%%%%%%%%%%%%%%%%%%%%%%%%
% PREAMBLE: Circus model checking toolkit, v 1.1, June 2007
%           Copyright 2004-2007 Leo Freitas
%           Department of Computer Science 
%           University of York, YO10 5DD UK
%           leo@cs.york.ac.uk, +44-1904-434753
%           http://www.cs.york.ac.uk/circus
%           http://www.cs.york.ac.uk/~leo
%%%%%%%%%%%%%%%%%%%%%%%%%%%%%%%%%%%%%%%%%%%%%%%%%%%%%%%%%%%%%%%%%%%%%%%%%%%%%%%%

%%%%%%%%%%%%%%%%%%%%%%%%%%%%%%%%%%%%%%%%%%%%%%%%%%%%%%%%%%%%%%%%%%%%%%%%%%%%%%%%
% Circus toolkit for the model checker. It extends the basic toolkit with some
% extra definitions used by the model checker VC generation engine.
%
\begin{zsection}
  \SECTION circus\_mc\_toolkit \parents circus\_toolkit
\end{zsection}

%%%%%%%%%%%%%%%%%%%%%%%%%%%%%%%%%%%%%%%%%%%%%%%%%%%%%%%%%%%%%%%%%%%%%%%%%%%%%%%%
% Automata normalisation

%   Distributed disjointness between set of sets.
%   TODO: Find the equivalent in LaTeX for this Unicode.
%%Zinchar \gendj U+2A46
\begin{zed}
\function 40 \leftassoc (\_~\gendj~\_)
\end{zed}

\begin{gendef}[X]
    \_ \gendj \_: \power~(\power~X) \cross \power~(\power~X) \fun \power~X
\where
    \forall S, P: \power~(\power X) @ P \gendj S = (\bigcap~P)~\setminus~(\bigcup~(S~\setminus~P))
\end{gendef}

%   Non-empty regions of a non-empty set of sets. 
%   That is, the corresponding set of sets with nondeterminism clearly separated.
%   It is used in the normalisation algorithm for Circus automata.
%   LaTeX char equivalent to \otimes
%%Zprechar \regions U+2A02
\begin{gendef}[X]
    \regions: \power_1~(\power~X) \fun \power_1~(\power~X)
\where
    \forall S: \power_1~(\power~X) @ \\
      \t1 \regions S = \{~ P: \power_1~(\power~X) | P \subseteq S @ P \gendj S \}~\setminus~\{~ \emptyset ~\}
\end{gendef}

%   Calculates the minimal set of a set of sets.
%   That is, the sets that are minimal under subset inclusion.
%   It is used in the normalisation algorithm for Circus automata
%   while calculating minimal sets of coalesced (power) states.
\begin{gendef}[X]
    minimal: \power~(\power~X) \fun \power~(\power~X)
\where
   \forall SS: \power~(\power~X) @ \\
    \t1 minimal~SS = \{ S: \power~X | S \in SS \land \\
                \t2 \lnot~(~\exists R: \power~X | R \in SS @ R \subset S~) \}
\end{gendef}

%   Set difference distributed over set of sets.
%   It is used for in the operational semantics of parallelism. 
%   LaTeX char equivalent to \searrow.
%   The associativity is the same as \setminus.
%   The binding power of this operator template is slightly tigher than \setminus (30).
%%Zinchar \dsetminus U+21F2
\begin{zed}
    \function 35 \leftassoc (\_ \dsetminus \_)
\end{zed}

\begin{gendef}[X]
    \_ \dsetminus \_ : \power~(\power~X) \cross \power X \fun \power~(\power~X)           
\where
    \forall P: \power~(\power~X); p: \power~X @ P \dsetminus p = \{~ q: \power~X | q \in P @ q \setminus p ~\}
\end{gendef}

%   Set intersection distributed over set of sets.
%   It is used for in the operational semantics of parallelism.
%   LaTeX char equivalent to \lozenge(?).
%   Alternative unicode to use U+22C4, which has LaTeX equivalent as \diamond.
%   The associativity is the same as \cap.
%   The binding power of this operator template is slightly tigher than \cap (40).
%   We haven't used 45 because it has already been taken by a \rightassoc operator
%   in sequence_toolkit.tex (\_ \extract \_).
%%Zinchar \dcap U+25C7
\begin{zed}
    \function 44 \leftassoc (\_ \dcap \_)
\end{zed}

\begin{gendef}[X]
    \_ \dcap \_ : \power~(\power~X) \cross \power X \fun \power~(\power~X)
\where
    \forall~ P: \power~(\power~X); p: \power~X @ P \dcap p = \{~ q: \power~X | q \in P @ q \cap p ~\}
\end{gendef}
