\documentclass{article}

\usepackage{my-a4,circus}

% Unifying Theory macros
\newcommand{\healthiness}[1]{\mathbf{#1}}
\newcommand{\pseudoId}{\hbox{\large\texttt{\emph{I}}}\!\!\hbox{\large\texttt{\emph{I}}}} %
\newcommand{\infixIf}{\ \raisebox{-.1ex}{\hbox{\large$\triangleleft$}}\ } %
\newcommand{\infixElse}{\ \raisebox{-.1ex}{\hbox{\large$\triangleright$}}\ } %

\bibliographystyle{plain}

\begin{document}

\title{\Circus: a concurrent refinement language}

\author{ %
  \begin{tabular}[t]{c}
    Jim Woodcock
    \\ %
    University of Kent at Canterbury
    \\ %
    England - UK
  \end{tabular}
  \quad
  \begin{tabular}[t]{c}
    Ana Cavalcanti
    \\ %
    Universidade Federal
    \\ %
    de Pernambuco
    \\ %
    Brazil
  \end{tabular}
  } %

\bigskip

\maketitle

\bigskip

\begin{abstract}
  \noindent We introduce \Circus, a concurrent language appropriate for
  refinement, based on imperative CSP with added specification
  facilities in the style of Z.  We describe this language of \Circus\
  and give a simple example of the specification of a reactive buffer.
  We also formalise the model of \Circus\ in Hoare \& He's unifying
  theories of programming.
\end{abstract}

\bigskip \bigskip

\begin{quote}
  \begin{quote}
    \begin{quote}
      \begin{quote}
        \begin{description}
        \item[\Circus] \emph{n}.  An open space where different routes
          come together (as in \emph{Oxford Circus}). [L, = ring]
        \end{description}
      \end{quote}
    \end{quote}
  \end{quote}
\end{quote}

\newpage %

\tableofcontents

\newpage %

\section{Introduction}

With the current trend to integrate different languages and techniques
to obtain formalisms that are suited to a broader range of
applications, several combinations of
Z~\cite{ZStandard,Spi88,Spi92,Spi95,WD96} with some process
algebra~\cite{Hoa85,Mil89,Ros98} have been proposed.  In all these
works~\cite{Fis98}, the objective is to specify both state and
communication aspects of concurrent systems in a convenient way,
integrating languages, related theories, and tools.  The semantics of
the resulting languages are given by an extension of an existing
semantics definition for the process algebra.

We are interested in a formalism that is not only suitable for the
specification of concurrent systems, but that also has an associated
theory of refinement.  We aim, in particular, at a methodology of
program development that is based on refinement laws and is
calculational in style.  Several refinement calculi have been proposed
and used~\cite{BW98,Mor94,Mor87c}.  The benefits of calculating as
opposed to verifying programs are manifold.  Morgan's style of program
refinement, in particular, produces developments that can be presented
uniformly as sequences of small refinement steps.  The refinement laws
provide guidance in the development of programs.  A similar calculus
has been proposed for Z in~\cite{CW99,Cav97}.

The usual semantic model for imperative refinement calculi is based on
predicate transformers.  Theories of refinement for CSP, however, have
been based on the failures-divergences model~\cite{Hoa85,Ros98}.  The
works in~\cite{Smi97} and \cite{Fis97,Fis00} present the semantics for
combinations of Object-Z and CSP by providing a failures-divergences
model for Object-Z classes.  Data refinement has been briefly explored
for such a combination, but no refinement laws have been proposed.
Behavioural semantics have been given to abstract data types, using
the failures model~\cite{WDB00}.  In~\cite{CS00}, refinement rules
have been proposed to support the development of Java programs, but no
semantic model has been provided.

The semantic model proposed in~\cite{HH98} is a unifying theory, where
both state and communication aspects of concurrent systems are
captured by observational variables that represent an enriched
failures-divergences model.  A simple and elegant definition of
refinement is proposed in this work; therefore, it is natural to
consider it as a basis for the justification of refinement laws.

Our objective is to propose and formalise a refinement calculus for a
combination of Z and CSP.  Of all the process algebras,
CSP~\cite{Hoa85,Ros98} is perhaps the most successful for industrial
application.  Its key distinguishing feature is that the language has
been designed around the notion of refinement, making it suitable for
the development of large-scale systems.  Moreover, commercial tools
for analysis and simulation of CSP specifications are
available~\cite{FDR99, PROBE98}.

In this report, as a first step, we define \Circus, a language that
combines Z and CSP and whose semantic model is based on the unifying
theory of~\cite{HH98}.  Specification constructs usually found in
refinement calculi and the constructs of the Dijkstra's language of
guarded commands~\cite{Dij76} are also included.  Our goals in
designing \Circus\ are: ease of use for those familiar with Z and CSP;
encapsulation of the underlying model; and the possibility of reusing
existing theories, techniques, methods, and tools.

The language we propose is a unified programming language, as usual
for refinement calculi.  Specifications in \Circus\ are based largely
on the use of Z constructs and specification statements.  These
constructs can be combined with executable commands, like assignments,
conditionals, and loops; reactive behaviour, including communication,
parallelism, and choice, is defined with the use of CSP constructs.
All existing combinations of Z with a process algebra model concurrent
programs as communicating abstract data types, but we do not insist on
identifying events with operations that change the state.  The result
is a general programming language adequate for developing concurrent
programs.

In the next section we give an overview of the structure of \Circus\
programs and define its syntax.  In Section~\ref{section:EARB}, we
give an example in which \Circus\ is used to present two possible
definitions of a buffer.  It is our conjecture that the second
definition is a refinement of the initial, more abstract, one.
Section~\ref{section:TUT} provides a brief introduction to
\emph{Unifying Theories of Programming} before, in
Section~\ref{section:TMOC}, we give the formal semantics of \Circus.
Finally, in Sections~\ref{section:RW} and~\ref{section:FW} we discuss
some related and future works, respectively.

\section{\Circus}\label{section:C}

A \Circus\ program is formed by a sequence of paragraphs, like a Z
specification; each of these can either be a Z paragraph, a channel
definition, a channel set definition, or a process definition.
Figures~\ref{CircusSyntax1} and~\ref{CircusSyntax2} present the BNF
description of the syntax of \Circus.  We use
$\mathsf{CircusParagraph^*}$ to denote a list of 0 or more elements of
the syntactic category $\mathsf{CircusParagraph}$; similarly for
$\mathsf{PParagraph^*}$ and $\mathsf{CParameter^*}$.  The notation
$\mathsf{N^+}$ is used for a comma-separated list of identifiers and
similarly for $\mathsf{Expression^+}$.  The syntactic category
$\mathsf{N}$ is that of the valid Z identifiers.  The categories
called $\mathsf{Paragraph}$, $\mathsf{Schema\hbox{-}Exp}$,
$\mathsf{Expression}$, $\mathsf{Declaration}$, and
$\mathsf{Predicate}$, are, as expected, those of Z paragraphs, schema
expressions, expressions, declarations and predicates; their
definitions are standard and can be found in~\cite{Spi92}.

\begin{figure}[t] \small
  \begin{syntax}
    %----------------------------------------------------------------%
    \mathsf{Program} & %
    \quad \mathsf{::=} \quad & \mathsf{CircusParagraph^*}
    \also %
    %----------------------------------------------------------------%
    \mathsf{CircusParagraph} & %
    \mathsf{::=} & \mathsf{Paragraph}
    \\ %
    & \mathsf{|} & \mathsf{ChannelDefinition} %
    \ \mathsf{|} \ \mathsf{ChanSetDefinition}
    \\ %
    & \mathsf{|} & \mathsf{ProcessDefinition}
    \also %
    %----------------------------------------------------------------%
    \mathsf{ChannelDefinition} & %
    \mathsf{::=} & \mathsf{\circchannel\ CDeclaration}
    \also %
    %----------------------------------------------------------------%
    \mathsf{CDeclaration} & %
    \mathsf{::=} & \mathsf{SimpleCDeclaration} %
    \ \mathsf{|} \ \mathsf{SimpleCDeclaration; CDeclaration}
    \\ %
    %----------------------------------------------------------------%
    \mathsf{SimpleCDeclaration} & %
    \mathsf{::=} & \mathsf{N^+} %
    \ \mathsf{|} \ \mathsf{N^+: Expression} %
    \ \mathsf{|} \ \mathsf{Schema\hbox{-}Exp}
    \also %
    %----------------------------------------------------------------%
    \mathsf{ChanSetDefinition} & %
    \mathsf{::=} & \mathsf{\circchanset\ N == CSExpression}
    \also %
    %----------------------------------------------------------------%
    \mathsf{CSExpression} & %
    \mathsf{::=} & \mathsf{ \lchanset ~ \rchanset } %
    \ \mathsf{|} \ \mathsf{ \lchanset N^+ \rchanset} %
    \ \mathsf{|} \ \mathsf{N}
    \\ %
    & \mathsf{|} & \mathsf{CSExpression \cup CSExpression} %
    \ \mathsf{|} \ \mathsf{CSExpression \cap CSExpression}
    \\ %
    & \mathsf{|} & \mathsf{CSExpression \setminus CSExpression}
    \also %
    %----------------------------------------------------------------%
    \mathsf{ProcessDefinition} & %
    \mathsf{::=} & \mathsf{\circprocess\ N \defs Process}
    \also %
    %----------------------------------------------------------------%
    \mathsf{Process} & %
    \mathsf{::=} & \mathsf{\circbegin\ PParagraph^* @ Action\ \circend} %
    \ \mathsf{|} \ \mathsf{N}
    \\ %
    & \mathsf{|} & \mathsf{Process; Process} %
    \ \mathsf{|} \ \mathsf{Process} \extchoice \mathsf{Process} %
    \ \mathsf{|} \ \mathsf{Process} \intchoice \mathsf{Process}
    \\ %
    & \mathsf{|} & \mathsf{Process} \lpar \mathsf{CSExpression}
    \rpar \mathsf{Process} %
    \\ %
    & \mathsf{|} & \mathsf{Process} \interleave \mathsf{Process}
    %
    \ \mathsf{|} \ \mathsf{Process} \hide \mathsf{CSExpression}
    \\ %
    & \mathsf{|} & \mathsf{Declaration} \odot \mathsf{Process} %
    \ \mathsf{|} \ \mathsf{Process \lfloor \mathsf{Expression^+}
      \rfloor} %
    \ \mathsf{|} \ \mathsf{Process} [ \mathsf{N^+} := \mathsf{N^+} ]
    \\ %
    & \mathsf{|} & \Comp \mathsf{Declaration} \odot \mathsf{Process} %
    \\ %
    & \mathsf{|} & \Extchoice \mathsf{Declaration} \odot \mathsf{Process} %
    \ \mathsf{|} \ \Intchoice \mathsf{Declaration} \odot \mathsf{Process}
    \\ %
    & \mathsf{|} & \!\!\! \Parallel \mathsf{Declaration} \lpar
    \mathsf{CSExpression} \rpar \odot \mathsf{Process}
    %
    \ \mathsf{|} \ \Interleave \mathsf{Declaration} \odot
    \mathsf{Process}
    \\ %
    & \mathsf{|} & \mathsf{Declaration} @ \mathsf{Process} %
    \ \mathsf{|} \ \mathsf{Process} ( \mathsf{Expression^+} )
    \\ %
    & \mathsf{|} & \Comp \mathsf{Declaration} @ \mathsf{Process} %
    \ \mathsf{|} \ \Extchoice \mathsf{Declaration} @ \mathsf{Process} %
    \ \mathsf{|} \ \Intchoice \mathsf{Declaration} @ \mathsf{Process}
    \\ %
    & \mathsf{|} & \!\!\! \Parallel \mathsf{Declaration} \lpar
    \mathsf{CSExpression} \rpar @ \mathsf{Process} %
    %
    \ \mathsf{|} \ \Interleave \mathsf{Declaration} @ \mathsf{Process}
    \\ %
    & \mathsf{|} & \mathsf{[ N^+ ] Process} %
    \ \mathsf{|} \ \mathsf{Process[ Expression^+ ]}
    %----------------------------------------------------------------%
  \end{syntax}
  \caption{\Circus\ syntax}
  \label{CircusSyntax1}
\end{figure}

\begin{figure}[t] \small
  \begin{syntax}
    %----------------------------------------------------------------%
    \mathsf{PParagraph} & %
    \mathsf{::=} & \mathsf{Paragraph} %
    \ \mathsf{|} \ \mathsf{N \defs Action}
    \ \mathsf{|} \ \mathsf{NameSetDefinition}
    \also %
    %----------------------------------------------------------------%
    \mathsf{Action} & %
    \mathsf{::=} & \mathsf{Schema\hbox{-}Exp} %
    \ \mathsf{|} \ \mathsf{CSPActionExp} %
    \ \mathsf{|} \ \mathsf{Command}
    \also %
    %----------------------------------------------------------------%
    \mathsf{CSPActionExp} & %
    \mathsf{::=} & Skip %
    \ \mathsf{|} \ Stop %
    \ \mathsf{|} \ Chaos
    \\ %
    & \mathsf{|} & \mathsf{Communication} \then \mathsf{Action} %
    \ \mathsf{|} \ \mathsf{Predicate}\ \&\ \mathsf{Action}
    \\ %
    & \mathsf{|} & \mathsf{Action}; \mathsf{Action} %
    \ \mathsf{|} \ \mathsf{Action} \extchoice \mathsf{Action} %
    \ \mathsf{|} \ \mathsf{Action} \intchoice \mathsf{Action}
    \\ %
    & \mathsf{|} & \mathsf{Action} \lpar
    \mathsf{NSExpression} | \mathsf{CSExpression} |
    \mathsf{NSExpression} \rpar \mathsf{Action} %
    \ \mathsf{|} \ \mathsf{Action} \interleave \mathsf{Action}
    \\ %
    & \mathsf{|} & \mathsf{Action} \hide \mathsf{CSExpression} %
    \ \mathsf{|} \ \mu \mathsf{N} @ \mathsf{Action}
    \\ %
    & \mathsf{|} & \mathsf{Declaration} @ \mathsf{Action} %
    \ \mathsf{|} \ \mathsf{Action} (\mathsf{Expression^+})
    \\ %
    & \mathsf{|} & \Comp \mathsf{Declaration} @ \mathsf{Declaration} %
    \ \mathsf{|} \ \Extchoice \mathsf{Declaration} @ \mathsf{Action} %
    \ \mathsf{|} \Intchoice \mathsf{Declaration} @ \mathsf{Action}
    \\ %
    & \mathsf{|} & \!\!\! \Parallel \mathsf{Declaration} \lpar
    \mathsf{NSExpression} | \mathsf{CSExpression} |
    \mathsf{NSExpression} \rpar @ \mathsf{Action} %
    \\ %
    & \mathsf{|} & \Interleave \mathsf{Declaration} @ \mathsf{Action}
    \also %
    %----------------------------------------------------------------%
    \mathsf{Communication} & %
    \mathsf{::=} & \mathsf{N \ CParameter^*}
    \also %
    %----------------------------------------------------------------%
    \mathsf{CParameter} & %
    \mathsf{::=} & ?~\mathsf{N} %
    \ \mathsf{|} \ ?~\mathsf{N: Predicate} %
    \ \mathsf{|} \ !~\mathsf{Expression} %
    \ \mathsf{|} \ .~\mathsf{Expression}
    \also %
    %----------------------------------------------------------------%
    \mathsf{Command} & %
    \mathsf{::=} & \mathsf{N^+} : [~ \mathsf{Predicate} ,
    \mathsf{Predicate} ~] %
    \ \mathsf{|} \ \mathsf{N^+} := \mathsf{Expression^+}
    \\ %
    & \mathsf{|} & \circif\ \mathsf{GuardedActions}\ \circfi
    \ \mathsf{|} \ \circvar\ \mathsf{Declaration} @ \mathsf{Action}
    \also %
    %----------------------------------------------------------------%
    \mathsf{GuardedActions} & %
    \mathsf{::=} & \mathsf{Predicate} \then \mathsf{Action} %
    \\ %
    & \mathsf{|} & \mathsf{Predicate} \then \mathsf{Action} \extchoice
    \mathsf{GuardedActions}
    \also %
    %----------------------------------------------------------------%
    \mathsf{NameSetDefinition} & %
    \mathsf{::=} & \mathsf{\circnameset\ N == NSExpression}
    \also %
    %----------------------------------------------------------------%
    \mathsf{NSExpression} & %
    \mathsf{::=} & \mathsf{ \{ ~ \}} %
    \ \mathsf{|} \ \mathsf{ \{ N^+ \}} %
    \ \mathsf{|} \ \mathsf{N}
    \\ %
    & \mathsf{|} & \mathsf{NSExpression \cup NSExpression} %
    \ \mathsf{|} \ \mathsf{NSExpression \cap NSExpression}
    \\ %
    & \mathsf{|} & \mathsf{NSExpression \setminus NSExpression}
    \also %
  \end{syntax}
  \caption{\Circus\ syntax}
  \label{CircusSyntax2}
\end{figure}

To explain the main constructs of \Circus, we use a small example
taken from~\cite{Hoa85}:~we define a process that outputs the
Fibonacci sequence (see Figure~\ref{fibonacciGenerator}).
\begin{figure}[t]
  \begin{zed}
    \circprocess\ Fib \defs \circbegin
    \also %
    \t1
    \begin{block}
      FibState \defs [~ x, y: \nat ~]
      \also %
      InitFibState \defs [~ FibState' | x' = y' = 1 ~]
      \\ %
      InitFib \defs out!1 \then out!1 \then InitFibState
      \also %
      OutFibState \defs [~ \Delta FibState; next!: \nat | next! = y' =
      x + y \land x' = y ~]
      \\
      OutFib \defs \mu X @ (~ \circvar\ next: \nat @ OutFibState;
      out!next \then X ~)
      \also %
      @ InitFib; OutFib
    \end{block}
    \also %
    \circend
  \end{zed}
  \caption{A Fibonacci generator}
  \label{fibonacciGenerator}
\end{figure}

\subsection{Channels}

A channel definition declares channels to which the processes may
refer:~it gives the name of each of the channels and the type of the
values it can communicate.  Our example process outputs through a
channel $out$ that communicates natural numbers.
\begin{zed}
  \circchannel\ out: \nat
\end{zed}
More than one channel may be declared in such a paragraph.  When a
channel is not used to communicate values, but just as a synchronising
event, its declaration consists of only its name:~no type is defined.
We can also use a schema to declare channels.  Such a schema groups
channel declarations, but does not have a predicate part, as the only
restriction we can impose on a channel is the type of the values it
communicates.  The notion of type here is more general than that of
the maximal type of Z; the values output on a channel have to belong
to its declared type.

Sets of previously defined channels may be introduced in a
$\circchanset$ paragraph.  We give a name to the channel set and a
channel-set expression that determines the members of this set.  The
syntactic category $\mathsf{CSExpression}$ of channel set expressions
contains the empty set of channels $\lchanset ~ \rchanset$, channel
enumerations enclosed in $\lchanset$ and $\rchanset$, and set
expressions formed by the usual set operators.  These sets of channels
are used in process expressions.

\subsection{Processes}

A process definition declares its name and gives a process
specification.  The most basic sort of process specification is formed
by a sequence of process paragraphs and a distinguished nameless
action at the end delimited by $\circbegin$ and $\circend$.  A process
paragraph can be a Z paragraph or an action definition; together, they
define the state and the behaviour of the process.  The \Circus\
process that generates the Fibonacci sequence in
Figure~\ref{fibonacciGenerator} is defined in this way.

The internal state of the process is described in the schema
$FibState$:~it contains two natural numbers, $x$ and $y$; the latter
records the last value output, and the former records the value output
before the last. A proof-obligation requires us to prove that the
invariant is different from $false$; in this case and in most examples
this proof-obligation is trivial.

The definitions that follow are action specifications.  The behaviour
of $Fib$ is described by the last, unnamed action; $Fib$ behaves first
as described by the action $InitFib$, and then as described by the
action $OutFib$.  We use the CSP sequence operator.

The action $InitFib$ outputs the number $1$ twice and then records
this by initialising the state components.  It uses the prefix
operator of CSP twice to output $1$ through $out$ and the schema
$InitFibState$ to initialise the state.  This is a schema that follows
the standard style of Z of defining initialisation operations.

The action $OutFib$ is defined recursively with the use of the CSP
operator $\mu$.  It consists of a local variable definition, an
operation on the state, an output on the $out$ channel, and a
recursive call.  The declaration of $next$ is required so that it is
in scope for both the operation schema $OutFibState$ and the
outputting action $out!next \then X$.

The schema $OutFibState$ actually defines the value of $next!  $,
which represents the value of $next$ in the state after the execution
of $OutFibState$.  The outputting action refers to $next$, the value
of this variable in the state before its execution.  In pure Z, dash
and shriek decorations are used to refer to after-state and output
variables, respectively.  In the above example, however, we can use
either $next!$ or $next'$ to refer to the after-state value of $next$.
Our choice has the purpose of emphasising the fact that $next$ is a
local variable, and so not really part of the state of $Fib$, and its
value is output in the next action.  In \Circus, dashes and shrieks
can be used interchangeably.

In summary, the action $OutFib$ first behaves like $OutFibState $.
This changes the state:~it records in $y$ the next output value $x +
y$ and records in $x$ the value of the previously output value $y$.
This action also initialises the value of $next$ to be $x + y$.
Afterwards, $OutFib$ outputs the value of $next$ and then proceeds
recursively.

It is possible to give a simpler definition to $OutFib$.  As already
explained, we define the output using the schema component $next!$;
the following action outputs this value.  As a consequence, we have to
bring $next$ into scope, which we do using a local variable
declaration.  This construction is very useful for implicit
specifications, but here the output is deterministic, so we can write
the action as follows.
\begin{zed}
  OutFibState \defs [~ \Delta FibState | y'= x + y \land x' = y ~]
  \\
  OutFib \defs \mu X @ out!(x + y) \then OutFibState; X
\end{zed}
In this case, the action $OutFibState$ only changes the state.  As for
the action $OutFib$, it first outputs the value $x + y$, then it
changes the state, and finally proceeds recursively.

\subsection{Process Expressions}

A process definition like that of $Fib$ is explicit, in that it uses Z
and CSP constructs to define the state and the behaviour of the
process.  We can also use CSP operators to define processes in terms
of others previously defined.

We can use sequence, internal and external choice, parallelism, and
interleaving.  For instance, we can define a process $FibTwice$ as
follows.
\begin{zed}
  FibTwice \defs Fib \interleave Fib
\end{zed}
In this case we are using the interleaving operator:~communications in
either process occur independently, with no need for one process to
synchronise with the other.

The state of the resulting process includes all the components of the
state of the operand processes.  In the above example, since both
operands are $Fib$, the state of $FibTwice$ includes two copies of the
components of $FibState$.

As expected, the behaviour of the resulting process is defined by
composing the actions that determine the behaviour of the operand
processes using the operator applied.  The process $FibTwice$ outputs
the Fibonacci sequence through the channel $out$ twice.  The sequences
are merged in an unpredictable way, as the communications occur
independently.  Further examples of the use of the above process
operators are presented in the next section.

Our parallel operator is alphabetised:~an extra argument determines
the channels on which the operand processes are required to
synchronise (following~\cite{Ros98}, rather than~\cite{Hoa85}).
Communications through channels that are not listed occur
independently.

We can also use hiding to define a process.  In this case, the state
and behaviour of the resulting process is exactly like that of the
operand process, except only that communications through the specified
channels are hidden from the environment.

A \Circus\ process operator that is not available in CSP is that of
indexing.  We can, for instance, define a process as $i: T \odot P $.
It behaves exactly like $P$, but the channels it uses are different.
For each channel $c$ of $P$, we have a channel $c\_i$, which must be a
fresh channel name.  It communicates pairs of values:~the first
element, the index, is a value $i$ of type $T$, and the second element
is a value of the original type of $c$.  The declarations of the
channels $c\_i$ are an implicit consequence of the indexing operation.

In $i: T \odot P$, the index is a parameter; correspondingly, we have
an instantiation operation.  If $P$ is an indexed process, then
$P\lfloor e\rfloor$ behaves exactly like $P$, but communicates,
through all the channels, pairs whose first element is the value of
the expression $e$.  The value of $e$ is the value of the index.

For example, suppose we want to generate two Fibonacci sequences, but
we want to identify the generator of each element.  We consider the
process $i: \{ 1,2 \} \odot Fib$.  It outputs through channel $out\_i$
the pairs $(i,1)$, $(i,1)$, $(i,2)$, $(i,3)$, $(i,5)$, and so on,
where in each case $i$ is either 1 or 2.  The process $(i: \{ 1,2 \}
\odot Fib)\lfloor 1\rfloor$ produces pairs through $out\_i$ whose
first elements are 1; similarly for $(i: \{ 1,2 \} \odot Fib)\lfloor
2\rfloor$.  Finally, $(i: \{ 1,2 \} \odot Fib)\lfloor 1\rfloor
\interleave (i: \{ 1,2 \} \odot Fib)\lfloor 2\rfloor$ produces an
arbitrary merge of the two sequences of pairs:~the first element of
the pair identifies the generator and the second is a Fibonacci
number.

We can declare an arbitrary number of indexes of arbitrary types with
the indexing operator.  The communicated values are tuples whose last
elements are the values originally communicated and the others are
values for the indexes; the instantiation operation, therefore, may
take a list of index values as argument.  Partial instantiation is
possible.

In CSP, indexing is achieved by renaming.  Channels do not have
types:~a communication of a value 2 through a channel $c$ is regarded
as an event $c.2$.  To achieve the effect of a channel $c$ that can
communicate arbitrary natural numbers, we need the infinite set of
events containing $c.0$, $c.1$, $ c.2$, and so on.  Indexing in CSP
amounts to defining a process $l.P$, where $l$ is a label, or rather,
a name.  This renames all channels resulting in a process that engages
in $l.c$ when $P$ engages in $c$.  We can also apply an arbitrary
injective function $f$ on event names to a process $P $ to obtain
another process that uses channel $f(c)$ in the same way that $P$ uses
$c$.

In \Circus\ channels have a name and a type; the reason for this
distinction is the need for strong typing of communications in the
spirit of Z.  We, therefore, have two operations:~the first is that of
indexing, explained above, which changes the type of the channels; the
second is that of renaming.  In the process $P[oldc := newc]$, the
communications of $P$ through channel $oldc$ are done through the
channel $newc$, which is implicitly declared by this operation, if it
has not already been declared.  Usually, indexing and renaming are
used in conjunction.  An example is presented in the next section.

We can define iterated combinations of processes using the operators
of sequence, internal or external choice, parallelism, or
interleaving.  For example, if $P$ is an indexed process, then
$\Interleave i: T \odot P\lfloor i\rfloor$ is the process defined by
interleaving each of the processes $P\lfloor v\rfloor$ formed by
instantiating $P $ with a value $v$ of $T$.  Except for sequence, all
the above operators are commutative and associative, so there is no
concern about the order of the elements of $T$ or about the grouping
of the processes.  For the sequence operator, we require that $T$ is a
sequence and define $\Semi\ i: T \odot P\lfloor i\rfloor$ to be the
sequence of the processes $P\lfloor v\rfloor$, with $v$ taken from $T$
in the order that they appear.  We illustrate the use of iterated
operators in the next section.  For the indexed parallel operator, we
also need to provide the list of synchronising channels.

Parametrisation can also be used in the definition of processes.  The
parameters are variables that can be used in the process specification
as values of their declared type.  For example, we could define $Fib $
to be a parametrised process with parameters $a$ and $b$ of type
$\nat$.  These should be declared just before the process
specification and could be used, for instance, to initialise the state
components $x$ and $y$ as the first values output.  In this case, the
instantiated process $Fib(5,8)$ outputs the Fibonacci sequence
starting from its fourth element.

Instantiation of parametrised processes can also be combined in
iterated operators with sequence, internal or external choice,
parallelism, or interleaving.  For a parametrised process $P$, for
instance, we have that $\Interleave i: T @ P(i)$ is the process
defined by interleaving each of the processes $P(v)$ formed by
instantiating $P$ with a value $v$ of $T$.

We can also define generic processes; the mechanism is similar to that
of a generic schema in Z.  In the process $[X]P$, the name $X$ is a
generic parameter that can be used as a type in the definition of $P$.
Later references to $P$ need to define a value for this parameter.  It
can be either inferred from the context or defined explicitly as in
$P[\nat]$, where $X$ is defined to be the set of natural numbers.

\subsection{Actions}

As already explained, an explicit process definition contains both Z
and CSP constructs.  Mainly, we have a Z specification where some
paragraphs are action definitions.  There is also a main action which
defines the behaviour of the process.  As exemplified in the
definition of the $Fib$ process, an action can be a schema, a CSP
process, a guarded command, or a combination of these constructs.

A schema expression defining an operation over the process state is an
action.  It changes the state, but does not communicate any value.

The action $Skip$ terminates immediately, without communicating any
value or changing the state; the action $Stop$ deadlocks, and $Chaos$
diverges.  The prefixing operator is standard, but can be associated
with the guard construction.  The action $p \guard c?x \then A$, for
instance, inputs a value through channel $c$, assigns it to the
variable $x$, and then behaves like the action $A$, if the condition
$p$ is true; otherwise, it blocks:~the predicate $p$ is an enabling
condition.  A guard or firing condition may be associated with any
kind of action.

In an action, all free variables have to be in scope; the state
components are always in scope, and input communications introduce
further variables into scope.  Input variables, however, may not be
used as the targets of assignment statements.

We can also use the CSP operators of sequence, internal and external
choice, parallelism, interleaving, hiding, and recursion.  We observe
that CSP operators are used both at the level of processes and at the
level of actions.  The operations that apply, however, are different
in each case.  At the level of processes, we do not handle
communications and, for the sake of simplicity, we do not have
recursive definitions.  The usefulness and the modelling of recursive
processes is a topic for further research.

Like parametrised processes, parametrised actions declare variables
that can be used locally in their definition.  Instantiations of these
actions give fixed values to these parameters.  Iterated sequences,
internal and external choices, parallelisms, and interleavings can
also be used.  The resulting action is formed by applying the used
operator to combine all the actions obtained by considering the values
in the type of the index.

In the parallel composition $A \lpar C \rpar B$, the user state is
affected by both $A$ and $B$.  It is the responsibility of the
programmer to guarantee that no conflict arises.  As a general policy,
operations that modify the state should not be run in parallel with
other operations that also modify the state; this restriction is
enforced by occam, for instance.  We must observe, however, that a
\Circus\ process, as opposed to an action, encapsulates its state.
Therefore, we can run processes in parallel without the need to worry
about interference.

An action can also be defined using Dijkstra's guarded commands.  An
action can be an assignment, possibly multiple, or a guarded
alternation.  For example, using an assignment, the $InitFib$ action
of the process $Fib$ can be defined as follows.
\begin{zed}
  InitFib \defs out!1 \then out!1 \then x,y := 1,1
\end{zed}
As exemplified in the definition of the action $OutFib$, we can also
use variable blocks.  In the interest of supporting a calculational
approach to development, an action can also be a specification
statement in the style of Morgan's refinement calculus~\cite{Mor94}.

\section{Example: a reactive buffer} \label{section:EARB}

In this section, we give a specification in \Circus\ of a simple
bounded reactive buffer that is used to store natural numbers.  We go
on to describe a possible implementation as a ring of cells with a
central controller and a cached head.

\subsection{Abstract behaviour}

As already explained, \Circus\ specifications are sequences of
paragraphs containing Z paragraphs, channel definitions, channel set
definitions, or process definitions.  Typically, standard Z paragraphs
are used to define given sets and global constants that are used in
several process definitions throughout a specification.

The reactive buffer is bounded in its length:~it may hold no more than
$maxbuff$ elements.
\begin{axdef}
  maxbuff: \nat_1
\end{axdef}
It is sensible to require that it can hold at least one value.

Usually, the definition of a process is preceded by a declaration of
channels.
\begin{zed}
  \circchannel\ input, output: \nat
\end{zed}
The process $Buffer$ has two channels: $input$ and $output
$.

\subsubsection{Process state}

The state of a process is defined as in Z.  For the $Buffer$, there
are two state components:~the contents of the buffer and its size, a
derived component.

\begin{zed}
  \circprocess\ Buffer \defs \circbegin
  \also %
  BufferState \defs [~ buff: \seq \nat; size: 0 \upto maxbuff | size =
  \# buff \leq maxbuff ~]
\end{zed}
The size of the buffer has to be less than or equal to $maxbuff$.

\subsubsection{Process actions}

In describing a \Circus\ process, a series of actions is defined,
showing the interaction between external event behaviour and
transformations on the internal state.  For the $Buffer$, there are
three actions:~initialisation, input, and output.  Initialisation sets
the buffer to empty.
\begin{zed}
  BufferInit \defs [~ BufferState' | buff' = \langle\rangle \land
  size' = 0 ~]
\end{zed}
The $Input$ action is enabled if there is space in the buffer for the
new input; the corresponding input operation on the state has this as
its precondition.  The new element is appended to the buffer's
contents and the size is updated.
\begin{schema}{InputCmd}
  \Delta BufferState
  \\ %
  x?: \nat
  \where %
  size < maxbuff
  \\ %
  buff' = buff \cat \langle x? \rangle \land size' = size + 1
\end{schema}
\begin{zed}
  Input \defs size < maxbuff \guard input?x \then InputCmd
\end{zed}
The action $Input$ guarantees the precondition by being guarded.

The output action is enabled if there is something in the buffer.  The
first element in the buffer is output; the remaining elements are
retained in order; the size of the buffer is suitably updated.
\begin{schema}{OutputCmd}
  \Delta BufferState
  \where %
  size > 0
  \\ %
  buff'= tail~buff \land size' = size - 1
\end{schema}
\begin{zed}
  Output \defs size > 0 \guard output!(head~buff) \then OutputCmd
\end{zed}
The output is actually defined in the $Output$ action.

In every \Circus\ process, there is always an unnamed action that
defines the externally-visible behaviour.
\begin{zed}
  @ BufferInit; \mu X @ (~ Input \extchoice Output ~) ; X
  \\ %
  \circend
\end{zed}
First, the buffer is initialised, and then it loops, offering to the
environment the choice to input and output.  The guards of these
actions guarantee that if the buffer is full, the input is blocked,
and if it is empty, the output is blocked.

\subsection{A cached-head ring buffer}

A well-known implementation of a \textsc{fifo}-buffer uses a ring:~a
circular array, with two indexes showing where the first and last
elements reside.  Our refinement uses a ring of cells, each
implemented as a \Circus\ process and holding just a single element,
and a central controller that keeps track of the indexes of the first
and last elements, offering the input and output services.

An interesting problem arises immediately:~the buffer must not refuse
to output if it is non-empty, so how should this be achieved?  One way
is to distribute control around the ring, so that the cell owning the
head of the buffer is enabled for output, but the others are disabled.
The cost of this solution is the overhead of the protocol for
distributed control.

Another solution is to cache the head of the ring in the controller,
and distribute only the tail of the buffer around the ring.  The
resulting protocol is very simple.

\subsubsection{Controller process}

We start the \Circus\ description of the cached ring with a
specification of the $Controller$ process.  It specifies the behaviour
of the buffer, but does not contain the buffer itself in its state,
just the cache; the ring is specified by another process.

First, there are two global declarations to define the indexes of the
ring cells.  The maximum size of the ring is one less than the size of
the buffer, as the head is cached.
\begin{axdef}
  maxring: \nat
  \where %
  maxring = maxbuff - 1
\end{axdef}
The indexes of the ring go from 0 to $maxring - 1$.
\begin{zed}
  RingIndex == 0 \upto maxring - 1
\end{zed}
The channels $input$ and $output$ are the external buffer channels.
The channels $read$ and $write$ are used for communication with the
ring cells.
\begin{zed}
  \circchannel\ read, write: RingIndex \times \nat
\end{zed}
The values communicated through $read$ and $write$ are pairs, where the
first element identifies a cell and the second element is the natural
number actually communicated.

\subsubsection{Controller state}

The state of the controller contains the size of the buffer, the size
of the ring, the cache, and two ring indexes, $top$ and $bot$, keeping
track of the index of the next available position and the index of the
first value stored, respectively.
\begin{zed}
  \circprocess\ Controller \defs \circbegin
\end{zed}
\begin{schema}{ControllerState}
  size: 0 \upto maxbuff
  \\ %
  ringsize: 0 \upto maxring
  \\ %
  cache: \nat
  \\ %
  top, bot: RingIndex
  \where %
  ringsize = max \{ 0, size - 1 \}
  \\ %
  ringsize \mod maxring = (top - bot) \mod maxring
\end{schema}
The size of the ring \pagebreak may be computed from the positions of
the $top $ and $bot$ indexes:~the number of elements between the two
indexes is $(top - bot) \mod maxring$.  Since this confuses two values
of the state (when the ring is full and when the ring is empty), it is
necessary to add the first equation that relates $ringsize$ and $size$.

\subsubsection{Controller actions}

Initially, the buffer is empty and so has zero size; we choose some
suitable values for $top$ and $bot$.
\begin{zed}
  InitController \defs [~ ControllerState' | size' = 0 \land bot' = 0
  \land top' = 0 ~]
\end{zed}
The input action depends on whether the buffer is empty or not.  If it
is empty, then the input must be kept in the cache; if it is
non-empty, then it must be passed on to the appropriate ring cell; if
it is full, then no input action is possible.

When the input is cached, the $top$ and $bot$ indexes do not change.
\begin{schema}{CacheInput}
  \Delta ControllerState
  \\ %
  x?: \nat
  \where %
  size = 0
  \\ %
  size' = 1 \land cache' = x?
  \\ %
  bot' = bot \land top' = top
\end{schema}
When the input is passed on to the ring, the $top$ index advances.
\begin{schema}{StoreInput}
  \Delta ControllerState
  \\ %
  \where %
  size > 0
  \\ %
  size' = size + 1 \land cache' = cache
  \\ %
  bot' = bot \land top' = (top + 1) \mod maxring
\end{schema}
The input action is enabled when there is space in the buffer; the
subsequent behaviour depends on whether the buffer is empty or not.  In
the case that it is non-empty, the controller transmits the input $x$
to the cell at the $top$ of the ring using the communication
$write.top!x$.
\begin{zed}
  InputController \defs
  \\ %
  \t1
  \begin{block}
    size < maxbuff \guard input?x \then
    \\ %
    \t1 \begin{block}
      size = 0 \guard CacheInput
      \\ %
      \extchoice
      \\ %
      size > 0 \guard write.top!x \then StoreInput
    \end{block}
  \end{block}
\end{zed}
We observe that there is not, for this action, a one-to-one
correspondence between the communication $input?x$ and operations that
change the state.

There is a similar case analysis for output:~the output always comes
from the cache, which must be replaced if the ring is non-empty.  In
the case that the ring is empty, we have $size = 1$; $size$ is reset;
nothing else changes.
\begin{zed}
  NoNewCache \defs [~ \Delta ControllerState | size = 1 \land size' =
  0 \land bot' = bot \land top' = top~]
\end{zed}
If the ring \pagebreak is non-empty, then a new element~(obtained from
the ring) is stored in the cache; $bot$ must be advanced.
\begin{schema}{StoreNewCache}
  \Delta ControllerState
  \\ %
  x?: \nat
  \where %
  size > 1
  \\ %
  size' = size - 1 \land cache' = x?
  \\ %
  bot' = (bot + 1) \mod maxring \land top' = top
\end{schema}
So, the output action is enabled when there is something in the
buffer; the subsequent behaviour depends on whether the ring is empty
or not.  In the case that the ring is non-empty, the controller
obtains the input $x$ from the cell at the $bot$ of the ring using the
communication $read.bot?x$.
\begin{zed}
  OutputController \defs
  \\ %
  \t1
  \begin{block}
    size > 0 \guard output!cache \then
    \\ %
    \t1
    \begin{block}
      size > 1 \guard read.bot?x \then StoreNewCache
      \\ %
      \extchoice
      \\ %
      size = 1 \guard NoNewCache
    \end{block}
  \end{block}
\end{zed}
Neither of the operation schemas describe the value that gets
communicated on the $output$ channel; it is the $cache$ that gets
output, and it is natural to describe this directly in the CSP part of
the action.

\subsubsection{Controller behaviour}

We conclude the description of the controller by describing its
overall behaviour:~after being initialised, it repeatedly offers
inputs and outputs.
\begin{zed}
  ControllerCycle \defs \mu X @ (~ InputController \extchoice
  OutputController ~); X
  \also %
  @ InitController; ControllerCycle
  \also %
  \circend
\end{zed}
We have now to specify the ring.

\subsubsection{Ring cell process}

Each ring cell has a $rd$ and a $wrt$ channel.
\begin{zed}
  \circchannel\ rd, wrt: \nat
\end{zed}
A cell contains one value.
\begin{zed}
  \circprocess\ RingCell \defs \circbegin\ CellState \defs [~ val:
  \nat ~]
\end{zed}
There are just two actions on the ring cell state.  The $Read$ action
is so simple that we do not bother to define a corresponding operation
on the state:~it merely outputs $val$.
\begin{zed}
  Read \defs rd!val \then Skip
\end{zed}
The $Write$ action updates $val$.
\begin{zed}
  CellWrite \defs [~ \Delta CellState; x?: \nat | val' = x? ~]
  \also %
  Write \defs wrt?x \then CellWrite
\end{zed}
The ring cell \pagebreak starts its behaviour with a $Write$ action;
subsequently, it allows either $Read$ or $Write$ actions.
\begin{zed}
  @ Write; \mu X @ ( Read \extchoice Write ); X
  \also %
  \circend
\end{zed}
The ring itself is formed by composing the cells in parallel.

\subsubsection{The cached-head ring buffer's behaviour}

Our first step is to assemble the ring.  From $RingCell$, we define an
indexed process, with indexes taken from the set of ring indexes.
\begin{zed}
  \circprocess\ IRingCell \defs (i: RingIndex \odot RingCell)[rd\_i,
  wrt\_i := read, write]
\end{zed}
The resulting indexed process operates on channels $rd\_i$ and
$wrt\_i$ of type $RingIndex \times \nat$.  We rename them to $read$
and $write$, respectively. The behaviour of an indexed ring cell is
exactly the same as that of a ring cell, except that the
communications $rd!val$ and $wrt?x$ are replaced by $read.i!val$ and
$write.i?x$, respectively.

There is no interaction between the ring's cells, so the ring is
constructed by interleaving the indexed ring cells.
\begin{zed}
  \circprocess\ Ring \defs \Interleave i: RingIndex \odot
  IRingCell\lfloor i\rfloor
\end{zed}
Finally, the cached-head ring is composed by putting the controller
and the ring in parallel, interacting along the internal $read$ and
$write$ channels.
\begin{zed}
  \circprocess\ CRing \defs (~ Controller \lpar \lchanset read, write
  \rchanset \rpar Ring ~) \hide \lchanset read, write \rchanset
\end{zed}
This completes our specification.

\section{Unifying Theories of Programming} \label{section:TUT}

In this section, we give an overview of Hoare \& He's unifying theory
of programming~\cite{HH98}, where the theory of relations is used as a
unifying basis for the science of programming across many different
computational paradigms: procedural and declarative, sequential and
parallel, closely-coupled and distributed, and hardware and software.
Programs, designs, and specifications are all interpreted as relations
between an initial observation and a single subsequent (intermediate
or final) observation of the behaviour of a device executing a
program.

In their unification, different theories share common ideas:
sequential composition is relational composition; the conditional is a
simple Boolean connective; nondeterminism is disjunction; and
parallelism is a restricted form of conjunction.  The miracle of the
refinement calculus is the empty relation and abortion is the
universal relation.  The definition of assertions as
conditional-aborts brings all of assertional reasoning within the
scope of the theory.  Both correctness and refinement are interpreted
as inclusion of relations, and all the laws of a relational calculus
are valid for reasoning about correctness in all theories and in all
languages.

Particular design calculi and programming languages are differentiated
by their alphabet, signature, and healthiness conditions.  The
\emph{alphabet} of a theory gives names for a range of external
observations of program behaviour.  By convention, the name of an
initial observation is undecorated, but the name of a similar
observation taken subsequently is decorated with a dash.  This allows
a relation to be expressed as in Z by its characteristic predicate.
The \emph{signature} provides syntax for denoting the objects of the
theory.  It names the relations corresponding to primitive operations
directly, and provides operators for combining them.  The
\emph{healthiness conditions} select the objects of a sub-theory from
those of a more expressive theory in which it is embedded.  Thus
programs form a subset of designs, and designs form a subset of
specifications.

The alphabet of each theory contains variables to describe all aspects
of program behaviour that are considered relevant.  In a purely
procedural paradigm, these stand for the initial and final values of
the global variables accessed and updated by a program block.  Some
variables are \emph{external}, because they are globally shared with
the real world in which the program operates, and so they cannot be
declared locally.  The first example is the Boolean variable
$okay$:~it means that the system has been properly started in a stable
state; $okay'$ means subsequent stabilisation in an observable state.
This permits a description of programs that fail due to nonterminating
loops or recursion.

In a theory of reactive processes, the variable $tr$ records past
interactions between a process and its environment; and the Boolean
variable $wait$ distinguishes the intermediate observations of waiting
states from final observations of termination.  During a wait, the
process can refuse to participate in certain events offered by the
environment; these are specified by the variable $ref$.

The operator that introduces concurrency into programs is one that
presents the greatest challenge for unification.  The solution that
has been found uniformly successful is to translate all forms of
parallelism into a disjoint form where it can be modelled simply by a
conjunction of relations.  Parallel composition operators $P \parallel
Q$ are, for instance, uniformly described as though each component
process operates on its own differently subscripted version of the
external variables.  The initial values of the pair of variables are
identical.  The value of the decorated, final, variables is the result
of the merge of the subscripted variables by an associative, symmetric
operator.

The signature of a theory varies in accordance with its intended use,
whether in specification, in design, or in programming.  A
specification language has the least restricted signature; for
example, it includes quantifiers and all the operators of the
relational calculus, even the converse.  Design calculi successively
remove unimplementable operators, starting with negation; all
operators are then monotonic, and recursion can safely be introduced
as a fixed-point operator.  In the programming language, only
implementable operations are left.  Operations of the chosen
programming language are defined in terms of their observable effects
using the more general specification language.

A healthiness condition, rather like a conservation law in physics,
distinguishes feasible descriptions of reality from infeasible ones.
By a suitable restriction of signature, design languages satisfy many
healthiness conditions, and programming languages satisfy even more.
There are typically healthiness conditions associated with each of the
external variables, and with groups of related variables.  Hoare \& He
have shown that all healthiness conditions of interest may be
expressed in the form $P = \phi(P)$, where $\phi$ is an idempotent
function mapping all relations to the healthy ones of a particular
theory.  These idempotents are called \emph{links}, and they may be
used to translate higher-level designs into lower level
implementations.

The laws of CSP are not true for all predicates over the observational
variables; there are eight healthiness conditions for CSP
processes:~three characterise \emph{reactive} processes in general;
two constrain reactive processes to be CSP ones; and a further three
are more specific still.  The first healthiness condition for a
reactive process~($\healthiness{R1}$) is that its execution can never
undo any event that has already been performed.  The second
healthiness condition ($\healthiness{R2}$) requires that a reactive
process's behaviour is oblivious of what has gone before.  The third
healthiness condition ($\healthiness{R3}$) is designed to make
sequential composition work as intended.  Suppose that we have the
sequential composition of two processes $P_1$ and $P_2$.  When $P_1$
terminates, the value of $wait'$ is false; therefore, $P_2$'s value of
$wait$ is also false, and control passes from $P_1$ to $P_2$.  On the
other hand, if $P_1$ is still waiting for interaction with its
environment, then its value of $wait'$ and $P_2$'s value of $wait$ are
both true; in this case, $P_2$ must leave the state unchanged.  Of
course, this is sensitive to the previous process's stability:~if
activated when $okay$ is false, then its only guarantee is to extend
the trace.

$\healthiness{R}$ is the set of reactive processes, which satisfy
these first three healthiness conditions.  An interesting subset of
$\healthiness{R}$ satisfies two additional conditions.  The first
($\healthiness{CSP1}$) states that, if a process has not started, then
we can make no prediction about its behaviour.  The second
($\healthiness{CSP2}$) states that we cannot require a process to
abort; this is characterised by the monotonicity of the $okay'$
variable.

$\healthiness{CSP}$ is the set of reactive processes that satisfy
these two healthiness conditions.  Further healthiness conditions are
required to capture the standard theory of CSP; each of them is given
as a simple unit law.  The first ($\healthiness{CSP3}$) requires that
a CSP process does not depend on the initial value of the $ref$
variable when $wait$ is false.  Of course, when $wait$ is true, then
it must behave as required by $\healthiness{R3}$.  The second
($\healthiness{CSP4}$) requires that the $ref'$ variable is irrelevant
after termination.  The third ($\healthiness{CSP5}$) requires that
refusal sets must be subset-closed.

The model we define in the next section for \Circus\ is based on this
theory.  We express this model in Z, which is a very suitable language
for the specification of relations.

\section{The Model of \Circus} \label{section:TMOC}

Our model for a \Circus\ program is a Z specification; the model of a
process is itself a Z specification, and the model of an action is a
schema.  With just a few syntactic adjustments, we can use
Z/EVES~\cite{ZEVES00} to analyse the model of \Circus\ as defined
here.

\subsection{Channel environment}

The semantics of a process depends on the channels in scope, which are
recorded in a channel environment.  We not only use Z specifications
and paragraphs as models, but also use the Z mathematical notation as
a meta-language to define the association of \Circus\ constructs to
these models.  The set $ChanName$ is a given set containing the valid
channel names.  The channel environment associates a channel name to
its type.
\begin{zed}
  ChanEnv == ChanName \ffun \mathsf{Expression}
\end{zed}

The channel environment corresponding to a \Circus\ program is
determined basically by its channel definitions.  The semantic
function \mbox{$\lbag\_ \rbag^{\cal CD}$} gives the meaning of channel
definitions as channel environments.
\begin{zed}
  \lbag\_ \rbag^{\cal CD}: \mathsf{ChannelDefinition} \pfun ChanEnv
  \also %
  \lbag \circchannel\ cdecl \rbag^{\cal CD} = \lbag cdecl \rbag^{\cal
    C}
\end{zed}
The function \mbox{$\lbag\_ \rbag^{\cal C}$} determines the
environment defined by a channel declaration.
\begin{zed}
  \lbag\_\rbag^{\cal C}: \mathsf{CDeclaration} \fun ChanEnv
\end{zed}
If a channel declaration just gives a channel name, but no type, this
channel is not used for communication, but only as a synchronising
event.  In this case, its type is recorded as $Sync$.
\begin{zed}
  \lbag n\rbag^{\cal C} = \{ n \mapsto Sync \}
\end{zed}
If the type is given in the channel declaration, it is recorded in the
environment.
\begin{zed}
  \lbag n: T \rbag^{\cal C} = \{ n \mapsto T \}
\end{zed}
If $n$ is a list of channel names, these definitions can be
generalised in a straightforward way.
\newpage

If the channel declaration is a schema expression, then the channels
declared are the components of this schema; it should not have a
predicate part, which is ignored here.
\begin{zed}
  \lbag sExp \rbag^{\cal C} =  \lbag SComp~sExp \rbag^{\cal C}
\end{zed}
The syntactic function $SComp$, when applied to a schema expression
$sExp$, yields the declaration of the components of the schema denoted
by $sExp$.  It is the environment corresponding to this declaration
that we record.

Finally, if the channel declaration actually introduces several
channels of possibly different types, we accumulate a record of all
these channels.
\begin{zed}
  \lbag sdecl; cdecl\rbag^{\cal C} = \lbag sdecl \rbag^{\cal C} \oplus
  \lbag cdecl \rbag^{\cal C}
\end{zed}
Redefinition of channel names is not a concern; the latest definition
is the one that matters.

\paragraph{Channel sets.}

A channel is not a value in our model.  For this reason, we cannot
define a set of channels.  In a \Circus\ program channel sets are used
to abbreviate process expressions like parallelism and hiding.  We
assume, therefore, that these process expressions are expanded by
replacing references to channel sets with the set of channels it
defines.  The channel set definitions can then be eliminated.

\subsection{Process environment}

A process definition may refer to other processes previously defined.
Therefore, its semantics also depends on a process environment, which
records associations of process names to their models.  The given set
$ProcName$ contains the valid process names, and
$\mathsf{ZSpecification}$ is the syntactic category of Z
specifications.  The set $ProcEnv$ is that of the finite sequences of
pairs that associate process names to their models. We use sequences
because the order in which processes are declared is relevant.  The Z
specification corresponding to a whole program includes the Z
specifications corresponding to the individual processes in the order
that they appear.
\begin{zed}
  ProcEnv == \seq (ProcName \times \mathsf{ZSpecification})
\end{zed}

The following relations inspect a process environment.  $N\ recorded\
\rho$ holds if the process name $N$ is recorded in the environment
$\rho$.
\begin{axdef}
  \_ recorded \_: \power (ProcName \times ProcEnv)
  \where %
  \forall N: ProcName; \rho: ProcEnv @ N\ recorded\ \rho \iff N \in
  \ran (\rho \comp fst)
\end{axdef}
The function $modelOf\ N\ in\ \rho$ gives the Z specification recorded
in $\rho$ as the model of $N$.
\begin{axdef}
  modelOf \_ in \_: ProcName \times ProcEnv \pfun
  \mathsf{ZSpecification}
  \where %
  \forall N: ProcName; \rho: ProcEnv @ N\ recorded\ \rho \implies
  (N,modelOf\ N\ in\ \rho) \in \ran \rho
\end{axdef}
Of course, $modelOf\ N\ in\ \rho$ is only defined if $N$ is actually
recorded in $\rho$.

A process definition enriches the environment by associating a process
name to the Z specification corresponding to the declared process.
Moreover, if the process specification involves indexing or channel
renaming, new channel names are implicitly declared, if they have not
yet.  Therefore, a process definition may also change the channel
environment.  The semantic function \mbox{$\lbag\_ \rbag^{\cal PD} $}
gives the meaning of a process definition as a process environment
that records just the single process it declares, and a channel
environment recording the new channels it introduces, if any.
\begin{zed}
  \lbag\_ \rbag^{\cal PD}: \mathsf{ProcessDefinition} \pfun ChanEnv
  \pfun ProcEnv \pfun (ChanEnv \times ProcEnv)
  \also %
  \forall \gamma: ChanEnv; \rho: ProcEnv @
  \\ %
  \t1 \lbag \circprocess\ N \defs P \rbag^{\cal PD}\gamma\ \rho = \LET
  Ps == \lbag P \rbag^{\cal P}\gamma\ \rho @ (Ps.1,\langle (N,
  Ps.2)\rangle)
\end{zed}
The semantics $Ps$ of $P$ is taken in the current channel and process
environments; it is a pair containing a channel environment and a Z
specification.  The semantics of the process definition is the pair
formed by the channel environment and the process environment that
associates $N$ to the Z specification.  The function \mbox{$\lbag\_
  \rbag^{\cal P}$} gives the meaning of processes and is defined later
on.

\subsection{Programs}

The meaning of a program is given by the function \mbox{$\lbag\_
  \rbag^{\cal PROG}$}.
\begin{zed}
  \lbag\_ \rbag^{\cal PROG}: \mathsf{Program} \pfun
  \mathsf{ZSpecification}
\end{zed}
The specification corresponding to a complete \Circus\ program $prog$
is formed mainly by its Z paragraphs and the paragraphs of the
specifications that model its process declarations.  More precisely,
the specification starts with the following four paragraphs.  We need
to define a boolean type, which is not directly available in Z; we use
variables of this type as predicates, for the sake of simplicity.
\newcommand{\freetypeFalse}{\freetype{False}}
\newcommand{\freetypeTrue}{\freetype{True}}
\begin{zed}
  Bool ::= \freetypeFalse | \freetypeTrue
\end{zed}
The second paragraph declares the given sets $Sync$ and $Event$.  The
first is the type of the synchronisation events, which are declared as
untyped channels.  The given type $Event$ includes the possible
communications of the program.  A communication is characterised by
the channels through which it occurs and the value communicated.  As
we formalise later on, as channels are declared, we define injective
functions into $Event$ that determine the communications that can
occur through those channels.
\begin{zed}
  [ Sync, Event ]
\end{zed}
The third paragraph is a schema that specifies the components that
comprise the state of a process, in addition to the user state
components.  These are the variables of the unifying theory model
presented in Section~\ref{section:TUT} and the additional $trace$
variable.  Our processes do not have an alphabet as in~\cite{HH98};
instead we consider the alphabetised parallel operator
of~\cite{Ros98}.
\begin{zed}
  ProcessState \defs [~ trace,tr: \seq Event; ref: \power Event; okay,
  wait: Bool ~]
\end{zed}
Changes to the process state are constrained as specified in the
fourth paragraph:~valid process observations increase the trace.  The
extra component $trace$ records the events that occurred since the
last observation.
\begin{zed}
  ProcessStateObs \defs [~ \Delta ProcessState | tr \prefix tr' \land
  trace' = tr' - tr ~]
\end{zed}
The remaining paragraphs are determined by $\lbag prog \rbag^{\cal
  CPARL}\ \emptyset\ \emptyset$.  This is the semantics of the list of
paragraphs that compose the program itself, taken in the empty process
environment and the empty channel environment.

\subsection{Paragraphs}

A \Circus\ paragraph can contribute to the Z specification defined by
the whole program in three ways:~by extending the Z specification with
new paragraphs, by extending the channel environment, and by extending
the process environment.  Therefore, the function \mbox{$\lbag\_
  \rbag^{\cal CPAR}$}, which defines the semantics of a \Circus\
paragraph, yields a triple: a Z specification, a channel environment,
and a process environment.
\begin{zed}
  \lbag\_ \rbag^{\cal CPAR}: \mathsf{CircusParagraph} \pfun ChanEnv
  \pfun ProcEnv \pfun
  \\ %
  \t1 (\mathsf{ZSpecification} \times ChanEnv \times ProcEnv)
\end{zed}
For a Z paragraph $Zp$, the definition of \mbox{$\lbag\_ \rbag^{\cal
    CPAR}$} is as follows.
\begin{zed}
  \lbag Zp \rbag^{\cal CPAR}\ \gamma\ \rho = (tc~Zp,\gamma,\rho)
\end{zed}
Essentially, a Z paragraph is added to the Z specification as it is,
and does not affect the channel or the process environment.  Slight
changes to the Z paragraph may be needed because of schemas with
untyped components, which are assumed to be synchronisation event
declarations.  The function $tc$, when applied to a schema that
declares such component, yields the schema obtained by declaring the
types of these components to be $Sync$.  When applied to any other
sort of Z paragraph, $tc$ keeps it as it is:~behaves like the
identity.

A channel definition $cd$ gives rise to a few paragraphs in the Z
specification and enriches the channel environment.
\begin{zed}
  \lbag cd \rbag^{\cal CPAR}\ \gamma\ \rho = \LET \gamma' == \lbag cd
  \rbag^{\cal CD} @ (events~\gamma',\gamma \oplus \gamma',\rho)
\end{zed}
The environment $\gamma'$ records the channels declared in $cd$ and is
used to enrich the current environment.  For each channel $c$ recorded
to have type $T$ different from $Sync$ in the environment $\gamma'$,
we have that $event~\gamma'$ yields an axiomatic description that
declares $c$ to be an injective function from $T$ to $Event$.
\begin{axdef}
  c: T \inj Event
\end{axdef}
If $T$ is $Sync$, we define $c$ to be itself an event.
\begin{axdef}
  c: Event
\end{axdef}
These constants and injective functions are $Event$ constructors.

A process definition $pd$ determines a Z specification of its model in
the unifying theory, and enriches the process environment and possibly
the channel environment as well.
\begin{zed}
  \lbag pd \rbag^{\cal CPAR}\ \gamma\ \rho = \LET pds == \lbag pd
  \rbag^{\cal PD}\ \gamma\ \rho @
  \\ %
    \quad
      ((pds.2~1).2,\gamma \oplus pds.1,(\rho \filter \{n: ProcName;
        s: {\sf ZSpecification} | n \neq (pds.2\ 1).1\}) \cat pds.2)
\end{zed}
The semantics of $pd$ is a pair $pds$ containing the channel
environment that records the channels $pd$~(implicitly) declares, and
a process environment that records the process defined by $pd$.  The Z
specification corresponding to the process is the second element of
the pair in the first and unique position of the process environment,
which is itself the second element of $pds$.

The function \mbox{$\lbag\_ \rbag^{\cal CPARL}$} gives meaning to
lists of \Circus\ paragraphs as a Z specification.
\begin{zed}
  \lbag\_ \rbag^{\cal CPARL}: \mathsf{CircusParagraph^*} \pfun ProcEnv
  \pfun ChanEnv \pfun \mathsf{ZSpecification}
\end{zed}
If the list contains only one paragraph $cp$, then the specification
is the first element of the tuple determined by its semantics.
\begin{zed}
  \lbag cp \rbag^{\cal CPARL}\ \gamma\ \rho = (\lbag cp \rbag^{\cal
    CPAR}\ \gamma\ \rho).1
\end{zed}
If, on the other hand, we \pagebreak have a list $cp\ cpl$ containing
more than one \Circus\ paragraph, then the Z specification is formed by
the paragraphs corresponding to the first \Circus\ paragraph $cp$,
followed by the paragraphs corresponding to the rest of the list of
\Circus\ paragraphs $cpl$.  The semantics of these paragraphs, however,
is taken in enriched channel and process environments that record the
declaration(s) in the first \Circus\ paragraph.
\begin{zed}
  \lbag cp~cpl \rbag^{\cal CPARL}\ \gamma\ \rho = \LET cps == \lbag cp
  \rbag^{\cal CPAR}\ \gamma\ \rho @ cps.1\ (\lbag cpl \rbag^{\cal
    CPARL}\ cps.2\ cps.3)
\end{zed}
The semantics $cps$ of the first \Circus\ paragraph $cp$ is a
tuple:~the first element is the Z specification, which is part of the
specification corresponding to $cp~cpl$; and the second and the third
elements are the channel and process environments, which are used in
the evaluation of the following paragraphs.

We eliminate repeated names used across different process definitions
by prefixing each name with the name of the process in which it is
declared.  Also, in the model of a process, where several schemas are
defined, the names of these schemas should be fresh.  Below, in the
presentation of this model we leave this assumption implicit.

\subsection{Processes}

The semantic function \mbox{$\lbag\_ \rbag^{\cal P}$} gives the
meaning of a process declaration as a pair containing a channel
environment and a Z specification.
\begin{zed}
  \lbag\_ \rbag^{\cal P}: \mathsf{Process} \pfun ChanEnv \pfun ProcEnv
  \pfun (ChanEnv \times \mathsf{ZSpecification})
\end{zed}
The function takes the current channel and process environments as
arguments.  For a process name $N$, its definition is as follows.
\begin{zed}
  \lbag N \rbag^{\cal P}\gamma\ \rho = (\gamma,modelOf\ N\ in\ \rho)
\end{zed}
A well-formed process definition does not refer to undeclared
processes, so we can assume that $N$ is in $\rho$.

For an explicit process specification we define
\begin{zed}
  \lbag \circbegin\ ppl @ A\ \circend\rbag^{\cal P}\ \gamma\ \rho
\end{zed}
as a Z specification containing the following paragraphs.
\begin{itemize}
\item $ProcObs$, a schema describing the observations that may be made
  of the process.
\item The Z paragraphs as they are, except for those that are schemas
  that define operations as these are actions.
\item For each action, a schema constraining the process observations.
\end{itemize}
In order to define $ProcObs$ we specify a schema $State$ that defines
the process state.  It includes the components of the schema
$ProcessState$ previously defined and those of the state defined in
$ppl$.
\begin{zed}
  State \defs UserState \land ProcessState
\end{zed}
We assume that the state schema in $ppl$ is called $UserState$.  In
summary, the overall state of a \Circus\ process in our model includes
the components of the state in its specification, which we refer to as
user state, and the observation variables of the Unifying Theory.

A process observation corresponds to a state change.
\begin{zed}
  ProcObs \defs \Delta UserState \land ProcStateObs
\end{zed}
Restrictions on changes to both the user state and the observation
variables apply.

As we explain later on, the state can be extended by the declaration
of extra variables.  Therefore, we actually consider a family of
schemas $ProcObs(USt)$; for a schema $USt$, $ProcObs(USt)$ is the
schema defined as $ProcObs$ above, except that it includes
\mbox{$\Delta USt$}, instead of \mbox{$\Delta UserState$}.  The
semantics of \mbox{$\circbegin\ ppl @ A\ \circend$} includes exactly
the schema $ProcObs$ defined above, or rather, $ProcObs(UserState)$.

\subsection{Actions}

To each action \mbox{$N \defs A$} corresponds a schema named $N$.
\begin{zed}
  N \defs \lbag A \rbag^{\cal A}\ \gamma\ UserState\
\end{zed}
It is determined by the function \mbox{$\lbag\_ \rbag^{\cal A}$} that
takes as arguments the current channel environment and the name of the
schema that defines the user state.  As shown above, for the actions
in $ppl$, the parameter given is $UserState$, the state defined in
$ppl$ itself.
\begin{zed}
  \lbag\_ \rbag^{\cal A}: \mathsf{Action} \pfun ChanEnv \pfun
  \mathsf{N} \pfun \mathsf{Schema\hbox{-}Exp}
\end{zed}

The main action is nameless; the schema corresponding to it is given a
fresh name.  This schema is also determined by the function $\lbag\_
\rbag^{\cal A}$ as above.  We distinguish three cases in the
definition of the behaviour of an action:~the normal case, the cases
in which the previous operation diverged, and the case in which the
previous operation has not terminated.
\begin{zed}
  \lbag A \rbag^{\cal A}\gamma\ USt = \lbag A \rbag^{\cal A_N}\gamma\
  USt \lor Diverge(USt) \lor Wait(USt)
\end{zed}
The function \mbox{$\lbag\_ \rbag^{\cal A_N}$} characterises the
normal behaviour of an action.
\begin{zed}
  \lbag\_ \rbag^{\cal A_N}: \mathsf{Action} \pfun ChanEnv \pfun
  \mathsf{N} \pfun \mathsf{Schema\hbox{-}Exp}
\end{zed}
It is defined by induction below.

The schema, or rather, the family of schemas $Diverge(USt)$
characterise the behaviour of an action in the presence of divergence.
\begin{zed}
  Diverge(USt) \defs [~ ProcObs(USt) | \lnot okay ~]
\end{zed}
Divergence is characterised by the fact that $okay$ is false.  There
are no guarantees as to the final value of any of the state
components, except only that the $trace$ can only be extended: a
restriction enforced by $ProcObs$.

For $Wait(USt)$ we have the following definition.
\begin{zed}
  Wait(USt) \defs [~  \Xi State(USt) |   okay \land wait ~]
\end{zed}
The waiting state is characterised by the fact that both $okay$ and
$wait$ are true: there is no divergence, but the previous action has
not terminated.  In this case, the state does not change.

The normal case of the actions behaviour is characterised by the
schema below.
\begin{zed}
  Normal(USt) \defs [~ ProcObs(USt) | okay \land \lnot wait ~]
\end{zed}
In this case, $okay$ is true, but $wait$ is false:~there is no
divergence and the previous action has terminated.

In the sequel, we define the normal behaviour of the actions.

\subsubsection{Schema Expressions}

For a schema expression $SExp$, we have the following definition.
\begin{zed}
  \lbag SExp \rbag^{\cal A_N}\gamma\ USt = SExp \land OpNormal \lor
  OpDiverge
\end{zed}
The schema $OpNormal$ characterises the case in which $SExp$ is
activated in a state that satisfies its precondition;~the $trace$ is
not modified and $okay$ and $wait$ do not change;~the operation
terminates successfully.
\begin{zed}
  OpNormal \defs [~ Normal(USt) | trace' = \langle\rangle \land okay'
  \land \lnot wait' ~]
\end{zed}
If, on the other hand, the precondition of $SExp$ is not satisfied,
the action diverges.  This case is specified by $OpDiverge$.
\begin{zed}
  OpDiverge \defs [~ Normal(USt); SExp \lor \lnot SExp | \lnot \pre
  SExp \land \lnot okay' ~]
\end{zed}
We include $SExp \lor \lnot SExp$ to put possible input and output
variables in scope.  The output variables $o!$ need to be renamed to
$o'$, since in \Circus\ the shriek and dash decorations are
interchangeable.

\subsubsection{CSP Expressions}

The definition of the normal behaviour of $Skip$ is as follows.
\begin{zed}
  \lbag Skip \rbag^{\cal A_N}\gamma\ USt = [~ Normal(USt) \land \Xi
  USt | trace' = \langle\rangle \land okay' \land \lnot wait' ~]
\end{zed}
The user state is not changed, the trace is also not changed, and it
terminates.

For $Stop$, we have the following definition.
\begin{zed}
  \lbag Stop \rbag^{\cal A_N}\gamma\ USt = [~ Normal(USt) \land \Xi
  USt | trace' = \langle\rangle \land okay' \land wait' ~]
\end{zed}
Again, the user state does not change as does not the trace.  Deadlock
is characterised by the fact that $wait'$ is true.

The $Chaos$ action normal behaviour is specified as shown below.
\begin{zed}
  \lbag Chaos \rbag^{\cal A_N}\gamma\ USt \defs [~ Normal(USt) | \lnot
  okay' ~]
\end{zed}
We require $\lnot okay'$, which characterises divergence.

Sequencing is defined in terms of a function $sequence$ on
$ProcObs(USt)$ as follows.
\begin{schema}{\lbag A; B \rbag^{\cal A_N}\gamma\ USt}
  Normal(USt)
  \where %
  \theta ProcObs(USt) = \theta (\lbag A \rbag^{\cal A}\gamma\ USt)\
  sequence\ \theta (\lbag B \rbag^{\cal A}\gamma\ USt)
\end{schema}
The function $sequence$ takes two process observations, and returns
the process observation that characterises their sequential
composition.
\begin{axdef}
  \_ sequence \_: ProcObs(USt) \times ProcObs(USt) \pfun ProcObs(USt)
  \where %
  \forall a, b, c : ProcObs(USt) | c = a\ sequence\ b \iff
  \\ %
  \t1 before\ c = before\ a \land after\ a = before\ b \land after\ b
  = after\ c
\end{axdef}
The sequential composition of two process observations \pagebreak is
well-defined only if the final state of the first is equal to the
initial state of the second.

The functions $before$ and $after$ project out the initial and the
final state of a process observation, respectively.
\begin{axdef}
  before: ProcObs(USt) \fun State(USt)
  \where %
  \forall ProcObs(USt) @ before\ \theta ProcObs(USt) = \theta
  State(USt)
\end{axdef}

\begin{axdef}
  after: ProcObs(USt) \fun State(USt)
  \where %
  \forall ProcObs(USt) @ after\ \theta ProcObs(USt) = \theta
  State(USt)'
\end{axdef}
The schema $State(USt)$ is, as expected, similar to the schema $State$
defined earlier on, except that it includes $USt$ instead of
$UserState$.

In the definition of $sequence$, if $a$ diverges, then we have that
$\lnot a.okay'$ and consequently $\lnot b.okay$.  So, if $b$ satisfies
the healthiness condition \textbf{CSP1}, then the composite
$a~sequence~b$ diverges.  Similarly, if $a$ is waiting, then we have
$a.wait'$ and so $b.wait$.  So if $b $ satisfies the healthiness
condition \textbf{R3}, then $a~sequence~b$ waits.

Before defining the different forms of prefixing, we specify a
communication as a process observation $Comm(USt)$.  The occurrence of
a communication is an event, which is characterised by a channel name
and a value communicated through that channel, and in our model is an
element of $Event$.  The process observation $Comm(USt)$ takes as
input the set of events $accEvents?$ that can take place, and outputs
the event $e!$ that actually takes place.

We can make observations at two stages of the communication. The first
is when the communication has not actually taken place yet:~we are
waiting for it to happen.
\begin{schema}{CommWaiting(USt)}
  Normal(USt)
  \\ %
  accEvents?: \power Event
  \\ %
  \Xi USt
  \where %
  trace' = \langle\rangle \land accEvents? \cap ref' = \emptyset
  \\ %
  okay' \land wait'
\end{schema}
The trace has not been extended, the acceptable events cannot be
refused, and the user state is not changed.

We can also observe a communication after it has occurred.
\begin{schema}{CommDone(USt)}
  Normal(USt)
  \\ %
  accEvents?: \power Event
  \\ %
  e!: Event
  \\ %
  \Xi USt
  \where %
  e! \in accEvents?
  \\ %
  trace' = \langle e! \rangle
  \\ %
  okay' \land \lnot wait'
\end{schema}
The trace is extended with one of the possible events and again the
user state is not changed.

In summary, an observation of a communication falls in one of the
above cases.
\begin{zed}
  Comm(USt) \defs CommWaiting(USt) \lor CommDone(USt) \lor
  Diverge(USt) \lor Wait(USt)
\end{zed}
We use this definition in the model of prefixings.
\newpage

A communication is actually a more primitive concept than a prefixing,
which is actually the sequential composition of a communication and an
action.  For a prefixing \mbox{$c!e \then A$}, we have the following
semantics.
\begin{schema}{ \lbag c!e \then A \rbag^{\cal A_N}\gamma\ USt}
  Normal(USt)
  \where %
  \exists oc: Comm(USt) | oc.accEvents? = \{ c(e) \} @
  \\ %
  \t1 \theta ProcObs(USt) = (procObsC\ oc)\ sequence\ \theta (\lbag A
  \rbag^{\cal A}\gamma\ USt)
\end{schema}
First the communication takes place, then, in sequence, the behaviour
is that of $A$.  The only possible communication is $c(e)$.  The
function $procObsC$ projects out the components of a communication
that form a process observation.
\begin{axdef}
  procObsC: Comm(USt) \pfun ProcObs(USt)
  \where %
  \forall Comm(USt) @ procObsC\ \theta Comm(USt) = \theta ProcObs(USt)
\end{axdef}
The semantics of \mbox{$c.e \then A$} and \mbox{$c \then A$} can be
defined in a similar way.  As a matter of fact, $c!e \then A$ is the
same as $c.e \then A$.

For \mbox{$c?x \then A$} the definition is different as $c?x$
introduces the variable $x$ in scope for $A$.  In other words, the
action $A$ acts on an extended state that includes $x$ as a component.
The type of the extra component is that of the channel through which
the parameter is supposed to be input.  The value of this new state
component, however, is determined by the input communication and is
fixed in $A$.  The semantics of the action $A$ is considered in the
state $xUSt(USt)$.  The reference to \mbox{$\Delta xUSt$} in
$ProcObs(xUSt)$ is a reference to the schema below, which enforces the
fact that the input variable $x$ is not modified.
\begin{zed}
  xUSt(USt) \defs [~ USt ; x: \gamma\ c ~]
  \also %
  \Delta xUSt(USt) \defs [~ xUSt(USt); xUSt'(USt) | x = x' ~]
\end{zed}
\begin{schema}{\lbag c?x \then A \rbag^{\cal A_N}\gamma\ USt}
  Normal(USt)
  \where %
  \LET as == \theta (\lbag A \rbag^{\cal A}\gamma\ xUSt(USt)) @
  \\ %
  \ \
  \begin{block}
    \exists ic: Comm(USt) |
    \\ %
    \ \
    \begin{block}
      ic.accEvents? = \{~ x: \gamma\ c @ c(x) ~\} \land c(as.x') = ic.e! @
      \\ %
      \theta ProcObs(USt) = (procObsC\ ic)\ sequence\ (procObsPxP\ as)
    \end{block}
  \end{block}
\end{schema}
The accepted events for an input communication are all those formed by
applying the event constructor $c$ to a valid value of the type of
$c$.  This type is necessarily different from $Sync$, as otherwise the
prefixing \mbox{$c?x \then A$} is not well-formed.  The value actually
communicated is defined as the initial~(constant) value of $x$.

The projection function $procObsPxP$ extracts a process observation
out of a $ProcObs(xUSt)$.
\begin{axdef}
  procObsPxP: ProcObs(xUSt) \fun ProcObs(USt)
  \where %
  \forall PxProcObs(xUSt) @ procObsPxP\ \theta ProcObs(xUSt) = \theta
  ProcObs(USt)
\end{axdef}

The semantics of a prefixing $c?x: input \then A$, where the predicate
$input$ restricts the set of acceptable values that can be
communicated through $c$, can be defined in the same way as above.  We
need to change the definition of the set $ic.accEvents?$ to be $\{~ x:
\gamma\ c | input @ c(x) ~\}$.  In this way, the values $x$ that can be
communicated are those of the type of $c$ which satisfy $input$.

We can easily define the semantics of a prefixing like $c?x!e \then A$
where the channel $c$ is used simultaneously for input and output.
\begin{schema}{\lbag c?x!e \then A \rbag^{\cal A_N}\gamma\ USt}
  Normal(USt)
  \where %
  \LET as == \theta (\lbag A \rbag^{\cal A}\gamma\ xUSt(USt)) @
  \\ %
  \ \
  \begin{block}
    \exists ioc: Comm(USt) |
    \\ %
    \ \
    \begin{block}
      ioc.accEvents? = \{~ p: \gamma\ c| snd\ p = e[c\inv(ioc.e!)/x] @
      c(p) ~\} \land {}
      \\ %
      as.x = c\inv(ioc.e!) @
      \\ %
      \theta ProcObs(USt) = (procObsC\ ioc)\ sequence\ (procObsPxP\
      as)
    \end{block}
  \end{block}
\end{schema}
In this case, the type of $c$ is a pair and the values that can be
communicated are pairs whose second element is $e[c\inv(ioc.e!)/x]$.
If the input variable $x$ is used in the definition of the output
expression $e$, it is replace with the input value.  This is obtained
by applying the inverse of the $c$ event constructor to the event
$ioc.e!$ that actually takes place.  This value is also used to
initialise $x$.  We do not consider here all the possible combinations
of inputs and outputs that can be used.  Their semantics is lengthy,
but not illuminating.

In CSP, the semantics of input is given as an external choice over the
acceptable events.  Here, as already mentioned, we regard
communication as a primitive concept.  The resulting Z specification
is more compact and we avoid the need for an infinite choice.

The action $p \guard A$ is enabled only if the $p$ condition holds.
Its semantics is defined as follows.
\begin{schema}{\lbag p \guard A \rbag^{\cal A_N}\gamma\ USt}
  Normal(USt)
  \where %
  p \implies \lbag A\rbag^{\cal A}\gamma\ USt
  \\ %
  \lnot p \implies trace' = \langle\rangle \land okay' \land wait'
  \land USt = USt'
\end{schema}
If the $p$ condition holds, the action behaves as $A$; otherwise it
behaves as $Stop$.

The behaviour of an external choice $A \extchoice B$ can be observed
in two points:~before the choice is made and after the choice is made.
Before the choice is made, the trace is empty, the process is waiting,
and the user state has not been changed; both $A$ and $B$ have to
behave in this way.
\begin{schema}{ExtChoiceWaiting(USt)}
  Normal(USt)
  \\ %
  \Xi USt
  \where %
  trace' = \langle\rangle \land okay' \land wait'
  \\ %
  \lbag A\rbag^{\cal A}\gamma\ USt
  \\ %
  \lbag B \rbag^{\cal A}\gamma\ USt
\end{schema}
The refusal set is characterised by the restrictions of both $A$ and
$B$; this means that an event is refused only if it is refused by both
$A$ and $B$.
\newpage

If a choice has been made, then either the trace is not empty or the
action has diverged or terminated.
\begin{schema}{ExtChoiceNotWaiting(USt)}
  Normal(USt)
  \where %
  trace' \neq \langle\rangle \lor \lnot okay' \lor \lnot wait'
  \\ %
  \lbag A\rbag^{\cal A}\gamma\ USt \lor \lbag B \rbag^{\cal A}\gamma\
  USt
\end{schema}
The behaviour presented is either that of $A$ or that of $B$.
\begin{zed}
  \lbag A \extchoice B \rbag^{\cal A_N}\gamma\ USt \defs
  ExtChoiceWaiting(USt) \lor ExtChoiceNotWaiting(USt)
\end{zed}
The behaviour of the external choice is given by the disjunction of
the above two cases.

The semantics of the internal choice $A \intchoice B$ is given by the
disjunction of the semantics of $A$ and $B$.
\begin{zed}
  \lbag A \intchoice B \rbag^{\cal A_N}\gamma\ USt \defs \lbag
  A\rbag^{\cal A}\gamma\ USt \lor \lbag B \rbag^{\cal A}\gamma\ USt
\end{zed}
The action $A \intchoice B$ behaves either like $A$ or like $B$ and
the choice is independent of the environment.

Before specifying the semantics of parallelism and hiding, we need to
define the set of events determined by a channel set $C$.  This is the
set of all communications through a channel in $C$, which in our model
are the events constructed using the injections into $Event$ defined
by the channels in $C$.  The definition of this set is shown below; it
depends on the current channel environment.
\begin{zed}
  \lbag \_ \rbag: \mathsf{CSExpression} \pfun ChanEnv \pfun \power
  Event
  \also %
  \lbag \lchanset ~ \rchanset \rbag\gamma = \emptyset
  \also %
  \forall C: \mathsf{CSExpression} @ C \neq \lchanset ~ \rchanset
  \implies \exists c: \mathsf{N} @ c \in C \land {}
  \\ %
  \t1
  \begin{block}
    \gamma\ c \neq Sync \implies \lbag C \rbag\gamma = \{~ v: \gamma\ c
    @ c(v) ~\} \cup \lbag C\setminus \{ c \} \rbag\gamma \land {}
    \\ %
    \gamma\ c = Sync \implies \lbag C \rbag\gamma = \{ c \} \cup \lbag
    C\setminus \{ c \} \rbag\gamma
  \end{block}
\end{zed}
We consider only the empty channel set and a channel set defined by
enumeration, as we assume in our semantics that more elaborate channel
set expressions are expanded and eliminated.

The definition of the $\_ \parallel \_ sync \_$ operator is as
follows.
\begin{axdef}
  \_ \parallel \_ sync \_ : \seq Event \times \seq Event \times \power
  Event \fun \power (\seq Event)
  \where %
  \forall t_1, t_2: \seq Event; s: \power Event; e_1,e_2: Event @
  \\ %
  \t1
  \begin{block}
    t_1 \parallel t_2\ sync\ s = t_2 \parallel t_1\ sync\ s \land {}
    \\ %
    \langle\rangle \parallel \langle\rangle\ sync\ s = \{
    \langle\rangle \} \land {}
    \\ %
    (e_1 \in s \implies
    \\ %
    \t1
      \begin{block}
        \langle e_1 \rangle \parallel \langle\rangle\ sync\ s =
        \emptyset \land {}
        \\ %
        (\langle e_1\rangle \cat t_1) \parallel (\langle
        e_1\rangle\cat t_2)\ sync\ s = \{~ t_3: (t_1 \parallel t_2\
        sync\ s) @ \langle e_1\rangle\cat t_3 ~\} \land {}
        \\ %
        e_2 \notin s \implies (\langle e_1\rangle\cat t_1) \parallel
        (\langle e_2\rangle\cat t_2)\ sync\ s =
        \\ %
        \t3 \{~ t_3: ((\langle e_1\rangle\cat t_1) \parallel t_2\ sync\
        s) @ \langle e_2\rangle\cat t_3 ~\} \land {}
        \\ %
        e_2 \in s \land e_1 \neq e_2 \implies (\langle e_1\rangle\cat
        t_1) \parallel (\langle e_2\rangle\cat t_2)\ sync\ s =
        \emptyset) \land {}
      \end{block}
    \\ %
    (e_1 \notin s \implies
    \\ %
    \t1
      \begin{block}
        \langle e_1 \rangle \parallel \langle\rangle\ sync\ s = \{
        \langle e_1 \rangle \} \land {}
        \\ %
        e_2 \notin s \implies (\langle e_1\rangle\cat t_1) \parallel
        (\langle e_2\rangle\cat t_2)\ sync\ s =
        \\ %
        \t3
        \begin{block}
          \{~ t_3: (t_1 \parallel (\langle e_2\rangle\cat t_2)\ sync\
          s) @ \langle e_1\rangle\cat t_3 ~\}\ \cup
          \\ %
          \{~ t_3: ((\langle e_1\rangle\cat t_1) \parallel t_2\ sync\
          s) @ \langle e_2\rangle\cat t_3 ~\}
        \end{block}
      \end{block}
  \end{block}
\end{axdef}
A definition of a similar function can be found in~\cite{Ros98}.
\newpage

We define the semantics of parallelism as shown below.
\begin{schema}{\lbag A \lpar C \rpar B \rbag^{\cal A_N}\gamma\ USt}
  Normal(USt)
  \where %
  \exists tracea, traceb, tra, trb: \seq Event; refa, refb: \power
  Event;
  \\ %
  \ \
    okaya, okayb, waita, waitb: Bool @
  \\ %
  \t1
  \begin{block}
    (\lbag A\rbag^{\cal A}\gamma\
    USt)[tracea,tra,refa,okaya,waita/trace',tr',ref',okay',wait']
    \land {}
    \\ %
    (\lbag B\rbag^{\cal A}\gamma\
    USt)[traceb,trb,refb,okayb,waitb/trace',tr',ref',okay',wait' ]
    \land {}
    \\ %
    trace' \in tracea \parallel traceb\ sync\ (\lbag C\rbag\gamma)
    \land {}
    \\ %
    ref' = (refa \cup refb) \cap \lbag C\rbag\gamma\ \cup\ (refa \cap
    refb) \setminus \lbag C\rbag\gamma \land {}
    \\ %
    okay' = okaya \land okayb \land {}
    \\ %
    wait' = waita \lor waitb
  \end{block}
\end{schema}
We include the schemas that define the semantics of $A$ and $B$.
These actions start in the same state, so their restrictions on the
initial state are conjoined.  The final state of the parallel
composition depends on the final state of the individual actions; we
rename the individual final state components to use in the definition
of the parallel composition.

An event can be refused by the parallel composition if either $A$ or
$B$ can refuse it and they have to synchronise on it, or rather, it is
in the synchronisation set $C$.  If it is not, then it can only be
refused if both $A$ and $B$ can refuse it.  The parallel composition
diverges if either $A$ or $B$ does, and it terminates when both $A$
and $B$ do.  The trace is the combination of the traces of $A$ and $B$
where events in the channel set determined by $C$ are synchronised.

The definition of interleaving is similar to that of parallelism.
\begin{schema}{\lbag A \interleave B \rbag^{\cal A_N}\gamma\ USt}
  Normal(USt)
  \where %
  \exists tracea, traceb, tra, trb: \seq Event; refa, refb: \power
  Event;
  \\ %
  \ \
    okaya, okayb, waita, waitb: Bool @
  \\ %
  \t1
  \begin{block}
    (\lbag A\rbag^{\cal A}\gamma\
    USt)[tracea,tra,refa,okaya,waita/trace',tr',ref',okay',wait']  \land {}
    \\ %
    (\lbag B\rbag^{\cal A}\gamma\
    USt)[traceb,trb,refb,okayb,waitb/trace',tr',ref',okay',wait' ] \land {}
    \\ %
    trace' \in tracea \interleave traceb \land {}
    \\ %
    ref' = refa \cap refb \land {}
    \\ %
    okay' = okaya \land okayb \land {}
    \\ %
    wait' = waita \lor waitb
  \end{block}
\end{schema}
An event can be refused by $A \interleave B$ if it can be refused by
both $A$ and $B$.  The definition of $\_ \interleave \_$ is simple.
\begin{axdef}
  \_ \interleave \_ : \seq Event \times \seq Event \fun \power (\seq
  Event) \where \forall t_1, t_2: \seq Event; e_1,e_2: Event @
  \\ %
  \t1
  \begin{block}
    \langle\rangle \interleave t_1 = t_1 \interleave \langle \rangle =
    \{ t_1 \} \land {}
    \\ %
    (\langle e_1\rangle\cat t_1) \interleave (\langle e_2\rangle\cat
    t_2) =
    \\ %
    \t1 \{~ t_3: (t_1 \interleave (\langle e_2\rangle\cat t_2)) @
    \langle e_1\rangle\cat t_3 ~\} \cup \{~ t_3: ((\langle e_1\rangle\cat
    t_1) \interleave t_2) @ \langle e_2\rangle\cat t_3 ~\}
  \end{block}
\end{axdef}
This function is defined in~\cite{Ros98}.
\newpage

The semantics of hiding is as follows.
\begin{schema}{\lbag A \hide C \rbag^{\cal A_N}\gamma\ USt}
  Normal(USt)
  \where %
  \exists tracep, trp: \seq Event; refp: \power Event @
  \\ %
  \t1
  \begin{block}
    \lbag A\rbag^{\cal A}\gamma\ USt[tracep,trp,refp/trace',tr',ref'] \land {}
    \\ %
    trace' = tracep \filter (Event \setminus \lbag C \rbag\gamma)
    \\ %
    refp = ref' \cup (\lbag C \rbag\gamma)
  \end{block}
\end{schema}
The traces and refusals of $A \hide C$ are determined in terms of
those of $A$:~we eliminate all events that represent communications
through the channels in $C$ from the trace and from the refusals.  In
order to deal adequately with the semantics of hiding, we need to have
infinite sequences in our model, as suggested in~\cite{Ros98}.  We
leave the complications of this as a future work for now.

The definition of the semantics of a recursive action $\mu X @ A$ is
standard.  We must observe, however, that the action $A$ may use $X$
as an action and, as such, is regarded as a function from actions to
actions.  For clarity, we refer to this action as $F(X)$.
\begin{schema}{\lbag \mu X @ F(X) \rbag^{\cal A_N}\gamma\ USt}
  Normal(USt)
  \where %
  \theta ProcObs(USt) \in \bigcup \{~ a: SProcess(USt) | a \subseteq
  \lbag F(\_) \rbag^{\cal F}\gamma\ USt \limg a \rimg ~\}
\end{schema}
In this context, a process is represented as a set of process
observations that satisfies the healthiness conditions.  To specify the
set $SProcess$ of such sets, we formalise the healthiness conditions.

The first healthiness condition~($\healthiness{R1}$) is already
enforced by $ProcStateObs$, which requires $tr \prefix tr'$.  This
guarantees that $tr' - tr$ is well-defined.  The sets of process
observations that satisfy $\healthiness{R2}$ can be defined as
follows.
\begin{zed}
  \healthiness{R2} == \{~ p: \power ProcObs(USt) | \forall po_1: p @
  \exists po_2: p @
  \\ %
  \t3
  \begin{block}
    po_2.tr = \langle\rangle \land po_2.tr' = po_1.tr' - po_1.tr \land {}
    \\ %
    po_2.ref = po_1.ref \land po_2.ref' = po_1.ref' \land {}
    \\ %
    po_2.okay = po_1.okay \land po_2.okay' = po_1.okay' \land {}
    \\ %
    po_2.wait = po_1.wait \land po_2.wait' = po_1.wait' \land {}
    \\ %
    userStateObs\ po_2 = userStateObs\ po_1 ~\}
  \end{block}
\end{zed}
This requires that the observations are independent from previous
events:~for each observation, there is another one that is similar and
occurs in a state where no communication has happened yet.  The
function $userStateObs$ extracts the user state components out of a
process observation.
\begin{axdef}
  userStateObs: ProcObs(USt) \fun \Delta USt
  \where %
  \forall ProcObs(USt) @ userStateObs\ \theta ProcObs(USt) = \theta
  (\Delta USt)
\end{axdef}
We formalise the healthiness condition $\healthiness{R3}$ as shown
below.
\begin{zed}
  \healthiness{R3} == \{~ p: \power ProcObs(USt) | \forall po: p @
  po.wait \land po.okay \land before\ po = after\ po ~\}
\end{zed}
This requires that if the previous process has not diverged and not
finished, then the current process does not touch the state.
\newpage

The set of reactive processes $\healthiness{R}$ is defined as the
intersection of $\healthiness{R2}$ and $\healthiness{R3}$.
\begin{zed}
  \healthiness{R} == \healthiness{R2} \cap \healthiness{R3}
\end{zed}
Two further healthiness conditions are needed:~$\healthiness{CSP1}$
and~$\healthiness{CSP2}$.  The first is as follows.
\begin{zed}
  \healthiness{CSP1} == \{~ p: \healthiness{R} | p = p \cup Q ~\}
  \also %
  Q == \{~ po: ProcObs(USt) | \lnot po.okay \land tr \prefix tr' ~\}
\end{zed}
This further restricts the reactive processes to those for which no
restrictions are enforced if $okay$ is false; in this situation we are
only guaranteed that the trace is extended.

The healthiness condition $\healthiness{CSP2}$ is formalised as
follows.
\begin{zed}
  \healthiness{CSP2} == \{~ p: \healthiness{R} | p = p \semi J ~\}
  \also %
  J == \{~ po: ProcObs(USt) | ( po.okay \implies po.okay' ) \land {}
  \\ %
  \t2
  \begin{block}
    fUserState\ po = iUserState\ po \land {}
    \\ %
    po.tr' = po.tr \land po.wait' = po.wait \land po.ref' = po.ref ~\}
  \end{block}
\end{zed}
The sequential composition of sets of process observations is defined
in terms of the composition of individual process observations.
\begin{axdef}
  \_ \semi\_: \power ProcObs(USt) \times \power ProcObs(USt) \fun
  \power ProcObs(USt)
  \where %
  \forall p_1, p_2: \power ProcObs(USt); po_3: ProcObs(USt) @ po_3 \in
  p_1 \semi p_2 \iff
  \\ %
  \t1 \exists po_1: p_1; po_2: p_2 @ po_3 = po_1\ sequence\ po_2
\end{axdef}
The projection functions $fUserState$ and $iUserState$ extract the
components of the final and initial user state of a $ProcObs$.
\begin{axdef}
  fUserState, iUserState: ProcObs(USt) \fun USt
  \where %
  \forall ProcObs(USt) @
  \\ %
  \t1
  \begin{block}
    fUserState\ \theta ProcObs(USt) = \theta USt' \land \null
    \\ %
    iUserState\ \theta ProcObs(USt) = \theta USt
  \end{block}
\end{axdef}
The $\healthiness{CSP}$ processes are reactive processes that satisfy
$\healthiness{CSP1}$ and $\healthiness{CSP2}$.
\begin{zed}
  \healthiness{CSP} = \healthiness{CSP1} \cap \healthiness{CSP2}
\end{zed}
Finally, our processes of interest satisfy three other healthiness
conditions.  They are given as unit laws.  The first,
$\healthiness{CSP3}$, requires that $Skip$ is the left unit of
sequential composition.
\begin{zed}
  \healthiness{CSP3} == \{~ p: \healthiness{CSP} | p = Skip \semi p ~\}
\end{zed}
The set of process observations $Skip$ is defined as follows.
\begin{zed}
  Skip == \{~ po: ProcObs(USt) |
  \\ %
  \t3
  \begin{block}
    (po.okay \land \lnot po.wait \land {}
    \\ %
    \t1 fUserState\ po = iUserState\ po \land po.trace' =
    \langle\rangle \land po.okay' \land po.wait') \lor
    \\ %
    \lnot po.okay \lor
    \\ %
    (po.okay \land po.wait \land before\ po = after\ po) ~\}
  \end{block}
\end{zed}
The healthiness condition $\healthiness{CSP4}$ requires $Skip$ to be
the right unit of sequential composition.
\begin{zed}
  \healthiness{CSP4} == \{~ p: \healthiness{CSP} | p = p \semi Skip ~\}
\end{zed}
The last healthiness condition, $\healthiness{CSP5}$, requires that
$Skip$ is the right unit of interleaving.
\begin{zed}
  \healthiness{CSP4} == \{~ p: \healthiness{CSP} | p = p \interleave
  Skip ~\}
\end{zed}
The definition of interleaving is as follows.
\begin{axdef}
  \_ \interleave\_: \power ProcObs(USt) \times \power ProcObs(USt)
  \fun \power ProcObs(USt)
  \where %
  \forall p_1, p_2: \power ProcObs(USt); po_3: ProcObs(USt) @ po_3 \in
  p_1 \interleave p_2 \iff
  \\ %
  \t1 \exists po_1: p_1; po_2: p_2 @ po_3 = po_1 \interleave po_2
\end{axdef}
For individual process observations, we define interleaving as shown
below.
\begin{axdef}
  \_ \interleave\_: ProcObs(USt) \cross ProcObs(USt) \fun ProcObs(USt)
  \where %
  \forall po_1, po_2, po_3: ProcObs(USt) |
  \\ %
  \t1
  \begin{block}
    \t1
    \begin{block}
      before\ po_1 = before\ po_2 = before\ po_3 \land {}
      \\ %
      fUserState\ po_1 = fUserState\ po_2 = fUserState\ po_3 @
    \end{block}
    \\ %
    po_3 = po_1 \interleave po_2 \iff
    \\ %
    \t1
    \begin{block}
      po_3.trace' = po_1.trace' \interleave po_2.trace' \land {}
      \\ %
      po_3.ref' = po_1.ref' \cap po_2.ref' \land {}
      \\ %
      po_3.okay' = po_1.okay' \land po_2.okay' \land {}
      \\ %
      po_3.wait' = po_1.wait' \lor po_2.wait'
    \end{block}
  \end{block}
\end{axdef}
We use the interleaving operation on traces defined previously.

The set $SProcess$ is defined as the intersection of
$\healthiness{CSP3}$, $\healthiness{CSP4}$, and $\healthiness{CSP5}$;
it forms a complete lattice.
\begin{zed}
  SProcess(USt) == \healthiness{CSP3} \cap \healthiness{CSP4} \cap
  \healthiness{CSP5}
\end{zed}

We use the semantic function \mbox{$\lbag \_ \rbag^{\cal F}$}, which
gives the semantics of a function on actions as a function on process
observations.
\begin{zed}
  \lbag\_ \rbag^{\cal F}: ( Action \fun Action ) \pfun ChanEnv \pfun
  SchemaName \pfun
  \\ %
  \t2 (ProcObs(USt) \fun ProcObs(USt))
\end{zed}
The function \mbox{$\lbag F(\_) \rbag^{\cal F}\gamma\ USt$} can be
defined in terms of \mbox{$\lbag \_ \rbag^{\cal A}$} as follows.
\begin{axdef}
  \lbag F(\_) \rbag^{\cal F}\gamma\ USt: ProcObs(USt) \fun
  ProcObs(USt)
  \where %
  \forall po: ProcObs(USt) @ \LET \theta (\lbag X \rbag^{\cal
    A}\gamma\ USt) == po\ @
  \\ %
  \t1 \exists \lbag F(X) \rbag^{\cal A}\gamma\ USt @ f\ po = \theta
  ProcObs(USt)
\end{axdef}
To determine \mbox{$\lbag F(X) \rbag^{\cal A}\gamma\ USt$}, we need to
know \mbox{$\theta (\lbag X \rbag^{\cal A}\gamma\ USt)$}.  This is
specified above as the argument given to $\lbag F(\_) \rbag^{\cal
  F}\gamma\ USt$.

By way of illustration, consider the semantics of $\mu X @ out!2 \then
X$, for an environment $\gamma$ in which $out$ is a channel of type
integer. The function $\lbag out!2 \then \_\rbag^{\cal F}\ \gamma\
USt$ is as follows.
\begin{axdef}
  \lbag out!2 \then \_\rbag^{\cal F}\ \gamma\ USt: ProcObs(USt) \fun
  ProcObs(USt)
  \where %
  \forall po: ProcObs(USt) @ \exists Output @ f~po = \theta
  ProcObs(USt)
\end{axdef}
\newpage
\noindent
The schema $Output$ is defined according to the semantics of $out!2
\then X$ as shown below.
\begin{schema}{Output}
  Normal(USt)
  \where %
  \exists oc: Comm(USt) | oc.accEvents? = \{ out(2) \} @
  \\%
  \t1
    \theta ProcObs(USt) = (procObsC\ oc)\ sequence\ po
\end{schema}
The semantics of $X$, which is required in the specification of
$Output$ is taken to be $po$.

The calculation of the semantics of other recursive actions, which
apply to $X$ operators whose semantics are given in terms of
\mbox{$\lbag X \rbag^{\cal A}\gamma\ USt$} instead of
\mbox{$\theta(\lbag X \rbag^{\cal A}\gamma\ USt)$}, requires more
effort.  We need to manipulate the definition of \mbox{$\lbag F(X)
  \rbag^{\cal A}\gamma\ USt$} to express it in terms of \mbox{$\theta
  (\lbag X \rbag^{\cal A}\gamma\ USt)$}.

A parametrised action $D @ A$ declares extra variables that can be
used in $A$.  As for input variables, we regard them as part of the
state, but their values cannot be changed; they are fixed by
instantiation.  To define the semantics of $D @ A$, we define the
state $DUSt$ as follows.
\begin{zed}
  DUSt(USt) \defs [~ USt; D ~]
\end{zed}
We also define a schema $\Delta DUSt(USt)$, which requires that the
variables declared in $D$ are not changed.
\begin{zed}
  \Delta DUSt(USt) \defs [~ DUSt(USt); DUSt'(USt) | \alpha~D' =
  \alpha~ D ~]
\end{zed}
We use the function $\alpha$ that extracts from a declaration the set
of variables it introduces.  We also decorate the declaration $D$ to
obtain a declaration that is exactly like $D$ except that it
introduces the dashed counterparts of the variables declared in $D$.
In Z, decorations can be applied to schemas and variables, but here we
also decorate declarations.  Finally, we write $\alpha~D = \alpha~ D'$
to denote the conjunction of equalities $x = x'$ for each variable $x$
in $\alpha D$.  With these definitions, the semantics of $D @ A$ can
be specified as follows.
\begin{zed}
  \lbag D @ A\rbag^{\cal A_N}\gamma\ USt \defs \lbag A\rbag^{\cal
    A}\gamma\ DUSt(USt)
\end{zed}
The semantics is that of $A$ taken in the extended state.

For an instantiation $A(e)$, we just have to define the initial value
of the extra state components in the semantics of $A$ and hide them.
\begin{zed}
  \lbag A(e) \rbag^{\cal A_N}\gamma\ USt \defs [~ Normal(USt) \land
  \lbag A\rbag^{\cal A_N}\gamma\ USt | \alpha D = e ~] \hide \alpha D,
  \alpha D'
\end{zed}
The semantics of $A$ is a process observation involving an extended
user state.  We assume the extra components of this state are declared
in $D$, and so $\alpha D$ gives the set of these components.  We must
observe that $e$ is a list of expressions and the equality stands for
a conjunction of equalities.  The correspondence between the variables
in $\alpha D$ and the expressions in $e$ is positional and based in
the order of the declaration of the extra variables.  The declaration
$D$ may not be explicit in $A(e)$ and is an important context
information that is implicitly used in the above definition.

Iterated operators are over finite sets.  They can be defined in terms
of the basic operator and the instantiation operator in the obvious
way.  For example, the action $\Extchoice i: T @ A(i)$ is defined as
$A(v_1) \extchoice \ldots \extchoice A(v_n)$, where the $v_i$ are the
values in $T$.

\subsubsection{Commands}

The behaviour of a specification statement $x :[~ pre, post ~]$ is
highly dependent on whether its precondition holds or not. If it does,
then the postcondition must be established and the operation must
terminate successfully, but the trace is not affected and only the
variables $x$ in the frame can be changed.
\begin{schema}{\lbag x :[~ pre, post ~] \rbag^{\cal A_N}\gamma\ USt}
  Normal(USt)
  \where %
  pre \implies post \land trace' = \langle\rangle \land okay' \land
  \lnot wait' \land \theta (USt' \setminus x') = \theta (USt \setminus
  x)
  \\ %
  \lnot pre \implies \lnot okay'
\end{schema}
We decorate the list of variables $x$ to obtain the list of
corresponding dashed variables.  In an abuse of notation we use $x$
and $x'$ as sets to define the set of user state components that
cannot be changed:~all but those in $x$.  The conjunction of
equalities $\alpha USt' \setminus x' = \alpha USt \setminus x$
enforces this restriction.

If the precondition holds, but the postcondition cannot be
established~(under the given circumstances), then we have an
infeasible~(miraculous) operation.  In such a circumstance, the
predicate of the above schema is $false$.  Therefore, the semantics of
$x:[~ pre, post ~]$ is a partial relation.

The semantics of the assignment $x := e$ is rather simple.  This
action does not change the trace and, since we assume the expressions
are always well-defined, it does not diverge and terminates.  Of
course, it sets the final value of $x$ to $e$.
\begin{schema}{\lbag x := e\rbag^{\cal A_N}\gamma\ USt}
  Normal(USt)
  \where %
  trace' = \langle\rangle \land okay' \land \lnot wait' \land x' = e
  \land \theta (USt' \setminus x') = \theta (USt \setminus x)
\end{schema}
If the assignment is multiple, the equality $x' = e$ actually stands
for a conjunction where the variables of the list $x$ are equated to
the correspond expressions of the list $e$.

The semantics of the conditional is also standard.
\begin{schema}{\lbag \circif \extchoice i @ g_i \then A_i\
    \circfi\rbag^{\cal A_N} \gamma\ USt}
  Normal(USt) %
  \where %
  (\lor i @ g_i) \implies (\lor i @ g_i \land \lbag A_i\rbag^{\cal
    A}\gamma\ USt)
  \\ %
  \lnot (\lor i @ g_i) \implies \lnot okay'
\end{schema}
Any of the guarded actions whose guard is true can be chosen for
execution.

The semantics of variable declaration is given by existential
quantification.
\begin{zed}
  \lbag \circvar\ x: T @ A\rbag^{\cal A_N}\gamma\ USt \defs [~
  Normal(USt) | \exists x, x': T @ \lbag A\rbag^{\cal A}\gamma\
  xUSt(USt) ~]
\end{zed}
The declaration of $x$ actually puts both $x$ and $x'$ into scope.
The semantics of the action in the scope of the variable block is
taken in the extended user state that includes $x$.  The definition of
$xUSt(USt)$ has been presented before, in the context of the
definition of the semantics of input communications.  In this case,
however, we do not restrict state changes on $xUSt$, as the variable
$x$ can be used freely in its scope.

\subsection{Process expressions}

A process expression combines existing processes using CSP operators.
For the binary operators $\mathsf{op}$, except parallelism and
interleaving, the semantics of $P\ \mathsf{op}\ Q$ can be given in
terms of an explicit process specification.
\begin{zed}
  P\ \mathsf{op}\ Q = \circbegin
  \also %
  \t1
  \begin{block}
    State \defs P.State \land Q.State
    \also %
    P.PPar \uparrow Q.State
    \also %
    Q.PPar \uparrow P.State
    \also %
    @ P.Act\ \mathsf{op}\ Q.Act
  \end{block}
  \also %
  \circend
\end{zed}
The schemas $P.State$ and $Q.State$ are the schemas that define the
user state of $P$ and $Q$.  This is a piece of context information
that is relevant to the semantics of a process.  The state of $P\
\mathsf{op}\ Q$ conjoins the states of $P$ and $Q$.  We assume that
name clashes are avoided through renaming.

We also refer to $P.PPar$ and $Q.PPar$, which are the process
paragraphs that compose the definitions of $P$ and $Q$, except
$P.State$ and $Q.State$ and their main actions.  These are all
included in $P\ \mathsf{op}\ Q$.  We must notice, however, that the
schemas in $P.PPar$ that specify an operation on $P.State$ are not by
themselves actions in $P\ \mathsf{op}\ Q$; we need to lift them to act
on the extended state defined by $State$.  This is the aim of the
operator $\uparrow$, which simply conjoins each such schema with $\Xi
Q.State$:~actions of $P$ are not supposed to affect the state
components that are inherited from $Q$.  Similar comments apply to the
actions in $Q.PPar$ and we include in $P\ \mathsf{op}\ Q$ the sequence
of process paragraphs $Q.PPar \uparrow State$.

The actions $P.Act$ and $Q.Act$ are those that specify the behaviour
of $P$ and $Q$, or rather, their main actions.  The specification of
the action that defines the behaviour of $P\ \mathsf{op}\ Q$ combines
the actions $P.Act$ and $Q.Act$ using $\mathsf{op}$.  We observe that
$P.Act$~($Q.Act$) operates on the part of the user state that is due
to $P$~($Q$) and cannot change the components that are originally part
of the $Q$~($P$) state.  It does not refer to them directly or
indirectly, except through schema actions which have been conjoined
with $\Xi Q.State$~($\Xi P.State$).

For parallelism and interleaving, the conjunction with $\Xi Q.State$
and $\Xi P.State$ is not appropriate, because state restrictions and
updates of $P.Act$ and $Q.Act$ have to be both satisfied.  If $P.Act$
is required not to change the state components of $P$, and vice-versa,
then their parallel composition can be inconsistent.  For this reason,
the definitions of $P \lpar C\rpar Q$ and $P \interleave Q$ are
similar to that given above for the other binary operators, but the
process paragraphs of the resulting process are slightly different.
The state and the main action are defined in the same way, but the
other paragraphs are defined in terms of a new operator as $P.PPar
\uparrow_{PAR} Q.State$ and $Q.PPar \uparrow_{PAR} P.State$.  The
paragraphs of $P$ are lifted to the extended state by conjoining the
schemas with $\Delta Q.State$; and similarly for the paragraphs of
$Q$.  In this way, no restrictions on the extra state components are
imposed.

The semantics of a hiding expression $P \hide C$ is even simpler.
\begin{zed}
  P \hide C \defs \circbegin
  \also %
  \t1
  \begin{block}
    P.State
    \also %
    P.PPar
    \also %
    @ P.Act \hide C
  \end{block}
  \also %
  \circend
\end{zed}
The process paragraphs of $P$ are included as they are; only the main
action is modified to include the hiding.

The indexed process $i: T \odot P$ implicitly declares channels
$c\_i$, for each channel $c$ used in $P$.  Its semantics, therefore,
affects the channel environment.
\begin{zed}
  \lbag i: T \odot P\rbag^{\cal P}\ \gamma\ \rho = \LET \gamma' ==
  \{~ c: used~P @ c\_i \mapsto T \times \gamma\ c ~\} @
  \\ %
  \t1 %
  \lbag i: T @ (P[c: (used~P) @ c\_i]) \rbag^{\cal P}\ (\gamma \oplus
  \gamma')\ \rho
\end{zed}
The environment $\gamma'$ records the channels implicitly declared.
The set $used~P$ includes the channels used in $P$; they can be
determined from the definition of $P$:~if $P$ is given by a process
expression, then it is the union of the channels used in the operand
processes, including those that have been hidden; and if $P$ is given
explicitly, then it contains the channels mentioned in the actions.
Of course, renamings and indexings have to be taken into account.  For
each channel $c$ that is used in $P$, $\gamma'$ records a channel
$c\_i$ which communicate pairs of values:~the index and whatever value
was communicated originally.  Therefore its type is the cartesian
product of $T$, the type of the index, and the type of $c$.

The semantics of $i: T \odot P$ is that of a parametrised process
taken in the extended channel environment that includes $\gamma'$.
The parameter is the index, and the process $P[c: (used~P) @ c\_i])$
is that obtained from $P$ by changing all the references to a used
channel $c$ by a reference to the channel $c\_i$.  Communications
through these channels are also changed so that the index $i$ is also
communicated.  The generalisation of this and the previous definition
for an arbitrary indexed process $D \odot P$ whose indexes are
declared by $D$ is lengthy, but straightforward.

The semantics of instantiation is given by substitution.
\begin{zed}
  P(e) = P[e/i]
\end{zed}
The parameter of $P$ is a contextual information needed in this
definition.  We observe that, as in CSP, substitution, when applied to
a process name, yields the result of applying the substitution to the
process definition.  For instance, in CSP, if we define $P$ to be $a
\then Q$ and $Q$ to be $\mu X @ b \then X$, then $P[c/b]$ is $a \then
(\mu X @ c \then X)$.  The iterated operators can be defined in the
obvious way using instantiation.

Renaming of channels also affects the channel environment.
\begin{zed}
  \lbag P[oldc := newc]\rbag^{\cal P}\ \gamma\ \rho = \LET \gamma' ==
  \{ newc \mapsto \gamma\ oldc \} @ \lbag P[newc/oldc]\rbag^{\cal P}\
  (\gamma \oplus \gamma')\ \rho
\end{zed}
The environment $\gamma'$ records the channel $newc$, whose type is
that of $oldc$.  The semantics of $P[oldc := newc]$ is that of
$P[newc/oldc]$ taken in the extended channel environment that includes
$\gamma'$.

The semantics of parametrisation is rather simple:~the parameters are
regarded as loose constants.  The channel environment in the pair
defined by $\lbag D @ P \rbag^{\cal P}\ \gamma\ \rho$ is that in
$\lbag P\rbag^{\cal P}\ \gamma\ \rho$.  The Z specification is also
that in $\lbag P \rbag^{\cal P}\ \gamma\ \rho$, preceded by an
axiomatic description that introduces $D$.  The instantiation of
parametrised processes has the same definition of instantiation of
indexed processes:~substitution.  Again, the iterated operators have
the straightforward definition.

The semantics of a generic process $[X]P$ is given by a rewriting of
the Z specification denoted by $P$; the channel environment is the
same.
\begin{zed}
  \lbag [X]P\rbag^{\cal P}\ \gamma\ \rho = \LET Ps == \lbag
  P\rbag^{\cal P}\ \gamma\ \rho @ (Ps.1,generic~X~Ps.2)
\end{zed}
The function $generic$, when applied to a~(list of) generic
parameter(s) $X$ and to a Z specification, turns each of its
paragraphs into a similar generic paragraph with parameter $X$.

Instantiation $P[E]$ of a generic process $P$ is defined by the
instantiation of all definitions in the Z specification corresponding
to $P$.  Again, the channel environment is not affected.
\begin{zed}
  \lbag P[E]\rbag^{\cal P}\ \gamma\ \rho = \LET Ps == \lbag
  P\rbag^{\cal P}\ \gamma\ \rho @ (Ps.1,instantiate~E~Ps.2)
\end{zed}
The function $instantiate$ instantiates with argument $E$ all the
generic definitions in $Ps.2$.  The instantiated definitions are given
fresh names, as usual.

\section{Related Work} \label{section:RW}

We have presented a unified language of specification, design, and
programming that combines Z and CSP, and is suitable for refinement.
Many approaches to the integration of Z or one of its extensions with
a process algebra have been presented elsewhere.  Our main objective
is not to propose yet another language, but to provide support for the
formal development of concurrent programs.  Nevertheless, we believe
the language presented here has some interesting characteristics of
its own.

In~\cite{Fis98}, Fischer surveys several integrations of Z with
process algebras.  That work considers the combinations of Z with CCS
in~\cite{Gal96,TA97}; the combinations of Z with CSP
in~\cite{Fis97,RWW94}; and the combination of Z with
Object-Z~\cite{CDDKRS90} in~\cite{Fis97}.  Several issues involved in
the integration of Z with a process algebra are discussed and the
above approaches are analysed with respect to these issues.

Fischer differentiates two styles for combining Z and a process
algebra:~syntactic and semantic.  In the former, the combination has a
single syntax, with semantic definitions lifted from the two
languages.  In the latter, the Z specification is identified with a
process.

We adopt the first style:~the model of \Circus\ is the unifying theory
of programming of Hoare \& He.  All the other combinations of Z and
CSP cited above adopt the semantic approach.  The syntactic approach
provides a deeper integration of the notations; the disadvantage is
that we have to define the semantics of both the Z and CSP constructs,
but we are already using an existing semantic model.  As we express
this model using Z, we are able use Z tools to analyse \Circus\
specifications.

Another issue is the relation between CSP events and Z operations;
Fischer identifies three possibilities.  Firstly, there is the single
event approach, in which there is a one-to-one correspondence between
operations and events; this leads to abstractions where operations are
atomic transactions.  Refinement to code must preserve this abstract
atomicity in some way.  Secondly, there is the double event approach,
where, in general, operations are divided into two parts:~a front-end
involving input and a back-end involving output.  This corresponds to
a natural way of using CSP, but a less natural way of using Z.
Finally, there is the multiple event approach, in which there is no
fixed connection between operations and events.

In \Circus, we have adopted the multiple event approach:~there is no
identification of operations with events.  Indeed, one can see
operations as occurring in the background between events.  Fischer
points out the single event approach as more abstract.  We observe,
however, that with the multiple event approach we can both adopt a
single event style of specification, as in our abstract buffer
example, and consider programs where events and state changes are not
associated.  For a refinement language this is certainly more
adequate.  Furthermore, we do not split the input and output of
parameters of the Z operations when we define processes.  The
combination preserves the intuition about Z definitions.

In most state-based formalisms, an operation may be given a partial
specification, and so a meaning must be given to what happens when the
operation is activated outside its precondition.  There are two
obvious choices:~the operation waits, which is Fischer's blocking
semantics; or the operation aborts, Fischer's non-blocking semantics.
In the standard texts on Z, operations are given a non-blocking
semantics; in \Circus, operations have the usual semantics: blocking
is specified by guards in actions.  Therefore, we can capture both
aspects of an operation (the enabling and the divergence conditions),
while maintaining the usual semantics of Z operations.

A further issue is related to the typing of inputs.  The Z type system
is very simple:~types are maximal, and there are only four
constructors (given set, powerset, cartesian product, and schema
product); however, it is common practice to constrain components to
range over subsets of types.  As inputs are non-blocking, then the
activation of a Z operation with an input outside its constraint leads
to abort; in CSP, such an input is blocked.  In \Circus, this is
exactly what happens, with each language playing its usual r\^{o}le.

Finally, regarding the issue of refinement, we observe that the major
objective of \Circus\ is to provide a theory of refinement and an
associated calculus.  Refinement has been studied for combinations of
Object-Z and CSP~\cite{SD01}, but nothing in the style of a calculus
has been proposed.

In~\cite{HR84} a semantics for occam is presented in a style similar
to that presented here:~programs are defined as predicates over traces
and refusals, and over a state variable that plays the role of our
$okay$ and $wait$ variables.  The subset of occam considered includes
$Stop$, $Skip$, assignment, communication, conditional, loop,
sequence, alternation, and variable declarations.  Even though the
parallel construct is mentioned and laws are presented, the semantics
for parallelism is left as an exercise.

\section{Future Work} \label{section:FW}

At the moment, we are working on a number of more substantial case
studies on the use of \Circus.  We are considering an IP-packet filter
firewall~\cite{WM99}, the steam boiler control system
of~\cite{ABL96,Bau93,WC01d}, a smart card system for electronic
finance~\cite{SCW00}, and a railway signalling
application~\cite{Woo99}.

The model of \Circus\ is a Z specification that can be analysed with
almost no changes using the Z/EVES theorem prover~\cite{ZEVES00}.
Analysing the model and its properties using Z/EVES is in our research
agenda.  We are also building a tool that calculates the Z
specification corresponding to a \Circus\ program, producing a
specification that is suitable for analysis using Z/EVES.

We are already considering the extension of \Circus\ to include the
operators of Timed CSP~\cite{Dav93}.  The resulting language is
expected to be adequate to the specification of data, behavioural, and
timing aspects of real-time systems.  We intend to define its model by
extending the unifying theory of programming to cover aspects of time.
In order to solve the difficulty with the semantics of the hiding
operator, we also plan to extend our model to allow infinite
traces~\cite{Ros98}.

Our main goal, however, is the proposal and proof of refinement laws
for \Circus.  We want data refinement rules that allow, for instance,
the proof that the process $CRing$ defined in
Section~\ref{section:EARB} refines the process $Buffer$.  We also want
rules that allows the stepwise refinement of $CRing$ to code in a
calculational way.

\section*{Acknowledgments}

We would like to thank Augusto Sampaio and Adnam Sherif for many
discussions and suggestions.  The work of Ana Cavalcanti is
financially supported by CNPq, grant~\mbox{520763/98-0}.  Jim Woodcock
gratefully acknowledges the support of CNPq and the University of
Oxford for his visit to the Federal University of Pernambuco during
July and December 2000.

\bibliography{espfor}

\end{document}

% LocalWords:  CircusParagraph ChannelDefinition ChanSetDefinition CDeclaration
% LocalWords:  ProcessDefinition SimpleCDeclaration Exp CSExpression PParagraph
% LocalWords:  CSPActionExp CParameter GuardedActions FibState InitFibState bot
% LocalWords:  InitFib OutFibState OutFib FibTwice maxbuff BufferState InputCmd
% LocalWords:  BufferInit OutputCmd maxring RingIndex ControllerState ringsize
% LocalWords:  InitController CacheInput StoreInput InputController NoNewCache
% LocalWords:  StoreNewCache OutputController ControllerCycle wrt RingCell Sync
% LocalWords:  CellState CellWrite IRingCell CRing ChanEnv ChanName cdecl cdecl
% LocalWords:  sExp SComp sExp sdecl sdecl ProcEnv ProcName ZSpecification fst
% LocalWords:  modelOf PD Ps Ps PROG ProcessState tr ref ProcessStateObs CPAR
% LocalWords:  Zp tc Zp pd pd CPARL cp cp cpl cps UserState ProcObs USt SExp oc
% LocalWords:  ProcStateObs SExp OpNormal OpDiverge CommWaiting accEvents Comm
% LocalWords:  CommDone oc procObsC xUSt ic procObsPxP PxProcObs ioc snd tracea
% LocalWords:  ExtChoiceWaiting ExtChoiceNotWaiting traceb refa refb okaya sync
% LocalWords:  okayb waita waitb tracep refp SProcess po userStateObs DUSt op
% LocalWords:  userStateObs fUserState iUserState SchemaName PPar oldc newc le
% LocalWords:  rogram yntax haos ib ext ut rocess hannels pecifications nput CD
% LocalWords:  uffer escription ead ontroller ingCell ellState val ing decl cd
% LocalWords:  onstructs ecorded aragraph cd pds aragraphs ps pl serState ync
% LocalWords:  equence ync pre ewc ldc
