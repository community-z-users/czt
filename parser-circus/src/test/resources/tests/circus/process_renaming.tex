\begin{zsection}
  \SECTION\ process\_renaming \parents\ circus\_toolkit
\end{zsection}

\begin{circus}
   \circchannel\ [X] g: X \cross \nat \\
   \circchannel\ c, d: \nat \cross \nat
\end{circus}

If the process being renamed is a call, no parenthesis are needed
%
\begin{circus}
   \circprocess\ RenamingProcess1 \circdef A \lcircrename c := d \rcircrename
\end{circus}
%
whereas, if the process being renamed is a compound processes,
surrounding parenthesis are required.
%
\begin{circus}
   \circprocess\ RenamingProcess2 \circdef (A \extchoice B) \lcircrename c := g[\nat] \rcircrename
\end{circus}
%
On the other hand, trying to mix them up can be confusing, as the next example shows.
%
\begin{circus}
     \circprocess\ RenamingProcess3 \circdef A \extchoice (B) \lcircrename c := d \rcircrename
\end{circus}
%
It represents the external choice between A, and process B with channel c renamed to d.

Unfortunately, because of the nature of renaming, with respect to other Circus (and Z)
constructs (i.e., schema substitution, action substitution, etc), the grammar becomes
very tricky to normalise without conflicts. Thus, even with precedences being used
(i.e., renaming being the strongest), the following example fails to parse
%
%%%%%\begin{circus}
\[
     \circprocess\ RenamingProcess4 \circdef A \extchoice B \lcircrename c := d \rcircrename
\]
%%%%%\end{circus}

Renaming also works for processes with generic actuals themselves.
%
\begin{circus}
     \circprocess\ RenamingProcess5 \circdef A[\nat] \lcircrename c := d \rcircrename \\
     \circprocess\ RenamingProcess6 \circdef B \extchoice (A[\nat]) \lcircrename c := d \rcircrename
\end{circus}
