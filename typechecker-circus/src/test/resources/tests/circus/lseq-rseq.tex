
% Note the implicit header containing
% begin{zsection}
% \SECTION Specification \parents standard\_toolkit, circus\_toolkit
% end{zsection}

% notice you don't need - to texify, just uncomment the lines :-)
%\documentclass{article}
%\usepackage{circus}
%\begin{document}
% for it to parse / typecheck! You do to texify it.
%

First you need to find the right unicode chars for \verb'\lseq' and \verb'\rseq'
They cannot be the same as \verb'\langle' and \verb'\rangle'.
%Zprechar \lseq  %U+????  % this is \langle U+27E8
%Zpostchar \rseq %U+????  % this is \rangle U+27E9

So, I just use a "string" version instead. Note that the LaTeX
markup directive starts with a double percentual sign followed
by "%%Zxy cmd spell", where x = "'pre'/'post'/'in'/''",
y = "char/word", cmd = "latex command", spell = "U+???? / 'string'" [for char/word].
%%Zpreword \lseq  lseq
%%Zpostword \rseq rseq

Finally, although it would parse nicely, \verb'\lseq' and \verb'\rseq'
wouldn't render properly since they are not known LaTeX, you need to
define the command. This is irrespective of parsing / typechecking
\newcommand{\lseq}{\mathrm{lseq}}
\newcommand{\rseq}{\mathrm{rseq}}
Note also that, nothing stops you from defining the command as
\verb'\renewcommand{\lseq}{\langle}'
\verb'\renewcommand{\rseq}{\rangle}'
and the parser doesn't care that the underlying LaTeX rendering
is inconsistent with the LaTeX markup directive. This would only
be noticeable if you were to convert between the various formats
(e.g., UNICODE, XML, LaTeX), where you would get the $\lseq$ / $\rseq$
from the LaTex markup directive! It is good practice to keep those in synch.

Now you've characterised the new LaTeX command to be parsed, you
need to tell the parser what does it represent: an operator template
with variable arguments, where we give both the template and its definition:
\begin{zed}
\function (\lseq \listarg \rseq)
\end{zed}
\begin{zed}
\lseq \listarg \rseq [ X ]  == \lambda s : \seq X @ s
\end{zed}

To illustrate its use, we define "reverse" just like "rev" but using our new commands.
\begin{gendef}[X]
   reverse: \seq~X \fun \seq~X
\where
   reverse~ \lseq \rseq = \langle \rangle
   \\
   \forall x: X; s: \seq~X @ reverse~(\lseq x \rseq \cat s) = (reverse~s)\cat \lseq x \rseq
\end{gendef}

And some examples of usage that should be true, and certainly typecheck :-)
\begin{theorem}{seqEuiv1}
   \vdash? \lseq 1 \rseq = \langle 1 \rangle
\end{theorem}

\begin{theorem}{revEquiv1}[X]
    \vdash? \forall s: \seq~X @ reverse~s = rev~s
\end{theorem}

\begin{theorem}{revEquiv2}[X]
    \vdash? \forall x: X; s: \seq~X @ reverse~(\lseq x \rseq \cat s) = rev~(\langle x \rangle \cat s)
\end{theorem}

\end{document}
