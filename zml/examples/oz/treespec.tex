\documentclass{article}
\usepackage{ltcadiz}
\begin{document}
\begin{center}
\begin{large}
Object-Z Specification of {\em IntTree}
\end{large}
\end{center}

\begin{zsection}
  \SECTION treespec \parents oz\_toolkit, standard\_toolkit
\end{zsection}

\noindent
The class {\em IntTree} provides access to a binary tree of integers that
can be accessed through a cursor, which keeps track of the ``current node.''
The class {\em Node} represents a node in the tree, which is either a
``null'' node, or consists of a value and a left and right child.
In {\em IntTree}, the free type $DIR$ represents a ``direction'', which
is either $PARENT$, $LEFT$ (child), or $RIGHT$ (child).
The attribute $root$ represents the root node of the tree
and $curr$ the current node.
Initially the tree consists of a single node, the root node, whose value
is 0.
The function $addChild(d,v)$ adds a child node to the current node in
direction $d$; the value of the child node is $v$ and the current node
is set to the new node.
If a child already exists in direction $d$, then $addChild(d,v)$ signals
the exception $NodeExc$, and if an attempt is made to add a parent to the
root node then it signals $RootNodeExc$.
The function $clear$ removes the left and right sub-tree from the current node.
$setRoot$ moves the cursor to the root node of the tree.
The function $setValue(v)$ sets the value of the current node to $v$ and
$getValue$ returns the value of the current node.
The function $canMove(d)$ returns whether you can move in direction $d$
from the current node, and $moveCurrent(d)$ moves the cursor in direction
$d$, signaling $NotNodeExc$ if you cannot move in that direction.

\begin{zed}
Node == NullNode \classuni ValNode
\end{zed}

\begin{class}{NullNode}
\begin{state}
	\Delta \\
	subnodes : \finset Node
\where
	subnodes = \{ self \}
\end{state}
\begin{init}
  true
\end{init}\\
\end{class}

\begin{class}{ValNode}
\begin{state}
        val : \num \\
	left, right : Node \\
	\Delta \\
	subnodes : \finset Node
\where
	\langle left.subnodes, \{ self \}, right.subnodes \rangle
		\partition subnodes
\end{state}\\
\begin{init}
        val = 0 \\
	left \in NullNode \\
	right \in NullNode
\end{init}\\
\begin{op}{setNode}
        \Delta (val,left,right) \\
	v? : \num ; \\
	l?, r? : Node
\where
        val' = v? \\
        left' = l? \\
        right' = r?
\end{op}
\end{class}


\begin{class}{IntTree}
\visibility (DIR,\Init,addChild,clear,setRoot,setValue,getValue,canMove,moveCurrent) \\
DIR ::= PARENT | LEFT | RIGHT\classbreak
\begin{axdef}
	canMoveFunc: DIR \cross Node \cross Node \fun \bool
\where
	\forall d: DIR; root,curr: Node @ canMoveFunc (d,root,curr) = true \iff \\
                \t2 curr \in ValNode \land\\
		\t2 (d = PARENT \land root \neq curr \lor \\
		\t2 d = LEFT \land curr.left \notin NullNode \lor \\
		\t2 d = RIGHT \land curr.right \notin NullNode)
\end{axdef}\classbreak
\begin{state}
	root, curr : Node
\where
	curr \in root.subnodes
\end{state}\classbreak
\begin{init}
	root \in ValNode \\
        root.\Init \\
        curr = root
\end{init}\classbreak
\begin{op}{addChildAux}
        \Delta (curr) \\
	newnode : ValNode \\
	null1,null2 : NullNode \\
	oldleft,oldright: Node \\
	oldval: \num
\where
	\{ newnode \} \cap root.subnodes = \emptyset \\
	\{ null1,null2 \} \cap root.subnodes = \emptyset \\
	null1 \neq null2 \\
	curr' = newnode \\
        curr \in ValNode \implies\\
           oldleft = curr.left \land oldright = curr.right \land oldval = curr.val
\end{op}\classbreak
addChild \sdef addChildAux @ ([curr \in ValNode] \land\\
	\t2 (newnode.setNode[null1/l?,null2/r?]) \land \\
	\t2 (([ d? : DIR | d? = LEFT ]
		\land curr.setNode[oldval/v?,newnode/l?,oldright/r?]) \gch \\
	\t2 ([ d? : DIR | d? = RIGHT ]
		\land curr.setNode[oldval/v?,oldleft/l?,newnode/r?])))\\
\begin{op}{addChildNodeExc}
        d? : DIR\\
        v? : \num
\where
        canMoveFunc (d?,root,curr)
\end{op}\classbreak
\begin{op}{addChildRootNodeExc}
        d? : DIR; \\
        v? : \num
\where
        d? = PARENT \land curr = root
\end{op}\classbreak
\begin{op}{clearAux}
	null1,null2 : NullNode \\
	oldval: \num
\where
	\{ null1,null2 \} \cap root.subnodes = \emptyset \\
	null1 \neq null2 \\
	curr \in ValNode \land oldval = curr.val \lor\\
        curr \in NullNode
\end{op}\classbreak
clear \sdef clearAux @ ([curr \in ValNode] @ curr.setNode[oldval/v?,null1/l?,null2/r?]) \\
\begin{op}{setRoot}
        \Delta (curr)
\where
	curr' = root
\end{op}\classbreak
setValue \sdef [ l, r : Node | curr \in ValNode \land l =
  curr.left \land r = curr.right ] @ curr.setNode[l/l?,r/r?]
\classbreak
\begin{op}{getValue}
        v! : \num
\where
        curr \in ValNode \implies v! = curr.val
\end{op}\classbreak
\begin{op}{canMove}
	d? : DIR \\
        v! : \bool
\where
	v! = canMoveFunc(d?,root,curr)
\end{op}\classbreak
\begin{op}{moveCurrent}
        \Delta (curr) \\
        d? : DIR
\where
	d? = PARENT \implies
		\exists n : root.subnodes | n \in ValNode \implies  n.left = curr \lor n.right = curr
			@ curr' = n \\
	d? = LEFT \implies curr \in ValNode \land curr' = curr.left \\
	d? = RIGHT \implies curr \in ValNode \land curr' = curr.right
\end{op}\classbreak
\begin{op}{moveCurrentNotNodeExc}
        d? : DIR
\where
	\lnot canMoveFunc (d?,curr,root)
\end{op}
\end{class}
\end{document}
