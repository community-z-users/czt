\documentclass[11pt,a4paper]{article}
\usepackage{oz}
\usepackage{rc}
\usepackage{wizard}
\begin{document}

\begin{zsection}
  \SECTION pirate \parents oz\_toolkit, standard\_toolkit
\end{zsection}

\begin{class}{Pirate}
\also
	\begin{state}
		curPosn: \nat \\
		numSteps: \nat \\
		firstVisit: \seq \nat
	\where
		curPosn \leq 10 \\
		\#firstVisit \leq 11
	\end{state} \\
	\begin{init}
		curPosn = 0 \\
		numSteps = 0 \\
		firstVisit = \langle \rangle
	\end{init} \\
	\begin{schema}{stepLeft}
		\Delta (curPosn, numSteps)
	\where	
	        curPosn \neq 0 \\
		curPosn \neq 10 \\
		curPosn' = curPosn - 1 \\
		numSteps' = numSteps + 1 \\
	\end{schema} \\
	\begin{schema}{stepRight}
		\Delta (curPosn, numSteps, firstVisit)
	\where
	        curPosn \neq 10 \\
        	curPosn' = curPosn + 1 \\
	        numSteps' = numSteps + 1 \\
		(\#firstVisit \leq curPosn) \implies firstVisit' = firstVisit \cat
			\langle numSteps \rangle \\
		(\#firstVisit > curPosn) \implies firstVisit' = firstVisit
	\end{schema} \\
	step == stepLeft \gch stepRight
	\end{class}

\begin{class}{WalkPlank}
\also
        \begin{state}
                p : Pirate
        \end{state} \\
	\begin{init}
		p.\Init
	\end{init}\\
	\begin{schema}{fell}
        \where
		p.curPosn = 10
	\end{schema} \\
	\begin{schema}{getVisits}
		numSteps! : \seq \nat
	\where
		fell \\
		numSteps! = p.firstVisit
	\end{schema} \\
	\\
        walk == (p.step \semi walk) \gch fell
\end{class}

\begin{class}{Pirates}
\also
	\begin{axdef}
		maxPirates : \nat
	\where
		maxPirates = 10000
	\end{axdef}
	\begin{state}
		pirates : \iseq WalkPlank \\
		total : \seq \nat
	\where
		\#total = 11 \\
	\end{state} \\
	\begin{init}
		\#pirates = maxPirates \\
		\forall j : \dom total @ total(j) = 0 \\ 
		\forall p : pirates @ p.\Init
	\end{init} \\
        allwalk == \dcnj p : \ran pirates @ p.walk\\
	\begin{schema}{add}
		\Delta (total) \\
		numSteps? : \seq \nat
	\where
		\forall j : \dom total @
			total'(j) = total(j) + numSteps?(j)
	\end{schema} \\
	\begin{schema}{outputAverage}
		average! : \seq \num
	\where
		\forall j : \dom average! @
			average!(j) = total(j) \div maxPirates
	\end{schema} \\
	getAverages == (\dsqc p : \ran pirates @ p.getVisits \parallel add) \\
		\t2 \semi \\
		\t2 outputAverage
\end{class}

\end{document}
