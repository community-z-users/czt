\documentclass{article}
\usepackage{oz}   % oz or z-eves or fuzz styles

\begin{zsection}
  \SECTION unfold \parents zpattern\_toolkit, standard\_toolkit
\end{zsection}

Declare the jokers used in these rules.

\begin{zedjoker}{DeclList} D, D1, D2, D3, D4 \end{zedjoker}
\begin{zedjoker}{Pred} P, P1, P2, P3, P4 \end{zedjoker}
\begin{zedjoker}{Expr} E, E1, E2, E3, E4, E5, E6, E7, E8, E9 \end{zedjoker}
\begin{zedjoker}{Name} v, v1, v2, v3, v4 \end{zedjoker}

To clearly distinguish our rules for unfolding declaration lists and
schema expressions from other rules, we introduce two new infix
operators: $\sexprUnfoldsTo$ and $\declListUnfoldsTo$.  
We do not really need to define their semantics in order to use them within
rules, but to reassure readers, we define their semantics to be just
equality.  In fact, the intention is that the right hand argument
will be the unfolded and normalized form of the left hand argument.

\begin{verbatim}
%%Zinword \sexprUnfoldsTo sexprUnfoldsTo
%%Zinword \declListUnfoldsTo declListUnfoldsTo
\end{verbatim}

\begin{zed}
  \relation ( \_ \sexprUnfoldsTo \_ )
\end{zed}
\begin{zed}
  \relation ( \_ \declListUnfoldsTo \_ )
\end{zed}


\begin{gendef}[SCHEMA]
  \_ \sexprUnfoldsTo \_ : SCHEMA \rel SCHEMA \\
  \_ \declListUnfoldsTo \_ : SCHEMA \rel SCHEMA
\where
  \forall s1,s2:SCHEMA @ s1 \sexprUnfoldsTo s2 \iff s1=s2 \\
  \forall s1,s2:SCHEMA @ s1 \declListUnfoldsTo s2 \iff s1=s2 \\
\end{gendef}

\section{Rules for Unfolding Predicates}

%% Syntax error on the conclusion of these rules.
%\begin{zedrule}{implies}
%   (P1 \implies P2) \iff ((\lnot P1) \lor P2)
%\end{zedrule}

%\begin{zedrule}{iff}
%   (P1 \iff P2) \iff ((P1 \land P2) \lor ((\lnot P1) \land (\lnot P2))
%\end{zedrule}

\section{Unfolding Declaration Lists}

We use the $\declListUnfoldsTo$ operator for unfolding declaration
lists.   The left-hand side is always of the form $[DeclList|true]$
and the right-hand side (which is usually a generated output) is
a normalized schema with no duplicated names in the signature, and
all the non-basetype conditions in the predicate part.

The next few rules implement the $\declListUnfoldsTo$ operator,
which unfolds and normalizes declaration lists.

\begin{zedrule}{VarDecl}
   \proviso E : \power E2 \\
   [D1 | true] \declListUnfoldsTo [D2 | P2]
\derives
   [v:E; D1 | true] \declListUnfoldsTo [v:E2; D2 |  v \in E \land P2]
\end{zedrule}

\begin{zedrule}{ConstDecl}
   \proviso E : E2 \\
   [D1 | true] \declListUnfoldsTo [D2 | P2]
\derives
   [v==E; D1 | true] \declListUnfoldsTo [v:E2; D2 |  v = E \land P2]
\end{zedrule}

\begin{zedrule}{IncludeDecl}
   E \sexprUnfoldsTo [D1 | P1] \\
   [D | true] \declListUnfoldsTo [D2 | P2] \\
   \proviso [D1 | true] \land [D2 | true] : \power [D3]
\derives
   [E; D | true] \declListUnfoldsTo [D3 |  P1 \land P2]
\end{zedrule}

\begin{zedrule}{EmptyDeclList}
   [~ | true] \declListUnfoldsTo [~ | true]
\end{zedrule}


\section{Unfolding Schema Expressions}

This section defines the unfolding of schema expressions,
using the $se \sexprUnfoldsTo stext$ operator, where $se$
is a schema expression and $stext$ is the resulting normalized
schema construction ($[Decls|Preds]$, where the types in $Decls$
are only base types).

Top-level schema expressions are unfolded into sets of bindings.
However, we have a special case for $\exists$, so that we can
put all the variables into the bound variable list of the
set comprehension, then create a binding using a subset of
those variables.  Putting all the variables at the same level
of scope gives the evaluation optimization algorithms more
freedom to reorder things later.

\begin{zedrule}{TopLevelExistsSchema}
   [D|P] \sexprUnfoldsTo [D1 | P1] \\
   E2 \sexprUnfoldsTo [D2 | P2] \\
   \proviso [D1 | true] \land [D2 | true] : \power [D3] \\
   \proviso [D4|true] == [D2|true] \schemaminus [D1|true] \\
   \proviso E == binding ~ [D4|true]
\derives
   (\exists D | P @ E2) = \{D3 | P1 \land P2 @ E\}
\end{zedrule}

\begin{zedrule}{TopLevelSchemaExpr}
  E \sexprUnfoldsTo [D1|P1] \\
  \proviso E2 == binding [D1 | true]
\derives
  E = \{ D1 | P1 @ E2 \}
\end{zedrule}


\begin{zedrule}{SchemaConstruction}
   [D1 | true] \declListUnfoldsTo [D2 | P2]
\derives
   [D1 | P] \sexprUnfoldsTo [D2 | P2 \land P]
\end{zedrule}

\begin{zedrule}{SchemaRef}
  \proviso ?~ v == E2 \\
  E2 \sexprUnfoldsTo [D2 | P2]
\derives
  v \sexprUnfoldsTo [D2 | P2]
\end{zedrule}

This rule unfolds any remaining $\Delta$ schemas.
If the specification explicitly defined the $\Delta$ schema,
then the above SchemaRef rule would have unfolded it.
\begin{zedrule}{DeltaRef}
  \proviso v2 == \Delta \unprefix v \\
  [v2; v2~'] \sexprUnfoldsTo [D1|P1]
\derives
  v \sexprUnfoldsTo [D1|P1]
\end{zedrule}

This rule unfolds any remaining $\Xi$ schemas.
If the specification explicitly defined the $\Xi$ schema,
then the above SchemaRef rule would have unfolded it.
\begin{zedrule}{XiRef}
  \proviso v2 == \Xi \unprefix v \\
  [v2; v2~'] \sexprUnfoldsTo [D1|P1] \\
  \proviso [v2|true] : \power [D2]
\derives
  v \sexprUnfoldsTo [D1|P1 \land \theta [D2|true] = \theta [D2|true]~']
\end{zedrule}

\begin{zedrule}{SchemaPrime}
   E \sexprUnfoldsTo [D1 | P1] \\
   \proviso [D2|P2] == [D1|P1]~' \\
\derives
   E~' \sexprUnfoldsTo [D2 | P2]
\end{zedrule}

The type proviso in the ExistsSchema rule checks
that $D1$ and $D2$ are type compatible.
\begin{zedrule}{ExistsSchema}
   [D|P] \sexprUnfoldsTo [D1 | P1] \\
   E2 \sexprUnfoldsTo [D2 | P2] \\
   \proviso [D1 | true] \land [D2 | true] : \power [D3] \\
   \proviso [D4|true] == [D2|true] \schemaminus [D1|true]
\derives
   (\exists D | P @ E2) \sexprUnfoldsTo [D4 | (\exists D1 @ P1 \land P2)]
\end{zedrule}

This rule applies the above schema unfolding rules to unfolding
the bound variables of quantifiers, because the rewrite tactics
transform schema text using the $\schemaEquals$ relation.
\begin{zedrule}{Quantifiers}
   [D|P] \sexprUnfoldsTo [D1|P1]
\derives
   [D|P] \schemaEquals [D1|P1]
\end{zedrule}


\begin{zsection}
  \SECTION birthdaybook \parents standard\_toolkit, unfold
\end{zsection}


\begin{document}
This is the BirthdayBook specification, from 
Spivey~\cite{spivey:z-notation2}.  We extend it slightly
by adding an extra operation, $RemindOne$, that is non-deterministic.

\begin{zed}
   [NAME, ~ DATE] 
\end{zed}

The $BirthdayBook$ schema defines the \emph{state space} of 
the birthday book system. 

\begin{schema}{BirthdayBook}
    known: \power NAME \\
    birthday: NAME \pfun DATE
\where
    known=\dom birthday
\end{schema}

This $InitBirthdayBook$ specifies the initial state
of the birthday book system.  It does not say explicitly that
$birthday'$ is empty, but that is implicit, because its domain
is empty.

\begin{schema}{InitBirthdayBook}
    BirthdayBook~'
\where
    known' = \{ \}
\end{schema}

\begin{zed}
X == [BirthdayBook~' | known' = \{ \}]
\end{zed}

Next we have several operation schemas to define the normal (non-error)
behaviour of the system.

\begin{schema}{AddBirthday}
    \Delta BirthdayBook \\
    name?: NAME \\
    date?: DATE
\where
    name? \notin known
\\
    birthday' = birthday \cup \{name? \mapsto date?\}
\end{schema}

\begin{schema}{FindBirthday}
    \Xi BirthdayBook \\
    name?: NAME \\
    date!: DATE 
\where
    name? \in known
\\
    date! = birthday(name?)
\end{schema}

\begin{schema}{Remind}
    \Xi BirthdayBook \\
    today?: DATE \\
    cards!: \power NAME
\where
    cards! = \{ n: known | birthday(n) = today? \}
\end{schema}

This $RemindOne$ schema does not appear in Spivey, but is
included to show how non-deterministic schemas can be animated.
It reminds us of just one person who has a birthday on the given 
day.
\begin{schema}{RemindOne}
    \Xi BirthdayBook \\
    today?: DATE \\
    card!: NAME
\where
    card! \in known \\
    birthday ~ card! = today?
\end{schema}


Now we strengthen the specification by adding error handling.

\begin{zed} 
    REPORT ::= ok | already\_known | not\_known
\end{zed}

First we define auxiliary schemas that capture various success
and error cases.

\begin{schema}{Success}
    result!: REPORT
\where
    result! = ok
\end{schema}

\begin{schema}{AlreadyKnown}
    \Xi BirthdayBook \\
    name?: NAME \\
    result!: REPORT
\where
    name? \in known \\
    result! = already\_known
\end{schema}


\begin{schema}{NotKnown}
    \Xi BirthdayBook \\
    name?: NAME \\
    result!: REPORT
\where
    name? \notin known \\
    result! = not\_known
\end{schema}

Finally, we define robust versions of all the operations
by specifying how errors are handled.  
For illustration purposes, we leave the $RemindOne$ operation non-robust.

\begin{zed} 
    RAddBirthday == (AddBirthday \land Success) \lor AlreadyKnown \\
    RFindBirthday == (FindBirthday \land Success) \lor NotKnown \\
    RRemind == Remind \land Success
\end{zed}

\bibliography{spec}
\begin{thebibliography}{1}
\bibitem{spivey:z-notation2}
J.~Michael Spivey.
\newblock {\em The Z Notation: A Reference Manual}.
\newblock International Series in Computer Science. Prentice-Hall International
  (UK) Ltd, second edition, 1992.
\end{thebibliography}
\end{document}
