\documentclass{article}
\usepackage{oz}   % oz or z-eves or fuzz styles
\newenvironment{machine}[1]{
    \begin{tabular}{@{\qquad}l}\textbf{\kern-1em machine}\ #1\\ }{
    \\ \textbf{\kern-1em end} \end{tabular} }
\newcommand{\machineInit}{\\ \textbf{\kern-1em init} \\}
\newcommand{\machineOps}{\\ \textbf{\kern-1em ops} \\}

\begin{document}
This is the BirthdayBook specification, from 
Spivey~\cite{spivey:z-notation2}.  We extend it slightly
by adding an extra operation, $RemindOne$, that is non-deterministic.

%    [NAME, DATE] 
\begin{zed}
   NAME == 1 \upto 5
\end{zed}
\begin{zed}
   DATE == 10 \upto 15
\end{zed}

The $BirthdayBook$ schema defines the \emph{state space} of 
the birthday book system. 

\begin{schema}{BirthdayBook}
    known: \power NAME \\
    birthday: NAME \pfun DATE
\where
    known=\dom birthday
\end{schema}

This $InitBirthdayBook$ specifies the initial state
of the birthday book system.  It does not say explicitly that
$birthday'$ is empty, but that is implicit, because its domain
is empty.

\begin{schema}{InitBirthdayBook}
    known': \power NAME \\
    birthday': NAME \pfun DATE
\where
    known' =\dom birthday' \\
    known' = \{ \}
\end{schema}

Next we have several operation schemas to define the normal (non-error)
behaviour of the system.

\begin{schema}{AddBirthday}
    known, known': \power NAME \\
    birthday, birthday': NAME \pfun DATE \\
    name?: NAME \\
    date?: DATE
\where
    known = \dom birthday \\
    known' = \dom birthday' \\
    name? \notin known
\\
    birthday' = birthday \cup \{name? \mapsto date?\}
\end{schema}

\begin{schema}{FindBirthday}
    known, known': \power NAME \\
    birthday, birthday': NAME \pfun DATE \\
    name?: NAME \\
    date!: DATE 
\where
    known = \dom birthday \\
    known' = \dom birthday' \\
    known' = known \\
    birthday' = birthday \\
    name? \in known
\\
    date! = birthday(name?)
\end{schema}

\begin{schema}{Remind}
    known, known': \power NAME \\
    birthday, birthday': NAME \pfun DATE \\
    today?: DATE \\
    cards!: \power NAME
\where
    known = \dom birthday \\
    known' = \dom birthday' \\
    known' = known \\
    birthday' = birthday \\
    cards! = \{ n: known | birthday(n) = today? \}
\end{schema}

This $RemindOne$ schema does not appear in Spivey, but is
included to show how non-deterministic schemas can be animated.
It reminds us of just one person who has a birthday on the given 
day.
\begin{schema}{RemindOne}
    known, known': \power NAME \\
    birthday, birthday': NAME \pfun DATE \\
    today?: DATE \\
    card!: NAME
\where
    known = \dom birthday \\
    known' = \dom birthday' \\
    known' = known \\
    birthday' = birthday \\
    card! \in known \\
    birthday ~ card! = today?
\end{schema}


Now we strengthen the specification by adding error handling.

\begin{zed} 
    REPORT ::= ok | already\_known | not\_known
\end{zed}

First we define auxiliary schemas that capture various success
and error cases.

\begin{schema}{Success}
    result!: REPORT
\where
    result! = ok
\end{schema}

\begin{schema}{AlreadyKnown}
    known, known': \power NAME \\
    birthday, birthday': NAME \pfun DATE \\
    name?: NAME \\
    result!: REPORT
\where
    known = \dom birthday \\
    known' = \dom birthday' \\
    known' = known \\
    birthday' = birthday \\
    name? \in known \\
    result! = already\_known
\end{schema}


\begin{schema}{NotKnown}
    known, known': \power NAME \\
    birthday, birthday': NAME \pfun DATE \\
    name?: NAME \\
    result!: REPORT
\where
    known = \dom birthday \\
    known' = \dom birthday' \\
    known' = known \\
    birthday' = birthday \\
    name? \notin known \\
    result! = not\_known
\end{schema}

Finally, we define robust versions of all the operations
by specifying how errors are handled.  
For illustration purposes, we leave the $RemindOne$ operation non-robust.

\begin{zed} 
    RAddBirthday == (AddBirthday \land Success) \lor AlreadyKnown \\
    RFindBirthday == (FindBirthday \land Success) \lor NotKnown \\
    RRemind == Remind \land Success
\end{zed}

Finally, we can (optionally) define an explicit state machine,
by identifying the state and initialisation schemas and the
operations.  This is not necessary for the animator, but is
useful to make the roles of the various schemas explicit, rather
than relying on informal naming conventions.  
This \emph{machine} construct is a Java-specific extension of Z
and will be ignored by other Z tools (because they do not
recognise the \verb!\begin{machine}..\end{machine}! \LaTeX environment.

Jaza uses this construct for translating the machine into the B
specification language or other similar languages and (in the future) for
constructing GUI animation interface for the machine.


\begin{machine}{BirthdayBook}
  BirthdayBook
\machineInit
  InitBirthdayBook
\machineOps
    RAddBirthday; 
    RFindBirthday;
    RRemind;
    RemindOne
\end{machine}

  
Here is an alternative style of defining a machine -- as a
schema with fields named $state$ and $init$, plus other
fields for the operations of the machine.  
Advantages of this approach include:
\begin{enumerate}

\item This \emph{machine} construct is a standard Z schema, 
  so all the usual Z operators can be used to construct machines, 
  add and hide operations etc.  

\item The operation names are local to this schema rather than global, 
  so short, non-qualified names can be used.  Furthermore, several
  machines can have the same operation name.

\item Other (non-schema) fields inside the machine can represent
  global constants of the machine, and the invariant can be used
  to constrain these constants.

\item A generic schema can represent a parameterised machine.  
\end{enumerate}

Disadvantages: 
\begin{enumerate}
\item The machine name cannot be the same as its state schema.  
\item Some restrictions are probably necessary on the invariant
  of the machine, because it is difficult to see what should be
  the meaning of a predicate that constrains two of the operations.
  (This has no obvious translation into B.  It is not expressive
  enough to act as a history invariant.)
\end{enumerate}

\begin{schema}{BBook}
    state   : BirthdayBook \\
    init    : InitBirthdayBook \\
    add     : RAddBirthday \\
    find    : RFindBirthday \\
    remind  : RRemind \\
    remind1 : RemindOne
\end{schema}


% \bibliographystyle{plain}
\bibliography{spec}
\begin{thebibliography}{1}
\bibitem{spivey:z-notation2}
J.~Michael Spivey.
\newblock {\em The Z Notation: A Reference Manual}.
\newblock International Series in Computer Science. Prentice-Hall International
  (UK) Ltd, second edition, 1992.
\end{thebibliography}
\end{document}

%%% Local Variables: 
%%% mode: latex
%%% TeX-master: t
%%% End: 
