\documentclass[11pt]{article}
\usepackage{ltcadiz}

\begin{document}

\begin{zsection}
  \SECTION graph \parents oz\_toolkit, standard\_toolkit
\end{zsection}

\zedindent=2mm
\begin{class}{Graph[DATA]}
\also
\begin{state}
nodes:  \finset DATA \\ 
edges: DATA \rel DATA
\where
 \dom (edges) \cup  \ran (edges) \subseteq nodes \\ 
 \lnot (\exists x: DATA@ (x, x) \in edges)
\end{state}\classbreak
\begin{init}
nodes = \emptyset \\ 
edges = \emptyset
\end{init} \classbreak
\begin{op}{AddNode}
\Delta(nodes)\\
d?: DATA 
\where
 d? \notin nodes \\ 
 nodes' = nodes \cup \{d?\} 
\end{op} \classbreak
\begin{op}{RemoveNode}
\Delta (nodes)\\
d?: DATA 
\where
  \lnot \exists n: nodes @ (n, d?) \in edges \lor 
    (d?, n) \in edges\\ 
 nodes' = nodes \setminus \{d?\} 
\end{op} \classbreak
\begin{op}{AddEdge}
\Delta (edges)\\
x?, y?: DATA 
\where
 \{x?, y?\} \subseteq nodes \land  x? \neq y? \land 
         (x?, y?) \notin edges \\ 
 edges' = edges \cup \{(x?, y?)\} 
\end{op}\classbreak
\begin{op}{RemoveEdge}
\Delta (edges)\\
x?, y?: DATA 
\where
 (x?, y?) \in edges \\ 
 edges' = edges \setminus \{(x?, y?)\} \\ 
\end{op} \classbreak
\begin{op}{FindPath}
x?, y?: DATA \\ 
s!:  \iseq nodes 
\where
 (x?, y?) \in edges \plus  \land x? \neq y? \land\\
       \exists p : \iseq nodes @ head(p) = x? \land last(p) = y? \land\\
       \forall i: 1 \upto ( \# p - 1)@ (p(i), p(i + 1)) \in edges \\ 
 head(s!) = x? \land last(s!) = y?\\ 
\forall i: 1 \upto ( \# s! - 1)@ (s!(i), s!(i + 1)) \in edges 
\end{op} 
\end{class} 
\zedindent=\leftmargini
\end{document}
