\documentclass{article}
\usepackage{czt}
\newcommand{\sexprUnfoldsTo}{\mathrel{=_{se}}}
\newcommand{\declListUnfoldsTo}{\mathrel{=_d}}
\newcommand{\unprefix}{\mathrel{unprefix}}
\newcommand{\schemaEquals}{\mathrel{=_S}}

\title{Schema Unfolding Rules in CZT}
\author{Mark Utting}
\begin{document}
\maketitle

\begin{zsection}
  \SECTION unfold \parents proviso\_toolkit, standard\_toolkit
\end{zsection}

This Z section contains rules for unfolding and normalising declarations
and schema expressions.  For example, given a schema expression such as 
$E \land [x:\nat | p~x]$ where $E$ is the name of a schema, these rules
define a normalisation process that will transform this schema expression
into something of the form $[D|P]$, where $D$ is a list of variable
declarations whose types are all base types and $P$ is a predicate that
will contain constraints like $x \in \nat$ and $p~x$.

We start by declaring the jokers used in all the rules in this section.

To clearly distinguish our rules for unfolding declaration lists and
schema expressions from other rules, we introduce two new infix
operators: $\sexprUnfoldsTo$ and $\declListUnfoldsTo$.  
We do not really need to define their semantics in order to use them within
rules, but to reassure readers, we define their semantics to be just
equality.  In fact, the intention is that the right hand argument
will be the unfolded and normalized form of the left hand argument.

\begin{verbatim}
%%Zinword \sexprUnfoldsTo sexprUnfoldsTo
%%Zinword \declListUnfoldsTo declListUnfoldsTo
\end{verbatim}

\begin{zed}
  \relation ( \_ \sexprUnfoldsTo \_ )
\end{zed}
\begin{zed}
  \relation ( \_ \declListUnfoldsTo \_ )
\end{zed}


\begin{gendef}[SCHEMA]
  \_ \sexprUnfoldsTo \_ : SCHEMA \rel SCHEMA \\
  \_ \declListUnfoldsTo \_ : SCHEMA \rel SCHEMA
\where
  \forall s1,s2:SCHEMA @ s1 \sexprUnfoldsTo s2 \iff s1=s2 \\
  \forall s1,s2:SCHEMA @ s1 \declListUnfoldsTo s2 \iff s1=s2 \\
\end{gendef}


\section{Unfolding Declaration Lists}

We use the $\declListUnfoldsTo$ operator for unfolding declaration
lists.   The left-hand side is always of the form $[DeclList|true]$
and the right-hand side (which is usually a generated output) is
of the form $[D|P]$, where the declaration list $D$ has no duplicated
names and all its types are Z base types, and the predicate $P$
contains any additional typing constraints that were in DeclList.

The next few rules specify the $\declListUnfoldsTo$ operator.
The rules are intended to recurse through a declaration list from left
to right, with the base case of an empty declaration list being handled
by the EmptyDeclList rule.  Note that multiple declarations such as
$x,y,z:T$ are expanded out into separate declarations before rules
are applied, so the rules that follow cover all possible kinds
of declarations.

The VarDecl1 rule is a special case of VarDecl2, which applies when $E$ is
already a base type.  Since VarDecl1 comes before VarDecl2 in this file,
the rewrite tactic gives it higher priority, and this avoids introducing
redundant tautologies (such as $E \in \arithmos$, which is guaranteed to
be true by the type system) into the predicate.  This is an example of
how we can influence the behaviour of the rewrite tactic by placing more
specific rules before more general ones.  Of course, in the interactive
prover, the user could choose to apply either rule when $E$ is a base type.

\begin{zedrule}{VarDecl1}
   E \hasType \power E \\
   [D1 | true] \declListUnfoldsTo [D2 | P2]
\derives
   [v:E; D1 | true] \declListUnfoldsTo [v:E; D2 |  P2]
\end{zedrule}

\begin{zedrule}{VarDecl2}
   E \hasType \power E2 \\
   [D1 | true] \declListUnfoldsTo [D2 | P2]
\derives
   [v:E; D1 | true] \declListUnfoldsTo [v:E2; D2 |  v \in E \land P2]
\end{zedrule}

\begin{zedrule}{ConstDecl}
   E \hasType E2 \\
   [D1 | true] \declListUnfoldsTo [D2 | P2]
\derives
   [v==E; D1 | true] \declListUnfoldsTo [v:E2; D2 |  v = E \land P2]
\end{zedrule}

This rule creates a subgoal $E \sexprUnfoldsTo \ldots$ that will
invoke the schema expression rules in the next section.  So the
two sets of $\declListUnfoldsTo$ and $\sexprUnfoldsTo$ rules are
mutually recursive.

\begin{zedrule}{IncludeDecl}
   E \sexprUnfoldsTo [D1 | P1] \\
   [D | true] \declListUnfoldsTo [D2 | P2] \\
   ([D1 | true] \land [D2 | true])\hasType \power [D3 | true] 
\derives
   [E; D | true] \declListUnfoldsTo [D3 |  P1 \land P2]
\end{zedrule}

\begin{zedrule}{EmptyDeclList}
   [~ | true] \declListUnfoldsTo [~ | true]
\end{zedrule}


\section{Unfolding Schema Expressions}

This section defines the unfolding of schema expressions,
using the $SE \sexprUnfoldsTo STEXT$ operator, where $SE$
is a schema expression and $STEXT$ is the resulting normalized
schema construction ($[Decls|Preds]$, where the types in $Decls$
are only base types).

Top-level schema expressions are unfolded into schema construction
expressions, which can then be expanded into sets of bindings.
However, we have a special case for $\exists$, so that we can
put all the variables into the bound variable list of the
set comprehension, then create a binding using a subset of
those variables.  Putting all the variables at the same level
of scope gives the evaluation optimization algorithms more
freedom to reorder things later.

\begin{zedrule}{TopLevel}
  E \hasType \power [D2 | true] \\
  E  \sexprUnfoldsTo [D | P]
\derives
  E = [D | P]
\end{zedrule}

\begin{zedrule}{Theta}
  E \hasType \power [D | true] \\
  \proviso E2 == \theta [D | true]
\derives
  \theta E = E2
\end{zedrule}

NOTE: we need thirteen versions of this next rule -- one for
each possible decoration.   In the future we will add
support for Stroke-List jokers, so that all decorations can
be handled by a single rule.

\begin{zedrule}{ThetaPrime}
  E \hasType \power [D | true] \\
  \proviso E2 == \theta [D | true] '
\derives
  \theta E~' = E2
\end{zedrule}

\begin{zedrule}{ThetaDecor9}
  E \hasType \power [D | true] \\
  \proviso E2 == \theta [D | true]~_{9}
\derives
  \theta E~_{9} = E2
\end{zedrule}

\begin{zedrule}{SchemaConstruction1}
  [D1 | true] \declListUnfoldsTo [D2 | true]
\derives
  [D1 | P] \sexprUnfoldsTo [D2 | P]
\end{zedrule}

\begin{zedrule}{SchemaConstruction2}
  [D1 | true] \declListUnfoldsTo [D2 | P2]
\derives
  [D1 | P] \sexprUnfoldsTo [D2 | P2 \land P]
\end{zedrule}

\begin{zedrule}{SchemaRef}
  E \hasDefn E2 \\
  E2 \sexprUnfoldsTo [D2 | P2]
\derives
  E \sexprUnfoldsTo [D2 | P2]
\end{zedrule}

This rule unfolds any remaining $\Delta$ schemas.
If the specification explicitly defined the $\Delta$ schema,
then the above SchemaRef rule would have unfolded it.

\begin{zedrule}{DeltaRef}
  \proviso v2 == \Delta \unprefix v \\
  [v2; v2~'] \sexprUnfoldsTo [D1|P1]
\derives
  v \sexprUnfoldsTo [D1|P1]
\end{zedrule}

This rule unfolds any remaining $\Xi$ schemas.
If the specification explicitly defined the $\Xi$ schema,
then the above SchemaRef rule would have unfolded it.

\begin{zedrule}{XiRef}
  \proviso v2 == \Xi \unprefix v \\
  [v2; v2~'] \sexprUnfoldsTo [D1|P1] \\
  [v2|true] \hasType \power [D2 | true] 
\derives
  v \sexprUnfoldsTo [D1|P1 \land \theta [D2|true] = \theta [D2|true]~']
\end{zedrule}

NOTE: we also currently need thirteen versions of this next rule 
-- one for each possible decoration. 

\begin{zedrule}{SchemaPrime}
   E \sexprUnfoldsTo [D1 | P1] \\
   \proviso [D2|P2] == [D1|P1]~' \\
\derives
   E~' \sexprUnfoldsTo [D2 | P2]
\end{zedrule}

\begin{zedrule}{SchemaDecor9}
   E \sexprUnfoldsTo [D1 | P1] \\
   \proviso [D2|P2] == [D1|P1]~_{9} \\
\derives
   E~_{9} \sexprUnfoldsTo [D2 | P2]
\end{zedrule}

The type proviso in the ExistsSchema rule checks
that $D1$ and $D2$ are type compatible.  The $\schemaminus$
operator in the proviso calculates $D2 - D1$.  That is,
$D4$ will contain all the declarations that appear in $D2$
but do not appear in $D1$. 

\begin{zedrule}{ExistsSchema}
   [D|P] \sexprUnfoldsTo [D1 | P1] \\
   E2 \sexprUnfoldsTo [D2 | P2] \\
   ([D1 | true] \land [D2 | true]) \hasType \power [D3 | true] \\
   \proviso [D4|true] == [D2|true] \schemaminus [D1|true]
\derives
   (\exists D | P @ E2) \sexprUnfoldsTo [D4 | (\exists D1 @ P1 \land P2)]
\end{zedrule}

The semantics of schema negation requires the schema to be
normalised before the predicate is negated.

\begin{zedrule}{SchemaNegation}
  E \sexprUnfoldsTo [D | P]
\derives
  (\lnot E) \sexprUnfoldsTo [D | \lnot P]
\end{zedrule}

\begin{zedrule}{SchemaConjunction}
  E1 \sexprUnfoldsTo [D1 | P1] \\
  E2 \sexprUnfoldsTo [D2 | P2] \\
  ([D1 | true] \land [D2 | true]) \hasType \power [D3 | true]
\derives
  (E1 \land E2) \sexprUnfoldsTo [D3 | P1 \land P2]
\end{zedrule}

\begin{zedrule}{SchemaDisjunction}
  E1 \sexprUnfoldsTo [D1 | P1] \\
  E2 \sexprUnfoldsTo [D2 | P2] \\
  ([D1 | true] \land [D2 | true]) \hasType \power [D3 | true]
\derives
  (E1 \lor E2) \sexprUnfoldsTo [D3 | P1 \lor P2]
\end{zedrule}

\begin{zedrule}{SchemaImplication}
  E1 \sexprUnfoldsTo [D1 | P1] \\
  E2 \sexprUnfoldsTo [D2 | P2] \\
  ([D1 | true] \land [D2 | true])\hasType \power [D3 | true]
\derives
  (E1 \implies E2) \sexprUnfoldsTo [D3 | P1 \implies P2]
\end{zedrule}

\begin{zedrule}{SchemaEquivalence}
  E1 \sexprUnfoldsTo [D1 | P1] \\
  E2 \sexprUnfoldsTo [D2 | P2] \\
  ([D1 | true] \land [D2 | true]) \hasType \power [D3 | true]
\derives
  (E1 \iff E2) \sexprUnfoldsTo [D3 | P1 \iff P2]
\end{zedrule}

Schema projection, $E1 \project E2$, is similar to schema conjunction,
but the resulting schema has only the names that are declared in $E2$.
Any other names that are declared in $E1$ are hidden existentially.

\begin{zedrule}{SchemaProjection}
  E1 \sexprUnfoldsTo [D1 | P1] \\
  E2 \sexprUnfoldsTo [D2 | P2] \\
  ([D1 | true] \land [D2 | true]) \hasType \power [D3 | true] \\
  \proviso [D4 | true] == [D3 | true] \schemaminus [D2 | true]
\derives
  E1 \project E2 \sexprUnfoldsTo [D2 | (\exists D4 @ P1 \land P2)]
\end{zedrule}

To calculate the precondition of a schema, we must existentially
hide all the output names, such as $x!$ or $x'$.  The $postnames$ 
operator in the following rule takes a normalised schema as input and
returns (in $D2$) just the declarations whose names have a final decoration 
of $!$ or $'$.  These names are removed from $D4$, so that $D3$ contains 
only the declarations of names that should appear in the precondition, 
such as $x$, $x?$ and $x_2$.

\begin{zedrule}{SchemaPrecondition}
  E \sexprUnfoldsTo [D4 | P2] \\
  \proviso [D2 | true] == postnames [D4 | true] \\
  \proviso [D3 | true] == [D4 | true] \schemaminus [D2 | true] \\
\derives
  \pre E \sexprUnfoldsTo [D3 | (\exists D2 @ P2)]
\end{zedrule}

\begin{zedrule}{SchemaComposition}
  E1 \hasType \power [D1 | true] \\
  E2 \hasType \power [D2 | true] \\
  \proviso [D3 | true] == primenames [D1 | true] \\
  \proviso [D4 | true] == undecorate [D3 | true] \\
  \proviso [D5 | true] == [D4 | true] \schemaminus [D2 | true] \\
  \proviso E6 == [D4 | true] \schemaminus [D5 | true] % matching names
\derives
  E1 \semi E2 =
  (\exists E6~_9 @ (\exists E6~' @ [E1 | \theta E6~' = \theta E6~_9])
                   \land
                   (\exists E6   @ [E2 | \theta E6   = \theta E6~_9]))
\end{zedrule}

TODO: schema hiding, schema piping,
schema forall, schema exists unique.


\section{Rules for Unfolding Declarations}

When rewriting terms involving quantifiers, we want to
rewrite and expand the schema text $E$ within quantifiers
such as $\exists E @ P$ or $(\lambda E@E2)$, as well as rewriting
the predicates and expressions within the body of the quantifiers.
The rewrite tactics try to transform such schema text using the 
$\schemaEquals$ relation.  So the following rule applies the above 
schema unfolding rules to unfolding any schema text that appears within 
the bound variable part of quantifiers.
\begin{zedrule}{Quantifiers}
   [D|P] \sexprUnfoldsTo [D1|P1]
\derives
   [D|P] \schemaEquals [D1|P1]
\end{zedrule}


\section{Rules for Unfolding Predicates}

If a schema expression $E$ is used as a predicate, it is equivalent to
checking that $\theta E$ (a binding constructed from names that are
currently in scope) satisfies the predicate part of $E$ 
(including any subtypes in the declarations).  So this rule
transforms any expression that is used as predicate into $\theta E \in E$.
\begin{zedrule}{SchemaPred}
  E \iff \theta E \in E
\end{zedrule}

%% Syntax error on the conclusion of these rules.
%\begin{zedrule}{implies}
%   (P1 \implies P2) \iff ((\lnot P1) \lor P2)
%\end{zedrule}

%\begin{zedrule}{iff}
%   (P1 \iff P2) \iff ((P1 \land P2) \lor ((\lnot P1) \land (\lnot P2))
%\end{zedrule}

\end{document}
