\begin{zsection}
  \SECTION normalization\_rules \\~~ \parents oracle\_toolkit
\end{zsection}

Declarations in Z can consist of complicated expressions.  A schema is
said to be normalized if its declarations are ``simple'': all
declarations are variable declarations, there are no duplicated names
declared, and all expressions occuring in the declarations are Z
carrier sets.  This section defines a $normalize$ operator, a helper
relation $pred$ and corresponding rules to transform an arbitrary
schema expression into a normalized schema expression that is
semantically equivalent.

\begin{gendef}[SIG1,SIG2]
  normalize : SIG1 \fun SIG2 \\
\end{gendef}

\begin{zed}
  \relation ( pred~\_ )
\end{zed}

\begin{zedrule}{Normalize}
   [D|P] :: \power [D1|true] \\
\derives
  normalize [D|P] = \\~~ [D1 | pred([D|true]) \land P]
\end{zedrule}
\TBDpar{IT}{Fix indent of joker and rule paragraphs to be same as other Z paragraphs.}

The Normalize rule given above defines the normalize operation.  Given
a schema expression $[D|P]$ (where $D$ is a declaration list and $P$
is a predicate), the normalized schema is of the form
$$[D1 | pred([D|true])\land P]$$ The declaration list $D1$ is
obtained from the type of the given schema expression.  This exploits
the fact that the carrier set of the schema type contains a normalized
declaration list.  The helper relation $pred$ identifies additional
typing constraints that were in the original declaration list but are
omitted in the normalized declaration list.  This predicate is
conjoined with the original predicate~$P$.

The following rules define the helper relation $pred$.  The rules are
intended to recurse through a declaration list from left to right,
with the base case of an empty declaration list being handled by the
PredEmptyDecl rule.  Note that multiple declarations such as $x,y,z:T$
are expanded out into separate declarations before rules are applied,
so the rules that follow cover all possible kinds of declarations.

The PredVarDecl1 rule is a special case of PredVarDecl2.  It applies
when the expression~$E$ is already a carrier set.  Since VarDecl1
comes before VarDecl2 in this file, the rewrite engine (see
Section~\ref{sec:rewrite}) gives it higher priority, and this avoids
introducing redundant tautologies (such as $E \in \arithmos$, which is
guaranteed to be true by the type system) into the predicate.  This is
an example of how we can influence the behaviour of the rewrite engine
by placing more specific rules before more general ones.  Of course,
in the interactive prover, the user could choose to apply either rule
when $E$ is a carrier set.


\begin{zedrule}{PredVarDecl1}
  E :: \power E
\derives
  pred [N:E;D | true] \iff \\~~ pred [D|true]
\end{zedrule}

\begin{zedrule}{PredVarDecl2}
  pred [N:E;D | true] \iff \\~~ N \in E \land pred [D|true]
\end{zedrule}

\begin{zedrule}{PredConstDecl}
  pred [N==E;D | true] \iff \\~~ N = E \land pred [D|true]
\end{zedrule}

\begin{zedrule}{PredIncludeDecl}
   pred [E; D | true] \iff \\~~ E \land pred [D|true]
\end{zedrule}

\begin{zedrule}{PredEmptyDecl}
  pred [|true] \iff true
\end{zedrule}
