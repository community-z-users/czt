\documentclass{article}
\usepackage{czt}

\parindent 0pt
\parskip 1ex plus 3pt

\title{Simple Schema Unfolding Rules in CZT}
\author{Mark Utting}
\begin{document}
\maketitle

This document defines simple rules for unfolding schema inclusions.

\begin{zsection}
  \SECTION simpleUnfold \parents proviso\_toolkit
\end{zsection}

Declare the jokers used in these rules.

We define an infix operator for unfolding schemas.

\newcommand{\unfoldsTo}{\mathrel{\leadsto}}

\begin{verbatim}
%%Zinword \unfoldsTo unfoldsTo
\end{verbatim}

\begin{zed}
  \relation ( \_ \unfoldsTo \_ )
\end{zed}

\begin{gendef}[SCHEMA]
  \_ \unfoldsTo \_ : SCHEMA \rel SCHEMA
\where
  \forall s1,s2:SCHEMA @ s1 \unfoldsTo s2 \iff s1=s2
\end{gendef}


\section*{Stage 1: Unfolding Simple Schema Inclusions}

The first step in the process is to move the predicates into
the normalised result.
\begin{zedrule}{Start}
   [D1 | true] \unfoldsTo [D2 | P2] \\
   P2 \land P \iff P3
\derives
   [D1 | P] = [D2 | P3]
\end{zedrule}

Now we handle each possible kind of declaration, from left to right.

This rule matches when the first declaration is a VarDecl.
Note that if the first declaration declares multiple variables,
like $a,b,c:\nat$, then the matcher should automatically split this
so that the $v:S$ pattern matches $a:\nat$ and the $b,c:\nat$
becomes the first declaration within $D1$.

\begin{zedrule}{VarDeclBaseType}
   E \hasType \power E \\
   [D1 | true] \unfoldsTo [D2 | P2]
\derives
   [v:E; D1 | true] \unfoldsTo [v:E; D2 | P2]
\end{zedrule}

\begin{zedrule}{VarDecl}
   E \hasType \power E2 \\
   [D1 | true] \unfoldsTo [D2 | P2] \\
   v \in E \land P2 \iff P3
\derives
   [v:E; D1 | true] \unfoldsTo [v:E2; D2 | P3]
\end{zedrule}

These rules handle simple schema inclusions, where the name $v$ is
directly defined in the specification.

\begin{zedrule}{SchemaRef}
  v \hasDefn [D1 | P1] \\
  [D1 | true] \unfoldsTo [D2 | P2] \\
  P1 \land P2 \iff P3
\derives
  v \unfoldsTo [D2 | P3]
\end{zedrule}


\begin{zedrule}{IncludeDecl}
   v \unfoldsTo [D1 | P1] \\
   [D | true] \unfoldsTo [D2 | P2] \\
   ([D1 | true] \land [D2 | true]) \hasType \power [D3 | true]\\
   P1 \land P2 \iff P3
\derives
   [v; D | true] \unfoldsTo [D3 | P3]
\end{zedrule}

The base case is an empty declaration list, $[|true]$.
\begin{zedrule}{Empty}
   [~ | true] \unfoldsTo [~ | true]
\end{zedrule}


\section*{Stage 2: Allowing Decorated Schema Names}

This extra rule extends the above rules by handling primed schema
references.

To implement this rule, we must implement an additional kind of
proviso: \verb!CalculateProviso!.  This defines a visitor which
traverses the right-hand-side expression and calculates a few known
operators, then matches the result against the left-hand-side expression.

For this rule, the only operator we must implement is the
\verb!visitDecorExpr! method, which adds the given decoration to all the
declared names within the schema.

\begin{zedrule}{DecorDecl}
   v \unfoldsTo [D1 | P1] \\
   [D1|P1]~' \is [D2|P2]\\
   [D | true] \unfoldsTo [D3 | P3] \\
   ([D2 | true] \land [D3 | true]) \hasType \power [D4 | true]\\
   P2 \land P3 \iff P4
\derives
   [v~'; D | true] \unfoldsTo [D4 | P4]
\end{zedrule}

\begin{zedrule}{TrueLeft}
  true \land P \iff P
\end{zedrule}

\begin{zedrule}{TrueRight}
  P \land true \iff P
\end{zedrule}

\begin{zedrule}{Impl}
  P \iff P
\end{zedrule}

\begin{zedrule}{ExistsSchema}
   E1 \unfoldsTo [D1 | P1] \\
   E2 \unfoldsTo [D2 | P2] \\
   [D2|true] \schemaminus [D1|true] \is [D3|true]
\derives
   (\exists E1 @ E2) \unfoldsTo [D3 | (\exists [D1|P1] @ P2)]
\end{zedrule}


\section*{Examples}
Here is an example of the results we should get.
(These could be used as tests for the rules implementation).

If we define a series of simple schemas like this:

\begin{zed}
  S1 == [x:\nat | x < 10] \\
  S2 == [S1; y:\nat | y < x] \\
  S3 == [S1; z:\nat | z < x]
\end{zed}

Using the $SchNorm1$ rule set, we should be able to
do the following transformations:

\begin{zed}
  \vdash? [x : \nat | true] \unfoldsTo [x:\arithmos | x \in \nat]
\end{zed}
\begin{zed}
  \vdash? [S1|true] \unfoldsTo [x:\arithmos | x<10 \land x\in\nat]
\end{zed}
\begin{zed}
  \vdash? [S1; y:\nat|true] \unfoldsTo
          [x:\arithmos~; y:\arithmos | x<10 \land x\in\nat \land y\in\nat]
\end{zed}
\begin{zed}
  \vdash? [S2|true] \unfoldsTo
          [x:\arithmos~; y:\arithmos | y<x \land (x<10 \land
                                       x\in\nat \land y\in\nat)]
\end{zed}

Using Stage 2, we should be able to
do the following transformations:

\begin{zed}
\vdash? [S1~' | true] \unfoldsTo [x':\arithmos | x'<10 \land x'\in\nat]
\end{zed}

This next test is not working yet.
\begin{verbatim}
\vdash? [S1;S1~' | true] \unfoldsTo
        [x:\arithmos~; x':\nat | x<10 \land x\in\nat \land x'<10]
\end{verbatim}

%\begin{zed}
\vdash? [S2~' | true] \unfoldsTo
        [x':\arithmos~; y':\arithmos | y'<x' \land (x'<10 \land
                                       x'\in\nat \land y'\in\nat)]
\end{zed}

(In fact, this last example is a bit difficult to implement safely,
because it is not clear just from the syntax whether the CalculateProviso
should prime the $x$ or not.  The \verb!visitDecorExpr! could perhaps
use the typechecker to see if $x$ is defined within $S1$ or not, but
that would be too complex.  I propose that \verb!visitDecorExpr! should
be quite restrictive, and throw an exception if the schema that it
is trying to prime contains difficult operators (operators that
decoration does not distribute into, like schema composition,
piping, hiding, projection, and precondition).

\begin{zed}
\vdash? [x:\arithmos|true] \unfoldsTo [x:\arithmos|true]
\end{zed}

\begin{zed}
\vdash? (\exists [x:\arithmos|true] @ [y:\arithmos|true]) \unfoldsTo
        [y:\arithmos | (\exists [x:\arithmos|true] @ true)]
\end{zed}

\begin{zed}
\vdash? (\exists [x:\arithmos|true] @ [x:\arithmos|true]) \unfoldsTo
        [ | (\exists [x:\arithmos|true] @ true)]
\end{zed}

\begin{zed}
\vdash? (\exists [x,z:\arithmos | true] @ [y,z:\arithmos|true]) \unfoldsTo
        [y:\arithmos | (\exists [x:\arithmos~; z:\arithmos|true] @ true)]
\end{zed}


Checks provisos:

\begin{theorem}{TestDecorateOracle}
\vdash?  [a : \arithmos~; b:\nat | a < b]~' \is
         [a': \arithmos ; b' : \nat | a' < b' ]
\end{theorem}

\end{document}
