\documentclass[draft,a4paper,10pt,wd]{isov2}
\usepackage{ltcadiz}
\begin{document}
\normannex{Mathematical toolkit}
\sclause{More notations for relations}

\begin{zsection}
\SECTION relation\_toolkit \parents set\_toolkit
\end{zsection}

\ssclause{First component projection}

\begin{gendef}[X,Y]
first : X \cross Y  \fun X
\where
\forall x : X; y : Y @ first~(x,y) = x
\end{gendef}

For any ordered pair $(x,y)$,
$first~(x,y)$ is the $x$ component of the pair.

\ssclause{Second component projection}

\begin{gendef}[X,Y]
second : X \cross Y \fun Y
\where
\forall x : X; y : Y @ second~(x,y) = y
\end{gendef}

For any ordered pair $(x,y)$,
$second~(x,y)$ is the $y$ component of the pair.

\ssclause{Maplet}

%%Zinchar \mapsto U+21A6
\begin{zed}
\function 10 \leftassoc (\_ \mapsto \_)
\end{zed}

\begin{gendef}[X,Y]
\_ \mapsto \_ : X \cross Y \fun X \cross Y
\where
\forall x : X ; y : Y @ x \mapsto y = ( x , y )
\end{gendef}

The maplet forms an ordered pair from two values;
$x \mapsto y$ is just another notation for $(x, y)$.

\ssclause{Domain}

%%Zpreword \dom dom
\begin{gendef}[X,Y]
\dom : (X \rel Y) \fun \power X
\where
\forall r : X \rel Y @ \dom r = \{~ p : r @ p.1 ~\}
\end{gendef}

The domain of a relation $r$ is the set of first components of the ordered
pairs in $r$.

\ssclause{Range}

%%Zpreword \ran ran
\begin{gendef}[X,Y]
\ran : (X \rel Y) \fun \power Y
\where
\forall r : X \rel Y @ \ran r = \{~ p : r @ p.2 ~\}
\end{gendef}

The range of a relation $r$ is the set of second components of the ordered
pairs in $r$.

\ssclause{Identity relation}

%%Zpreword \id id
\begin{zed}
\generic (\id \_)
\end{zed}

\begin{zed}
\id X == \{~ x : X @ x \mapsto x ~\}
\end{zed}

The identity relation on a set $X$ is the relation that relates every
member of $X$ to itself.

\ssclause{Relational composition}

%%Zinchar \comp U+2A3E
\begin{zed}
\function 40 \leftassoc (\_ \comp \_)
\end{zed}

\begin{gendef}[X,Y,Z]
\_ \comp \_ : (X \rel Y) \cross (Y \rel Z) \fun (X \rel Z)
\where
\forall r : X \rel Y ; s : Y \rel Z @
 r \comp s = \{~ p : r ; q : s | p.2 = q.1 @ p.1 \mapsto q.2 ~\}
\end{gendef}

The relational composition of a relation $r : X \rel Y$ and $s : Y \rel Z$
is a relation of type $X \rel Z$ formed by taking
all the pairs $p$ of $r$ and $q$ of $s$,
where the second component of $p$ is equal to the first component of $q$,
and relating the first component of $p$ with the second component of $q$.

\ssclause{Functional composition}

%%Zinchar \circ U+2218
\begin{zed}
\function 40 \leftassoc (\_ \circ \_)
\end{zed}

\begin{gendef}[X,Y,Z]
\_ \circ \_ : (Y \rel Z) \cross (X \rel Y) \fun (X \rel Z)
\where
\forall r : X \rel Y ; s : Y \rel Z @ s \circ r = r \comp s
\end{gendef}

The functional composition of $s$ and $r$ is the same as
the relational composition of $r$ and $s$.

\ssclause{Domain restriction}

%%Zinchar \dres U+25C1
\begin{zed}
\function 65 \rightassoc (\_ \dres \_)
\end{zed}

\begin{gendef}[X,Y]
\_ \dres \_ : \power X \cross ( X \rel Y ) \fun ( X \rel Y )
\where
\forall a : \power X ; r : X \rel Y @ a \dres r = \{~ p : r | p.1 \in a ~\}
\end{gendef}

The domain restriction of a relation $r : X \rel Y$ by a set $a : \power X$ is
the set of pairs in $r$ whose first components are in $a$.

\ssclause{Range restriction}

%%Zinchar \rres U+25B7
\begin{zed}
\function 60 \leftassoc (\_ \rres \_)
\end{zed}

\begin{gendef}[X,Y]
\_ \rres \_ : ( X \rel Y) \cross \power Y \fun ( X \rel Y )
\where
\forall r : X \rel Y ; b : \power Y @ r \rres b = \{~ p : r | p.2 \in b ~\}
\end{gendef}

The range restriction of a relation $r : X \rel Y$ by a set $b : \power Y$ is
the set of pairs in $r$ whose second components are in $b$.

\ssclause{Domain subtraction}

%%Zinchar \ndres U+2A64
\begin{zed}
\function 65 \rightassoc (\_ \ndres \_)
\end{zed}

\begin{gendef}[X,Y]
\_ \ndres \_ : \power X \cross ( X \rel Y ) \fun ( X \rel Y )
\where
\forall a : \power X ; r : X \rel Y @ a \ndres r = \{~ p : r | p.1 \notin a ~\}
\end{gendef}

The domain subtraction of a relation $r : X \rel Y$ by
a set $a : \power X$
is the set of pairs in $r$ whose first components are not in $a$.

\ssclause{Range subtraction}

%%Zinchar \nrres U+2A65
\begin{zed}
\function 60 \leftassoc (\_ \nrres \_)
\end{zed}

\begin{gendef}[X,Y]
\_ \nrres \_ : ( X \rel Y) \cross \power Y \fun ( X \rel Y )
\where
\forall r : X \rel Y ; b : \power Y @
r \nrres b = \{~ p : r | p.2 \notin b ~\}
\end{gendef}

The range subtraction of a relation $r : X \rel Y$ by a set $b : \power Y$
is the set of pairs in $r$ whose second components are not in $b$.

\ssclause{Relational inversion}

%%Zpostchar \inv U+223C
\begin{zed}
\function (\_ \inv)
\end{zed}

\begin{gendef}[X,Y]
\_ \inv : ( X \rel Y ) \fun ( Y \rel X )
\where
\forall r : X \rel Y @ r \inv = \{~ p : r @ p.2 \mapsto p.1 ~\}
\end{gendef}

The inverse of a relation is the relation obtained by reversing every
ordered pair in the relation.

\ssclause{Relational image}

%%Zinchar \limg U+2987
%%Zpostchar \rimg U+2988
\begin{zed}
\function (\_ \limg \_ \rimg)
\end{zed}

\begin{gendef}[X,Y]
\_ \limg \_ \rimg : ( X \rel Y ) \cross \power X \fun \power Y
\where
\forall r : X \rel Y ; a : \power X @ r \limg a \rimg = \{~ p : r | p.1 \in a @ p.2 ~\}
\end{gendef}

The relational image of a set $a : \power X$ through a relation $r : X \rel Y$
is the set of values of type $Y$ that are related under $r$ to a value in $a$.

\ssclause{Overriding}

%%Zinchar \oplus U+2295
\begin{zed}
\function 50 \leftassoc (\_ \oplus \_)
\end{zed}

\begin{gendef}[X,Y]
\_ \oplus \_ : ( X \rel Y ) \cross ( X \rel Y ) \fun ( X \rel Y )
\where
\forall r , s : X \rel Y @ r \oplus s = ( ( \dom s ) \ndres r ) \cup s
\end{gendef}

If $r$ and $s$ are both relations between $X$ and $Y$,
the overriding of $r$ by $s$
is the whole of $s$ together with those members of $r$
that have no first components that are in the domain of $s$.

\ssclause{Transitive closure}

%%Zpostword \plus ^+
\begin{zed}
\function (\_ \plus)
\end{zed}

\begin{gendef}[X]
\_ \plus : ( X \rel X ) \fun (X \rel X )
\where
\forall r : X \rel X @
r \plus = \bigcap \{~ s : X \rel X | r \subseteq s \land r \comp s \subseteq s ~\}
\end{gendef}

The transitive closure of a relation $r : X \rel X$ is
the smallest set that contains $r$ and
is closed under the action of composing $r$ with its members.

\ssclause{Reflexive transitive closure}

%%Zpostword \star ^*
\begin{zed}
\function (\_ \star)
\end{zed}

\begin{gendef}[X]
\_ \star : ( X \rel X ) \fun (X \rel X )
\where
\forall r : X \rel X @ r \star = r \plus \cup \id X
\end{gendef}

The reflexive transitive closure of a relation $r : X \rel X$ is the
relation formed by extending the transitive closure of $r$ by the identity
relation on $X$.

\end{document}
