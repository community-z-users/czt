\documentclass[draft,a4paper,10pt,wd]{isov2}
\usepackage{ltcadiz}
\begin{document}
\normannex{Mathematical toolkit}
\sclause{Sequences}

\begin{zsection}
\SECTION sequence\_toolkit \parents function\_toolkit, number\_toolkit
\end{zsection}

\ssclause{Number range}

%%Zinword \upto ..
\begin{zed}
\function 20 \leftassoc (\_ \upto \_)
\end{zed}

\begin{axdef}
\_ \upto \_ : \arithmos \cross \arithmos \pfun \power \arithmos
\where
(\num \cross \num) \dres (\_ \upto \_) \in \num \cross \num \fun \power \num
\also
\forall i , j : \num @ i \upto j = \{~ k : \num | i \leq k \leq j ~\}
\end{axdef}

The number range from $i$ to $j$ is the set of all integers
greater than or equal to $i$,
which are also less than or equal to $j$.

\ssclause{Iteration}

\begin{gendef}[X]
iter : \num \fun (X \rel X) \fun (X \rel X)
\where
\forall r : X \rel X @ iter ~ 0 ~ r = \id X
\also
\forall r : X \rel X; n : \nat @ iter~(n+1)~r = r \comp (iter~n~r)
\also
\forall r : X \rel X; n : \nat @ iter~(\negate n)~r = iter~n~(r \inv)
\end{gendef}

$iter$ is the iteration function for a relation.
The iteration of a relation $r : X \rel X$ for zero times is the
identity relation on $X$.
The iteration of a relation $r : X \rel X$ for $n+1$ times is the
composition of the relation with its iteration $n$ times.
The iteration of a relation $r : X \rel X$ for $\negate n$ times is the
iteration for $n$ times of the inverse of the relation.

\begin{zed}
\function (\_~^{~\_~})
\end{zed}

\begin{gendef}[X]
\_~^{~\_~} : (X \rel X) \cross \num \fun (X \rel X)
\where
\forall r : X \rel X; n : \nat @ r~^{~n~} = iter~n~r
\end{gendef}

$iter~n~r$ may be written as $r ^{ n }$.

\ssclause{Number of members of a set}

%%Zprechar \# U+0023
\begin{zed}
\function (\# \_)
\end{zed}

\begin{gendef}[X]
\# \_ : \finset X \fun \nat
\where
\forall a : \finset X @ \# a = (\mu n : \nat | ( \exists f : 1 \upto n \inj a @ \ran f = a ) )
\end{gendef}

The number of members of a finite set is the upper limit of the number
range starting at $1$ that can be put into bijection with the set.

\ssclause{Minimum}

\begin{zed}
\function (min~\_)
\end{zed}

\begin{axdef}
min~\_ : \power \arithmos \pfun \arithmos
\where
\power \num \dres (min~\_) = \{~ a : \power \num ; m : \num | m \in a \land ( \forall n : a @ m \leq n ) @ a \mapsto m ~\}
\end{axdef}

If a set of integers has a member that is less than or equal
to all members of that set,
that member is its minimum.

\ssclause{Maximum}

\begin{zed}
\function (max~\_)
\end{zed}

\begin{axdef}
max~\_ : \power \arithmos \pfun \arithmos
\where
\power \num \dres (max~\_) = \{~ a : \power \num ; m : \num | m \in a \land ( \forall n : a @ n \leq m ) @ a \mapsto m ~\}
\end{axdef}

If a set of integers has a member that is greater than or equal
to all members of that set,
that member is its maximum.

\ssclause{Finite sequences}

%%Zpreword \seq seq
\begin{zed}
\generic (\seq \_)
\end{zed}

\begin{zed}
\seq X == \{~ f : \nat \ffun X | \dom f = 1 \upto \# f ~\}
\end{zed}

A finite sequence is a finite indexed set of values of the same type,
whose domain is a contiguous set of positive integers starting at $1$.

$\seq X$ is the set of all finite sequences of values of $X$,
that is, of finite functions from the set $1 \upto n$,
for some $n$, to elements of $X$.

\ssclause{Non-empty finite sequences}

\begin{zed}
\seq_1 X == \seq X \setminus \{ \emptyset \}
\end{zed}

$\seq_1 X$ is the set of all non-empty finite sequences of values of $X$.

\ssclause{Injective sequences}

%%Zpreword \iseq iseq
\begin{zed}
\generic (\iseq \_)
\end{zed}

\begin{zed}
\iseq X == \seq X \cap (\nat \pinj X)
\end{zed}

$\iseq X$ is the set of all injective finite sequences of values of $X$,
that is, of finite sequences over $X$ that are also injections.

\ssclause{Sequence brackets}

%%Zprechar \langle U+27E8
%%Zpostchar \rangle U+27E9
\begin{zed}
\function (\langle \listarg \rangle)
\end{zed}

\begin{zed}
\langle \listarg \rangle [ X ]  == \lambda s : \seq X @ s
\end{zed}

The brackets $\langle$ and $\rangle$ can be used for enumerated sequences.

\ssclause{Concatenation}

%%Zinchar \cat U+2040
\begin{zed}
\function 30 \leftassoc (\_ \cat \_)
\end{zed}


\begin{gendef}[X]
\_ \cat \_ : \seq X \cross \seq X \fun \seq X
\where
\forall s , t : \seq X @
s \cat t = s \cup \{~ n : \dom t @ n + \# s \mapsto t~n ~\}
\end{gendef}

Concatenation is a function of a pair of finite sequences of
values of the same type whose result is a sequence that
begins with all elements of the first sequence and
continues with all elements of the second sequence.

\ssclause{Reverse}

\begin{gendef}[X]
rev : \seq X \fun \seq X
\where
\forall s : \seq X @ rev~s = \lambda n : \dom s @ s ( \# s - n + 1 )
\end{gendef}

The reverse of a sequence is the sequence obtained by taking its
elements in the opposite order.

\ssclause{Head of a sequence}

\begin{gendef}[X]
head : \seq_1 X \fun X
\where
\forall s : \seq_1 X @ head~s = s~1
\end{gendef}

If $s$ is a non-empty sequence of values,
then $head~s$ is the value that is first in the sequence.

\ssclause{Last of a sequence}

\begin{gendef}[X]
last : \seq_1 X \fun X
\where
\forall s : \seq_1 X @ last~s = s ( \# s )
\end{gendef}

If $s$ is a non-empty sequence of values,
then $last~s$ is the value that is last in the sequence.

\ssclause{Tail of a sequence}

\begin{gendef}[X]
tail : \seq_1 X \fun \seq X
\where
\forall s : \seq_1 X @ tail~s = \lambda n : 1 \upto (\# s - 1) @ s ( n + 1 )
\end{gendef}

If $s$ is a non-empty sequence of values,
then $tail~s$ is the sequence of values that is obtained from $s$
by discarding the first element and renumbering the remainder.

\ssclause{Front of a sequence}

\begin{gendef}[X]
front : \seq_1 X \fun \seq X
\where
\forall s : \seq_1 X @ front~s = \{ \# s \} \ndres s
\end{gendef}

If $s$ is a non-empty sequence of values,
then $front~s$ is the sequence of values that is obtained from $s$
by discarding the last element.

\ssclause{Squashing}

\begin{gendef}[X]
squash : ( \num \ffun X ) \fun \seq X
\where
\forall f : \num \ffun X @ squash~f =
	\{~ p : f @ \# \{~ i : \dom f | i \leq p.1 ~\} \mapsto p.2 ~\}
\end{gendef}

$squash$ takes a finite function $f : \num \ffun X$
and renumbers its domain to produce a finite sequence.

\ssclause{Extraction}

%%Zinchar \extract U+21BF
\begin{zed}
\function 45 \rightassoc (\_ \extract \_)
\end{zed}

\begin{gendef}[X]
\_ \extract \_ : \power \num \cross \seq X \fun \seq X
\where
\forall a : \power \num ; s : \seq X @
	a \extract s = squash ( a \dres s )
\end{gendef}

The extraction of a set $a$ of indices from a sequence
is the sequence obtained from the original by discarding any indices
that are not in the set $a$,
then renumbering the remainder.

\ssclause{Filtering}

%%Zinchar \filter U+21BE
\begin{zed}
\function 40 \leftassoc (\_ \filter \_)
\end{zed}

\begin{gendef}[X]
\_ \filter \_ : \seq X \cross \power X \fun \seq X
\where
\forall s : \seq X ; a : \power X @
	s \filter a = squash ( s \rres a )
\end{gendef}

The filter of a sequence by a set $a$
is the sequence obtained from the original by discarding any members
that are not in the set $a$,
then renumbering the remainder.

\ssclause{Prefix relation}

%%Zinword \prefix prefix
\begin{zed}
\relation (\_ \prefix \_)
\end{zed}

\begin{gendef}[X]
\_ \prefix \_ : \seq X \rel \seq X
\where
\forall s,t: \seq X @
s \prefix t \iff s \subseteq t
\end{gendef}

A sequence $s$ is a prefix of another sequence $t$
if it forms the front portion of $t$.

\ssclause{Suffix relation}

%%Zinword \suffix suffix
\begin{zed}
\relation (\_ \suffix \_)
\end{zed}

\begin{gendef}[X]
\_ \suffix \_ : \seq X \rel \seq X
\where
\forall s,t: \seq X @
s \suffix t \iff ( \exists u: \seq X @ u \cat s = t)
\end{gendef}

A sequence $s$ is a suffix of another sequence $t$
if it forms the end portion of $t$.

\ssclause{Infix relation}

%%Zinword \infix infix
\begin{zed}
\relation (\_ \infix \_)
\end{zed}

\begin{gendef}[X]
\_ \infix \_ : \seq X \rel \seq X
\where
\forall s,t: \seq X @
s \infix t \iff ( \exists u,v: \seq X @ u \cat s \cat v = t)
\end{gendef}

A sequence $s$ is an infix of another sequence $t$
if it forms a mid portion of $t$.

\ssclause{Distributed concatenation}

%%Zinword \dcat {\cat}/
\begin{gendef}[X]
\dcat : \seq \seq X \fun \seq X
\where
\dcat \langle \rangle = \langle \rangle
\also
\forall s : \seq X @ \dcat \langle s \rangle = s
\also
\forall q , r : \seq \seq X @ \dcat ( q \cat r ) = ( \dcat q ) \cat ( \dcat r )
\end{gendef}

The distributed concatenation of a sequence $t$
of sequences of values of type $X$
is the sequence of values of type $X$
that is obtained by concatenating the members of $t$ in order.

\end{document}
