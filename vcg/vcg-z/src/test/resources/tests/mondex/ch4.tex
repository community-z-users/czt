\ai4fmignore{
\begin{zsection}
  \SECTION ch4 \parents ch3
\end{zsection}
CHANGES:
* keyword renaming: status -> "statuS" 
* few hard spaces
* removed znote
* changed syntax to zed for MESSAGE
* changed \bot to mondexError message
* LogIfNecessary: pulled the IF-THEN-ELSE outwards as an equivalent formulation of PRED as EXPR
}

\chapter{\Betw\ model, single purse, operations}\label{ch4}

\section{Status}

\begin{LFType}[Between world status type]
\begin{zed}
   STATUS ::= eaFrom | eaTo | epr | epv | epa
\end{zed}~\end{LFType}

\section{Message details}\label{ch4.msgdetails}

\subsection{Start message counter party details}

As $CounterPartyDetails$ appears in ($\mu$-) equivalences
and expressions related to $PayDetails$, we also change
to be a finite schema.
%
\begin{LSDef}[Counter party details]
\begin{schema}{CounterPartyDetails}
   name: NAMES \\
   value: NAT \\
   nextSeqNo: NAT
\end{schema}~\end{LSDef}

\subsection{Payment log message details}

During the various definitions and theorems where $PayDetails$
appear, it often happens that some sets of its bindings restricted
in particular ways are finite. Unfortunately, that is not true
because the space of names and possible values are not bounded.
To amend that and make the claimed finiteness assumptions appearing
in the PRG $true$, we need change the values to be bounded as $NAT$.
%
\begin{LSDef}[Payment details]
\begin{schema}{PayDetails}
    TransferDetails \\
    fromSeqNo, toSeqNo: NAT
\where
    from \neq toN
\end{schema}~\end{LSDef}
%
There are two (wrong) assumptions made in the PRG:~one implicit,
and another explicit. The implicit one is that the set
%
\begin{gzed}
   \{~ pd: PayDetails | pd.from = name ~\} \in \finset~PayDetails
\end{gzed}
%
of $PayDetails$ bindings is finite. This true if, and only if, the available names
are finite and the values are bounded. Otherwise, we could have
payments from the same person $to$ infinitely many other persons with
infinitely many possible values. The explicit one is that the set
%
\begin{gzed}
   purse \in NAME \ffun ConPurse \implies \\ \t1
   \{~ pd: PayDetails | pd.from \in \dom~purse ~\} \in \finset~PayDetails
\end{gzed}
%
is finite. It is explicit because the PRG clearly mentions (p.40) for
the definition of $toInEpv$ being finite ``because $conAuthPurse$ is finite'',
which is not $true$.

\section{Clear exception log validation}

\begin{LGSet}[Exception log clearance code for $PayDetails$]
\begin{zed}
   [CLEAR]
\end{zed}~\end{LGSet}

To prove the consistency of $image$ below, we require that at
leat one payment details could be created, which is ensured by
the next lemma.
%
\begin{LNewLemma}[It is always possible to find a payment details]
\begin{theorem}{lExistsPayDetails}
    \exists pd: PayDetails @ true
\end{theorem}~\end{LNewLemma}

%%\begin{}[$PayDetails$ $image$ definition consistency]
%%\begin{}{tImageDefinitionConsistency}
%%    \forall c: CLEAR @ \exists image: \power_1~PayDetails \inj CLEAR @ true
%%\end{}~\end{}
%
%try lemma tImageDefinitionConsistency;
%use lExistsPayDetails;
%prenex;
%instantiate image == \{(\{ pd \}, c)\};
%invoke (\_ \inj \_);
%invoke (\_ \fun \_);
%prove by rewrite;
%apply extensionality;
%prove by rewrite;
TODO: Add discharging axiomatic definition to make
$image$ an injection on some choice of $c \in CLEAR$.
%
\begin{LADef}[$PayDetails$ exception log clearance code]
\begin{axdef}
   image: \power_1 PayDetails \inj CLEAR
%\where
%   true
\end{axdef}~\end{LADef}

\section{Messages}

\begin{LFType}[Between world message type]
%%Zchar \bot U+22A5
\begin{zed}
   MESSAGE ~~::= ~~ startFrom \ldata CounterPartyDetails \rdata \\
           \t1 | startTo \ldata CounterPartyDetails \rdata \\
           \t1 | readExceptionLog \\
           \t1 | req \ldata PayDetails \rdata \\
           \t1 | val \ldata PayDetails \rdata \\
           \t1 | ack \ldata PayDetails \rdata \\
           \t1 | exceptionLogResult \ldata NAMES \cross PayDetails \rdata \\
           \t1 | exceptionLogClear \ldata NAMES \cross CLEAR \rdata \\
           \t1 | mondexError
\end{zed}~\end{LFType}

\section{A concrete purse}

Although there are no finiteness restrictions
on expressions involving $ConPurse$, it appears in expressions
updating $PayDetails$, which includes finiteness restrictions.
Thus, differently from $AbPurse$, we need to impose restrictions
on the elements of $ConPurse$ affected by $PayDetails$.
%
\begin{LSDef}[Concrete purse definition]
\begin{schema}{ConPurse}
   balance: \nat \\
   exLog: \power~PayDetails \\
   name: NAMES \\
   nextSeqNo: NAT \\
   pdAuth: PayDetails \\
   % ZEves eclipse doesn't accept proof keywords as user declared names
   statuS: STATUS
\where
   \forall pd: exLog @ name \in \{~ pd.from, pd.toN ~\}
   \also
   statuS = epr \implies name = pdAuth.from \\
                \t4 \land pdAuth.value \leq balance \\
                \t4 \land pdAuth.fromSeqNo < nextSeqNo
   \also
   statuS = epv \implies pdAuth.toSeqNo < nextSeqNo
   \also
   statuS = epa \implies pdAuth.fromSeqNo < nextSeqNo
\end{schema}~\end{LSDef}

\section{Invisible operations}

\subsection{Increase purse}

\begin{LSDef}[Concrete purse signature for increase operations]
\begin{zed}
   ConPurseIncrease ~~\defs~~ ConPurse \hide (nextSeqNo)
\end{zed}~\end{LSDef}

\begin{LSDef}[Concrete purse successful increase operation]
\begin{schema}{IncreasePurseOkay}
   \Delta ConPurse \\
   m?, m!: MESSAGE
\where
  \Xi ConPurseIncrease
  \also
  nextSeqNo' \geq nextSeqNo
  \also
  m! = mondexError
\end{schema}~\end{LSDef}

\subsection{Exception logging}

This is the same as subsection $2.3.1$ (p.$13$) from PRG-$126$.
\begin{LSDef}[Exception log security property for concrete purses]
\begin{schema}{LogIfNecessary}
   \Delta ConPurse
\where
   % ZEves accepts IF-THEN-ELSE predicate, CZT only Expression
   %exLog' = exLog \cup (\IF statuS \in \{~epv, epa~\} \THEN \{~pdAuth~\} \ELSE \emptyset)
   %
   % Equiv. Encoding 1: use implies
   %statuS \in \{~epv, epa~\} \implies exLog' = exLog \cup \{~ pdAuth ~\}
   %\\
   %statuS \notin \{~epv, epa~\} \implies exLog' = exLog \cup \emptyset
   %\\
   % Equiv. Encoding 2: push the equality into the IF-THEN-ELSE
   \IF (statuS \in \{~epv, epa~\}) \THEN 
   		[|exLog' = exLog \cup \{~pdAuth~\}]
   \ELSE
   		[|exLog' = exLog \cup \emptyset]
\end{schema}~\end{LSDef}

\subsection{Abort purse}

\begin{LSDef}[Concrete purse signature for aborted operations]
\begin{zed}
   ConPurseAbort ~~\defs~~ ConPurse \hide (nextSeqNo, exLog, pdAuth, statuS)
\end{zed}~\end{LSDef}

\begin{LSDef}[Concrete purse successful abort operation]
\begin{schema}{AbortPurseOkay}
   \Delta ConPurse \\
   m?, m!: MESSAGE
\where
  \Xi ConPurseAbort \\
  LogIfNecessary
  \also
  statuS' = eaFrom \\
  nextSeqNo' \geq nextSeqNo \\
\end{schema}~\end{LSDef}

\section{Value transfer operations}\label{ch4.valTransferOp}

\begin{LSDef}[Concrete purse signature for start operations]
\begin{zed}
   ConPurseStart ~~\defs~~ ConPurse \hide (nextSeqNo, pdAuth, statuS)
\end{zed}~\end{LSDef}

\begin{LSDef}[Concrete purse signature for request operations]
\begin{zed}
   ConPurseReq ~~\defs~~ ConPurse \hide (balance, statuS)
\end{zed}~\end{LSDef}

\begin{LSDef}[Concrete purse signature for value transfer operations]
\begin{zed}
   ConPurseVal ~~\defs~~ ConPurse \hide (balance, statuS)
\end{zed}~\end{LSDef}

\begin{LSDef}[Concrete purse signature for acknowledgement operations]
\begin{zed}
   ConPurseAck ~~\defs~~ ConPurse \hide (statuS, pdAuth)
\end{zed}~\end{LSDef}

\subsection*{Automation for $StartFromPurse$}

The following $3$ rules are needed for discharging goals involving the types of elements
from counter party details schema. They appear because in the definite description of
\begin{gzed} StartFromPurseEafromOkay \end{gzed} we make direct use of the fields of $cpd$.
%
\begin{LFRT}[$CounterPartyDetails$ $name$ type]
\begin{theorem}{frule fCounterPartyDetailsNameType}
   x \in CounterPartyDetails \implies x.name \in NAMES
\end{theorem}~\end{LFRT}

\begin{LFRT}[$CounterPartyDetails$ $value$ upper bound]
\begin{theorem}{frule fCounterPartyDetailsValueUpperBound}
   x \in CounterPartyDetails \implies x.value \leq MAX\_NAT
\end{theorem}~\end{LFRT}

\begin{LFRT}[$CounterPartyDetails$ $value$ lower bound]
\begin{theorem}{frule fCounterPartyDetailsValueLowerBound}
   x \in CounterPartyDetails \implies 0 \leq x.value
\end{theorem}~\end{LFRT}

\begin{LFRT}[$CounterPartyDetails$ $nextSeqNo$ upper bound]
\begin{theorem}{frule fCounterPartyDetailsNextSeqNoUpperBound}
   x \in CounterPartyDetails \implies x.nextSeqNo \leq MAX\_NAT
\end{theorem}~\end{LFRT}

\begin{LFRT}[$CounterPartyDetails$ $nextSeqNo$ lower bound]
\begin{theorem}{frule fCounterPartyDetailsNextSeqNoLowerBound}
   x \in CounterPartyDetails \implies 0 \leq x.nextSeqNo
\end{theorem}~\end{LFRT}
%
The following $5$ rules are needed for discharging goals involving \begin{gzed} startFrom\inv m? \end{gzed}
The relationship between them is explained in the corresponding proof scripts below.
%
\begin{LFRT}[$CounterPartyDetails$ binding type]
\begin{theorem}{frule fCounterPartyDetailsMember}
    x \in  CounterPartyDetails \implies  x \in  \lblot name: NAME; nextSeqNo: \nat ; value: \nat \rblot
\end{theorem}~\end{LFRT}

\begin{LFRT}[$CounterPartyDetails$ binding membership]
\begin{theorem}{frule fCounterPartyDetailsInSetMember}
   x \in  \lblot name: NAMES; nextSeqNo: NAT ; value: NAT \rblot  \implies  x \in  CounterPartyDetails
\end{theorem}~\end{LFRT}

\begin{LGRT}[$startFrom$ $MESSAGE$ relation type]
\begin{theorem}{grule gMESSAGEStartFromRelType}
   startFrom \in  \lblot name: NAME; nextSeqNo: \nat ; value: \nat \rblot  \rel  MESSAGE
\end{theorem}~\end{LGRT}

\begin{LGRT}[$startFrom$ $MESSAGE$ partial function type]
\begin{theorem}{grule gMESSAGEStartFromPFunType}
   startFrom \in  \lblot name: NAME; nextSeqNo: \nat ; value: \nat \rblot  \pfun  MESSAGE
\end{theorem}~\end{LGRT}

\begin{LGRT}[$startFrom$ $MESSAGE$ partial injection type]
\begin{theorem}{grule gMESSAGEStartFromPInjType}
   startFrom \in \lblot name: NAME; nextSeqNo: \nat ;  value: \nat \rblot \pinj MESSAGE
\end{theorem}~\end{LGRT}
%
Interestingly, during the experiments with an old version of \zeves\ in an old machine,
we had the last theorem as
%
\begin{gzed}
  startFrom \in \lblot name: NAME; value: \nat; nextSeqNo: \nat \rblot \pinj MESSAGE
\end{gzed}
%
which should be irrelevant, but the old prover complained that ``\textit{You may only make binding sets of an already-declared schema type}''
(error code: \texttt{NoBindingSetAlphabet}). The solution is trivial: the earlier version requires that
binding variable names appears in alphabetical order, rather than in declaration order.
Similar changes in the order of binding variable names are needed in:
%
\begin{table}[ht]
\[
\begin{array}{|c|l|l|}
    \hline
    \multicolumn{1}{|c}{\mbox{\textbf{Chapter}}} & \multicolumn{1}{|c}{\mbox{\textbf{Binding}}} & \multicolumn{1}{|c|}{\mbox{\textbf{Theorem}}} \\
    \hline
    \ref{ch4}               & CounterPartyDetails     & gMESSAGEStartToPInjType \\
    \hline
    \ref{ch5}               & PayDetails              & lPayDetailsIsFinite \\
    \hline
    \ref{ch6}               & PayDetails              & fPayDetailsMember \\
    \hline
    \ref{ch6}               & PayDetails              & gMESSAGEReqRelType \\
    \hline
    \ref{ch6}               & PayDetails              & gMESSAGEReqPFunType \\
    \hline
    \ref{ch6}               & PayDetails              & gMESSAGEReqPInjType \\
    \hline
\end{array}
\]
\caption{Bindings changed to alphabetical rather than declaration order}
\end{table}

\subsection{StartFromPurse}\label{ch4.valTransferOp.StartFromPurse}

Maybe a schema inclusion with the interface is better for \zeves\ automation.
Let us see.
\begin{LSDef}[$startFrom$ operation invariant]
\begin{schema}{ValidStartFrom}
   ConPurse \\
   m?: MESSAGE
   \also
   cpd: CounterPartyDetails
\where
   m? \in \ran~startFrom
   \also
   cpd = (startFrom~\inv)~m?
   \also
   cpd.name \neq name \\
   cpd.value \leq balance
\end{schema}~\end{LSDef}

\begin{LSDef}[Successful $startFrom$ operation over concrete purses]
\begin{schema}{StartFromPurseEafromOkay}
   \Delta ConPurse \\
   m?, m!: MESSAGE
   \also
   cpd: CounterPartyDetails
\where
   ValidStartFrom \\
   statuS = eaFrom
   \also
   \Xi ConPurseStart
   \also
   %\znote{Due to the limit on natural number values}
   %\znote{we must make sure we are not at the limit}
   nextSeqNo < MAX\_NAT \\
   nextSeqNo' > nextSeqNo
   \also
   pdAuth' = (\mu PayDetails | \\
      \t1 from = name \\
      \t1 \land toN = cpd.name \\
      \t1 \land value = cpd.value \\
      \t1 \land fromSeqNo = nextSeqNo \\
      \t1 \land toSeqNo = cpd.nextSeqNo)
   \also
   statuS' = epr
   \also
   m! = mondexError
\end{schema}~\end{LSDef}

\begin{LSDef}[Successful $startFrom$ operation with recovery from abortion]
\begin{zed}
   StartFromPurseOkay ~~\defs~~ AbortPurseOkay \semi StartFromPurseEafromOkay \hide (cpd)
\end{zed}~\end{LSDef}

\subsection{StartToPurse}

\subsection*{Automation for $ValidStartTo$}

These theorems are similar to those for $startFrom$ in~\ref{ch4.valTransferOp}
%
\begin{LGRT}[$startTo$ $MESSAGE$ relation type]
\begin{theorem}{grule gMESSAGEStartToRelType}
   startTo \in  \lblot name: NAME; nextSeqNo: \nat ; value: \nat \rblot  \rel  MESSAGE
\end{theorem}~\end{LGRT}

\begin{LGRT}[$startTo$ $MESSAGE$ partial function type]
\begin{theorem}{grule gMESSAGEStartToPFunType}
   startTo \in  \lblot name: NAME; nextSeqNo: \nat ; value: \nat \rblot  \pfun  MESSAGE
\end{theorem}~\end{LGRT}

\begin{LGRT}[$startTo$ $MESSAGE$ partial injection type]
\begin{theorem}{grule gMESSAGEStartToPInjType}
   startTo \in \lblot name: NAME; nextSeqNo: \nat; value: \nat \rblot \pinj MESSAGE
\end{theorem}~\end{LGRT}

\subsection*{$StartToPurse$ definition}

\begin{LSDef}[$startTo$ operation invariant]
\begin{schema}{ValidStartTo}
   ConPurse \\
   m?: MESSAGE
   \also
   cpd: CounterPartyDetails
\where
   m? \in \ran~startTo
   \also
   cpd = (startTo~\inv)~m?
   \also
   cpd.name \neq name
\end{schema}~\end{LSDef}

\begin{LSDef}[Successful $startTo$ operation over concrete purses]
\begin{schema}{StartToPurseEafromOkay}
   \Delta ConPurse \\
   m?, m!: MESSAGE
   \also
   cpd: CounterPartyDetails
\where
   ValidStartTo \\
   statuS = eaFrom
   \also
   \Xi ConPurseStart
   \also
   %\znote{Due to the limit on natural number values}
   %\znote{we must make sure we are not at the limit}
   nextSeqNo < MAX\_NAT \\
   nextSeqNo' > nextSeqNo
   \also
   pdAuth' = (\mu PayDetails | \\
      \t1 toN = name \\
      \t1 \land from = cpd.name \\
      \t1 \land value = cpd.value \\
      \t1 \land toSeqNo = nextSeqNo \\
      \t1 \land fromSeqNo = cpd.nextSeqNo)
   \also
   statuS' = epv
   \also
   m! = req~pdAuth'
\end{schema}~\end{LSDef}

\begin{LSDef}[Successful $startTo$ operation with recovery from abortion]
\begin{zed}
   StartToPurseOkay ~~\defs~~ AbortPurseOkay \semi StartToPurseEafromOkay \hide (cpd)
\end{zed}~\end{LSDef}

\pagebreak
\subsection{$ReqPurse$}

\begin{LSDef}[Request operation invariant]
\begin{schema}{AuthenticReqMessage}
   ConPurse \\
   m?: MESSAGE
\where
   m? = req~pdAuth
\end{schema}~\end{LSDef}

\begin{LSDef}[Successful request operation over concrete purses]
\begin{schema}{ReqPurseOkay}
   \Delta ConPurse \\
   m?, m!: MESSAGE
\where
   AuthenticReqMessage \\
   statuS = epr
   \also
   \Xi ConPurseReq
   \also
   balance' = balance - pdAuth.value \\
   statuS' = epa
   \also
   m! = val~pdAuth
\end{schema}~\end{LSDef}

\subsection{ValPurse}

\begin{LSDef}[Value operation invariant]
\begin{schema}{AuthenticValMessage}
   ConPurse \\
   m?: MESSAGE
\where
   m? = val~pdAuth
\end{schema}~\end{LSDef}

\begin{LSDef}[Successful value operation over concrete purses]
\begin{schema}{ValPurseOkay}
   \Delta ConPurse \\
   m?, m!: MESSAGE
\where
   AuthenticValMessage \\
   statuS = epv
   \also
   \Xi ConPurseVal
   \also
   balance' = balance + pdAuth.value \\
   statuS' = eaTo
   \also
   m! = ack~pdAuth
\end{schema}~\end{LSDef}

\subsection{AckPurse}

\begin{LSDef}[Acknowledgement operation invariant]
\begin{schema}{AuthenticAckMessage}
   ConPurse \\
   m?: MESSAGE
\where
   m? = ack~pdAuth
\end{schema}~\end{LSDef}

\begin{LSDef}[Successful ack operation over concrete purses]
\begin{schema}{AckPurseOkay}
   \Delta ConPurse \\
   m?, m!: MESSAGE
\where
   AuthenticAckMessage \\
   statuS = epa
   \also
   \Xi ConPurseAck
   \also
   statuS' = eaFrom
   \also
   m! = mondexError
\end{schema}~\end{LSDef}

\section{Exception logging operations}\label{ch4.exlog}

\subsection{ReadExceptionLogPurse}

\begin{LSDef}[Successful read exception log over concrete purses]
\begin{schema}{ReadExceptionLogPurseEafromOkay}
   \Xi ConPurse \\
   m?, m!: MESSAGE
\where
   m? = readExceptionLog \\
   statuS = eaFrom
   \also
   m! \in \{~mondexError~\} \cup \{~ ld: exLog' @ exceptionLogResult~(name, ld) ~\}
\end{schema}~\end{LSDef}

\begin{LSDef}[Successful read exception log with recovery from abortion]
\begin{zed}
   ReadExceptionLogPurseOkay ~~\defs~~ AbortPurseOkay \semi ReadExceptionLogPurseEafromOkay
\end{zed}~\end{LSDef}

\subsection{ClearExceptionLogPurse}

\begin{LSDef}[Concrete purse signature for clearing the exception log]
\begin{zed}
   ConPurseClear ~~\defs~~ ConPurse \hide (exLog)
\end{zed}~\end{LSDef}

\begin{LSDef}[Successful clear exception log over concrete purses]
\begin{schema}{ClearExceptionLogPurseEafromOkay}
   \Delta ConPurse \\
   m?, m!: MESSAGE
\where
   exLog \neq \emptyset \\
   m? = exceptionLogClear~(name, image~exLog) \\
   statuS = eaFrom
   \also
   \Xi ConPurseClear
   \also
   exLog' = \emptyset
   \also
   m! = mondexError
\end{schema}~\end{LSDef}

\begin{LSDef}[Successful clear exception log with recovery from abortion]
\begin{zed}
   ClearExceptionLogPurseOkay ~~\defs~~ AbortPurseOkay \semi ClearExceptionLogPurseEafromOkay
\end{zed}~\end{LSDef}

\newpage

\section{Summary}\label{ch4.summary}

\ldefsummary %
\lthmsummary %
\lthmaddeddefsummary %
\lthmaddedthmsummary %
\lzevessummary %
%\lpscriptsummary
