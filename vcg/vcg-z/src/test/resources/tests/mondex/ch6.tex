\ai4fmignore{
\begin{zsection}
  \SECTION ch6 \parents ch5
\end{zsection}
CHANGES:
* to -> "toN"
* added missing hard space
* removed znote
* changed \bot -> mondexError
}

\chapter{\Betw\ model, initialisation and finalisation}\label{ch6}

\section{Summing values}\label{ch6.sumValue}

Originally from Appendix~D, p.$226$.
\begin{LADef}[Summing values from purse payments]
\begin{axdef}
   sumValue: \finset PayDetails \fun \nat
\where
   \Label{rule dSumValueBase}
      sumValue~\{~\} = 0
   \also
   \Label{rule dSumValueInduc}
      \forall pds: \finset PayDetails; PayDetails | \theta PayDetails \notin pds @ \\
          \t1 sumValue~( \{~ \theta PayDetails ~\} \cup pds) = \\
                \t3 value + sumValue~pds
\end{axdef}~\end{LADef}

\section{Initialisation}\label{ch6.init}

\begin{LSDef}[Between world initial state]
\begin{schema}{BetweenInitState}
  BetweenWorld~' 
\where 
  \{ readExceptionLog, mondexError \}
  \\ %
  \cup
  \\ %
  \bigcup \{~ cpd: CounterPartyDetails @ \{ startFrom~cpd, startTo~cpd \} ~\}
  \also %
  \subseteq ether'
\end{schema}~\end{LSDef}

\subsection*{Automation for BetwInitIn}

These rules are similar to those found in~\ref{ch4.valTransferOp} for $CounterPartyDetails$.
%
\begin{LFRT}[Payment details $from$ purse type]
\begin{theorem}{frule fPayDetailsFromType}
   x \in PayDetails \implies x.from \in NAMES
\end{theorem}~\end{LFRT}

\begin{LFRT}[Payment details $to$ purse type]
\begin{theorem}{frule fPayDetailsToType}
   x \in PayDetails \implies x.toN \in NAMES
\end{theorem}~\end{LFRT}

\begin{LFRT}[Payment details $value$ type]
\begin{theorem}{frule fPayDetailsValueType}
   x \in PayDetails \implies x.value \in NAT
\end{theorem}~\end{LFRT}

These rules are similar to those found in~\ref{ch4.valTransferOp}
for $startFrom$ and $startTo$, but for schema $CounterPartyDetails$
rather than $PayDetails$. Nevertheless, as $PayDetails$ has an
invariant, we cannot have an $frule$ like
$fCounterPartyDetailsInSetMember$. Thus, the proofs for the $grules$
is slightly different at the end. Also, we need to include the
relationship between $req$ and $PayDetails$ explicitly.
%
\begin{LFRT}[$PayDetails$ binding type]
\begin{theorem}{frule fPayDetailsMember}
    x \in  PayDetails \implies  x \in  \lblot from: NAME; fromSeqNo: \nat; \\
        \t1 toN: NAME; toSeqNo: \nat; value: \nat \rblot
\end{theorem}~\end{LFRT}

\begin{LGRT}[$req$ $MESSAGE$ relational type]
\begin{theorem}{grule gMESSAGEReqRelType}
   req \in  \lblot from: NAME; fromSeqNo: \nat; \\
        \t1 toN: NAME; toSeqNo: \nat; value: \nat \rblot  \rel MESSAGE
\end{theorem}~\end{LGRT}

\begin{LGRT}[$req$ $MESSAGE$ partial function type]
\begin{theorem}{grule gMESSAGEReqPFunType}
  req \in  \lblot from: NAME; fromSeqNo: \nat; \\
        \t1 toN: NAME; toSeqNo: \nat; value: \nat \rblot  \pfun MESSAGE
\end{theorem}~\end{LGRT}

\begin{LGRT}[$req$ $MESSAGE$ relational non-maximal type]
\begin{theorem}{grule gMESSAGEReqPayDetailsRelType}
   req \in PayDetails \rel MESSAGE
\end{theorem}~\end{LGRT}

\begin{LGRT}[$req$ $MESSAGE$ partial injection type]
\begin{theorem}{grule gMESSAGEReqPInjType}
   req \in \lblot from: NAME; fromSeqNo: \nat; \\
        \t1 toN: NAME; toSeqNo: \nat; value: \nat \rblot \pinj MESSAGE
\end{theorem}~\end{LGRT}

The next rule is necessary to tell \zeves\ about the (non-maximal) type of the result of
applying the given message to the inverse of the $req$ message. Although this result is trivially
obvious, it is quite hard to prove. A simple suggestion around this, would be to give a variable
(with declared type) named $v$ as
%
\begin{gzed}
    v \in PayDetails \land v = req\inv
\end{gzed}
%
\begin{LRRT}[$req$ $MESSAGE$ inverse maximal type]
\begin{theorem}{rule rMESSAGEReqInvMaxType}
   \forall m: MESSAGE | m \in \ran req @ (req~\inv)~m \in \\ \t1
      \lblot from: NAMES; fromSeqNo: NAT; toN: NAMES; toSeqNo: NAT; value: NAT \rblot
\end{theorem}~\end{LRRT}

\subsection*{$BetwInitIn$}

In this schema the use of $(req\inv m?)$ directly is a bad idea for \zeves\ automation,
as it incurs rather complex lemmas. To avoid this, one could simply declare a variable
to hold such value. In this way, \zeves\ would know the type of this expression, the main
problem appearing in the domain check. Nevertheless, for the sake of keeping to the original
as much as possible, we left it unchanged.
%
\begin{LSDef}[Between world initialisation input]
\begin{schema}{BetwInitIn}
  g? : AIN
  \also
  m? : MESSAGE
  \\    name? : NAMES
  \where
  m? \in \ran req \implies
  \\ \t1                g? = transfer~(~ \mu TransferDetails |
  \\ \t2                        from = ((req~\inv)~m?).from
  \\ \t2                        \land toN = ((req~\inv)~m?).toN
  \\ \t2                        \land value = ((req~\inv)~m?).value ~)
  \also
  m? \notin \ran req
  \implies
  g? = aNullIn
\end{schema}~\end{LSDef}

\section{Finalisation}\label{ch6.final}

\begin{LSDef}[Between world finalisation state]
\begin{schema}{BetwFinState}
  BetweenWorld
  \\    GlobalWorld
  \where
  \dom gAuthPurse = \dom conAuthPurse
  \also
  \forall name: \dom conAuthPurse @ \\
  \t1             (gAuthPurse~name).balance = (conAuthPurse~name).balance
  \also
  \t1             \land (gAuthPurse~name).lost =
  \\ \t2                        sumValue(~ (definitelyLost \cup maybeLost)
  \\ \t4                          {} \cap \{~ ld:PayDetails | ld.from = name ~\}~)
\end{schema}~\end{LSDef}

\begin{LSDef}[Between world finalisation output]
\begin{schema}{BetwFinOut}
  g! : AOUT
  \\      m!: MESSAGE
  \where
  g! = aNullOut
\end{schema}~\end{LSDef}

\newpage
\section{Summary}\label{ch6.summary}

\ldefsummary %
\lthmsummary %
\lthmaddeddefsummary %
\lthmaddedthmsummary %
\lzevessummary %
%\lpscriptsummary
