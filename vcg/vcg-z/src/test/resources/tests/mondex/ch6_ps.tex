\ai4fmignore{
\begin{zsection}
  \SECTION ch6\_ps \parents ch6
\end{zsection}
CHANGES:
* to -> "toN"
* added missing hard space
* removed znote
}

\chapter{Chapter~\ref{ch6} proofs}

\section{Summing values}\plink{ch6.sumValue}

\begin{LDCheck}\begin{zproof}[sumValue\$domainCheck]
   prove by rewrite;
\end{zproof}\end{LDCheck}

\section{Initialisation}\plink{ch6.init}

\begin{LDCheck}\begin{zproof}[BetweenInitState\$domainCheck]
   prove by rewrite;
\end{zproof}\end{LDCheck}


\begin{LPScript}\begin{zproof}[fPayDetailsFromType]
   with enabled (PayDetails\$member) prove by reduce ;
\end{zproof}\end{LPScript}

\begin{LPScript}\begin{zproof}[fPayDetailsToType]
   with enabled (PayDetails\$member) prove by reduce ;
\end{zproof}\end{LPScript}

\begin{LPScript}\begin{zproof}[fPayDetailsValueType]
   with enabled (PayDetails\$member) prove by reduce ;
\end{zproof}\end{LPScript}

\begin{LPScript}\begin{zproof}[fPayDetailsMember]
    with enabled (PayDetails\$member, PayDetails\$inSet) prove by reduce;
\end{zproof}\end{LPScript}

\begin{LPScript}\begin{zproof}[gMESSAGEReqRelType]
    use req\$declaration;
    invoke (\_ \fun \_);
    invoke (\_ \pfun \_);
    invoke (\_ \rel \_);
    %\znote{At this point fCounterPartyDetailsMember and}
    %\znote{fCounterPartyDetailsInSetMember are used}
    rewrite;
    %\znote{Trivial rewriting is needed in order}
    %\znote{not to loose information about power}
    %\znote{type of req.}
    trivial rewrite;
    prenex;
    rearrange;
    equality substitute r;
    apply inPower;
    prenex;
    instantiate e\_\_0 == e;
    with enabled (inCross2) prove by rewrite;
\end{zproof}\end{LPScript}

\begin{LPScript}\begin{zproof}[gMESSAGEReqPFunType]
    use req\$declaration;
    invoke (\_\fun \_);
    invoke (\_\pfun \_);
    %\znote{At this point gMESSAGEReqRelType is used}
    prove by rewrite;
    instantiate x\_\_0 == x, y1\_\_0 == y1, y2\_\_0 == y2;
    %\znote{At this point fPayDetailsMember is used}
    prove by rewrite;
    %\znote{Small difference from other similar proofs}
    %\znote{as there is no fPayDetailsInSetMember}
    invoke (\_ \rel \_);
    apply inPower;
    instantiate e == (x, y1);
    with enabled (inCross2) prove by rewrite;
\end{zproof}\end{LPScript}

\begin{LPScript}\begin{zproof}[gMESSAGEReqPayDetailsRelType]
    use req\$declaration;
    invoke (\_ \fun \_);
    invoke (\_ \pfun \_);
    prove by rewrite;
\end{zproof}\end{LPScript}

\begin{LPScript}\begin{zproof}[gMESSAGEReqPInjType]
    invoke (\_\pinj \_);
    rewrite;
    invoke (\_\pfun \_);
    %\znote{At this point gMESSAGEReqRelType is used}
    prove by rewrite;
    %\znote{At this point gMESSAGEReqPFunType is used}
    use pairInFunction[\lblot from: NAME;fromSeqNo: \nat;toN: NAME;toSeqNo:
                        \nat;value: \nat\rblot, MESSAGE]
                          [f := req, x := y1, y := x];
    use pairInFunction[\lblot from: NAME;fromSeqNo: \nat;toN: NAME;toSeqNo:
                        \nat;value: \nat\rblot, MESSAGE]
                          [f := req, x := y2, y := x];
    %\znote{At this point fPayDetailsMember is used}
    prove by rewrite;
    %\znote{Small difference from other similar proofs}
    use gMESSAGEReqPayDetailsRelType;
    invoke (\_ \rel \_);
    apply inPower;
    instantiate e == (y2, req~ y2);
    instantiate e == (y1, req~ y2);
    with enabled (inCross2) prove by rewrite;
\end{zproof}\end{LPScript}

\begin{LPScript}\begin{zproof}[rMESSAGEReqInvMaxType]
    apply inRan;
    prove by rewrite;
    use pairInFunction[\lblot from: NAME;fromSeqNo: \num;toN:
            NAME;toSeqNo: \num;value: \num \rblot, MESSAGE]
            [f := req, x := x, y := m];
    prove by rewrite;
    use memberFirstInDom[\lblot from: NAME;fromSeqNo: \num;toN:
            NAME;toSeqNo: \num;value: \num \rblot, MESSAGE]
            [R := req, x := x, y := req~ x];
    rearrange;
    with predicate (req \in \lblot from: NAME;fromSeqNo:
        \num;toN: NAME;toSeqNo: \num;value: \num \rblot
            \rel MESSAGE) rewrite;
    simplify;
    use applyInverse[\lblot from: NAME;fromSeqNo: \num;toN:
        NAME;toSeqNo: \num;value: \num \rblot, MESSAGE]
        [A := \lblot from: NAME;fromSeqNo: \num;toN:
            NAME;toSeqNo: \num;value: \num \rblot,
         B := MESSAGE, f := req, x := x];
    with enabled (PayDetails\$member) prove by reduce;
\end{zproof}\end{LPScript}

\begin{LDCheck}\begin{zproof}[BetwInitIn\$domainCheck]
    cases;
    prove by rewrite;
    next;
    prove by reduce;
    use rMESSAGEReqInvMaxType[m := m?];
    apply PayDetails\$inSet;
    prove by rewrite;
    next;
    use muInSet[\{ TransferDetails[from\_\_1/from, toN\_\_1/toN, value\_\_1/value]
        | from\_\_1 = ((req~\inv)~m?).from \land toN\_\_1 = ((req~\inv)~m?).toN
          \land value\_\_1 = ((req~\inv)~m?).value \}];
    prove by reduce;
    instantiate a == \theta TransferDetails[from := ((req~\inv)~m?).from,
        toN := ((req~\inv)~m?).toN, value := ((req~\inv)~m?).value];
    use rMESSAGEReqInvMaxType[m := m?];
    apply PayDetails\$inSet;
    prove by reduce;
    equality substitute
      (\mu m: \{ TransferDetails[from\_\_1/from, toN\_\_1/toN, value\_\_1/value]
         | from\_\_1 = ((req~\inv)~ m?).from \land toN\_\_1 = ((req~\inv)~m?).toN
           \land value\_\_1 = ((req~ \inv)~ m?).value \});
    apply TransferDetails\$thetaMember;
    prove by reduce;
    next;
\end{zproof}\end{LDCheck}

\section{Finalisation}\plink{ch6.final}

\begin{LDCheck}\begin{zproof}[BetwFinState\$domainCheck]
   prove by rewrite;
\end{zproof}\end{LDCheck}

\newpage
\section{Summary}\label{ch6.ps.summary}
\lpscriptsummary
