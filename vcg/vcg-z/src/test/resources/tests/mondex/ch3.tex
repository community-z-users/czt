\ai4fmignore{
\begin{zsection}
  \SECTION ch3 \parents standard\_toolkit
\end{zsection}
CHANGES
* removed znote 
* transfer details :  "to" -> "toN"; 
* various DecorExpr: TransferDefails? -> TransferDetails~?
* automation lemmas for MAX_NAT earlier than NAT definition
* few missing hard spaces
}

\chapter{\Abs\ model:~security policy}\label{ch3}

A positive side-effect for being faithful to the
original source is that we do not need to worry
much about understanding the protocol. As it happens,
we learned it by osmosis!

\section{The abstract state}\label{ch3.abstractState}

\begin{LSDef}[Abstract purse]
\begin{zed}
   AbPurse ~~\defs~~ [~ balance, lost: \nat ~]
\end{zed}~\end{LSDef}

\subsection{A theory for purse names}\label{ch3.pursenames}

In the definition of the various structures that follows in Mondex,
a unique purse $NAME$ often appears. Another situation that often appears
is that sets involving names must be finite. Nevertheless, nothing about the
structure of $NAME$ is given. Thus, such assumed finiteness properties
cannot be proved (see $AuxWorld$ in Chapter~\ref{ch5}, $BetwFinState$ in Chapter~\ref{ch6},
and many others throughout the PRG).
%
\begin{LGSet}[Unique purse name]
\begin{zed}
  [NAME]
\end{zed}~\end{LGSet}
%
For instance, one of those finiteness assumptions comes from expressions such as
%
\begin{gzed}
   PayDetails ~~\defs~~ [~ from, to: NAME; \cdots | from \neq to ~]
   \\
   name: NAME \shows \{~ pd: PayDetails | pd.from = name ~\} \in \finset~PayDetails
\end{gzed}
%
Because this set comprehension expression represents a set of bindings, even
if the $from$ purse is fixed to a $name$, we could have an infinite number of
$to$ purses this $from$ purse is associated with. Therefore, we must introduce
further structuring for the unique purse names in order for the hidden finiteness
assumptions used throughout the PRG to hold. Luckily, they are fairly obvious and
abstract. We call it $NAMES$ and it could be defined as
%
\begin{gzed}
   NAMES \in \finset~NAME
\end{gzed}
%
Nevertheless, names are mostly use for characterising payment details between two
distinct (unique) purses. In fact, in Section~\ref{ch10.retrieve-value-mig-quant},
the existentially quantified equivalence of the retrieve relation partition is
provable only if there exists at least two different names.

Taking this into account, let us define at least two different known names.
This is also useful in order to prove the definition of $NAMES$ is consistent.
%
\begin{LNewADef}[Existence of distinct names]
\begin{axdef}
  name1, name2: NAME
\where
  \Label{disabled rule dExistsDistinctNames}  \lnot name1 = name2
\end{axdef}~\end{LNewADef}
%
Next, we ensure that the definition of $NAMES$ we are about to introduce is
consistent.
%
\begin{LConsistency}[$NAMES$ definition existential consistency]
\begin{theorem}{tNamesExistentialConsistency}
    \exists NAMES: \finset_1~NAME @ \exists n1, n2: NAMES @ \lnot  n1 = n2
\end{theorem}~\end{LConsistency}
%
\begin{LConsistency}[$NAMES$ definition universal consistency]
\begin{theorem}{tNamesUniversalConsistency}
    \exists NAMES: \finset_1~NAME @ \forall n1: NAMES @ \exists n2: NAMES @ \lnot  n1 = n2
\end{theorem}~\end{LConsistency}
%
After proving those two theorems, we can then safely axiomatically define $NAMES$.
%
\begin{LNewADef}[Properties of unique purse names]
\begin{axdef}
  NAMES: \finset_1~NAME
\where
  \Label{disabled dNamesElem} \exists n1, n2: NAMES @ \lnot  n1 = n2 \\
  \Label{disabled dNamesDistinctElem} \forall n1: NAMES @ \exists  n2: NAMES @ \lnot  n1 = n2
\end{axdef}~\end{LNewADef}
%
With this structure model of names, we represent the finiteness assumptions
made throughout the PRG, as well as guarantee that payment transaction are
total.

\subsection*{Automation}

\begin{LRRT}[Every structured name belongs to the original $NAME$]
\begin{theorem}{rule rInNames}
    n \in NAMES \implies n \in NAME
\end{theorem}~\end{LRRT}

\begin{LToolkit}[Finite set weakening]
\begin{theorem}{rule rFiniteMember}[X]
    A \in \finset~X \implies A \in \power~X
\end{theorem}~\end{LToolkit}

The next theorem is an obvious conclusion from the declaration of $NAMES$ using $\finset_1$.
%
\begin{LNewLemma}[There exists at least one structured name]
\begin{theorem}{lExistsName}
    \exists name: NAMES @ true
\end{theorem}~\end{LNewLemma}

The next theorem is important to prove the finiteness (and existence)
of $PayDetails$, as at least two different names are required. That is,
we must have at least one distinct originator and receptor for any
authentic payment transaction to take place.
%
\begin{LNewLemma}[There exists two different names]
\begin{theorem}{lExistsDifferentNames}
    \exists n1, n2: NAMES @ \lnot n1 = n2
\end{theorem}~\end{LNewLemma}
%
Thus, structured names make payment transactions total, hence
always allows a payment to take place. Similarly, we sometimes know
one of the names and only assert the existence of the next one.

\begin{LNewLemma}[Given a name, there exists another distinct name]
\begin{theorem}{lExistsDifferentNameFromGivenName}
    \forall n1: NAMES @ \exists n2: NAMES @ \lnot n1 = n2
\end{theorem}~\end{LNewLemma}

\subsection{Finite space of numbers}\label{ch3.NAT}

Similarly, in order to make schemas finite as required,
we must also add a finite space of numbers by defining a
non-empty range over $\nat$, which is bounded by a $MAX\_NAT$
value defined as
%
\begin{LConsistency}[$NAT$ boundary consistency]
\begin{theorem}{tNATBoundaryConsistency}
   \exists MAX\_NAT: \nat @ true
\end{theorem}~\end{LConsistency}

\begin{LNewADef}[Upper bound for $\nat$]
\begin{axdef}
   MAX\_NAT: \nat
\end{axdef}~\end{LNewADef}

\subsection*{Automation}

Next, we add some trivial automation rules about these data types
in order to improve automation levels.
%
\begin{LGRT}[$MAX\_NAT$ maximal type]
\begin{theorem}{grule gMaxNatMaxType}
    MAX\_NAT \in  \num
\end{theorem}~\end{LGRT}
%
\begin{LGRT}[$MAX\_NAT$ type]
\begin{theorem}{grule gMaxNatNatType}
    MAX\_NAT \in  \nat
\end{theorem}~\end{LGRT}
%
\begin{LRRT}[$MAX\_NAT$ is a natural number (LHS)]
\begin{theorem}{rule rMaxNatType}
    MAX\_NAT \geq  0
\end{theorem}~\end{LRRT}
%
\begin{LRRT}[$MAX\_NAT$ is a natural number (RHS)]
\begin{theorem}{rule rMaxNatType2}
    0 \leq  MAX\_NAT
\end{theorem}~\end{LRRT}
%
This last rule is given just for the sake of
pattern matching on different scenarios, hence
making rules application smoother.
After that, we do the same for $NAT$.
%

\subsubsection{Finite natural numbers}

%
\begin{LConsistency}[$NAT$ definition consistency]
\begin{theorem}{tNATDefinitionConsistency}
   \exists NAT: \power~(0 \upto MAX\_NAT) @ true
\end{theorem}~\end{LConsistency}

%
\begin{LNewADef}[Bounded $\nat$]
\begin{zed}
    NAT == 0 \upto MAX\_NAT
\end{zed}~\end{LNewADef}

\begin{LNewThm}[$NAT$ definition is consistent]
\begin{theorem}{tNATConsistency}
    \exists  n: NAT @ true
\end{theorem}~\end{LNewThm}
%

\begin{LRRT}[$NAT$ equivalence]
\begin{theorem}{rule rInNAT}
    x \in NAT \iff x \in \num \land x \geq 0 \land x \leq MAX\_NAT
\end{theorem}~\end{LRRT}

\subsection{Transfer details}\label{ch3.transferdetails}

Now, with the definition of the finite elements, we redefine
$TransferDetails$ to be finite. Note that because we leave
$AbPurse$ components unbound (as $\nat$), no additional proof
obligations when values are added to the $balance$ or $lost$ fields
are generated. We do not change $AbPurse$ to be bounded because
there is no requirement that it should be finite.
%
\begin{LSDef}[Purse transference details]
\begin{schema}{TransferDetails}
	% ZEves eclipse doesn't allow proof keywords in user names. Sorry
   from, toN: NAMES\\
   value: NAT
\end{schema}~\end{LSDef}

\subsection*{Automation}

\begin{LFRT}[$TransferDetails$ $value$ type]
\begin{theorem}{frule fTransferDetailsValueType}
    \forall TransferDetails @ value \in NAT
\end{theorem}~\end{LFRT}

\begin{LFRT}[$TransferDetails$ binding maximal type]
\begin{theorem}{frule fTransferDetailsMemberMaxType}
    x \in  TransferDetails \implies  x \in  (\lblot from: NAME; toN: NAME; value: \num \rblot)
\end{theorem}~\end{LFRT}

\begin{LFRT}[$TransferDetails$ binding type]
\begin{theorem}{frule fTransferDetailsMember}
    x \in  TransferDetails \implies  x \in  (\lblot from: NAME; toN: NAME; value: \nat \rblot)
\end{theorem}~\end{LFRT}

\begin{LFRT}[$TransferDetails$ binding bembership]
\begin{theorem}{frule fTransferDetailsInSetMember}
    x \in  (\lblot from: NAMES; toN: NAMES; value: NAT \rblot) \implies  x \in  TransferDetails
\end{theorem}~\end{LFRT}

\subsection{Abstract world}

\begin{LSDef}[Abstract world of purses]
\begin{schema}{AbWorld}
  abAuthPurse: NAMES \pfun  AbPurse
\where
  %\znote{Partial finiteness functionality factored to the}
  %\znote{predicate part for Z/EVES automation purposes only.}
  abAuthPurse \in  NAMES \ffun  AbPurse
\end{schema}~\end{LSDef}

\section{Security properties}\label{ch3.security}

\subsection{Authentic purses}

This is the same as subsection $2.2.3$ (p.$12$) from PRG-$126$.
\begin{LSDef}[Authentic purses]
\begin{schema}{Authentic}
  AbWorld\\
  name?: NAMES
\where
  name? \in \dom~abAuthPurse
\end{schema}~\end{LSDef}

\subsection*{Automation}

\begin{LFRT}[Authentic $name$ type]
\begin{theorem}{frule fAuthenticNameType}
  \forall  Authentic @ name? \in  \dom~abAuthPurse
\end{theorem}~\end{LFRT}

\subsection{Sufficient funds}

This is the same as subsection $2.2.4$ (p.$12$) from PRG-$126$.
\begin{LSDef}[Abstract purses have sufficient funds]
\begin{schema}{SufficientFundsProperty}
  AbWorld\\
  TransferDetails~?
\where
  Authentic[from?/name?]\\
  value? \leq  (abAuthPurse~from?).balance
\end{schema}~\end{LSDef}

\section{Secure operations}\label{ch3.secure.op}

\subsection{Abstract inputs and outputs}

\begin{LFType}[Abstract input]
\begin{zed}
  AIN ::= aNullIn | transfer \ldata TransferDetails \rdata
\end{zed}~\end{LFType}

\begin{LFType}[Abstract output]
\begin{zed}
  AOUT ::= aNullOut
\end{zed}~\end{LFType}

\begin{LSDef}[Abstract operation signature]
\begin{schema}{AbOp}
  \Delta AbWorld\\
  a?: AIN\\
  a!: AOUT
\where
  a! = aNullOut
\end{schema}~\end{LSDef}

\subsection*{Automation}

\begin{LGRT}[Abstract input transfer constructor $\pfun$ maximal type]
\begin{theorem}{grule gAINTrasferPFunMaxType}
    transfer \in  \lblot from: NAME; toN: NAME; value: \num \rblot  \pfun  AIN
\end{theorem}~\end{LGRT}

\begin{LGRT}[Abstract input transfer constructor relational type]
\begin{theorem}{grule gAINTrasferRelType}
    transfer \in  \lblot from: NAME; toN: NAME; value: \nat \rblot  \rel  AIN
\end{theorem}~\end{LGRT}

\begin{LGRT}[Abstract input transfer constructor partial function type]
\begin{theorem}{grule gAINTrasferPFunType}
    transfer \in  \lblot from: NAME; toN: NAME; value: \nat \rblot  \pfun  AIN
\end{theorem}~\end{LGRT}

\begin{LGRT}[Abstract input transfer constructor partial injection type]
\begin{theorem}{grule gAINTransferType}
    transfer \in  \lblot from: NAME; toN: NAME; value: \nat \rblot  \pinj  AIN
\end{theorem}~\end{LGRT}

\begin{LRRT}[Abstract input transfer constructor result type]
\begin{theorem}{rule rAINTransferResult}
    \forall  TransferDetails @ transfer~(\theta  TransferDetails) \in  AIN
\end{theorem}~\end{LRRT}

\begin{LRRT}[Abstract input transfer constructor is a total function]
\begin{theorem}{rule rAINTransferIsTotal}
    \forall  TransferDetails @ \theta  TransferDetails \in  \dom~transfer
\end{theorem}~\end{LRRT}

\begin{LRRT}[Abstract input transfer constructor inverse type]
\begin{theorem}{rule rAINTransferDomRanRel}
    \forall  in: AIN | in \in  \ran~transfer @ in \in  \dom~(transfer~\inv)
\end{theorem}~\end{LRRT}

\subsection{Abstract ignore}

\begin{LSDef}[Abstract ignore operation]
\begin{schema}{AbIgnore}
   AbOp
\where
   abAuthPurse' = abAuthPurse
\end{schema}~\end{LSDef}

\subsection{Transfer}

\begin{LSDef}[Abstract purse transference operation]
\begin{zed}
  AbPurseTransfer ~~\defs~~ AbPurse \hide (balance, lost)
\end{zed}~\end{LSDef}

\begin{LSDef}[Abstract purse secure operations signature]
\begin{schema}{AbWorldSecureOp}
  AbOp\\
  TransferDetails~?
\where
  a? \in  \ran  transfer\\
  \theta  TransferDetails~? = (transfer~\inv)~a?\\
  \{~from?, toN?~\} \ndres  abAuthPurse' = \{~from?, toN?~\} \ndres  abAuthPurse
\end{schema}~\end{LSDef}

\subsection*{Automation for Transfer okay TD domain check}

% For the first case for abAuthPurse' from?

% For the second case for abAuthPurse' to?
% abAuthPurse' from? does not need balance \in nat because there is the relationship with respect to value? from SecureFundsProperty.
\begin{LRRT}[Expanded abstract purse $balance$ type]
\begin{theorem}{rule rAbPurseBalanceTypeOpen}
    \forall  purse: NAMES \ffun AbPurse; name: NAMES | name \in \dom~purse @ \\
        \t1 (purse~name).balance \in \nat
\end{theorem}~\end{LRRT}

% As the goal appears with the elements expanded (after invoke) we need the ugly definition
\begin{LRRT}[Expanded abstract purse $lost$ type]
\begin{theorem}{rule rAbPurseLostTypeOpen}
   \forall purse: NAMES \ffun AbPurse; name: NAMES | \\
      \t1 name \in \dom~purse @ (purse~name).lost \in \nat
\end{theorem}~\end{LRRT}

\begin{LRRT}[Theta instantiation idiom for abstract purses]
\begin{theorem}{rule rThetaPurseInstantiated}
  \forall purse: NAMES \ffun AbPurse; name: NAMES | name \in  \dom~purse @ \\
      \t1 \theta AbPurse[balance := (purse~name).balance, \\
            \t3 lost := (purse~name).lost] = purse~name
\end{theorem}~\end{LRRT}

\begin{LRRT}[Authentic $from?$ purse $TransferOkay$ mu equivalence]
\begin{theorem}{rule lAbAuthPurseFromTransferOkayMuEquivalence}
    \forall  AbWorldSecureOp | Authentic[from?/name?] \land  SufficientFundsProperty \\
        \t1 \land from? \neq toN? @ \\
        \t1\ (~\mu \Delta AbPurse | \\
            \t2 \theta  AbPurse = abAuthPurse~from? \land  \\
            \t2 balance' = balance - value? \land  \\
            \t2 lost' = lost \land  \\
            \t2 \Xi AbPurseTransfer @ \theta  AbPurse'~) = \\
        \t1 \theta AbPurse'[balance' := (abAuthPurse~from?).balance - value?, \\
            \t2 lost' := (abAuthPurse~from?).lost]
\end{theorem}~\end{LRRT}

\begin{LRRT}[Authentic $to?$ purse $TransferOkay$ mu equivalence]
\begin{theorem}{rule lAbAuthPurseToTransferOkayMuEquivalence}
    \forall  AbWorldSecureOp | Authentic[toN?/name?] \land SufficientFundsProperty \\
        \t1 \land from? \neq  toN? @ \\
        \t1\ (~\mu  \Delta AbPurse | \\
            \t2 \theta  AbPurse = abAuthPurse~toN? \land \\
            \t2 balance' = balance + value? \land  \\
            \t2 lost' = lost \land  \\
            \t2 \Xi AbPurseTransfer @ \theta  AbPurse'~) = \\
        \t1 \theta AbPurse'[balance' := (abAuthPurse~toN?).balance + value?, \\
            \t2 lost' := (abAuthPurse~toN?).lost]
\end{theorem}~\end{LRRT}

\begin{LRRT}[Authentic $from?$ purse $TransferLost$ mu equivalence]
\begin{theorem}{rule lAbAuthPurseFromTransferLostMuEquivalence}
    \forall  AbWorldSecureOp | Authentic[from?/name?] \land  SufficientFundsProperty \\
        \t1 \land from? \neq  toN? @ \\
        \t1\ (~\mu  \Delta AbPurse | \\
            \t2 \theta AbPurse = abAuthPurse~from? \\
            \t2 \land balance' = balance - value? \\
            \t2 \land lost' = lost + value? \\
            \t2 \land \Xi AbPurseTransfer @ \theta AbPurse'~) = \\
        \t1 \theta AbPurse'[balance' := (abAuthPurse~from?).balance - value?, \\
            \t2 lost' := (abAuthPurse~from?).lost + value?]
\end{theorem}~\end{LRRT}


%
%\begin{}{rule rAbPurseFromDashIsNOTAuthentic}
%\forall  x, y, z, w: NAMES; xP, yP, zP: AbPurse | x \neq  y \land  x \neq  z \land  y \neq  z \land  (\forall  n: NAMES @ n \in  \{x, y, z\}) \land  \langle \{x\}, \{y\}, \{z\}\rangle  \partition  NAMES \land  \langle \{xP\}, \{yP\}, \{zP\}\rangle  \partition  AbPurse @ \forall  AbWorldSecureOp | Authentic[from?/name?] \land  abAuthPurse = \{\} \land  abAuthPurse' = \{(x, xP), (y, yP)\} @ from? \in  \dom  abAuthPurse'
%\end{}
%

\subsubsection{Transfer okay TD}

The component $abAuhPurse$ is never directly initialised (as far as we could see).
Nevertheless, the retrieve relation from Chapter~\ref{ch10} establishes that
\begin{gzed}
 \dom~abAuthPurse = \dom~conAuthPurse
\end{gzed}
and in Chapter~\ref{ch6}, the range of $conAuthPurse$, but nothing is said about
its domain. In these circumstances, as $AbWorldSecureOp$ allows, some operation could
certainly remove $from?$ purse from the domain of $abAuthPurse'$, hence we need the
$\Delta Authentic$ below.

\begin{LSDef}[Successful secure transfer operation on abstract purses]
\begin{schema}{AbTransferOkayTD}
  AbWorldSecureOp
\where
  \Delta Authentic[from?/name?, from?/name?']\\
  \Delta Authentic[toN?/name?, toN?/name?']\\
  %Authentic[from?/name?]\\
  %Authentic[to?/name?]\\
  SufficientFundsProperty\\
  from? \neq toN?\\
  abAuthPurse' from? = (\mu  \Delta AbPurse | \theta  AbPurse = \\
      \t1 abAuthPurse~from? \land  balance' = balance - value? \land \\
      \t1 lost' = lost \land  \Xi AbPurseTransfer @ \theta  AbPurse~')\\
  abAuthPurse' toN? = (\mu  \Delta AbPurse | \theta  AbPurse = \\
      \t1 abAuthPurse~toN? \land  balance' = balance + value? \land  \\
      \t1 lost' = lost \land  \Xi AbPurseTransfer @ \theta  AbPurse~')
\end{schema}~\end{LSDef}

\begin{LSDef}[Successful secure transfer operation with hidden details]
\begin{zed}
   AbTransferOkay ~~\defs~~ AbTransferOkayTD \hide (toN?, from?, value?)
\end{zed}~\end{LSDef}

\subsubsection{Transfer lost TD}

We need the $\Delta Authentic$ here for $abAuthPurse' to?$,
because although no changes take place, it assumes $to?$ belongs
to the domain of $abAuthPurse'$.
%
\begin{LSDef}[Lost secure transfer operation on abstract purses]
\begin{schema}{AbTransferLostTD}
  AbWorldSecureOp
\where
  \Delta Authentic[from?/name?, from?/name?']\\
  \Delta Authentic[toN?/name?, toN?/name?']\\
  SufficientFundsProperty\\
  from? \neq toN?\\
  abAuthPurse'~from? \in \{~ \Delta AbPurse | \\
      \t1 \theta AbPurse = abAuthPurse~from? \\
      \t1 \land balance' = balance - value? \\
      \t1 \land lost' = lost + value? \\
      \t1 \land \Xi AbPurseTransfer \\
      \t2 @ \theta AbPurse~' ~\}\\
  abAuthPurse'~toN? = abAuthPurse~toN?
\end{schema}~\end{LSDef}

\begin{LSDef}[Lost secure transfer operation with hidden details]
\begin{zed}
   AbTransferLost ~~\defs~~ AbTransferLostTD \hide (toN?, from?, value?)
\end{zed}~\end{LSDef}

\subsubsection{Full transfer}

\begin{LSDef}[Total transfer operation for abstract purses]
\begin{zed}
   AbTransfer ~~\defs~~ AbTransferOkay \lor AbTransferLost \lor AbIgnore
\end{zed}~\end{LSDef}

\section{Abstract initial state}

\begin{LSDef}[Abstract world initialisation state]
\begin{zed}
  AbInitState ~~\defs~~ AbWorld~'
\end{zed}~\end{LSDef}

\begin{LSDef}[Abstract and global initialisation input relationship]
\begin{zed}
  AbInitIn ~~\defs~~ [~ a?, g?: AIN | a? = g? ~]
\end{zed}~\end{LSDef}

\section{Abstract instantiation}

Changed to use renaming to avoid duplicating the
factoring of partial finiteness of the $abAuthPurse$
function.
%
\begin{LSDef}[Global world of purses]
\begin{zed}
   GlobalWorld ~~\defs~~ AbWorld[gAuthPurse/abAuthPurse]
\end{zed}~\end{LSDef}

\begin{LSDef}[Abstract finalisation state]
\begin{schema}{AbFinState}
   AbWorld \\
   GlobalWorld
\where
   gAuthPurse = abAuthPurse
\end{schema}~\end{LSDef}

\begin{LSDef}[Abstract and global finalisation output relationship]
\begin{zed}
   AbFinOut ~~\defs~~ [~ a!, g!: AOUT | a! = g! ~]
\end{zed}~\end{LSDef}

%\pagebreak

\section{Summary}\label{ch3.summary}

We have substituted $NAME$ for $NAMES$ and $\nat$ for $NAT$ at various places
for the forementioned reasons. In here we provide the list where those changes
in declarations took place:
%
\begin{itemize}
   \item Chapter~3: $TransferDetails$, $AbWorld$, $Authentic$
   \item Chapter~4: $CounterPartyDetails$, $PayDetails$, $MESSAGE$, $ConPurse$
   \item Chapter~5: $Logbook$, $AuxWorld$, $PhiBOp$, $Ignore$, $AuthoriseExLogClearOkay$, $Archive$
   \item Chapter~6: $BetwInitIn$
   \item Chapter~7: $PhiCOp$, $CIgnore$, $CArchive$
   \item Chapter~8: $BetweenOpSig$, $ConOpSig$
\end{itemize}
%
Obviously, the theorems involved in these definitions are somewhat affected, but we
tried to keep to the minimum possible. Moreover, because of the limit on the maximal
values, we need added preconditions for the concrete world operations, precisely:
%
\begin{itemize}
   \item Chapter~4: $StartFromPurseEafromOkay$, $StartToPurseEafromOkay$.
   \item Chapter~8: corresponding (4) theorems for the added change in Chapter~4.
\end{itemize}
%
%
Maybe $AbWorld.abAuthPurse$ do no need $NAMES$.
%
$ConWorld.conAuthPurse$, $PhiBOp.names?$, $Ignore.names?$, $AuthoriseExLogClearOkay.names?$
need $NAMES$ due to the relationship with $Logbook$ and the $archive$.

\ldefsummary %
\lthmsummary %
\lthmaddeddefsummary %
\lthmaddedthmsummary %
\lzevessummary %
%\lpscriptsummary


%
%MU EQUIVALENCE FOR THE AUTHENTIC CASES..... lets see.
%
%
%try
%  \forall
%    n, m: \Global NAMES; nP, mP: AbPurse; AbWorldSecureOp
%      |       Authentic[from?/name?] \\
%        \land
%         (\forall x: \Global NAMES @ \Local x \in \{\Local n, \Local m\}) \\
%        \land
%         \Local abAuthPurse' = \{(\Local n, \Local nP), (\Local m, \Local mP)\}
%    @ \Local from? \in \Global \dom \Local abAuthPurse';
%prove by reduce;
%split n = from?;
%rewrite;
%with normalization rewrite;
%cases;
%instantiate x == n;
%rewrite;
%instantiate x == n;
%rewrite;
%rearrange;
%rewrite;
%split
%  \IF m = to?
%  \THEN \{\} = (\{ from? \} \cup \{ to? \}) \ndres abAuthPurse
%  \ELSE \{(m, mP)\} \cup \{(n, nP)\}
%        = (\{ from? \} \cup \{ to? \}) \ndres abAuthPurse;
%rewrite;
%rearrange;
%split m = to?;
%rewrite;
%cases;
%instantiate x == m;
%rewrite;
%instantiate x == n;
%rewrite;
%apply extensionality to predicate
%  \{\} = (\{ from? \} \cup \{ to? \}) \ndres abAuthPurse;
%prove by rewrite;
%
%=>
%try
%  \forall
%    x, y, z, w: NAMES; xP, yP, zP, wP: AbPurse | x \neq y \land x \neq z \land x \neq w \land y \neq w \land y \neq z \land z \neq w \land
%       (\forall n: NAMES @ n \in \{ x, y, z, w \}) @ \forall AbWorldSecureOp | Authentic[from?/name?] \land \\
%         abAuthPurse' = \{( x,  xP), ( y, yP)\}
%    @ \Local from? \in \Global \dom \Local abAuthPurse';
%prove by reduce;
%split n = from?;
%rewrite;
%with normalization rewrite;
%cases;
%instantiate x == n;
%rewrite;
%instantiate x == n;
%rewrite;
%rearrange;
%rewrite;
%split
%  \IF m = to?
%  \THEN \{\} = (\{ from? \} \cup \{ to? \}) \ndres abAuthPurse
%  \ELSE \{(m, mP)\} \cup \{(n, nP)\}
%        = (\{ from? \} \cup \{ to? \}) \ndres abAuthPurse;
%rewrite;
%rearrange;
%split m = to?;
%rewrite;
%cases;
%instantiate x == m;
%rewrite;
%instantiate x == n;
%rewrite;
%apply extensionality to predicate
%  \{\} = (\{ from? \} \cup \{ to? \}) \ndres abAuthPurse;
%prove by rewrite;
%cases;
%
%
%invoke;
%apply inDom;
%rewrite;
%prenex;
%rearrange;
%equality substitute abAuthPurse';
%apply extensionality to predicate
%  (\{ from? \} \cup \{ to? \}) \ndres (\{(m, mP)\} \cup \{(n, nP)\})
%  = (\{ from? \} \cup \{ to? \}) \ndres abAuthPurse;
%apply inNdres;
%apply cupSubset;
%apply unitSubset;
%rewrite;
%rewrite;
%rearrange;
%
%invoke;
%rearrange;
%equality substitute;
%rewrite;
%with normalization rewrite;
%cases;
%
%invoke AbWorldSecureOp;
%invoke AbOp;
%invoke \Delta AbWorld;
%invoke AbWorld;
%invoke TransferDetails;
%rearrange;
%invoke Authentic;
%invoke AbWorld;
%prove by rewrite;
%
%invoke Authentic;
%apply inDom;
%prove by rewrite;
%invoke AbWorldSecureOp;
%apply extensionality to predicate
%  (\{ from? \} \cup \{ to? \}) \ndres abAuthPurse'
%  = (\{ from? \} \cup \{ to? \}) \ndres abAuthPurse;
%apply inNdres;
%apply cupSubset;
%apply unitSubset;
%prove by rewrite;
%invoke;
%instantiate y\_\_1 == y;
%simplify;
