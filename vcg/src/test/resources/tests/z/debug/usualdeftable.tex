\documentclass{article}
\usepackage{czt}
\begin{document}

\begin{zsection}
\SECTION usualdeftable \parents set\_toolkit
\end{zsection}

Given sets
\begin{zed}
   [G, H]
\end{zed}

Usual free types (e.g., recursive, but not mutually recursive)
\begin{zed}
   FT ::= constant | func \ldata \nat \cross FT \rdata
\end{zed}

trivial/usual axdef
\begin{axdef}
   max: \nat
\end{axdef}

State schema with generics
\begin{schema}{State}[X]
   x: \nat \\
   y: \power~\nat \\
   f: \nat \fun \nat \\
   g: X
\where
   y \cross y \subseteq f \\
   x \in y \\
   y \cap y = y
\end{schema}

Init schema with generic instantiation
\begin{schema}{Init}
   State~'
\where
   x' = 0 \\
   y' = \{\} \\
   f' = \{\} \\
   g' = 0
\end{schema}

Usual state update with inputs and outputs, but with twist on generic instantiation
\begin{schema}{OpUpdate}
   \Delta State \\
   i?: \nat \\
   o!: \nat
\where
   g = g' = max \\ % resolves generic parameter implicitly
   y' = y \\
   i? \in y' \\
   % \{i?\} \dres f = \{i?\} \dres f' \\ - avoid dependency with relation\_toolkit
   f~x = i? \\
   f'~i? = x' = o!
\end{schema}

And similarly for a state query operation
\begin{schema}{OpQuery}
  \Xi State \\
  o!: \nat
\where
   g' = o! = f~x
\end{schema}

\begin{schema}{OpUpd2}
   \Delta State
\where
    x' = x = g \\
\end{schema}

This is not very usual, but onkay.
\begin{zed}
   OX == \pre OpQuery
\end{zed}

\begin{axdef}
   ox: OX
\end{axdef}

And now some name collusion as horizontal def
\begin{zed} 
   Man  == [ knowledge: G \fun G ] \\
   Lion == [ knowledge: \power~(G \cross G) ] \\
   Sphinx == [ Man; Lion ]
\end{zed}

\end{document}
