\documentclass{article}

%\newenvironment{theorem}{...}{...}
\usepackage{czt}
\usepackage[color]{circus}

% TODO: These commands also ought to become part of something like circus-time.sty
% 		In particular the selection of the appropriate UNICODE character from within
%		the font's symbol table
\makeatletter
\newcommand{\circwait}{\zpreop{\circuskeyword{wait}}}
\newcommand{\circstartby}{\zbinop{startby}} 
\newcommand{\circendby}{\zbinop{endby}} 
\newcommand{\lcirctime}{\zopenop{(}} 
\newcommand{\rcirctime}{\zcloseop{)}} 
\newcommand{\circtimeout}{\zbinop{timeout}}
\newcommand{\circtimedinterrupt}{\zbinop{interrupt}}
\newcommand{\circat}{\zordop{at}}
\makeatother

\begin{document}

\section{Preamble}

\begin{itemize}
   \item section name and its parents
   \item basic process header
   \item typed channel
   \item generic typed channels
   \item synchronisation channels
   \item various auxiliary declarations used
   \item WHAT KIND OF TEST?
\end{itemize}

\begin{zsection}
  \SECTION\ action\_grammar\_rules \parents\ circus\_toolkit
\end{zsection}

\begin{axdef}
    outside: \nat
\end{axdef}

\begin{circus}
   \circchannel\ c: \nat \cross \nat \cross \nat \cross \nat
\end{circus}

\begin{circus}
   \circprocess\ CircusTimeActionTests \circdef \circbegin
\end{circus}

\begin{axdef}
   n1, n2: \nat \\
   x?, y!, z?: \nat
\end{axdef}

\begin{axdef}
   f : \nat \fun \nat \cross \nat
\end{axdef}

\begin{zed}
   S == [~ y: \nat ~]
\end{zed}

\begin{circusaction}
	\t1 A \circdef \Skip
\end{circusaction}

\newpage
\section{Example 1 --- $CIRCWAIT$}

Production rule: 
\begin{verbatim}
	CIRCWAIT:cw expression:e
\end{verbatim}
%
\begin{itemize}
   \item simple wait with expressions of various kinds and no following action
   \item expressions can come from within and outside the process
   \item expressions that look {\bf funny like partially applied function or theta}
\end{itemize}

Shouldn't $Test2/3$ fail to typecheck somehow because their types too generous?
I guess these (wicked) expressions are for the typechecker. Also, what about 
parenthesis? Are they needed / required here?
%
\begin{circusaction}
   \t1 Test0 \circdef \circwait 10 + outside
   \also
   \t1 Test1 \circdef \circwait n1+n2 
   \also
   \t1 Test2 \circdef \circwait f 
   \also
   \t1 Test3 \circdef \circwait \theta S 
   \also
   \t1 Test4 \circdef \circwait [~ x: \nat | x > 10 ~] 
\end{circusaction}
%
We would also like to add some examples with erroneous productions like
\begin{verbatim}
	\t1 TestX \circdef \circwait 10 \then \Skip
\end{verbatim}
That is, one might get this (or others) as common mistakes because
of the other type (in next section)? That means adding spurious grammar
productions for common (error) patterns will help give better error messages.


Production rule:
\begin{verbatim}
	CIRCWAIT DECLWORD:dw COLON expression:e CIRCSPOT circusAction:ac
\end{verbatim}
%
\begin{itemize}
	\item complex wait patterns with ensuing actions
	\item complex wait patterns with duplicated names (i.e. already declared)
	\item complex wait patterns with possibly unacceptable names?
\end{itemize}

Here some tests reuse declared (global) names, which should be caught by the 
typechecker yet accepted by the parser given the production expects a DECLWORD,
which will also accept strokes (?, !, etc), that you might want to avoid as well?
%
\begin{circusaction}
   \t1 Test5 \circdef \circwait x : 10 + outside \circspot \Skip
   \also
   \t1 Test6 \circdef \circwait y : n1+n2 \circspot \Skip 
   \also
   \t1 Test7 \circdef \circwait z : f \circspot \Skip
   \also
   \t1 Test8 \circdef \circwait w: \theta S \circspot \Skip
   \also
   \t1 Test9 \circdef \circwait x?: S \circspot \Skip
   \also
   \t1 Test10 \circdef \circwait f: S \circspot \Skip
\end{circusaction}

        
\section{Example 2 --- $CIRCSTARTBY$ and $CIRCENDBY$}

Production rule:
\begin{verbatim}
   LCIRCTIME expression:e RCIRCTIME CIRCSTARTBY circusAction:ca
   circusAction:ca CIRCENDBY LCIRCTIME expression:e RCIRCTIME
\end{verbatim}

Again same issues about expressions apply.
%
\begin{circusaction}
   \t1 Test11 \circdef \lcirctime 10 + outside \rcirctime \circstartby \Skip
   \also
   \t1 Test12 \circdef \lcirctime n1+n2 \rcirctime \circstartby \Skip 
   \also
   \t1 Test13 \circdef \lcirctime f \rcirctime \circstartby \Skip
   \also
   \t1 Test14 \circdef \lcirctime \theta S \rcirctime \circstartby \Skip
   \also
   \t1 Test15 \circdef \lcirctime x? \rcirctime \circstartby \Skip
\end{circusaction}

\begin{circusaction}
   \t1 Test16 \circdef A \circendby \lcirctime 10 + outside \rcirctime 
   \also
   \t1 Test17 \circdef A \circendby \lcirctime n1+n2 \rcirctime 
   \also
   \t1 Test18 \circdef A \circendby \lcirctime f \rcirctime 
   \also
   \t1 Test19 \circdef A \circendby \lcirctime \theta S \rcirctime 
   \also
   \t1 Test20 \circdef A \circendby \lcirctime x? \rcirctime 
\end{circusaction}

\section{Example 3 --- $CIRCTIMEOUT$}

Production rule:
\begin{verbatim}
   circusAction:al CIRCTIMEOUT LCIRCTIME expression:e RCIRCTIME  circusAction:ar
\end{verbatim}


\begin{circusaction}
   \t1 Test21 \circdef A \circtimeout \lcirctime 10 + outside \rcirctime A
   \also
   \t1 Test22 \circdef A \circendby \lcirctime f  \rcirctime A \circseq A
\end{circusaction}


\section{Example 4 --- $CIRCTIMEDINTERRUPT$}

Production rule:
\begin{verbatim}
 circusAction CIRCTIMEDINTERRUPT LCIRCTIME expression RCIRCTIME circusAction
\end{verbatim}

\begin{circusaction}
   \t1 Test23 \circdef A \circtimedinterrupt \lcirctime 10 + outside \rcirctime A
   \also
   \t1 Test24 \circdef A \circtimedinterrupt \lcirctime f \rcirctime A \circseq A
\end{circusaction}
				

\section{Example 5 --- timed prefix}

Production rule:
\begin{verbatim}
 communication PREFIXTHEN:pt LCIRCTIME expression RCIRCTIME circusAction
\end{verbatim}

Given complexity of communication and channels (i.e. smart scanning), 
for timed prefix, we'd better have thorough tests. I will copy those from
\textsf{dot-field-singleenv.tex} as they already are thorough for communication
and expand/extend them here for timed communication.
%
\begin{circusaction}
   \t1 Test25 \circdef c?x?y?z \then \lcirctime 10 + outside \rcirctime \Skip
   \also
   \t1 Test26 \circdef c?x!n1.n2 \then \lcirctime x? \rcirctime \Skip 
   \also
   \t1 Test27 \circdef c!n1?x.n2 \then \lcirctime f \rcirctime \Skip
   \also
   \t1 Test28 \circdef c.n1!n2?x \then \lcirctime 20 \rcirctime d \then \lcirctime \theta S \rcirctime \Skip 
   \also
   \t1 Test29 \circdef d \then \lcirctime f \rcirctime e \then \lcirctime f \rcirctime \Skip 
   \also
   \t1 Test30 \circdef c~?i~!(f~i) \then \lcirctime f \rcirctime \Skip
       \also
   \t1 Test31 \circdef c.(S.y)?z \then \lcirctime 20 \rcirctime \Skip
       \also
   \t1 Test32 \circdef c.(x?)!(y!)!(z?) \then \lcirctime 20 \rcirctime \Skip 
\end{circusaction}     
%
\textbf{* $S \in \power~(\lblot y == \nat \rblot)$, hence $S.y \in \nat$.}
\\
\indent \textbf{* $x?, y!, z?$ are decorated names; usually they appear in schemas.}
%
If this proves too complicated, perhaps having it separate as another file 
is a good away to divide-and-conquer.

\paragraph{Description \\ \\}

\vspace{2pt}
\begin{tabular}{l|l}
   \hline
   Action   & Communication pattern \\
   \hline
   $Test25$  & In($x, \nat$), In($y, \nat$), In($z, \nat$) \\
   \hline
   $Test26$  & In($x, \nat$), Out($n1$), Dot($n2$) \\
   \hline
   $Test27$  & Out($n1$), In($x, \nat$), Dot($n2$) \\
   \hline
   $Test28$  & Dot($n1$), Out($n2$), In($x, \nat$), Synch \\
   \hline
   $Test29$  & Synch, Synch, Synch \\
   \hline
   $Test30$  & In($i, \nat$), Out($\nat \cross \nat$) \\
   \hline
   $Test31$  & Dot($S.y$), In($z, \nat \cross \nat$) \\
   \hline
   $Test32$  & Out($x?$), Out($y!$), Out($z?$) \\
   \hline
\end{tabular}
\\

QUESTION: what about prefix with restricting expression and time? The grammar has 
no production for it, and yet would seem sensible to have them for completion? Here
are how they are in \Circus.
%
\begin{circusaction}
   \t1 Test33 \circdef c?x \prefixcolon (x > 1)!(f~x) \then  \Skip
   \also
   \t1 Test34 \circdef c?x \then \\
                  \t4 c?z \prefixcolon (z > x.1).(f~(x.2+x.3)) \then \Skip 
\end{circusaction}

\paragraph{Description \\ \\}

\vspace{2pt}
\begin{tabular}{l|l}
   \hline
   Action   & Communication pattern \\
   \hline
   $Test33$  & In($x, \{~ v: \nat | v > 1 ~\}$), Out($\nat \cross \nat$) \\
   \hline
   $Test34$  & In($x, \nat \cross \nat \cross \nat$);~ In($z, \{~ v: \nat | v > x.1 ~\}$), Out($\nat \cross \nat$) \\
   \hline
\end{tabular}
\\

\textbf{* type on inputs are restricted according to given predicate.}
\\
\indent \textbf{* $Test34$ input on $z$ is from ``$?z \prefixcolon (z > x.1)$''.}

Finally we have generic channels version with time
%
\begin{circusaction}
   \t1 Test34 \circdef g[\nat \cross \nat \cross \nat]?x!n1.n2 \then \lcirctime 20 \rcirctime \Skip
        \also
   \t1 Test35 \circdef g.n1.(f~n1) \then \lcirctime 20 \rcirctime \Skip 
\end{circusaction}


\section{Example 6 --- AT timed prefix}

Production rule:
\begin{verbatim}
 communication ATTIME DECORWORD  PREFIXTHEN  circusAction
 communication ATTIME DECORWORD PREFIXTHEN LCIRCTIME expression RCIRCTIME circusAction
\end{verbatim}

Here I am adding names for the $\circat$ symbol that are either repeated or the ``wrong'' type.
%
\begin{circusaction}
   \t1 Test36 \circdef c?x?y?z \circat X \then \lcirctime 10 + outside \rcirctime \Skip
   \also
   \t1 Test37 \circdef c?x!n1.n2 \circat Z \then \lcirctime x? \rcirctime \Skip 
   \also
   \t1 Test38 \circdef c!n1?x.n2 \circat Y \then \lcirctime f \rcirctime \Skip
   \also
   \t1 Test39 \circdef c.n1!n2?x \circat x? \then \lcirctime 20 \rcirctime d \circat f \then \lcirctime \theta S \rcirctime \Skip 
   \also
   \t1 Test40 \circdef d \then \lcirctime f \rcirctime e \then \lcirctime f \rcirctime \Skip 
   \also
   \t1 Test41 \circdef c~?i~!(f~i) \circat W \then \Skip
       \also
   \t1 Test42 \circdef c.(S.y)?z \circat K \then \Skip
       \also
   \t1 Test43 \circdef c.(x?)!(y!)!(z?) \circat Q \then \lcirctime 20 \rcirctime \Skip 
\end{circusaction}    
				
\section{Example 6 --- Processes TODO}

Production rule:
\begin{verbatim}
   process:pl CIRCENDBY LCIRCTIME expression:e RCIRCTIME
   LCIRCTIME expression:e RCIRCTIME CIRCSTARTBY process:pr
   process:pl CIRCTIMEOUT  LCIRCTIME expression:e RCIRCTIME  process:pr
   process:pl CIRCTIMEDINTERRUPT:ti  LCIRCTIME expression:e RCIRCTIME  process:pr
\end{verbatim}

\newpage
\section{($!$ Prolegomena) --- basic process footer}

It just terminates

\begin{circusaction}
   \circspot \Skip
\end{circusaction}

\begin{circus}
   \circend
\end{circus}

\newpage
\section{\LaTeX}

You also need to update \textsf{circus-time.sty} to contain something akin to

\begin{verbatim}

% TODO: These commands also ought to become part of something like circus-time.sty
% 		In particular the selection of the appropriate UNICODE character from within
%		the font's symbol table
\makeatletter
\newcommand{\circwait}{\zpreop{\circuskeyword{wait}}}
\newcommand{\circstartby}{\zbinop{startby}} 
\newcommand{\circendby}{\zbinop{endby}} 
\newcommand{\lcirctime}{\zopenop{(}} 
\newcommand{\rcirctime}{\zcloseop{)}} 
\newcommand{\circtimeout}{\zbinop{timeout}}
\newcommand{\circtimedinterrupt}{\zbinop{interrupt}}
\newcommand{\circat}{\zordop{at}}
\makeatother

\end{verbatim}

Notice that the choices I've made quickly here don't make for pretty latex! You need to fix / arrange those
by looking at how it's done properly for Z and Circus.

\end{document} 