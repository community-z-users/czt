\documentclass{article}

\usepackage{czt, amssymb, circus-time}
\usepackage[color]{circus}

\begin{document}

\section{Preamble}

\begin{itemize}
   \item section name and its parents
   \item basic process header
   \item typed channel
   \item generic typed channels
   \item synchronisation channels
   \item various auxiliary declarations used
\end{itemize}

\begin{zsection}
  \SECTION\ action\_grammar\_rules \parents\ circustime\_toolkit
\end{zsection}

\begin{circus}
     \circchannel\ d: \nat \\
\end{circus}

\begin{axdef}
    outside: \nat
\end{axdef}

\begin{circus}
   \circchannel\ c: \nat \cross \nat \cross \nat \cross \nat
\end{circus}

\begin{circus}
   \circprocess\ CircusTimeActionTests \circdef \circbegin
\end{circus}

\begin{axdef}
   n1, n2: \nat \\
   x?, y!, z?: \nat
\end{axdef}

\begin{axdef}
   f : \nat \fun \nat \cross \nat
\end{axdef}

\begin{zed}
   S == [~ y: \nat ~]
\end{zed}

%Let, Mu, Conditional, Binding and Tuple expressions
\begin{zed}
  letExpr == \LET x == 1 @ x\\
  muExpr1 == (\mu x : \nat @ x)\\
  muExpr2 == (\mu x : \nat | true)\\
  condExpr == \IF true \THEN 1 \ELSE 2\\
  bindExpr == \lblot one == 1 \rblot\\
  tupleExpr == (1, \{2\}, \power \{3\})\\
\end{zed}

\begin{circusaction}
	\t1 A \circdef \Skip
\end{circusaction}


\section{Example 1 --- $CIRCWAIT$}

Production rule: 
\begin{verbatim}
	CIRCWAIT:cw expression:e
\end{verbatim}
%
\begin{itemize}
   \item simple wait with expressions of various kinds and no following action
   \item expressions can come from within and outside the process
   \item expressions that look {\bf funny like partially applied on several type expressions like theta, mu, condition, bind and tuple expressions}
\end{itemize}
%
\begin{circusaction}
   \t1 Test0 \circdef \circwait 10 + outside
   \also
   \t1 Test1 \circdef \circwait n1+n2 
   \also
   \t1 Test2 \circdef \circwait f 
   \also
   \t1 Test3 \circdef \circwait \theta S 
   \also
   \t1 Test4 \circdef \circwait [~ x: \nat | x > 10 ~] 
   \also
   \t1 Test5 \circdef \circwait letExpr
   \also
   \t1 Test6 \circdef \circwait muExpr1
   \also
   \t1 Test7 \circdef \circwait muExpr2
   \also
   \t1 Test8 \circdef \circwait condExpr
   \also
   \t1 Test9 \circdef \circwait bindExpr
   \also
   \t1 Test10 \circdef \circwait tupleExpr.1
   \also
   \t1 Test11 \circdef \circwait 1 \upto 10
   \also
   \t1 Test12 \circdef c \then \circwait outside
   \also
   \t1 Test13 \circdef \circwait outside \circseq \circwait outside
   \also
   \t1 Test14 \circdef c \then \circwait outside \circseq c \then \circwait outside
   \also
   \t1 Test15 \circdef d!x \then \circwait x + outside
   \also
   \t1 Test16 \circdef \circvres vart : \nat  \circspot \circwait vart 
   \also
   \t1 Test17 \circdef \circvres vart : \nat  \circspot c \then \circwait vart + outside
\end{circusaction}
%
The following examples presents with erroneous productions like
\begin{verbatim}
	\t1 TestX1 \circdef \circwait 10 \then \Skip
	\t1 TestX2 \circdef \circwait \lcirctime 10 \rcirctime 
\end{verbatim}

Production rule:
\begin{verbatim}
	CIRCWAIT DECLWORD:dw COLON expression:e CIRCSPOT circusAction:ac
\end{verbatim}
%
\begin{itemize}
	\item complex wait patterns with ensuing actions
	\item complex wait patterns with duplicated names (i.e. already declared)
        \item the strokes are accepatable by the parser, but that must be avoided during type checking
\end{itemize}

%
\begin{circusaction}
   \t1 Test18 \circdef \circwait x : 10 + outside \circspot \Skip
   \also
   \t1 Test19 \circdef \circwait y : n1+n2 \circspot \Skip 
   \also
   \t1 Test20 \circdef \circwait z : f \circspot \Skip
   \also
   \t1 Test21 \circdef \circwait w: \theta S \circspot \Skip
   \also
   \t1 Test22 \circdef \circwait x?: S \circspot \Skip
   \also
   \t1 Test23 \circdef \circwait f: S \circspot \Skip
   \also
   \t1 Test24 \circdef \circwait m: letExpr \circspot \Skip
   \also
   \t1 Test25 \circdef \circwait m: muExpr1 \circspot \Skip
   \also
   \t1 Test26 \circdef \circwait m: muExpr2 \circspot \Skip
   \also
   \t1 Test27 \circdef \circwait m: condExpr \circspot \Skip
   \also
   \t1 Test28 \circdef \circwait m: bindExpr \circspot \Skip
   \also
   \t1 Test29 \circdef \circwait m: tupleExpr \circspot \Skip
   \also
   \t1 Test30 \circdef \circwait m: 1 \upto 20 \circspot c \then \Skip
   \also
   \t1 Test31 \circdef c \then \circwait m: 1 \upto 20 \circspot c \then \Skip
   \also
   \t1 Test32 \circdef \circwait m: 1 \upto 20 \circspot c \then \circwait outside 
   \also
   \t1 Test33 \circdef d!u \then \circwait m: 1 \upto 20 \circspot \circwait outside + u 
   \also
   \t1 Test34 \circdef \circwait m: 1 \upto 20 \circspot \Skip \circseq \circwait n: 1 \upto 20 
   \circspot \Skip
   \also
   \t1 Test35 \circdef \circvres vart : \nat \circspot \circwait m: 1 \upto 20 \circspot \Skip 
   \also
   \t1 Test36 \circdef \circvres vart : \nat \circspot \circwait m: 1 \upto 20 \circspot c \then 
   \circwait vart + outside
   \also
   \t1 Test37 \circdef \circwait m: 1 \upto 20 \circspot \circwait n: 1 \upto 20 \circspot \Skip
\end{circusaction}

        
\section{Example 2 --- $CIRCSTARTBY$ and $CIRCENDBY$}

Production rule:
\begin{verbatim}
   LCIRCTIME expression:e RCIRCTIME CIRCSTARTBY circusAction:ca
   circusAction:ca CIRCENDBY LCIRCTIME expression:e RCIRCTIME
\end{verbatim}

\begin{itemize}
	\item StartBy and EndBy operatrs with actions
	\item Expressions are acceptable by parser (similar to wait expression)
\end{itemize}
%
\begin{circusaction}
   \t1 Test38 \circdef \lcirctime 10 + outside \rcirctime \circstartby \Skip
   \also
   \t1 Test39 \circdef \lcirctime n1+n2 \rcirctime \circstartby \Skip 
   \also
   \t1 Test40 \circdef \lcirctime f \rcirctime \circstartby \Skip
   \also
   \t1 Test41 \circdef \lcirctime \theta S \rcirctime \circstartby \Skip
   \also
   \t1 Test42 \circdef \lcirctime x? \rcirctime \circstartby \Skip
   \also
   \t1 Test43 \circdef \lcirctime letExpr \rcirctime \circstartby \Skip
   \also
   \t1 Test44 \circdef \lcirctime muExpr1 \rcirctime \circstartby \Skip
   \also
   \t1 Test45 \circdef \lcirctime muExpr2 \rcirctime \circstartby \Skip
   \also
   \t1 Test46 \circdef \lcirctime condExpr \rcirctime \circstartby \Skip
   \also
   \t1 Test47 \circdef \lcirctime bindExpr \rcirctime \circstartby \Skip
   \also
   \t1 Test48 \circdef \lcirctime tupleExpr.1 \rcirctime \circstartby \Skip
   \also
   \t1 Test49 \circdef \lcirctime 1 \upto 20  \rcirctime \circstartby \Skip
   \also
   \t1 Test50 \circdef \lcirctime outside  \rcirctime \circstartby c \then \Skip
   \also
   \t1 Test51 \circdef c \then \lcirctime outside  \rcirctime \circstartby \Skip
   \also
   \t1 Test52 \circdef c \then \lcirctime outside  \rcirctime \circstartby d \then \lcirctime    
   outside \rcirctime \circstartby \Skip
   \also
   \t1 Test53 \circdef d!u \then \lcirctime u  \rcirctime \circstartby \Skip
   \also
   \t1 Test54 \circdef  \circvres vart : \nat \circspot  c \then \lcirctime vart  \rcirctime 
   \circstartby \Skip
   \also
   \t1 Test55 \circdef  \circvres vart : \nat \circspot  c \then \lcirctime vart + outside  
   \rcirctime \circstartby \Skip
   \also
   \t1 Test56 \circdef  A \circseq \lcirctime 10 + outside \rcirctime \circstartby \Skip 
\end{circusaction}

An erroneous example for the gievn production 
\begin{verbatim}
	\t1 TestX3 \circdef A \lcirctime 10 + outside \rcirctime \circstartby \Skip 
	\t1 TestX4 \circdef  10 \circstartby A 
\end{verbatim}



\begin{circusaction}
   \t1 Test57 \circdef A \circendby \lcirctime 10 + outside \rcirctime 
   \also
   \t1 Test58 \circdef A \circendby \lcirctime n1+n2 \rcirctime 
   \also
   \t1 Test59 \circdef A \circendby \lcirctime f \rcirctime 
   \also
   \t1 Test60 \circdef A \circendby \lcirctime \theta S \rcirctime 
   \also
   \t1 Test61 \circdef A \circendby \lcirctime x? \rcirctime 
   \also
   \t1 Test62 \circdef A \circendby \lcirctime 1 \upto 20 \rcirctime 
   \also
   \t1 Test63 \circdef A \circendby \lcirctime letExpr \rcirctime 
   \also
   \t1 Test64 \circdef A \circendby \lcirctime muExpr \rcirctime 
   \also
   \t1 Test65 \circdef A \circendby \lcirctime muExpr1 \rcirctime 
   \also
   \t1 Test66 \circdef A \circendby \lcirctime condExpr \rcirctime 
   \also
   \t1 Test67 \circdef A \circendby \lcirctime bindExpr \rcirctime 
   \also
   \t1 Test68 \circdef A \circendby \lcirctime tupleExpr \rcirctime 
   \also
   \t1 Test69 \circdef c \then A \circendby \lcirctime outside \rcirctime 
   \also
   \t1 Test70 \circdef d!u \then A \circendby \lcirctime u \rcirctime 
   \also
   \t1 Test71 \circdef \circvres vart : \nat \circspot  c \then A \circendby \lcirctime vart 
   \rcirctime 
   \also
   \t1 Test72 \circdef \circvres vart : \nat \circspot  c \then A \circendby \lcirctime vart + 
   outside \rcirctime 
   \also
   \t1 Test73 \circdef  A \circendby \lcirctime outside \rcirctime \circseq A 
\end{circusaction}

An erroneous example for the gievn production 
\begin{verbatim}
	\t1 TestX5 \circdef A \circendby \lcirctime f \rcirctime \then A
        \t1 TestX6 \circdef A \circendby 10
\end{verbatim}


\section{Example 3 --- $CIRCTIMEOUT$}


Production rule:
\begin{verbatim}
   circusAction:al CIRCTIMEOUT LCIRCTIME expression:e RCIRCTIME  circusAction:ar
\end{verbatim}

\begin{itemize}
	\item Timeout operator with actions
	\item Expressions are acceptable by parser (similar to previous expression patterns)
\end{itemize}


\begin{circusaction}
   \t1 Test74 \circdef A \circtimeout {\lcirctime 10 + outside \rcirctime} A
   \also
   \t1 Test75 \circdef c \then A \circtimeout {\lcirctime outside  \rcirctime} A 
   \also
   \t1 Test76 \circdef c \then A \circtimeout {\lcirctime outside  \rcirctime} d \then A 
   \also
   \t1 Test77 \circdef A \circtimeout {\lcirctime f  \rcirctime} A \circseq A
   \also
   \t1 Test78 \circdef c \then A \circtimeout {\lcirctime f  \rcirctime} c \then A \circseq d  
   \then A
   \also
   \t1 Test79 \circdef d!u \then A \circtimeout {\lcirctime u  \rcirctime} A
   \also
   \t1 Test80 \circdef \circvres vart : \nat \circspot c \then A \circtimeout {\lcirctime vart 
   \rcirctime} A
   \also
   \t1 Test81 \circdef \circvres vart : \nat \circspot c \then A \circtimeout {\lcirctime vart + 
   outside \rcirctime} A
\end{circusaction}

An erroneous example for the gievn production 
\begin{verbatim}
	\t1 TestX7 \circdef A \circtimeout 10 A
\end{verbatim}



\section{Example 4 --- $CIRCTIMEDINTERRUPT$}

Production rule:
\begin{verbatim}
 circusAction CIRCTIMEDINTERRUPT LCIRCTIME expression RCIRCTIME circusAction
\end{verbatim}
	
\begin{itemize}
	\item Timedinterrupt operator with actions
	\item Expressions are acceptable by parser (similar to previous expression patterns)
\end{itemize}


\begin{circusaction}
   \t1 Test82 \circdef A \circtimedinterrupt {\lcirctime 10 + outside \rcirctime} A
   \also
   \t1 Test83 \circdef c \then A \circtimedinterrupt {\lcirctime outside  \rcirctime} A 
   \also
   \t1 Test84 \circdef c \then A \circtimedinterrupt {\lcirctime outside  \rcirctime} d \then A 
   \also
   \t1 Test85 \circdef A \circtimedinterrupt {\lcirctime f  \rcirctime} A \circseq A
   \also
   \t1 Test86 \circdef c \then A \circtimedinterrupt {\lcirctime f  \rcirctime} c \then A  
   \circseq d \then A
   \also
   \t1 Test87 \circdef d!u \then A \circtimedinterrupt {\lcirctime u  \rcirctime} A
   \also
   \t1 Test88 \circdef \circvres vart : \nat \circspot c \then A \circtimedinterrupt {\lcirctime 
   vart \rcirctime} A
   \also
   \t1 Test89 \circdef \circvres vart : \nat \circspot c \then A \circtimedinterrupt {\lcirctime 
   vart + outside \rcirctime} A
\end{circusaction}
	
An erroneous example for the gievn production 
\begin{verbatim}
	\t1 TestX8 \circdef A \circtimedinterrupt 10 A
\end{verbatim}
		
\section{Example 5 --- timed prefix}

Production rule:
\begin{verbatim}
 communication PREFIXTHEN:pt LCIRCTIME expression RCIRCTIME circusAction
\end{verbatim}

\begin{itemize}
	\item Timed prefix with communication channel is uesed to express the complexity of timed communication using circus-time operators to test the smart scanner.
	\item Expressions are acceptable by parser (similar to previous expression patterns)
\end{itemize}

%
\begin{circusaction}
   \t1 Test90 \circdef c?x?y?z \then {\lcirctime 10 + outside \rcirctime} \Skip
   \also
   \t1 Test91 \circdef c?x!n1.n2 \then \lcirctime x? \rcirctime \Skip 
   \also
   \t1 Test92 \circdef c!n1?x.n2 \then \lcirctime f \rcirctime \Skip
   \also
   \t1 Test93 \circdef c.n1!n2?x \then \lcirctime 20 \rcirctime d \then \lcirctime \theta S 
   \rcirctime \Skip 
   \also
   \t1 Test94 \circdef d \then \lcirctime f \rcirctime e \then \lcirctime f \rcirctime \Skip 
   \also
   \t1 Test95 \circdef c~?i~!(f~i) \then \lcirctime f \rcirctime \Skip
   \also
   \t1 Test96 \circdef c.(S.y)?z \then \lcirctime 20 \rcirctime \Skip
   \also
   \t1 Test97 \circdef c.(x?)!(y!)!(z?) \then \lcirctime 20 \rcirctime \Skip 
   \also
   \t1 Test98 \circdef c.(x?)!(y!)!(z?) \then \lcirctime 20 \rcirctime \lcirctime outside 
   \rcirctime \circstartby \Skip 
   \also
   \t1 Test99 \circdef c.(x?)!(y!)!(z?) \then \lcirctime 20 \rcirctime A \circendby \lcirctime 
   outside \rcirctime
   \also
   \t1 Test100 \circdef c.(x?)!(y!)!(z?) \then \lcirctime 20 \rcirctime \circwait outside
   \also
   \t1 Test101 \circdef c.(x?)!(y!)!(z?) \then \lcirctime 20 \rcirctime \circwait t : \nat 
   \circspot  A
   \also
   \t1 Test102 \circdef c.(x?)!(y!)!(z?) \then \lcirctime 20 \rcirctime d \then A \circtimeout   
   {\lcirctime outside \rcirctime}  \Skip
   \also
   \t1 Test103 \circdef c.(x?)!(y!)!(z?) \then \lcirctime 20 \rcirctime d \then \lcirctime 20 
   \rcirctime A \circtimeout {\lcirctime outside \rcirctime} \Skip
\end{circusaction}     
%
\textbf{* $S \in \power~(\lblot y == \nat \rblot)$, hence $S.y \in \nat$.}
\\
\indent \textbf{* $x?, y!, z?$ are decorated names; usually they appear in schemas.}
%
If this proves too complicated, perhaps having it separate as another file 
is a good away to divide-and-conquer.

\paragraph{Description \\ \\}

\vspace{2pt}
\begin{tabular}{l|l}
   \hline
   Action   & Communication pattern \\
   \hline
   $Test25$  & In($x, \nat$), In($y, \nat$), In($z, \nat$) \\
   \hline
   $Test26$  & In($x, \nat$), Out($n1$), Dot($n2$) \\
   \hline
   $Test27$  & Out($n1$), In($x, \nat$), Dot($n2$) \\
   \hline
   $Test28$  & Dot($n1$), Out($n2$), In($x, \nat$), Synch \\
   \hline
   $Test29$  & Synch, Synch, Synch \\
   \hline
   $Test30$  & In($i, \nat$), Out($\nat \cross \nat$) \\
   \hline
   $Test31$  & Dot($S.y$), In($z, \nat \cross \nat$) \\
   \hline
   $Test32$  & Out($x?$), Out($y!$), Out($z?$) \\
   \hline
\end{tabular}
\\

Timed comminication prefix  with restricting expression. 
%
\begin{circusaction}
   \t1 Test104 \circdef c?x \prefixcolon (x > 1)!(f~x) \then  \Skip
   \also
   \t1 Test105 \circdef c?x \then \\
                  \t4 c?z \prefixcolon (z > x.1).(f~(x.2+x.3)) \then \Skip 
   \also
   \t1 Test106 \circdef c?x \prefixcolon (x > 1).(f~(x.2+x.3)) \then \lcirctime outside \rcirctime 
   \Skip 
   \also
   \t1 Test107 \circdef c?x \prefixcolon (x > 1).(f~(x.2+x.3)) \then \lcirctime outside \rcirctime 
    c?z \prefixcolon (z > x.1).(f~(x.2+x.3)) \then \Skip 
   \also
   \t1 Test108 \circdef c?x \prefixcolon (x > 1).(f~(x.2+x.3)) \then \lcirctime outside \rcirctime 
    c?z \prefixcolon (z > x.1).(f~(x.2+x.3)) \then \lcirctime 20 \rcirctime \Skip 
   \also
   \t1 Test109 \circdef c?x \prefixcolon (x > 1).(f~(x.2+x.3)) \then \lcirctime outside \rcirctime 
    c?z \prefixcolon (z > x.1).(f~(x.2+x.3)) \then \lcirctime 20 \rcirctime \lcirctime outside 
   \rcirctime \circstartby A 
   \also
   \t1 Test110 \circdef c?x \prefixcolon (x > 1).(f~(x.2+x.3)) \then \lcirctime outside \rcirctime 
    c?z \prefixcolon (z > x.1).(f~(x.2+x.3)) \then \lcirctime 20 \rcirctime A \circendby  
   \lcirctime outside \rcirctime 
\end{circusaction}

\paragraph{Description \\ \\}

\vspace{2pt}
\begin{tabular}{l|l}
   \hline
   Action   & Communication pattern \\
   \hline
   $Test33$  & In($x, \{~ v: \nat | v > 1 ~\}$), Out($\nat \cross \nat$) \\
   \hline
   $Test34$  & In($x, \nat \cross \nat \cross \nat$);~ In($z, \{~ v: \nat | v > x.1 ~\}$), Out($\nat \cross \nat$) \\
   \hline
\end{tabular}
\\

\textbf{* type on inputs are restricted according to given predicate.}
\\
\indent \textbf{* $Test34$ input on $z$ is from ``$?z \prefixcolon (z > x.1)$''.}

Finally we have generic channels version with time
%
\begin{circusaction}
   \t1 Test111 \circdef g[\nat \cross \nat \cross \nat]?x!n1.n2 \then \lcirctime 20 \rcirctime 
   \Skip
   \also
   \t1 Test112 \circdef g.n1.(f~n1) \then \lcirctime 20 \rcirctime \Skip 
\end{circusaction}


\section{Example 6 --- AT timed prefix}

Production rule:
\begin{verbatim}
 communication ATTIME DECORWORD  PREFIXTHEN  circusAction
 communication ATTIME DECORWORD PREFIXTHEN LCIRCTIME expression RCIRCTIME circusAction
\end{verbatim}

\begin{itemize}
	\item Grammar for $\circat$ operator with actions
	\item Grammer for $\circat$ operator with expression and action 
	\item Expressions are acceptable by parser (similar to previous expression patterns)
	\item Comminication channel can allow previous communication patterns and restricted inputs 
\end{itemize}

%
\begin{circusaction}
   \t1 Test113 \circdef c?x?y?z \circat X \then \lcirctime 10 + outside \rcirctime \Skip
   \also
   \t1 Test114 \circdef c?x!n1.n2 \circat Z \then \lcirctime x? \rcirctime \Skip 
   \also
   \t1 Test115 \circdef c!n1?x.n2 \circat Y \then \lcirctime f \rcirctime \Skip
   \also
   \t1 Test116 \circdef c.n1!n2?x \circat x? \then \lcirctime 20 \rcirctime d \circat f \then \lcirctime \theta S \rcirctime \Skip 
   \also
   \t1 Test117 \circdef d \then \lcirctime f \rcirctime e \then \lcirctime f \rcirctime \Skip 
   \also
   \t1 Test118 \circdef c~?i~!(f~i) \circat W \then \Skip
       \also
   \t1 Test119 \circdef c.(S.y)?z \circat K \then \Skip
       \also
   \t1 Test120 \circdef c.(x?)!(y!)!(z?) \circat Q \then \lcirctime 20 \rcirctime \Skip 
   \also
   \t1 Test121 \circdef c!n1?x.n2 \circat Y \then \lcirctime f \rcirctime d \then \Skip
   \also
   \t1 Test122 \circdef c!n1?x.n2 \circat Y \then \lcirctime f \rcirctime d \then \lcirctime 
   outside \rcirctime \Skip
   \also
   \t1 Test122 \circdef c!n1?x.n2 \circat Y \then \lcirctime f \rcirctime d \then \lcirctime 
   outside \rcirctime \lcirctime 20 \rcirctime \circstartby \Skip  
   \also
   \t1 Test123 \circdef c!n1?x.n2 \circat Y \then \lcirctime f \rcirctime d \then \lcirctime 
   outside \rcirctime \lcirctime 20 \rcirctime \circstartby \circwait 20
   \also
   \t1 Test124 \circdef c!n1?x.n2 \circat Y \then \lcirctime f \rcirctime d \then \lcirctime 
   outside \rcirctime \lcirctime 20 \rcirctime \circstartby \circwait t : \nat \circspot d \then 
   \Skip   
   \also
   \t1 Test125 \circdef c!n1?x.n2 \circat Y \then \lcirctime f \rcirctime d\circat X \then 
   \lcirctime outside \rcirctime \lcirctime 20 \rcirctime \circstartby \circwait t : \nat 
   \circspot d \then \Skip

\end{circusaction}    

\newpage
\section{($!$ Prolegomena) --- basic process footer}

It just terminates

\begin{circusaction}
   \circspot \Skip
\end{circusaction}

\begin{circus}
   \circend
\end{circus}

\end{document} 
