\begin{zsection}
  \SECTION\ DDDR\_DDIR\_Section \parents\ circustime\_toolkit
\end{zsection}


\hline
\textbf{Type Definition}
\hline

\begin{zed}
  BOOL ::= ON | OFF \\
  OpMode ::= DDDR | DDIR
\end{zed}

\hline
\textbf{Range value of the different configuration parameters}
\hline

\begin{axdef}
  LRIRange == 350 \upto 2000\\            %\t2 \textit{Lower Rate Interval (in ms.)} \\
  AVIRange == 100 \upto 150\\             %\t2 \textit{Atria-Ventricle Interval (in ms.) }\\
  MSRRange == 350 \upto 1200\\            %\t2 \textit{Maximum Sensor Rate Interval (in ms.) }\\
  PVARPRange == 150 \upto 500\\           %\t2 \textit{Post Ventricle Atria Refectory Period (in ms.) }\\ 
  SlopRange == 1 \upto 16\\               %\t3 \textit{slop range for changing pacing length during rate modulation mode}\\
  PacingLengthRange == 1 \upto 3\\        %\t2 \textit {time duration for current discharging in both chambers (in ms.)} \\
  SenseARange == 25 \upto 75\\            %\t2 \textit{Atrium Sense range 0.25, 0.5, 0.75mV  (Volt * 100) (threshold) }\\
  SenseVRange == 100 \upto 1000\\         %\t2 \textit{Ventricle Sense range 1.0-10 mV (Volt * 100) (threshold)}\\
  MSRange == 2 \upto 10                   %\t3 \textit{motion sensor .2 to 1.0, sensing activity}
\end{axdef}


\hline
\textbf{A set of channels declaration }
\hline

\begin{circus}
  %\textit{Output Channel for Pacing} \\
  \circchannel paceA, paceV : BOOL\\

  \\

  %\textit{Channel for pacemaker cycle } \\
  \circchannel startCycle \\

  \\

  %\textit{Input channel for sensing in atrium chamber}\\
  \circchannel senseA :  SenseARange\\

  \\

  %\textit{Input channel for sensing in ventricle chamber}\\
  \circchannel senseV :  SenseVRange\\

  \\
    
  %\textit{Input channel for motion sensor} \\
  \circchannel inMotionSensor : MSRange\\
  
  \\

  %\textit{Environment channels to start and stop the pacemaker.}\\
  \circchannel startPacemaker \\
  \circchannel stopPacemaker \\

  \\
  
  %\textit{Emergency channel is used to stop the pacemaker due to environment interruption.}\\ 
  \circchannel emergencyStop \\
  
  %\textit{Extra channels to synchronize timing and pacemaker behavior}\\
  \circchannel timeoutA, timeoutV, noPaceA, noPaceV\\
  
  \circchannel\ [X] c, d: X 
  
\end{circus}





\hline
\textbf{A set of global constants of the cardiac pacemaker}
\hline
\begin{axdef}
  paceLength : PacingLengthRange\\
  AVI : AVIRange \\
  MSR : MSRRange \\
  PVARP : PVARPRange \\
  slop  : SlopRange  \\
  timeDuration : \nat
  \where \\
  MSR \geq PVARP+AVI
\end{axdef}


\textbf{Mode and Counter state}

\begin{schema}{ModeCounterState}
  currentMode : OpMode\\
  PCount : \nat \\
\end{schema}


\hline
\textbf{To begin \Circus processes }
\hline


\begin{circus}
	\circprocess\ Pacemaker\_DDDR\_DDIR \circdef \circbegin
\end{circus}



\textbf {Pacing in the atrium chamber}

\begin{circusaction}
  PaceA \circdef \\
  \t1 paceA!ON \then paceA!OFF \then  c!ON \then  \Skip 
\end{circusaction}


\textbf {Pacing in the ventricle chamber}

\begin{circusaction}
  PaceV \circdef \\
  \t1 paceV!ON \then paceV!OFF \then \Skip
\end{circusaction}

\textbf {Sensing process in the atrium chamber}



\begin{circusaction}
SenseAtrium \circdef \\ 
\t1 senseA?THR \then 
    ( noPaceA \then \Skip \\
      \extchoice SenseAtrium) \extchoice timeoutA \then  PaceA 
\end{circusaction}




\textbf {Sensing process in the ventricle chamber}



\begin{circusaction}
SenseVentricle \circdef \\
\t1 senseV?THR \then 
   ( noPaceV \then \Skip \\
   \extchoice SenseVentricle) \extchoice timeoutV \then PaceV
\end{circusaction}




\textbf {Pacing cycle in pacemaker in any mode(DDDR or DDIR)}
\begin{circusaction}
PacemakerOperation \circdef\\
  \t1 SenseAtrium \circseq \\
  \t1 SenseVentricle
\end{circusaction}

\begin{circusaction}
Pacemaker \circdef \\
\t1 \circmu X \circspot startCycle \then PacemakerOperation \circseq X 
\end{circusaction}


\textbf{ Main \Circus~ action }\\

\begin{circusaction}
  \circspot \circmu X \circspot (startPacemaker \then Pacemaker \\ 
   \circinterrupt \\
   (stopPacemaker \then X \\ 
  \extchoice \\
  emergencyStop \then \Stop)) 
\end{circusaction}

\begin{circus}
  \circend
\end{circus}





\hline
\textbf{To begin \Circus processes Timing Requirements}
\hline

\begin{circus}
	\circprocess\ PMTReq \circdef \circbegin
\end{circus}

\begin{schema}{PacemakerState}
  paceInterval : LRIRange  \\
  \where
  paceInterval > PVARP + AVI \\
  paceInterval > MSR \\
\end{schema}


\textbf{State components of the cardiac pacemaker, which are modified during system operations}

\begin{circusaction}
  \circstate ~ PacemakerState
\end{circusaction}

\begin{circusaction}
TReq \circdef \\
\t1 \circvres PCount,iteration : \nat ; \circvres currentMode : OpMode \\
\t2 \circspot \circmu X \circspot (TReqCycle(PCount, currentMode) \\
\linter \{PCount,currentMode \} | \{paceInterval,iteration\}  \rinter \\
RateModulation(iteration)) \circseq X \\
\end{circusaction}



\begin{circusaction}
	TReqCycle \circdef \circvres PCount : \nat ; \circvres currentMode : OpMode \circspot\\
	startCycle \then TReqSelection(PCount, currentMode)\\
\end{circusaction}


\textbf{ Timing requirement selection according to operating mode}\\
\begin{circusaction}
TReqSelection \circdef \\
\t1 \circvres PCount : \nat; \circvres currentMode : OpMode \circspot \\
\t2 \circif  currentMode = DDDR \circthen \\
\t3 DDDR\_TReq (PCount,0,0,currentMode) \\ 
\t2 \circelse currentMode = DDIR \circthen \\
\t3 DDIR\_TReq (PCount,0,currentMode)  \\ 
\t2 \circfi
\end{circusaction}




\textbf{Mode changing schema : DDDR $\leftrightarrow$ DDIR}

\begin{schema}{ModeSwitching}
 \Delta ModeCounterState\\
  passedTime? : \nat 
\where
(passedTime? > MSR \land PCount \notin \{5,8\} \land PCount'= PCount + 1 \land currentMode' = currentMode)\\
\lor \\
(passedTime? \leq MSR \land PCount'= 0 \land currentMode' = currentMode)\\
\lor \\
(passedTime? > MSR \land PCount = 5 \land currentMode = DDDR \land currentMode' = DDIR \land PCount' = 0) \\
\lor \\
(passedTime? > MSR \land PCount = 8 \land currentMode = DDIR \land currentMode' = DDDR \land PCount' = 0) \\
\end{schema}


\textbf{DDDR timing cycle}

\begin{circusaction}
DDDR\_TReq \circdef \\
\t1 \circvres PCount, passedTime, atriumSenseTime : \nat; \circvres currentMode : OpMode \\
  \t2 \circspot (
  \circwait(PVARP) \circseq \\
   passedTime := passedTime + PVARP \circseq \\
   SenseAtrium\_DDDR\_DDIR\_TReq(passedTime) \circseq \\
   atriumSenseTime := passedTime \circseq \\
   SensepaceVentricle\_DDDR(passedTime, atriumSenseTime) \circseq \\  
   ModeSwitching) \\ 
\end{circusaction}


% might be parametric error 
\begin{circusaction}
SenseAtrium\_DDDR\_DDIR\_TReq \circdef \\
\t1\circvres passedTime : \nat \\
\t2 \circspot senseA?THR \then \\
\t1 \circif  THR \geq 25 \circthen\\
\t2 noPaceA \then \Skip \\
\t1 \circelse \lnot  THR \geq 25 \circthen\\
\t2 \circif  (passedTime \geq (paceInterval - AVI) ) \circthen \\
\t3 timeoutA \then TReqPaceA(passedTime) \\
\t2 \circelse \lnot (passedTime \geq (paceInterval - AVI)) \circthen \\
\t3 \circwait(1) \circseq passedTime := passedTime + 1 \circseq\\
\t3 SenseAtrium\_DDDR\_DDIR\_TReq(passedTime) \\
\t2 \circfi\\ 
\t1 \circfi \\ 
\end{circusaction}

\begin{circusaction}
	TReqPaceA \circdef \circvres passedTime : \nat \circspot \\
	paceA?x \then \circwait(paceLength) \circseq paceA?y \then passedTime := passedTime + paceLength 
\end{circusaction}



\begin{circusaction}
SensepaceVentricle\_DDDR \circdef \\
\t1 \circvres passedTime, atriumSenseTime : \nat \\
\t2\circspot senseV?THR \then \\
\t1 \circif  THR \geq 100 \circthen\\
\t2 noPaceV \then \Skip \\
\t1 \circelse \lnot  THR \geq 100 \circthen\\
\t2 \circif  ( passedTime \geq (atriumSenseTime + AVI) ) \circthen \\
\t3 timeoutV \then TReqPaceV(passedTime) \\
\t2 \circelse \lnot (passedTime \geq (atriumSenseTime + AVI)) \circthen \\
\t3  \circwait(1) \circseq passedTime := passedTime + 1 \circseq\\
\t3  SensepaceVentricle\_DDDR(passedTime,atriumSenseTime) \\
\t2 \circfi\\ 
\t1 \circfi \\ 
\end{circusaction}



\begin{circusaction}
	TReqPaceV \circdef \circvres passedTime : \nat \circspot \\
	paceV?x \then \circwait(paceLength) \circseq paceV?y \then passedTime := passedTime + paceLength 
\end{circusaction}




\textbf{DDIR timing cycle}

\begin{circusaction}
	DDIR\_TReq \circdef \\
  \t1 \circvres PCount, passedTime : \nat ; \circvres currentMode : OpMode \\
  \t2 \circspot 
  \circwait(PVARP) \circseq \\
   passedTime := passedTime + PVARP \circseq \\
   SenseAtrium\_DDDR\_DDIR\_TReq(passedTime) \circseq \\
   SensepaceVentricle\_DDIR(passedTime) \circseq \\
   ModeSwitching  
\end{circusaction}


\textbf {Sensing process in the ventricle chamber : DDIR}\\

\begin{circusaction}
SensepaceVentricle\_DDIR \circdef \\
\t1 \circvres passedTime : \nat \\
\t2 \circspot senseV?THR \then \\
\t1 \circif  THR \geq 100 \circthen\\
\t2 noPaceV \then \Skip \\
\t1 \circelse \lnot  THR \geq 100 \circthen\\
\t2 \circif  (passedTime \geq paceInterval ) \circthen \\
\t3 timeoutV \then TReqPaceV(passedTime)\\
\t2 \circelse \lnot (passedTime \geq paceInterval ) \circthen \\
\t3 \circwait(1) \circseq passedTime := passedTime + 1 \circseq \\
\t3 SensepaceVentricle\_DDIR(passedTime) \\
\t2 \circfi\\ 
\t1 \circfi \\ 
\end{circusaction}


%%%% rate modulation section


\textbf{Motion sensor schema to update the pace interval} 

\begin{schema}{ChangePaceInterval}
\Delta PacemakerState \\
iteration : \nat \\
iteration' : \nat \\
motionSensorValue? : MSRange 
\where
(motionSensorValue? > 5 \land paceInterval'= 60000 \div (60 + slop * iteration))\\
\lor \\
(motionSensorValue? \leq 5 \land paceInterval'= 60000 \div (120 - slop * iteration))\\
\land \\
(iteration < ((120-60) \div slop) \land iteration' = iteration + 1)   \\
\lor \\
(iteration \geq ((120-60) \div slop) \land iteration'= 0 \land paceInterval'=paceInterval ) \\  
\end{schema}

\textbf{Motion sensor process of the pacemaker} 

\begin{circusaction}
RateModulation \circdef \\
\t1 \circvres\ iteration :\nat  \circspot \\
\t2 inMotionSensor~?~motionSensorValue \then ChangePaceInterval
\end{circusaction}



\textbf{ Main \Circus~ action }\\

\begin{circusaction}
	\circspot \circmu X \circspot paceInterval := 1000 \circseq ( startPacemaker \then TReq(0,0,DDDR) \\ 
   \circinterrupt \\
  (stopPacemaker \then \circwait(0..timeDuration) \circseq X \\ 
  \extchoice \\
  emergencyStop \then \Stop))
\end{circusaction}


\begin{circus}
  \circend
\end{circus}


\textbf{The requirements of the abstract system are specified by the system process below. }\\

\begin{circus}
\circprocess PacemakerAbstract \circdef  Pacemaker\_DDDR\_DDIR \lpar \lchanset startCycle, senseA, senseV, noPaceA, noPaceV, timeoutA, timeoutV \rchanset \rpar PMTReq
\end{circus}
