%%%%%%%%%%%%%%%%%%%%%%%%%%%%%%%%%%%%%%%%%%%%%%%%%%%%%%%%%%%%%%%%%%%%%%%%%%%%%%%%
% PREAMBLE: Circus prelude, v 1.3, June 2007
%           Copyright 2004-2007 Leo Freitas
%           Department of Computer Science
%           University of York, YO10 5DD UK
%           leo@cs.york.ac.uk, +44-1904-434753
%           http://www.cs.york.ac.uk/circus
%           http://www.cs.york.ac.uk/~leo
%%%%%%%%%%%%%%%%%%%%%%%%%%%%%%%%%%%%%%%%%%%%%%%%%%%%%%%%%%%%%%%%%%%%%%%%%%%%%%%%

%%%%%%%%%%%%%%%%%%%%%%%%%%%%%%%%%%%%%%%%%%%%%%%%%%%%%%%%%%%%%%%%%%%%%%%%%%%%%%%%
% Like the Z standard prelude, the Circus prelude defines the basic keywords
% and operators for the Circus language. It MUST NOT depend on any other toolkit,
% otherwise a cyclic recursion might occur. The right way to specify user toolkits
% for Circus is to have the circus_prelude or circus_toolkit as a parent.
% circus_toolkit inherits the circus_prelude and standard_toolkit.
%
\begin{zsection}
   \SECTION circustime\_prelude
\end{zsection}

%%%%%%%%%%%%%%%%%%%%%%%%%%%%%%%%%%%%%%%%%%%%%%%%%%%%%%%%%%%%%%%%%%%%%%%%%%%%%%%%
% Support for Circus Time

%   Wait action
%%Zpreword \circwait wait

%   Time End By
%%Zinchar \circendby U+25B6

%   Time Start By
%%Zinchar \circstartby U+25C0

%   Timeout
%%Zinchar \circtimeout U+25B9

%   Time Interrupt
%%Zinchar \circtimedinterrupt U+25B5

%   Time Angular brackets 
%%Zchar \lcirctime U+27EC
%%Zchar \rcirctime U+27ED

% At-time Symbol
%%Zchar \circat U+0040

