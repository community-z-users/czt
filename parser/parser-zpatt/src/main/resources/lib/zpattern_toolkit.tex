\documentclass[10pt]{article}
\usepackage{czt}

\title{CZT Z-Pattern Toolkit}
\author{Mark Utting}
\begin{document}
\maketitle

This toolkit is for use with the \emph{Z pattern} extension of Z,
which is used by CZT to define transformation and reasoning rules
for Z.  These rulesets are usually defined by tool-builders rather
than end-users of Z.

\LaTeX\ files that define Z pattern rulesets should use the
\verb!czt.sty! file, which defines several \LaTeX macros and
environments for defining Z pattern rules, such as 
the \verb!zedjoker! environment for declaring
Z pattern jokers and the \verb!zedrule! environment for defining
rules.  See the rules/src/test/resources/*.tex files for examples.

\begin{zsection}
  \SECTION zpattern\_toolkit \parents standard\_toolkit
\end{zsection}

This toolkit section defines any operators or functions used in the
provisos of rules.  It also defines the \LaTeX\ and Unicode markup of
the symbols used in the rule syntax.

\begin{verbatim}
%%Zinchar \proviso U+25B8
%%Zinchar \derives U+2501
\end{verbatim}

We define some syntax for use within the provisos that manipulate
schemas and signatures.  The $[D1|true] \schemamerge [D2|true]$
operator calculates the union of the two DeclLists. 
$[D1|true] \schemaminus [D2|true]$ returns the declarations that are in
$D1$ but not in $D2$.  Both these operators have the precondition that
$D1$ and $D2$ must be in normalized form (VarDecls only with one name
per VarDecl, types that are base types, no duplicated names).

\begin{verbatim}
%%Zinword \schemamerge schemamerge
%%Zinword \schemaminus schemaminus
%%Zinword \unprefix unprefix
%%Zinword \schemaEquals schemaEquals
\end{verbatim}

\begin{zed}
  \function 30 \leftassoc ( \_ \schemamerge \_ ) \\
  \function 30 \leftassoc ( \_ \schemaminus \_ ) \\
  \function 30 \leftassoc ( \_ \unprefix \_ ) \\
  \relation ( \_ \schemaEquals \_ )
\end{zed}


This definition captures part of the semantics of
the $\schemaminus$ operator and gives it the correct
type.  Note that this definition will create an $\exists$
within the predicate of SIG3, but that exists will be
trivially true if we assume that all base types are non-empty. 

TODO: add this predicate: $SIG3 = \exists SIG2 @ SIG1$.

\begin{gendef}[SIG1,SIG2,SIG3]
  \_ \schemaminus \_ : SIG1 \cross SIG2 \fun SIG3 \\
  \_ \schemamerge \_ : SIG1 \cross SIG2 \fun SIG3
\end{gendef}

\begin{gendef}[SIG1,SIG2]
  \_ \schemaEquals \_ : SIG1 \rel SIG2
\end{gendef}
\end{document}
