\begin{zsection}
  \SECTION circus\_toolkit \parents standard\_toolkit
\end{zsection}

% Mark-up directives for core language symbols are as follows.
% The choice for pre, in, or post, has to do with how one wants
% the character to be rendered. See Z Std A.2.3.2.
%
% By using keychars such as \circguard instead of %, one allows
% appropriate rendering by both LaTeX and Unicode mark-ups.
 
% Language keychars: 
% TODO: Try to use just \circthen for both guards and prefixing?
%
%%Zprechar \circmu U+03BC
%%Zinchar \circthen U+2192
%%Zinchar \circelse U+25A1
%%Zinchar \defs U+2259
%%Zinchar \circguard U+0026

% Sequence
%
%%Zinchar \circsemi ;
%%Zprechar \Semi U+2A1F

% Brackets
%
%%Zprechar \lchanset U+2983
%%Zpostchar \rchanset U+2984

% Process Indexing 
%
%%Zinchar \circindex U+2299
%%Zchar \circlinst U+230A
%%Zchar \circrinst U+230B

% Parallel operators
%
%%Zinchar \lpar U+27E6
%%Zinchar \rpar U+27E7
%%Zinchar \linter U+27E6
%%Zinchar \rinter U+27E7
%%Zinchar \interleave U+2980
%%Zprechar \Parallel U+2225
%%Zprechar \Interleave U+2AFC 

% Choice operators
%
%%Zinchar \extchoice U+25FB
%%Zinchar \intchoice U+2293
%%Zprechar \Extchoice U+25A1
%%Zprechar \Intchoice U+2293

% Prefixing
%
%%Zinchar \then U+2192

% Alternatives          Observation
%
% \lpar         U+301A  Both options are like [|, rather than |[
% \rpar         U+301B  Both options are like |], rather than ]|
% \linter       U+301A  At the moment just like \lpar (find other later).
% \rinter       U+301B  At the moment just like \rpar (find other later).
% \circindex    U+2A00  (2A00=bigger version). Use circus.sty \circindex instead of LaTeX \odot.
% \circlinst    U+23A3
% \circrinst    U+23A6, U+300D
% \Parallel     U+2016
% \Interleave   U+2980  Small version of |||
% \defs         U+225D  = with def on the top of it.
% \Semi         U+2A3E


% Language keywords:
%

% Guarded commands
%
%%Zpreword \circif if
%%Zpreword \circfi fi

% Channels
%
%%Zpreword \circchannel channel                            
%%Zpreword \circchannelfrom channelfrom
%%Zpreword \circchannelset channelset

% Processes
%
%%Zpreword \circprocess process
%%Zpreword \circbegin begin
%%Zinword  \circstate state
%%Zpostword \circend end

% Definitions
%
% TODO: Find the equivalent in LaTeX for this Unicode.
%%Zinchar \gendj U+2A46
\begin{zed}
\function 40 leftassoc (\_~\gendj~\_)
\end{zed}

\begin{gendef}[X]
    \_ \gendj \_: \power~(\power~X) \cross \power~(\power~X) \fun \power~X
\where
    \forall S, P: \power~(\power X) @ P \gendj S = (\bigcap~P)~\setminus~(\bigcup~(S~\setminus~P))
\end{gendef}

% LaTeX equivalent is \otimes
%%Zprechar \regions U+2A02
\begin{gendef}[X]
    \regions: \power_1~(\power~X) \fun \power_1~(\power~X)
\where
    \forall S: \power_1~(\power~X) @ \\
      \t1 regions~S = \{~ P: \power_1~(\power~X) | P \subseteq S @ P \gendj S \}~\setminus~\{~ \emptyset ~\}
\end{gendef}

\begin{gendef}[X]
    minimal: \power~(\power~X) \fun \power~(\power~X)
\where
   \forall SS: \power~(\power~X) @ \\
    \t1 minimal~SS = \{ S: \power~X | S \in SS \land \\
                \t2 \lnot~(~\exists R: \power~X | R \in SS @ R \subset S~) \}
\end{gendef}

%TODO: Fix these functions when get to parallelism in the parser.
%
%<<<<<<< circus_toolkit.tex
%%TODO: Fix UNICODE CHARACTER
% Zinchar searrow U+2200
%%\begin{zed}
%%\function 30 \leftassoc (\_ searrow \_)
%%\end{zed}
%
%\begin{gendef}[X]
%%\_ searrow \_ : \power~(\power~X) \cross \power X \fun \power~(\power~X)
%    searrow:\power~(\power~X) \cross \power X \fun \power~(\power~X)
%\where
%    \forall P: \power~(\power~X); p: \power~X @ searrow~(P, p) = \{~ q: \power~X | q \in P @ q \setminus p ~\}
%\end{gendef}
%
%%TODO: Fix UNICODE CHARACTER
% Zinchar \diamond U+2201
%%\begin{zed}
%%\function 40 \leftassoc (\_ \diamond \_)
%%\end{zed}
%=======
% Zinchar \searrow U+2200
%%\begin{zed}
%% \function 30 \leftassoc (\_ \searrow \_)
%%\end{zed}
%>>>>>>> 1.5
%
%<<<<<<< circus_toolkit.tex
%\begin{gendef}[X]
%%\_ \diamond \_ : \power~(\power~X) \cross \power X \fun \power~(\power~X)
%  diamond: \power~(\power~X) \cross \power X \fun \power~(\power~X)
%\where
%    %\forall~ P: \power~(\power~X); p: \power~X @ P \diamond p = \{~ q: \power~X | q \in P @ q \cap p ~\}
%    \forall~ P: \power~(\power~X); p: \power~X @ diamond~(P, p) = \{~ q: \power~X | q \in P @ q \cap p ~\}
%\end{gendef}
%=======
%>>>>>>> 1.5
