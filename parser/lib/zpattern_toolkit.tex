\newcommand{\schemaminus}{\mathbin{\textbf{schemaminus}}}
\newcommand{\schemamerge}{\mathbin{\textbf{schemamerge}}}

\begin{zsection}
  \SECTION zpattern\_toolkit \parents standard\_toolkit
\end{zsection}

This toolkit section defines any operators or functions used in the
provisos of rules.  At the moment, it just defines the \LaTeX\ markup
of the symbols used in the rule syntax.

\begin{verbatim}
%%Zinchar \proviso U+25B8
%%Zinchar \derives U+2500
\end{verbatim}

To clearly distinguish our partially unfolded schemas from ordinary
equalities, we introduce a new infix operator called $\unfoldsTo$.  We
do not really need to define its semantics in order to use it within
rules, but to reassure readers, we define its semantics to be just
equality.  In fact, the intention is that the right hand argument
($s2$) will be the unfolded form of the left hand argument ($s1$).

\begin{verbatim}
%%Zinword \unfoldsTo unfoldsTo
\end{verbatim}

\begin{zed}
  \relation ( \_ \unfoldsTo \_ )
\end{zed}

We also define some syntax for use within the provisos that manipulate
schemas and signatures.  The $\schemamerge$ operator calculates
the union of two DeclLists. 

$[D1] \schemaminus [D2]$ returns the declarations that are in
$D1$ but not in $D2$.  It has the precondition that $D1$ and $D2$
must be in normalized form (base types, VarDecls only with one 
name per VarDecl).

\begin{verbatim}
%%Zinword \schemamerge schemamerge
%%Zinword \schemaminus schemaminus
\end{verbatim}

\begin{zed}
  \function 30 \leftassoc ( \_ \schemamerge \_ ) \\
  \function 30 \leftassoc ( \_ \schemaminus \_ )
\end{zed}

\begin{gendef}[SCHEMA]
  \_ \unfoldsTo \_ : SCHEMA \rel SCHEMA
\where
  \forall s1,s2:SCHEMA @ s1 \unfoldsTo s2 \iff s1=s2
\end{gendef}


This definition captures part of the semantics of
the $\schemaminus$ operator and gives it the correct
type.  Note that this definition will create an $\exists$
within the predicate of SIG3, but that exists will be
trivially true if we assume that all base types are non-empty. 

TODO: add this predicate: $SIG3 = \exists SIG2 @ SIG1$.

\begin{gendef}[SIG1,SIG2,SIG3]
  \_ \schemaminus \_ : SIG1 \cross SIG2 \fun SIG3
\end{gendef}
