\documentclass{article}
\usepackage{ltcadiz}
\begin{document}

\section{Introduction}

The prelude is a Z section.
It is an implicit parent of every other section.
It assists in defining the meaning of number literal expressions
(see 12.2.6.9),
and the list arguments of operators
(see 12.2.12).
via syntactic transformation rules later in this International Standard.
The prelude is presented here using the mathematical lexis.

\section{Formal definition of prelude}

\begin{zsection}
\SECTION prelude
\end{zsection}

The section is called $prelude$ and has no parents.

% Mark-up directives for core language symbols are as follows.
% 
%%Zchar	\{ U+007B
%%Zchar	\} U+007D
%%Zinchar \where U+007C
%%Zchar \Delta U+0394
%%Zchar \Xi U+039E
%%Zprechar \theta U+03B8
%%Zprechar \lambda U+03BB
%%Zprechar \mu U+03BC
%%Zchar \ldata U+300A
%%Zchar \rdata U+300B
%%Zchar \lblot U+2989
%%Zchar \rblot U+298A
%%Zchar \vdash U+22A2
%%Zinchar \land U+2227
%%Zinchar \lor U+2228
%%Zinchar \implies U+21D2
%%Zinchar \iff U+21D4
%%Zprechar \lnot U+00AC
%%Zprechar \forall U+2200
%%Zprechar \exists U+2203
%%Zinchar \in U+2208
%%Zinchar \spot U+2981
%%Zinchar \hide U+29F9
%%Zinchar \project U+2A21
%%Zinchar \semi U+2A1F
%%Zinchar \pipe U+2A20
%%Zpreword \IF if
%%Zinword \THEN then
%%Zinword \ELSE else
%%Zpreword \LET let
%%Zpreword \pre pre
%%Zinword \function function
%%Zinword \generic generic
%%Zinword \relation relation
%%Zinword \leftassoc leftassoc
%%Zinword \rightassoc rightassoc
%%Zinword \listarg {,}{,}
%%Zinword \varg \_
%%Zprechar \power U+2119
%%Zinchar \cross U+00D7
%%Zchar \arithmos U-0001D538
%%Zchar \nat U+2115

% Mark-up directives for the rest of the Greek alphabet are as follows.
% 
%%Zchar \alpha U+03B1
%%Zchar \beta U+03B2
%%Zchar \gamma U+03B3
%%Zchar \delta U+03B4
%%Zchar \epsilon U+03B5
%%Zchar \zeta U+03B6
%%Zchar \eta U+03B7
%%Zchar \iota U+03B9
%%Zchar \kappa U+03BA
%%Zchar \nu U+03BD
%%Zchar \xi U+03BE
%%Zchar \pi U+03C0
%%Zchar \rho U+03C1
%%Zchar \sigma U+03C3
%%Zchar \tau U+03C4
%%Zchar \upsilon U+03C5
%%Zchar \phi U+03C6
%%Zchar \chi U+03C7
%%Zchar \psi U+03C8
%%Zchar \omega U+03C9
%%Zchar \Gamma U+0393
%%Zchar \Theta U+0398
%%Zchar \Lambda U+039B
%%Zchar \Pi U+03A0
%%Zchar \Sigma U+03A3
%%Zchar \Upsilon U+03A5
%%Zchar \Phi U+03A6
%%Zchar \Psi U+03A8
%%Zchar \Omega U+03A9

NOTE 1
\begin{zed}
\generic (\power \_)
\end{zed}
$\power$ has already been introduced in this International Standard
(see 8.3),
so this operator template is not necessary.
However, it may be a convenient way of introducing to a tool
the association of $\power$ with the appropriate token and precedence,
especially in preparation for the toolkit's $\power_1$ (see B.3.6).
A tool may introduce $\cross$ here similarly,
that being the only other Z core notation whose precedence
lies amongst those of user-defined operators.

%\begin{zed}
%\generic 8 \rightassoc (\_ \cross \_)
%\end{zed}

\begin{zed}
[\arithmos]
\end{zed}

The given type $\arithmos$, pronounced ``arithmos'', provides
a supply of values for use in specifying number systems.

\begin{axdef}
\nat : \power \arithmos
\end{axdef}

The set of natural numbers, $\nat$, is a subset of $\arithmos$.

%the original definition doesn't seem right... should be an axdef?
%Changed by Tim Miller
%\begin{zed}
%number_literal_0: \nat
%number_literal_1: \nat
%\end{zed}

\begin{axdef}
  number\_literal\_0 : \nat\\
  number\_literal\_1 : \nat
\end{axdef}

$0$ and $1$ are natural numbers, all uses of which are transformed
to references to these declarations (see 12.2.6.9).

\begin{zed}
\function 30 \leftassoc (\_ + \_)
\end{zed}

\begin{axdef}
\_ + \_ : \power ((\arithmos \cross \arithmos) \cross \arithmos)
\where
\forall m, n: \nat @ \exists_1 p: ( \_ + \_ ) @ p.1 = (m, n)\\
\forall m, n: \nat @ m + n \in \nat\\
\forall m, n: \nat | m + 1 = n + 1 @ m = n\\
\forall n: \nat @ \lnot n + 1 = 0\\
\forall w: \power \nat | 0 \in w \land (\forall y: w @ y + 1 \in w) @ w = \nat\\
\forall m: \nat @ m + 0 = m\\
\forall m, n: \nat @ m + (n + 1) = (m + n) + 1
\end{axdef}

Addition is defined here for natural numbers.
(It is extended to integers in the mathematical toolkit,
annex~B.)
Addition is a function.
The sum of two natural numbers is a natural number.
The operation of adding 1 is an injection on natural numbers,
and produces a result different from 0.
There is an induction constraint that all natural numbers are
either 0 or are formed by adding 1 to another natural number.
0 is an identity of addition.
Addition is associative.

%NOTE 2~~The definition of addition is equivalent to the following definition,
%which is written using notation from the mathematical toolkit
%(and so is unsuitable as the normative definition here).
%\begin{axdef}
%\_~+~\_ : \arithmos \cross \arithmos \rel \arithmos
%\where
%(\nat \cross \nat) \dres (\_~+~\_) \in (\nat \cross \nat) \fun \nat\\
%\lambda n : \nat @ n + 1 \in \nat \inj \nat\\
%disjoint \langle\{0\}, \{n : \nat @ n+1\}\rangle\\
%\forall w : \power \nat | \{0\} \cup \{n:w @ n+1\} \subseteq w @ w = \nat\\
%\forall m : \nat @ m + 0 = m\\
%\forall m, n : \nat @ m + (n + 1) = m + n + 1\\
%\end{axdef}

\end{document}
