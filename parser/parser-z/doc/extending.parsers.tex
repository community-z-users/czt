\documentclass{article}

\usepackage{vmargin}

\setpapersize{A4}
%\setmarginsrb{leftmargin}{topmargin}{rightmargin}{bottommargin}{headheight}{headsep}{footheight}{footskip}
%\setmarginsrb{20mm}{10mm}{20mm}{10mm}{12pt}{11mm}{0pt}{11mm}
%\setmarginsrb{25mm}{20mm}{25mm}{20mm}{12pt}{11mm}{0pt}{10mm}
\setmarginsrb{40mm}{20mm}{40mm}{20mm}{12pt}{11mm}{0pt}{10mm}

\newcommand{\Circus}{\textsf{\textit{Circus}}}

\begin{document}

\title{Extending CZT parsers}
\author{Leo Freitas}
\date{March 2006, January 2013}

\maketitle

\begin{abstract}
    \noindent This article documents the process of extending the CZT parsing architecture
    for a new Z extension. In particular, it uses \Circus\ and Circus time as target language
    examples.
\end{abstract}

\section{Introduction}\label{introduction}

In this document we define the general guidelines for extending the CZT parsing
architecture. It includes various parsers, scanners, and pretty printers. The
architecture allows one to use Apache ANT to generate the Java, CUP, and JFlex
source files related to theses tools from various XML schema files containing
the appropriate markup for each different language. Thus, we allow some reuse
within the rather usually monolithic structure of parsing tools.

The main tools generated are usually related to \LaTeX\ and Unicode. At the
moment we have  extensions for Z, Object-Z and TCOZ, Z/EVES, \Circus, and Circus time. 
New languages can be added as needed. For instance, in this document we use \Circus\ as the new
language we are extending the CZT parsing architecture for. Furthermore, we
assume the paths given are relative to the value of the \texttt{CZT\_HOME}
environment variable.

Although the modifications for different extensions spread across different
files and directories, it is left to an irreducible minimum in order to allow
extension with least effort possible. This is because parsers and scanner code
are very difficult to extend through inheritance due to the very nature of its
algorithm. Thus, our compromise here for reusability is using XML templates as
an intermediate medium between the various versions. That is, from the XML
templates, one is able to generate a set of different parsers and scanners,
hence leaving room for better maintenance and extensibility.

\subsection{Dependencies}

The parsing architecture depends on other projects like \texttt{corejava}, and
\texttt{session}. It's important to understand how these dependencies work, and
the best way to do that is to look at the Maven dependency graphs of each project
involved. Therefore, to extend a parser one should be aware of the
dependencies and documentation of those two packages. Moreover, the JWSP files
might be necessary if one want to read/write XML files using JAXB.

During the execution of a parsing session, some warning or information messages
about \texttt{clover.jar} or typechecking might appear. These can be ignored
and have to do with CZT testing architecture, and session management
typechecking configuration respectively.

\section{Defining Unicode Characters}

+ devtools/circus2charclass.xsl!

+ talk about the 13 layered scanners!

+ talk about ContextFreeScanner + Latex2Unicode!

\section{Defining keywords}\label{defining-keywords}

To instruct the parsers to recognise keywords of a language as special tokens,
one need to edit the following files:

\begin{enumerate}
    \item \texttt{./parser/lib/circus\_toolkit.tex}

        The \texttt{parser/lib} directory contains the toolkits for the various
        languages. Even if the target language do not have particular
        definitions, the keywords must appear somewhere for the parser/scanners
        to be aware of it.

        It does not need to be in the toolkit file necessarily, neither on this directory.
        Nevertheless, if one puts it somewhere else, the \texttt{czt.path}
        environment variable must be to the appropriate directory where the file
        containing the keywords lie. It can be either a Unicode
        (\texttt{.utf8}, or \texttt{.utf16}), or a \LaTeX\ file.

    \item \texttt{./devtools/circuschar.xml} \\
          \texttt{./corejava/xsl/circuschar2stringclass.xsl}

        These files are the link between Unicode characters and strings into Java.
        The char file contains Unicode characters, whereas the string file
        contains Unicode strings. Usually, keywords are to be found here.

    \item \texttt{./parser/templates/KeywordScanner.xml}
        This XML template is the place to define the keywords for the various
        scanners, as well as to link them to the underlying parsers.

    \item \texttt{./parser/templates/Parser.xml}
        This XML template is where the language grammar rules are defined using
        the CUP parser generator.

        The Java variable name of the string constant defined in the
        \texttt{} file \textbf{*must*} be the same as the CUP terminal name used
        for this special token. This establishes a programmable link between the
        CUP parsing and other Java files and scanners.

\end{enumerate}

For example, to define the $\mathbf{channel}$ keyword for declaration of
channels in \Circus\, one needs to update these files as follows:

\begin{enumerate}
    \item \texttt{./parser/lib/circus\_toolkit.tex}

      \hspace{5pt}\texttt{\%\%Zpreword $\backslash$circchannel channel}


    \item \texttt{./corejava/xsl/circuschar2stringclass.xsl}

      \hspace{5pt}\texttt{String CIRCCHAN = ``channel''}

    \item \texttt{./parser/templates/KeywordScanner.xml}

      \hspace{5pt}\texttt{<add:circus>}

      \hspace{5pt}\texttt{import net.sourceforge.czt.circus.util.CircusString;}

      \hspace{5pt}\texttt{</add:circus>}

      \hspace{5pt}$\vdots$

      \hspace{5pt}\texttt{public class KeywordScanner implements Scanner \{}

      \hspace{5pt}$\vdots$

        \hspace{15pt}\texttt{private Map<String, Integer> createKeywordMap() \{}

          \hspace{25pt}\texttt{Map<String, Integer> result = new HashMap<String, Integer>();}

          \hspace{25pt}$\vdots$

          \hspace{25pt}\texttt{<add:circus>}

          \hspace{25pt}\texttt{//Maps the Java Unicode string with the Parser symbols}

          \hspace{25pt}\texttt{result.put(CircusString.CIRCCHAN, new Integer(Sym.CIRCCHAN));}

          \hspace{25pt}\texttt{</add:circus>}

          \hspace{25pt}$\vdots$

        \hspace{15pt}\texttt{\}}

      \hspace{5pt}\texttt{\}}

    \item \texttt{./parser/templates/Parser.xml}

      \hspace{5pt}\texttt{//Line 1055 (v1.214)}
      \hspace{5pt}\texttt{scan with \{: return local\_next\_token(); :\};}
      \hspace{5pt}\texttt{<add:circus>}

        \hspace{10pt}\texttt{terminal CIRCCHAN;}

      \hspace{5pt}\texttt{</add:circus>}

\end{enumerate}

Firstly, in \texttt{circus\_toolkit.tex} we define the mapping between \LaTeX\
and Unicode representation. As in this example, they do not need to be the
same. Moreover, we also define which markup directive is to be used for this
keyword, which in this case is a \textbf{pre}fix key\textbf{word}
(\textit{i.e.}, \texttt{\%\%Zpreword}). Next, in
\texttt{circuschar2stringclass.xsl} we define Unicode representation for the
\texttt{CIRCCHAN} terminal token for the CUP file generated from
\texttt{Parser.xml}. Finally, in \texttt{KeywordScanner.xml} we define the last
link between Unicode to \LaTeX\ for the \texttt{$\backslash$circchannel}
\LaTeX\ command as just the \texttt{``channel''} Unicode string.


\section{Setting a debugging environment}\label{setting-debug-environment}



\section{Conclusion}\label{conclusion}

+ Simplifies the designs and extension.

+ Demands a series of well-documented steps.


\end{document}


+ conflicting tokens!
