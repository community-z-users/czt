As described in the ISO Z standard (Note 4, Section 8.4, p37), the Z
grammar has several ambiguities.  The expression ``i [``
can either turn out to be a generic instantitation:

\begin{zed}
  c1 == i [e1]
\end{zed}

Alternatively, it can turn out to be an application to a schema construction:

\begin{zed}
  c2 == i [e1; e2]
\end{zed}

Here are some more examples:

\begin{zed}
  c3 == x [a, b, c]
\end{zed}

should be a generic instantiation of x.

\begin{zed}
  c4 == x [ a;b;c ]
\end{zed}

and

\begin{zed}
  c5 == x [ a;b;c | true]
\end{zed}

\begin{zed}
  c6 == x [ a;b;c : \nat]
\end{zed}

should be function applications of function x to a set of bindings (a schema).

In the next test, the first should be a generic instantiation, while the 
rest should be a function application.
\begin{zed}
  c7 == x [ a[\nat]]\\
  c8 == x [ a[\nat] | true]\\
  c9 == x [ a[\nat]; b[\nat]]\\
  c10 == x [ a == \nat]
\end{zed}

Here are some examples provided by Ian Toyn.  The first one is an
instantiation with \mu expression.  The second and third are
applications to schemas formed from sets of bindings.

\begin{zed}
  c11 == \emptyset [ \mu a == 42 @ \{ a \} ] \\
  c12 == \emptyset [ \{ \lblot x == 42 \rblot \} ;
                     \{ \lblot x == 42 \rblot \} ] \\
  c13 == \emptyset [ \{ ( \mu a == \lblot x == 42 \rblot ) \} ;
                     \{ \lblot x == 42 \rblot \} ]
\end{zed}
