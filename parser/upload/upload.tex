\documentclass{article}
\usepackage{oz}   % oz or z-eves or fuzz styles
\newenvironment{machine}[1]{
    \begin{tabular}{@{\qquad}l}\textbf{\kern-1em machine}\ #1\\ }{
    \\ \textbf{\kern-1em end} \end{tabular} }
\newcommand{\machineInit}{\\ \textbf{\kern-1em init} \\}
\newcommand{\machineOps}{\\ \textbf{\kern-1em ops} \\}

\begin{document}
This is the BirthdayBook specification, from 
Spivey~\cite{spivey:z-notation2}.  We extend it slightly
by adding an extra operation, $RemindOne$, that is non-deterministic.

\begin{zed}
   NAME == \{1,2,3,4,5\}
\end{zed}
\begin{zed}
   DATE == \{10,11,12\}
\end{zed}

The $BirthdayBook$ schema defines the \emph{state space} of 
the birthday book system. 

% TODO: fix the parser to allow birthday : NAME \pfun DATE
\begin{schema}{BirthdayBook}
    known,a',b: \power NAME \\
    birthday: \power (NAME \cross DATE)  
\end{schema}


\bibliography{spec}
\begin{thebibliography}{1}
\bibitem{spivey:z-notation2}
J.~Michael Spivey.
\newblock {\em The Z Notation: A Reference Manual}.
\newblock International Series in Computer Science. Prentice-Hall International
  (UK) Ltd, second edition, 1992.
\end{thebibliography}
\end{document}
