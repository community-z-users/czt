\documentclass{article}
\usepackage{circus}

% To avoid formulas in $ $ to be broken across lines
\relpenalty=10000 \binoppenalty=10000

\title{A complete \Circus\ example of ATMs using bags}
\author{Ana Cavalcanti
        and
        Marie-Claude Gaudel}

\date{23 April 2009}

\begin{document}

\maketitle

A \Circus\ model is a sequence of paragraphs, just like in Z, but they can also declare channels
and processes. We initialise the specification by giving it a name and what other specifications
does it depend on.
\begin{zsection}
  \SECTION atm\_bags \parents zeves\_bag\_toolkit
\end{zsection}

In our example, we first have a paragraph that declares the sets
$CARD$ and $PIN$ of valid cards and pin numbers.  We use the Z
notation for introducing given sets.
\begin{zed}
  [CARD, PIN]
\end{zed}
We next declare some channels.  Requests for money are accepted by
the cash machine through the channel $incard$, which also takes a pin
number and an amount to be withdrawn:~the inputs are triples.
\begin{circus}
  \circchannel\ incard: CARD \cross PIN \cross \nat_1
\end{circus}
The amount is a positive natural number.  Cards are returned through
a channel $outcard$, unless there is a problem with the card and it
is retained.
\begin{circus}
  \circchannel\ outcard: CARD
\end{circus}
The notes kept in and dispensed by the cash machine are those whose
denominations are in the set $Note$ defined below (in the standard Z
notation).
\begin{zed}
  Note == \{10,20,50\}
\end{zed}
For simplicity, we consider just a few notes, and do not address the
fact that the amount requested must be decomposable in terms of the
notes available.  If it is not, the machine fails to dispense the
cash. In our model, cash is represented as a bag~\cite{Spi92} of
notes:~elements of the the set $Note$.
\begin{zed}
  Cash == \bag Note
\end{zed}
If there is enough money in the machine and a way of providing the
requested amount, the cash is output through a channel $cash$.
\begin{circus}
  \circchannel\ cash: Cash
\end{circus}
The cash machine has two main components:~a card verifier, which
accepts requests and decides whether the card should be returned and
the cash dispensed, and a cash controller, which dispenses the cash
if possible, and refills the note bank. These components interact
through the channels below.
\begin{circus}
  \circchannel\ disp: \nat; ok
\end{circus}
The channel $disp$ is used by the card verifier to tell the cash
controller to dispense a given amount, and $ok$ is used by the cash
controller to tell the card verifier that it has concluded its
operation. The channel $ok$ does not have a type; it is not used to
communicate values, but just for synchronisation.  This is also the
case of the channel $refill$ defined below.
\begin{circus}
  \circchannel\ refill
\end{circus}
This channel is used to accept requests to refill the machine

A \Circus\ process models a system or a component.  Just like in CSP,
it interacts with its environment and other processes via channels.
In \Circus, however, a process encapsulates a state defined just like
in Z.  Our model defines the process $CashMachine$; its only state
component is a function $noteBank$ that records, for each
denomination, the amount of notes available.
\begin{circus}
  \circprocess\ CashMachine \circdef \circbegin
\end{circus}
The state is defined by a schema, namely $CMState$, which declares
$noteBank$ as a total function.  In this example, we do not have an
elaborate state invariant (which is restricted to the functional
property of $noteBank$).
\begin{circusaction}
  \circstate\ CMState == [~noteBank: Note \fun \nat~]
\end{circusaction}

Inductive definition of summation for bags of integers,
could just change $\num$ to $Cash$ if want to make it specific.
\begin{axdef}
   \Sigma: \bag~\num \fun \num
\where
   \Sigma~\lbag \rbag = 0
   \\
   \forall x: \num @ \Sigma~(\lbag x \rbag) = (\lbag x \rbag \bcount x)
   \\
   \forall b, c: \bag~\num @ \Sigma~(b \uplus c) = \Sigma~b + \Sigma~c
\end{axdef}
%
\begin{axdef}
   LSigma: \bag~\num \fun \num
\where
   LSigma~\lbag \rbag = 0
   \\
   \forall x: \num; b: \bag~\num @ \Sigma~(\lbag x \rbag \uplus b) = (\lbag x \rbag \bcount x) + LSigma~b
\end{axdef}


The $DispenseNotes$ operation that dispenses notes takes an amount
$a?$ as input and produces the bag of notes $notes!$ as output; it
updates the $noteBank$ accordingly.  It is defined using the Z
notation:~a schema that defines a relation on $CMState$. The
declaration $\Delta CMState$ introduces the variables $noteBank$ to
represent the value of the state component before the operation, and
$noteBank'$ to represent its value after the operation.
\begin{schema}{DispenseNotes}
  \Delta CMState
  \\ %
  a?: \nat
  \\ %
  notes!: Cash
\where
  \Sigma~notes! = a?
  \\ %
  \forall n: Note @ (notes!~\bcount~n) \leq noteBank~n
  \\ %
  \t1 \land %<-- missing conjunction for \forall's scope for n
  noteBank'~n = (noteBank~n) - (notes!~\bcount~n)
\end{schema}
The value of $notes!$ is nondeterministically chosen:~it is any bag
$notes!$ whose sum $\Sigma~notes!$ of its elements is equal to $a?$,
and such that, for each note denomination $n$, the number
$notes!~\bcount~n$ of occurrences of $n$ is less than or equal to the
number of notes of denomination $n$ in the bank.

If there is no such bag, we have an error:~the output is the empty
bag, and the state is not changed.  This scenario is defined by the
schema $DispenseError$ below; it includes $\Xi CMState$ to declare
implicitly the variables $noteBank$ and $noteBank'$ and define~(also
implicitly) that their values are equal.
\begin{schema}{DispenseError}
  \Xi CMState
  \\ %
  a?: \nat
  \\ %
  notes!: Cash
\where
  \lnot \exists ns: Cash @
  % \\ %<-- new line will break the scope without \land
  %%\quad %<-- \also or \Also for vertical space; \t1, ..., \t9 for horizontal
  \Sigma~ns = a? \land
  %<-- this doesn't typecheck. \bcount expects a  "bag \bcount bag-type"
  %    it was given bag1 \bcount bag2. Is n to be of type "cash" or "Note"?
  \forall n: Note @ (ns~\bcount~n) \leq noteBank~n
  \\ % <-- \exists scope ends at last "n" above
  notes! = \lbag ~\rbag
\end{schema}
The total operation to $Dispense$ cash is the schema disjunction of
the operations $DispenseNotes$ and $DispenseError$.
\begin{zed}
  Dispense == DispenseNotes \lor DispenseError
\end{zed}
For conciseness, we omit the definition of the operation $\Sigma$ for
bags.

The function $pin$ defines the pin number of each valid card.  It is
declared using a Z axiomatic description, but its scope is restricted
to the process.
\begin{axdef}
  pin: CARD \fun PIN
\end{axdef}
For simplicity, we assume that the pin numbers are constant.

The first component of the cash machine is the card verifier, which
is defined below using CSP notation.  It first accepts a request
$incard?c.(pin~c)?a$; this is an input of any card $c$, the
particular pin number $pin~c$, and any amount $a$. It then decides
whether to retain the card, output the card using the channel
$outcard$ and no money, or ask the cash controller to dispense the
requested amount. The decision is entirely nondeterministic:~it is
defined by factors outside of this model:~status of the card, balance
on the corresponding account, and so on.  If the card verifier
decides to ask for the cash to be dispensed, then it waits for an
$ok$ from the controller to indicate that it is finished (and the
verifier can proceed recursively to accept a new request).
\begin{circusaction}
  CardV \circdef
  \circblockbegin
  incard?c.(pin~c)?a \then
            \\ %
      \t1 \circblockbegin
          (CardV
          \\
          \intchoice
          \\ %
          outcard.c \then CardV
          \\ %
          \intchoice
          \\ %
          disp!a \then ok \then outcard.c \then CardV)
      \circblockend
  \circblockend
\end{circusaction}
Nondeterminism here is in the pattern of interaction, and it is
explicitly indicated using the CSP construct $\intchoice$ for
internal choice.

The cash controller $CashC$ offers the choice to $refill$ the bank or
$disp$ense some money.  For simplicity, we assume that when the
machine is refilled, it then has $cap$ notes of each denomination.
\begin{axdef}
  cap: \nat
\end{axdef}
This is a constant that reflects the capacity of the cash machine.

In the definition of $CashC$, we use an assignment to $noteBank$,
instead of a data operation defined by a Z schema, to define the
value of the state after a synchronisation on $refill$.  This
illustrates the possibility of use of programming constructs as well
as abstract specifications in \Circus.  In particular, it is possible
to define an executable \Circus\ model.

If the cash controller receives a request $disp?a$ to dispense an
amount $a$ of cash, it uses the operation $Dispense$ defined
previously to determine how cash is to be dispensed.  To use that
operation, $CashC$ declares a local variable $notes$. The input
variable $a$ and the local variable $notes$ now in scope are
associated with the input and output of $Dispense$, which then
assigns an appropriate value to $notes$. If that value is not the
empty bag, then the cash is dispensed using the channel $cash$, and
then the message $ok$ is sent (to $CardV$). If, on the other hand,
the bag is empty, then it is not possible to output the amount of
cash requested and the $ok$ message is sent directly.
\begin{circusaction}
  CashC \circdef
  \\ \t1
  \circblockbegin
      refill \then (noteBank := \{~10 \mapsto cap, 20 \mapsto cap, 50 \mapsto cap~\} \circseq CashC)
      \\ %
      \extchoice
      \\ %
      disp?a \then
        \circblockbegin   % <-- haven't added the new line, so block will be aligned with the last \then
            \circvar\ notes: Cash \circspot
                \\ \t1 %
                \circblockbegin
                     \lschexpract Dispense \rschexpract \circseq~{}
                     \also
                     \circblockbegin
                        \lcircguard
                         notes \neq \lbag\rbag
                         \rcircguard
                            \circguard\ cash!notes \then ok \then CashC
                         \\ %
                         \extchoice
                         \\ %
                         \lcircguard notes = \lbag\rbag \rcircguard
                             \circguard\ ok \then CashC
                     \circblockend \\
                \circblockend \\
        \circblockend \\
   \circblockend
\end{circusaction}
This illustrates the free combination of specification and
programming constructs, and the free combination of data operations
and communications. There is no direct association between
interactions and state changes.

So far, we defined just components of (the model of) $CashMachine$.
In particular, the schemas $DispenseNotes$, $DispenseError$, and
$Dispense$, which are data operations, and $CardV$ and $CashC$ are
actions of $CashMachine$.  As we have seen, they have access to the
state of the process, and are defined using a combination of Z, CSP,
and guarded command constructs.  They are used in the specification
below of a main~(nameless) action that defines the behaviour of
$CashMachine$. This is a parallel composition of the $CardV$ and
$CashC$ components, synchronising on the channels $disp$ and $ok$.

In \Circus, to avoid conflicts in the access to variables, a parallel
composition of actions defines the disjoint sets of variables to
which each of the parallel actions have write access. All the actions
can read the value of all the variables before the parallelism
starts, but can modify only the variables in their associated sets.
In our example, $CardV$ does not update the state, and so is
associated with the empty set $\{~\}$ of variables.  On the other
hand,  $CashC$ updates $noteBank$, and so it is associated with
\{noteBank\}.

The parallel composition also defines the channels on which
communication requires interaction from both parallel actions.  In
our example, they are $disp$ and $ok$.  This means, for example, that
$CardV$ can engage on communications using $incard$ and $outcard$
independently from $CashC$, and that $CashC$ can communicate on
$refill$ and $cash$ independently from $CardV$.  This freedom is an
extra source of nondeterminism.

The channels $disp$ and $ok$ are used only for communication between
the internal components $CardV$ and $CashC$ of $CashMachine$. Such
communications are of no interest to the user of a cash machine, and
so they are hidden.  Just like in CSP, communications on hidden
channels are not visible.
%
To conclude, the main action of $CashMachine$ is as follows.
\begin{circusaction}
  \circspot (CardV \lpar \{ \} | \lchanset disp,ok \rchanset | \{noteBank\} \rpar CashC) \circhide \lchanset disp,ok \rchanset
\end{circusaction}
\begin{circus}
    \circend
\end{circus}
The cash machine can be refilled while a request for cash is being
processed, but not when cash is being actually dispensed.

Interaction with $CashMachine$ is only possible via the channels
$incard$, $outcard$, $refill$, and $cash$, in the way defined by its
main action above.  There is no possibility of direct access to its
state, which is encapsulated.  In \Circus, we can combine basic
processes, which are defined as above using Z and CSP constructs; the
combinators are the usual CSP operators for internal and external
choice, parallelism, interleaving and so on.

\end{document}
