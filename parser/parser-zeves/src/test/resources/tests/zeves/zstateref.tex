
\begin{zsection}
   \SECTION zstateref \parents zeves\_toolkit
\end{zsection}

\begin{axdef}
   foo: \nat
\where
   foo = 1
\end{axdef}

\begin{axdef}
   bar: \nat
\where
   bar > foo
\end{axdef}


\begin{schema}{S}
  x: \nat
\where
   bar > x
\end{schema}

\begin{schema}{SInit}
   S~'
\where
   x' = 0
\end{schema}

\begin{schema}{AState}
   x: \power \nat
\where
  x \neq \emptyset
\end{schema}

\begin{schema}{ASInit}
   AState~'
\where
   x' = \{ 0 \}
\end{schema}

\begin{schema}{ASFin}
   AState
\where
   true
\end{schema}

\begin{schema}{CState}
  y: \seq \nat
\where
   y \neq \langle \rangle
\end{schema}

\begin{schema}{CSInit}
   CState~'
\where
  y' = \langle 0 \rangle
\end{schema}

\begin{schema}{CSFin}
   CState
\where
   true
\end{schema}

\begin{schema}{Retr}
   AState \\
   CState
\where
   \ran~y = x
\end{schema}

One schema with no z state info
\begin{schema}{Nothing}
   \Delta AState
\where
   true
\end{schema}

\begin{schema}{NothingAgain}
   \Delta CState
\where
   true
\end{schema}

\begin{schema}{AOp1}
   \Delta AState \\
   i?: \nat
\where
   x' = x \cup \{i?\}
\end{schema}

\begin{schema}{COp1}
   \Delta CState \\
   i?: \nat
\where
   y' = y \cat \langle i? \rangle
\end{schema}

\begin{schema}{AOp2}
   \Xi AState \\
   o!: \nat
\where
    o! \in  x' 
\end{schema}

\begin{schema}{COp2}
   \Xi CState \\
   o!: \nat
\where
   o! \in \ran~y'
\end{schema}

PS: horizontally defined schemas *cannot* have state tags. That's because
unboxed paragraphs accept multiple items and would mean more complicated lexing.

\begin{schema}{AOp1Sig}
 i? : \nat \\
 x : \power \nat 
\where
 AState
\end{schema}

\begin{schema}{COp1Sig}
 i? : \nat \\
 y : \seq \nat 
\where
 CState
\end{schema}

\begin{schema}{AOp2Sig}
 x : \power \nat 
\where
 AState
\end{schema}

\begin{schema}{COp2Sig}
 y : \seq \nat 
\where
 CState
\end{schema}

\begin{schema}{COp2DashSig}
   COp2Sig~' \\
   o! : \nat
\end{schema}

\begin{theorem}{ForwardSInitialisation} 
   \forall CSInit @ \exists Retr~' @ ASInit
\end{theorem}

\begin{theorem}{ForwardSFeasibilityAOp1} 
   \forall AOp1Sig; COp1Sig | \pre~AOp1 \land Retr @ \pre~COp1
\end{theorem}

\begin{theorem}{ForwardSCorrctnessAOp1} 
   \forall AOp1Sig; COp1Sig; CState~' | \pre~AOp1 \land Retr \land COp1 @ 
   	\exists AState~' | Retr~' @ AOp1
\end{theorem}

\begin{theorem}{BackwardSFeasibilityAOp1} 
   \forall COp1Sig | (\forall AOp1Sig | Retr @ \pre AOp1) @ \pre COp1
\end{theorem}

\begin{theorem}{BackwardSCorrectnessAOp1} 
  \forall COp1Sig | (\forall AOp1Sig | Retr @ \pre AOp1) @ 
   	(\forall AState~'; COp1 | Retr~' @ (\exists AState | Retr @ AOp1))
\end{theorem}


\begin{theorem}{ForwardSFeasibilityAOp2} 
   \forall AOp2Sig; COp2Sig | \pre~AOp2 \land Retr @ \pre~COp2
\end{theorem}

To avoid undeclared identifiers quantify over COp2 as well
\begin{theorem}{ForwardSCorrctnessAOp2} 
  \forall AOp2Sig; COp2Sig; COp2 | \pre~AOp2 \land Retr \land COp2 @ 
   	\exists AState~' | Retr~' @ AOp2
\end{theorem}

\begin{theorem}{BarwardSInitialisation} 
   \forall CSInit; Retr~' @ ASInit 
\end{theorem}

\begin{theorem}{BackwardSFeasibilityAOp2} 
   \forall COp2Sig | (\forall AOp2Sig | Retr @ \pre AOp2) @ \pre COp2
\end{theorem}

\begin{theorem}{BackwardSCorrectnessAOp2} 
  \forall COp2Sig | (\forall AOp2Sig | Retr @ \pre AOp2) @ 
   	(\forall AState~'; COp2 | Retr~' @ (\exists AState | Retr @ AOp2))
\end{theorem}