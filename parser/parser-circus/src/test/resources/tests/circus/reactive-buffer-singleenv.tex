\documentclass{article}

\usepackage{vmargin}
\usepackage{circus}

\setpapersize{A4}
%\setmarginsrb{leftmargin}{topmargin}{rightmargin}{bottommargin}{headheight}{headsep}{footheight}{footskip}
%\setmarginsrb{20mm}{10mm}{20mm}{10mm}{12pt}{11mm}{0pt}{11mm}
%\setmarginsrb{25mm}{20mm}{25mm}{20mm}{12pt}{11mm}{0pt}{10mm}
\setmarginsrb{40mm}{20mm}{40mm}{20mm}{12pt}{11mm}{0pt}{10mm}

\begin{document}

\title{\Circus\ Example --- Reactive Buffer (unboxed)}
\author{Leonardo Freitas}
\date{June 2006}

\maketitle

\begin{zsection}
  \SECTION\ unboxed\_reactive\_buffer \parents\ circus\_toolkit
\end{zsection}

\section{Abstract buffer specification}

\begin{axdef}
  maxbuff: \nat_1
\end{axdef}

\begin{circus}
   \circchannel\ input, output: \nat
\end{circus}

\begin{circus}
  \circprocess\ Buffer \circdef \circbegin
  \also %
    \t1 \circstate\ BufferState ~~==~~ [~ buff: \seq \nat; size: 0 \upto maxbuff | size = \#~buff \leq maxbuff ~] \\ %
    \t1 BufferInit ~~==~~ [~ BufferState~' | buff' = \langle\rangle \land size' = 0 ~] \\
    \t1 Input ~~\circdef~~ \lcircguard size < maxbuff \rcircguard \circguard input~?x \then InputCmd \\
    \t1 InputCmd ~~==~~ [~ \Delta BufferState; x?: \nat | size < maxbuff \land buff' = buff \cat \langle x? \rangle \land size' = size + 1 ~] \\
    \t1 OutputCmd ~~==~~ [~\Delta BufferState | size > 0 \land buff'= tail~buff \land size' = size - 1 ~] \\
    \t1 Output ~~\circdef~~ \lcircguard size > 0 \rcircguard \circguard output~!(head~buff) \then OutputCmd \\
  \circspot BufferInit \circseq\ (~\circmu X \circspot (~Input \extchoice Output~) \circseq\ X~) \\
  \circend
\end{circus}


\section{A cached-head ring buffer implementation}

\subsection{Controller process}

\begin{axdef}
  maxring: \nat
\where %
  maxring = maxbuff - 1
\end{axdef}

\begin{zed}
    RingIndex ~~==~~ 0 \upto maxring - 1
\end{zed}

\begin{circus}
   \circchannel\ read, write: RingIndex \cross \nat
\end{circus}

\begin{circus}
  \circprocess\ Controller ~~\circdef~~ \circbegin
  \also
    \t1 \circstate\ ControllerState ~~==~~ [~ size: 0 \upto maxbuff; ringsize: 0 \upto maxring; cache: \nat; top, bot: RingIndex | ringsize = max \{ 0, size - 1 \} \land ringsize \mod maxring = (top - bot) \mod maxring ~] \\
    \t1 InitController ~~==~~ [~ ControllerState~' | size' = 0 \land bot' = 0 \land top' = 0 ~] \\
    \t1 CacheInput ~~==~~ [~ \Delta ControllerState; x?: \nat | size = 0 \land size' = 1 \land cache' = x? \land bot' = bot \land top' = top~] \\
    \t1 StoreInput ~~==~~ [~ \Delta ControllerState | size > 0 \land size' = size + 1 \land cache' = cache \land bot' = bot \land top' = (top + 1) \mod maxring ~] \\
    \t1 InputController ~~\circdef~~ \lcircguard size < maxbuff \rcircguard \circguard input~?x \then (~(\lcircguard size = 0 \rcircguard \circguard CacheInput) \extchoice (\lcircguard size > 0 \rcircguard \circguard write.top~!x \then StoreInput)~) \\
    \t1 NoNewCache ~~==~~ [~ \Delta ControllerState | size = 1 \land size' = 0 \land bot' = bot \land top' = top~] \\
    \t1 StoreNewCache ~~==~ [~ \Delta ControllerState; x?: \nat | size > 1 \land size' = size - 1 \land cache' = x? bot' = (bot + 1) \mod maxring \land top' = top ~] \\
    \t1 OutputController ~~\circdef~~ \lcircguard size > 0 \rcircguard \circguard output~!cache \then (~(\lcircguard size > 1 \rcircguard \circguard read.bot~?x \then StoreNewCache) \extchoice (\lcircguard size = 1 \rcircguard \circguard NoNewCache)~) \\
    \t1 ControllerCycle ~~\circdef~~ \circmu\ X \circspot (~ InputController \extchoice OutputController ~) \circseq X
  \also %
   \circspot InitController \circseq ControllerCycle \\
  \circend
\end{circus}

\subsection{Ring cell process}

\begin{circus}
  \circchannel\ rd, wrt: \nat
\end{circus}

Because \LaTeX\ is only an intermediary representation to Unicode,
we MUST not use the \Circus\ (Unicode) keywords within a process.
So for example, $begin$, $process$, \textit{etc.} are not allowed.
Thus, we have changed the state name from $val$ (keyword for parameters
passed by value to $val\_$).
\begin{circus}
  \circprocess\ RingCell \circdef \circbegin
  \also
    \t1 \circstate\ CellState ~~==~~ [~ val\_: \nat ~] \\
    \t1 Read \circdef rd~!val\_ \then \Skip \\
    \t1 CellWrite ~~==~~ [~ \Delta CellState; x?: \nat | val\_' = x? ~] \\
    \t1 Write \circdef wrt~?x \then CellWrite \\
    \circspot Write \circseq (~\circmu\ X \circspot ( Read \extchoice Write ) \circseq\ X~) \\
  \circend
\end{circus}
%
There is no way around this. The choice is perhaps to change the keyword
if it often happens within specifications

\subsubsection{The cached-head ring buffer's behaviour}

This is an example of indexed process renaming,
on-the-fly replicated indexed process call, and
process hiding with on-the-fly basic channel set expressions.
Renaming causes a series of problems with generic actuals (i.e. may need look-ahead to differentiate
from variable substitution) and on-the-fly actions.
Once the conflicts were solved, the production in itself was not yet right, as it was not parsing the expected program.
Like in FDR, we have chosen special parenthesis to make it clear that it is a renaming. This also avoid confusion when
generic processes and renaming are used together.
%
\begin{circus}
    % (~i: RingIndex \circindex RingCell~)[rd\_i, wrt\_i := read, write]
    % i: RingIndex \circindex RingCell
  \circprocess\ IRingCell \circdef (~i: RingIndex \circindex RingCell~) \lcircrename rd\_i, wrt\_i := read, write \rcircrename
  \also
  \circprocess\ Ring \circdef (~ \Interleave i: RingIndex \circindex IRingCell ~) \lcircindex i \rcircindex
  \also
  \circprocess\ CRing \circdef (~ Controller \lpar \lchanset read, write \rchanset \rpar Ring ~) \circhide \lchanset read, write \rchanset
\end{circus}
%
Note that
\begin{circus}
  \circprocess\ IRingCellJustTesting \circdef (~i: RingIndex \circspot RingCell~) \lcircrename rd\_i, wrt\_i := read, write \rcircrename
  \also
  \circprocess\ IRingCellJustTesting \circdef (P \extchoice Q) \lcircrename c\_i := d \rcircrename

\end{circus}

\end{document}
