\documentclass{article}

%\newenvironment{theorem}{...}{...}
\usepackage{czt}
\usepackage[color]{circus}

\begin{document}

\section{Preamble}

\begin{itemize}
   \item section name and its parents
   \item basic process header
   \item typed channel
   \item generic typed channels
   \item synchronisation channels
   \item various auxiliary declarations used
\end{itemize}

\begin{zsection}
  \SECTION\ dot\_field\_multenv \parents\ circus\_toolkit
\end{zsection}

\begin{circus}
   \circchannel\ c: \nat \cross \nat \cross \nat \cross \nat
\end{circus}

\begin{circus}
   \circprocess\ DotTestMulti \circdef \circbegin
\end{circus}

\begin{axdef}
   n1, n2: \nat \\
   x?, y!, z?: \nat
\end{axdef}

\begin{axdef}
   f : \nat \fun \nat \cross \nat
\end{axdef}

\begin{zed}
   S == [~ y: \nat ~]
\end{zed}

\newpage
\section{Example 1 --- various simple patterns}

\begin{itemize}
   \item multiple actions in one environment
   \item tabs and various forms of new line
   \item multiple field patterns: in, out, dot
   \item field type directly mapped to channel type
   \item synchronisation channel $d$ on chained prefixing
\end{itemize}

\begin{circusaction}
   \t1 Test0 \circdef c?x?y?z \then \Skip
   \also
   \t1 Test1 \circdef c?x!n1.n2 \then \Skip \\
   \t1 Test2 \circdef c!n1?x.n2 \then \Skip
   \also
   \t1 Test3 \circdef c.n1!n2?x \then d \Skip
\end{circusaction}

\begin{circusaction}
   \t1 Test4 \circdef d \then e \then \Skip
\end{circusaction}

\paragraph{Description \\ \\}

\vspace{2pt}
\begin{tabular}{l|l}
   \hline
   Action   & Communication pattern \\
   \hline
   $Test0$  & In($x, \nat$), In($y, \nat$), In($z, \nat$) \\
   \hline
   $Test1$  & In($x, \nat$), Out($n1$), Dot($n2$) \\
   \hline
   $Test2$  & Out($n1$), In($x, \nat$), Dot($n2$) \\
   \hline
   $Test3$  & Dot($n1$), Out($n2$), In($x, \nat$), Synch \\
   \hline
   $Test4$  & Synch, Synch, Synch \\
   \hline
\end{tabular}

\paragraph{\LaTeX\\ \\}

\noindent To avoid parsing the \LaTeX\ markup within the \verb'\begin{verbatim}'
environment we omit the slash before the begin/end environment.

\begin{verbatim}
  begin{circusaction}
   \t1 Test0 \circdef c?x?y?z \then \Skip
       \also
   \t1 Test1 \circdef c?x!n1.n2 \then \Skip \\
   \t1 Test2 \circdef c!n1?x.n2 \then \Skip
       \also
   \t1 Test3 \circdef c.n1!n2?x \then d \Skip
  end{circusaction}
  begin{circusaction}
   \t1 Test4 \circdef d \then e \then \Skip
  end{circusaction}
\end{verbatim}

\newpage
\section{Example 2 --- complex output expressions}

\begin{itemize}
   \item hard spaces - make no semantic difference
   \item application expressions on output fields
   \item parenthesised expressions form \textbf{one} field
   \item last fields get remainder type dimensions
   \item trickery to allow strokes on field expr --- \textbf{mandatory} parenthesis
   \item schema binding selection as output $(S.y$) --- \textbf{mandatory} parenthesis
   \item function application result as output --- \textbf{mandatory} parenthesis
\end{itemize}

\begin{circusaction}
   \t1 Test5 \circdef c~?i~!(f~i) \then \Skip
       \also
   \t1 Test6 \circdef c.(S.y)?z \then \Skip
       \also
   \t1 Test7 \circdef c.(x?)!(y!)!(z?) \then \Skip
\end{circusaction}

\paragraph{Description \\ \\}

\vspace{2pt}
\begin{tabular}{l|l}
   \hline
   Action   & Communication pattern \\
   \hline
   $Test5$  & In($i, \nat$), Out($\nat \cross \nat$) \\
   \hline
   $Test6$  & Dot($S.y$), In($z, \nat \cross \nat$) \\
   \hline
   $Test4$  & Out($x?$), Out($y!$), Out($z?$) \\
   \hline
\end{tabular}
\\

\textbf{* $S \in \power~(\lblot y == \nat \rblot)$, hence $S.y \in \nat$.}
\\
\indent \textbf{* $x?, y!, z?$ are decorated names; usually they appear in schemas.}

\paragraph{\LaTeX\\ \\}

\begin{verbatim}
  begin{circusaction}
   \t1 Test5 \circdef c~?i~!(f~i) \then \Skip
       \also
   \t1 Test6 \circdef c.(x.y)?z!w \then \Skip
       \also
   \t1 Test7 \circdef c.(x?)!(y!)!(z?) \then \Skip
  end{circusaction}
\end{verbatim}

\newpage
\section{Example 3 --- complex input with restrictions}

\begin{itemize}
   \item input prefix restrictions --- \textbf{mandatory} parenthesis
   \item prefix restrictions and complex expressions
   \item fields depending on previous value input
   \item chained expressions depending on previous input
   \item tuple selection within field restriction and output
   \item action broken across multiple lines
\end{itemize}

\begin{circusaction}
   \t1 Test8 \circdef c?x \prefixcolon (x > 1)!(f~x) \then \Skip
    \also
   \t1 Test9 \circdef c?x \then \\
                  \t4 c?z \prefixcolon (z > x.1).(f~(x.2+x.3)) \then \Skip
\end{circusaction}

\paragraph{Description \\ \\}

\vspace{2pt}
\begin{tabular}{l|l}
   \hline
   Action   & Communication pattern \\
   \hline
   $Test8$  & In($x, \{~ v: \nat | v > 1 ~\}$), Out($\nat \cross \nat$) \\
   \hline
   $Test9$  & In($x, \nat \cross \nat \cross \nat$);~ In($z, \{~ v: \nat | v > x.1 ~\}$), Out($\nat \cross \nat$) \\
   \hline
\end{tabular}
\\

\textbf{* type on inputs are restricted according to given predicate.}
\\
\indent \textbf{* $Test9$ input on $z$ is from ``$?z \prefixcolon (z > x.1)$''.}

\paragraph{\LaTeX\\ \\}

\begin{verbatim}
  begin{circusaction}
   \t1 Test8 \circdef c?x \prefixcolon (x > 1)!(f~x) \then \Skip
\also
   \t1 Test9 \circdef c?x \then \\
                  \t4 c?z \prefixcolon (z > x.1).(f~(x.2+x.3)) \then \Skip
  end{circusaction}
\end{verbatim}

\newpage
\section{Example 4 --- generic channels}

\begin{itemize}
   \item explicitly given generic actuals
   \item implicitly inferred generic actuals (?)
\end{itemize}

\begin{circusaction}
   \t1 Test10 \circdef g[\nat \cross \nat \cross \nat]?x!n1.n2 \then \Skip
        \also
   \t1 Test10 \circdef g.n1.(f~n1) \then \Skip
\end{circusaction}

\paragraph{\LaTeX\\ \\}

\begin{verbatim}
  begin{circusaction}
   \t1 Test10 \circdef g[\nat \cross \nat \cross \nat]?x!n1.n2 \then \Skip
       \also
   \t1 Test10 \circdef g.n1.(f~n1) \then \Skip
  end{circusaction}
\end{verbatim}


\newpage
\section{($!$ Prolegomena) --- basic process footer}

It just terminates

\begin{circusaction}
   \circspot \Skip
\end{circusaction}

\begin{circus}
   \circend
\end{circus}


\end{document} 