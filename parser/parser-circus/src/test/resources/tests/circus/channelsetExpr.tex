\documentclass{article}

\usepackage{vmargin}
\usepackage{circus}

\setpapersize{A4}
\setmarginsrb{40mm}{20mm}{40mm}{20mm}{12pt}{11mm}{0pt}{10mm}

\begin{document}

\title{\Circus\ channel set grammar extended}
\author{Leo Freitas}
\date{December 2016}

\maketitle

\section{Introduction}

\begin{zsection}
  \SECTION\ circusChannelSetExpr \parents\ circus\_toolkit
\end{zsection}

%
\begin{circus}
  \circchannel\ a, b: \nat
\end{circus}%
%
\begin{circus}
   \circchannel\ [X, Y]\ g, h, i: (X \rel Y)
\end{circus}

\begin{circus}
   \circchannel\ c, d 
\end{circus}%


% SOLVED: This is a type checking problem. Channel type inference is done locally. The parser remains as it is.
%
%\begin{issue}[type inference on generic channels usage]
%    In practice, how should generic actuals be instantiated for channels?
%    The implications are clear at parallel composition and implementation of
%    ``generically defined'' communications.
%
%    For example, in:
%
%    \[
%        \circchannel\ [X] c, d: X \\
%
%        \vdots \\
%
%        c?x \then d!x \then \Skip \lpar \lchanset c, d \rchanset \rpar c.v \then d?x \then \Skip
%    \]
%
%    Could the typechecker infer the types of both $c$ and $d$ to be the
%    type of $v$? In a multisychronisation scenario, however, this would
%    become a typechecking nightmare!
%
%    The suggestion is then to enforce generic actuals instantiation on
%    generically defined channels, so that no type inference is needed,
%    as in:
%
%    \[
%       c[\nat]?x \then d[\nat]!x \then \Skip \lpar \lchanset[\nat], d[\nat] \rchanset \rpar c[\nat].v \then d[\nat]?x \then \Skip
%    \]
%
%    This involves a clumsy and too verbose specification, but simpler
%    typechecking. What would be the compromise? Just leave it out for the
%    moment?!
%\end{issue}


% Justs ignore these for now.
%\subsection{Channels through schemas --- \grammar{channelFromDecl}}

\section{\Circus\ channel set paragraphs --- \grammar{channelSetPara}}

Channel sets defines an expression to represent a particular set of channels.
They are normally used as reference names for other \Circus\ operators that
have channel references as parts of their AST, such as the parallel and the
hiding operators.

Channel sets are given as a name and a corresponding channel set expression.
%
\begin{circus}
    \circchannelset\ C == \emptyset \\
    \circchannelset\ C0 == \lchanset \rchanset \\
    \circchannelset\ C1 == \lchanset a, b, c, d \rchanset \\
    \circchannelset\ C2 == \lchanset x, y, x1, y1 \rchanset \\
% Generic channels are not working yet
    \circchannelset\ [A, B] C3 == \lchanset x2[A \cross B], y2[A \fun B], x1[A], y1[B] \rchanset \\
    \circchannelset\ C4 == (C1 \cup C2 \cap C3) \setminus (C0 \cup C1 \cup C)
\end{circus}%

The channel set expressions mostly used are: set union, intersection and
difference expressions;~set extension expressions;~reference expressions;~and
application expressions.

Considering \code{BasicChannelSetExpr} a special case of set extension (with special brackets),
the channel set expressions are a subset of the whole Z expression tree. More precisely, table~\ref{csexpr}
shows which one of the available Z expression productions are valid channel set expressions. On the third column
of the table table:~``\textbf{Y(N)}'' means the Z expression is (is not) valid;~``\textbf{Y*}'' means the Z expression
is valid provided one changes the normal ($\{~\}$) brackets to the special ($\lchanset~\rchanset$) channel set
brackets;~``\textbf{Y!}'' means an undecided inclination towards accepting;~and~``\textbf{?}'' means yet unknown.
%
\begin{table}[h]
\begin{tabular}{|l|l|c|}
\hline
\multicolumn{1}{|c|}{\textbf{Description}} & \multicolumn{1}{c|}{\textbf{Example}} & \textbf{Valid CSE?} \\
\hline
Conditional & $\IF pred \THEN expr \ELSE expr$ & \textbf{Y} \\
\hline
Universal quantification & $\forall decl | pred @ expr$ & \textbf{N} \\
\hline
Existential quantification & $\exists decl | pred @ expr$ & \textbf{N} \\
\hline
Unique existential quant. & $\exists_1 decl | pred @ expr$ & \textbf{N} \\
\hline
Function construction & $\lambda decl | pred @ expr$ & \textbf{Y} \\
\hline
Definite description & $\mu decl | pred @ expr$ & \textbf{Y} \\
\hline
Substitution expression & $\LET abbrv @ expr$ & \textbf{Y} \\
\hline
Schema equivalence & $S \iff T$ & \textbf{N} \\
\hline
Schema implication & $S \implies T$ & \textbf{N} \\
\hline
Schema disjunction & $S \lor T$ & \textbf{N} \\
\hline
Schema conjunction & $S \land T$ & \textbf{N} \\
\hline
Schema negation & $\lnot S$ & \textbf{N} \\
\hline
Schema composition & $S \Comp T$ & \textbf{N} \\
\hline
Schema piping & $S \pipe T$ & \textbf{N} \\
\hline
Schema hiding & $S \hide T$ & \textbf{N} \\
\hline
Schema projection & $S \project T$ & \textbf{N} \\
\hline
Schema precondition & $\pre S$ & \textbf{N} \\
\hline
Powerset & $\power X$ & \textbf{?} \\
\hline
Cartesian product & $X \cross Y$ & \textbf{N} \\
\hline
Prefix application (PRE) & $f \_$ & \textbf{?} \\
\hline
Prefix application (L, ERE) & $a \_ b \_$ & \textbf{?} \\
\hline
Prefix application (L, SRE) & $a \_ ,, b \_$ & \textbf{?} \\
\hline
Postfix application (POST) & $X~\star$ & \textbf{?} \\
\hline
Postfix application (ELP, ERP) & $\_ a \_ b$ & \textbf{?} \\
\hline
Postfix application (ELP, SREP) & $\_ a \_,, b$ & \textbf{?} \\
\hline
Infix application (I) & $\_ \cup \_$ & \textbf{Y} \\
\hline
Infix application (EL, ERE) & $\_ a \_ b \_$ & \textbf{?} \\
\hline
Infix application (EL, SRE) & $\_ a \_ ,, b \_$ & \textbf{?} \\
\hline
Nofix application (L, ER) & $a \_ b$ & \textbf{?} \\
\hline
Nofix application (L, SR) & $\langle \_,,\rangle$ & \textbf{Y} \\
\hline
Set extension & $\{~a, b~\}$ & \textbf{Y*} \\
\hline
Set comprehension & $\{~ decl | pred ~\}$ & \textbf{?} \\
\hline
Characteristic set comp. & $\{~ decl | pred @ expr ~\}$ & \textbf{?} \\
\hline
Tuple extension & $(x, y, z)$ & \textbf{N} \\
\hline
Characteristic definite desc. & $(\mu decl | pred)$ & \textbf{?} \\
\hline
Binding extension & $\lblot x == 10, y == \{~\} \rblot$ & \textbf{N} \\
\hline
Empty schema construction & $[~]$ & \textbf{N} \\
\hline
Schema construction & $[ decl | pred ]$ or $[ | pred]$ & \textbf{N} \\
\hline
Binding selection & $S.x$ or $\lblot x == 10 \rblot$.x & \textbf{Y!} \\
\hline
Tuple selection & $t.1$ or $(x,y).1$ & \textbf{Y!} \\
\hline
Schema decoration & $S'$ or $S_1$ & \textbf{Y!} \\
\hline
Binding construction & $\theta S$ & \textbf{Y!} \\
\hline
Function appl with Sch Expr. & $f~[ decl | pred ]$ & \textbf{?} \\
\hline
Generic instantiation & $seq[\nat, \num]$, or $(\_ \fun \_)[X, Y]$ & \textbf{Y} \\
\hline
Schema renaming & $S[x/y, a/b]$ & \textbf{Y!} \\
\hline
Number literal & $10$ & \textbf{N} \\
\hline
Reference & $Name:  $ & \textbf{Y} \\
\hline
\end{tabular}
\caption{Filtering the Z expression tree for channel sets}
\end{table}
%
At the moment, the parser implementation only filters set extension and comprehension, hence
accepting all other Z expression productions.

\begin{issue}[Including all Z restrictions]
   Add the other restrictions from table~\ref{csexpr}
\end{issue}

% SOLVED: Generalise channel set expressions to allow any Z expression - Jim
%         Should it simplify typechecking in a great extent?
%
%\begin{issue}[generalising channel set expressions]
%  The restricted form of channel sets complicate the grammar in a great
%  extent because it demands filtering through the (very complex) expression
%  subtree, which also mixes with the predicate subtree.
%
%  If application expressions were not to be considered, the limitation would be
%  rather simple, as the other channel set expressions are easy to (hard) code. Nevertheless,
%  as one might like (or even require) to define particular functions for
%  channel sets to be used over replicated operators, application expressions may not (or cannot) be
%  avoided.
%
%  For example, this scenario might appear as a requirement when one needs to perform
%  synchronisation between disjoint set of channels amongst more than two processes,
%  in order to avoid multisynchronisation complexities.
%
%  Therefore, because of application expression, we need to include channel sets with mostly
%  all expressibility power of Z expressions. The best choice seems to reuse the
%  Z grammar for abbreviations, and encoded as an AST of \code{ChannelSet}
%  encapsulating a \code{ConstDecl}. This is useful for simplifying typechecking
%  (and perhaps other purposes).
%
%  This choice will also normalise the language with respect to generically
%  defined channels, as Z abbreviations already take care of generic formal and
%  actual parameters. \textbf{This would simplify the parser grammar as well as the AST structure in a grater
%  extend.} Furthermore, a nice side effect is the ability to have generic
%  operators as possible channel set functions!
%
%  On the other hand, if one wants to keep the restrictions, one possible
%  approach would be to leave the parser with Z abbreviations, but restrict
%  its use through the type checking of channel sets.
%\end{issue}

\begin{issue}[How are the type rules for channel set references?]
  That is, could channel set references have generic actuals, as in?
  %
  \[
    \circchannel\ [X] c, d: X \\
    \circchannel\ [Y] e, f: Y \\

    \circchannelset\ CS1 = \lchanset c, d \rchanset \\
    \circchannelset\ CS2 = \lchanset e, f \rchanset \\
    \circchannelset\ CS3 = CS1[\nat] \cup CS2[\power~\num]
  \]
  %
  This is a typechecking problem that shouldn't affect the parser, unless
  one wanted to restrict the generic instantiation as a parsing error, something
  that would require the use of \code{RefName} rather than \code{RefExpr}.

  Even further, as occurred with channel declaration from schemas, where we could
  possibly have channels as well as the schema generics, shouldn't we allow this
  for channel sets as well (\textit{i.e.}, shouldn't it has generic actions?
  %
  \[
      \circchannelset\ [A, B] CS4 = CS1[\nat \cross A] \cup CS2[A \cross B]
  \]
  %
  If this is a desired feature, the AST for ChannelSet do require modification to
  include the list of \code{DeclName} for the generic formals.
\end{issue}

\begin{issue}[decorated channel sets]
    Decorated channel sets are also valid in the parser at the moment, as it makes the
    production rules easier accepting strokes.
    %
    \[
       CS1_0  \implies  \lchanset c_0, d_0 \rchanset
    \]
    %
    This could be removed if needed, though.
    At the moment, I am assuming it as a typechecking rather than a parsing problem.
\end{issue}


\begin{issue}[Channel set expressions subtree --- restrictions]
    To simplify the implementation of channel set expressions, we simply encapsulate
    a Z expression (\grammar{expression}) filtered ``adequately'' (\textit{i.e.}, do not containing
    set extensions or comprehension) and include the production for \code{BasicChannelSetExpr}.

    From the expressive options to include Z applications, this is the simplest one. Nevertheless,
    as \code{BasicChannelSetExpr} is not embedded into the Z expression subtree, one cannot mix
    it with normal Z expressions. The consequence is a fairly uncompromising (verboseness) restriction.
    For example, something like
    %
    \[
        \circchannelset CS1 == \lchanset a, b \rchanset \\
        \circchannelset CS2 == \lchanset c \rchanset \\
        \circchannelset CS3 == CS1 \cup CS2
    \]
    %
    is valid because no Z expression is mixed with \code{BasicChannelSetExpr}, whereas something like
    %
    \[
        \circchannelset CS3 == \lchanset a, b \rchanset \cup \lchanset c \rchanset
    \]
    %
    is not valid because such mix occurs.

    One first (abandoned) alternative solution was to include the \code{BasicChannelSetExpr} production
    under a Z expression and use flags to say when to consider it or not. The consequences are neither
    elegant, nor desirable: clumsy code, difficulty in properly restricting the channel set expression
    tree properly, etc.
\end{issue}

\section{Conclusion}

\section{Future work}

The ideal solution for channel set expressions would be to have their own expression subtree.
This would be the neatest choice as it neither requires filtering on the Z expression subtree,
nor imposes the restrictions on mixed expressions just mentioned above. Nevertheless, it is
a quite hard choice as sorting the precedences and solving the conflicts of such subtree is
very difficult (see the \textsf{parser/tests/circus/cs\_expression.Parser.xml} file for details).

Another task for the near future is to sort out the JAXB problem from \code{ChannelDecl}.

\end{document}
