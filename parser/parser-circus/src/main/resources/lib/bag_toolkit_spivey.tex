\documentclass{article}

\usepackage{czt}

\begin{document}

This file contains the bag toolkit definitions as in Spivey's book

\section{Bags section}

Bags inherits from \verb'sequence\_toolkit', as it requires
both function, number, and sequence toolkits. This inclusion
is meant to be minimal.
%
\begin{zsection}
\SECTION bag\_toolkit\_spivey \parents sequence\_toolkit
\end{zsection}

\section{Bag definitions}

In what follows, we add each subsection section
according to Spivey's book, and add two further
sections for the operator templates and bag displays.

\subsection{Operator templates}

The following operator templates correspond to:
bag constructor; bag element count; bag scaling;
bag membership; sub-bag relation; bag union;
bag difference; and bag display left and right brackets.

%%Zpreword \bag bag
%%Zinchar \bcount U+266F
%%Zinchar \otimes U+2297
%%Zinchar \inbag U+22FF

% unfortunately, for Unicode, 228F is taken (\circrefines)
%%Zinword \subbageq subbageq    

%%Zinchar \uplus U+228E
%%Zinchar \uminus U+2A41

% unfortunately, for Unicode, 301A and 27E6 are taken (\linter, \lpar)
% unfortunately, for Unicode, 301B and 27E7 are taken (\rinter, \rpar)
%%Zpreword \lbag lbag         
%%Zpostword \rbag rbag        
\begin{zed}
    \generic (\bag \_) \\
    \function 50 \leftassoc (\_ \bcount \_) \\
    \function 40 \leftassoc (\_ \otimes \_) \\
    \relation (\_ \inbag \_) \\
    \relation (\_ \subbageq \_) \\
    \function 30 \leftassoc (\_ \uplus \_) \\
    \function 30 \leftassoc (\_ \uminus \_) \\
    \function (\lbag \listarg \rbag) \\
\end{zed}

\subsection{Basic definitions}

\begin{zed}
  \bag~X == X \pfun \nat_1
\end{zed}

\begin{gendef}[X]
  count: \bag X \bij (X \fun \nat) \\
  \_ \bcount \_: \bag X \cross X \fun \nat \\
  \_ \otimes \_: \nat \cross \bag X \fun \bag X
\where
  \forall B: \bag X @ count~B = (\lambda x: X @ 0) \oplus B
\also
  \forall x: X; B: \bag X @ B \bcount x = (count ~ B)~x
\also
\forall n: \nat; B: \bag X; x: X @  (n \otimes B) \bcount x = n * (B \bcount x)
\end{gendef}

Before we can define bag displays, we require the
definition of $items$, which appears later. For the
same reason, the laws about displays also appear later.

\begin{theorem}{singletonScalling}[X]
    \vdash? \forall B: \bag X @ 1 \otimes B = B
\end{theorem}
\begin{theorem}{singletonScalling2}[X]
    \vdash? \forall n, m: \nat ; B: \bag X @ \\
         \t1 (n * m) \otimes B = n \otimes (m \otimes B)
\end{theorem}

\subsection{Bag membership and subbags}

\begin{gendef}[X]
  \_ \inbag \_ : X \rel \bag X \\
  \_ \subbageq \_: \bag X \rel \bag X \\
\where
  \forall x: X; B: \bag X @ (x \inbag B \iff x \in \dom B)
\also
  \forall A, B: \bag X @ A \subbageq B \iff
	(\forall x: X @ A \bcount x \leq B \bcount x)
\end{gendef}

\begin{theorem}{bagMembership}[X]
    \vdash? \forall x: X; B: \bag X @ x \inbag B \iff B \bcount x > 0
\end{theorem}
\begin{theorem}{subBagSubsetDom}[X]
    \vdash? \forall B, C: \bag X | B \subbageq C @ \dom~B \subseteq \dom~C
\end{theorem}
\begin{theorem}{subBagReflexive}[X]
    \vdash? \forall B: \bag X @ B \subbageq B
\end{theorem}
\begin{theorem}{subBagAntiSymmetric}[X]
    \vdash? \forall B, C: \bag X | B \subbageq C \land C \subbageq B @ B = C
\end{theorem}
\begin{theorem}{subBagTransitive}[X]
    \vdash? \forall B, C, D: \bag X | B \subbageq C \land C \subbageq D @ B \subbageq D
\end{theorem}

\subsection{Bag union and difference}

\begin{gendef}[X]
  \_ \uplus \_, \_ \uminus \_:  \bag X \cross \bag X \fun \bag X
\where
  \forall A, B: \bag X; x: X @
      (A \uplus B) \bcount x = (A \bcount x) + (B \bcount x)
\also
  \forall A, B: \bag X; x: X @
     (A \uminus B) \bcount x =max \{ 0, (A \bcount x) - (B \bcount x) \}
\end{gendef}

\begin{theorem}{domBagUnion}[X]
   \vdash? \forall B, C: \bag X @ \dom~(B \uplus C) = \dom~B \cup \dom~C
\end{theorem}
\begin{theorem}{bagUnionCommutes}[X]
   \vdash? \forall B, C: \bag X @ B \uplus C = C \uplus B
\end{theorem}
\begin{theorem}{bagUnionAssociates}[X]
   \vdash? \forall B, C, D: \bag X @ (B \uplus C) \uplus D = B \uplus (C \uplus D)
\end{theorem}
\begin{theorem}{bagUnionDiffElim}[X]
   \vdash? \forall B, C: \bag X @ (B \uplus C) \uminus C = B
\end{theorem}
\begin{theorem}{bagUnionScallingProperty}[X]
   \vdash? \forall n, m: \nat; B: \bag X @ (n + m) \otimes B = n \otimes B \uplus m \otimes B
\end{theorem}
\begin{theorem}{bagDiffScallingProperty}[X]
   \vdash? \forall n, m: \nat; B: \bag X | n \geq m @ (n - m) \otimes B = n \otimes B \uminus m \otimes B
\end{theorem}
\begin{theorem}{bagUnionScallingDistributes}[X]
   \vdash? \forall n: \nat; B, C: \bag X @ n \otimes (B \uplus C) = (n \otimes B) \uplus (n \otimes C)
\end{theorem}
\begin{theorem}{bagDiffScallingDistributes}[X]
   \vdash? \forall n: \nat; B, C: \bag X @ n \otimes (B \uminus C) = (n \otimes B) \uminus (n \otimes C)
\end{theorem}

\subsubsection*{Bag items}

\begin{gendef}[X]
  items: \seq X \fun \bag X
\where
  \forall s: \seq X; x : X @ (items~s) \bcount x = \# \{ i: \dom~s | s(i) = x \}
\end{gendef}

\begin{theorem}{domBagItems}[X]
   \vdash? \forall s: \seq~X @ \dom~(items~s) = \ran~s
\end{theorem}
\begin{theorem}{itemsCatBagUplus}[X]
   \vdash? \forall s, t: \seq~X @ items~(s \cat t) = items~s \uplus items~t
\end{theorem}
\begin{theorem}{itemsEquiv}[X]
   \vdash? \forall s, t: \seq~X @ items~s = items~t \iff (\exists f: \dom~s \bij \dom~t @ s = t~ \circ ~f)
\end{theorem}

\subsection{Bag displays}

Now that we have defined $items$, we can define bag displays
in terms of it, as mentioned by Spivey at the beginning of Section 4.6
of his book.
%
\begin{zed}
    \lbag \listarg \rbag [X] == (\lambda x: \seq X @ items~x)
\end{zed}

An example of usage of bag displays for empty and singleton bags
%
\begin{theorem}{emptyBagIsEmptySet}[X]
    \vdash? \lbag \rbag = \emptyset[X \cross \nat]
\end{theorem}
\begin{theorem}{unitBagDef}[X]
    \vdash? \forall x : X @ \lbag x \rbag =  \{ (x, 1) \}
\end{theorem}

And the bag display theorems accumulated from the previous sections
%
\begin{theorem}{domBags}[X]
    \vdash? \forall x, y: X @ \dom~(\lbag x, y \rbag) = \{ x, y \}
\end{theorem}
\begin{theorem}{emptyBagScalling}[X]
    \vdash? \forall n: \nat; B: \bag X @ n \otimes \lbag \rbag = 0 \otimes B = \lbag \rbag
\end{theorem}
\begin{theorem}{emptyBagSubBag}[X]
    \vdash? \forall B: \bag X @ \lbag \rbag \subbageq B
\end{theorem}
\begin{theorem}{emptyBagUnitBagUnion}[X]
    \vdash? \forall B: \bag X @ \lbag \rbag \uplus B = B \uplus \lbag \rbag = B
\end{theorem}
\begin{theorem}{emptyBagUnitBagDiff}[X]
    \vdash? \forall B: \bag X @ B \uminus \lbag \rbag = B
\end{theorem}
\begin{theorem}{emptyBagZeroBagDiff}[X]
    \vdash? \forall B: \bag X @ \lbag \rbag \uminus B =  \lbag \rbag
\end{theorem}
\begin{theorem}{items2BagDisplay}[X]
    \vdash? \forall x, y: X @ items~\langle x, y \rangle = \lbag x, y \rbag
\end{theorem}

\section{Using this toolkit}

To use this file, just add
%
\begin{verbatim}
\documentclass{article} % or whatever else
\usepackage{czt}

begin{zsection}
   \SECTION my\_section \parents bag\_toolkit\_spivey
end{zsection}

% your specification goes here

\end{document}
\end{verbatim}
%
within a file caled \verb'my\_section.tex'.
