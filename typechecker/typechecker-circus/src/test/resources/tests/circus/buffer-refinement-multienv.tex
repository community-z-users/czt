\documentclass{article}

\usepackage[cntbysection,colour]{circus}

% count definitions please
\CountDefstrue

\begin{document}

\section{Introduction}

\begin{verbatim}
%-----------------------------------------------------------------------------
%-- A REACTIVE BUFFER    - Case Study                                        -
%-- Action Refinement: controller and ring partitions                        -
%-- Example extracted from the paper "A Refinement Strategy for Circus"      -
%-----------------------------------------------------------------------------
\end{verbatim}


You must always include the circus\_toolkit, and also check inside
it to see the LaTeX commands the parser recognises (or to create
your own commands)

\section{Channels}

Section name is to be the same as the name of the file, is possible. It helps IDEs
%
\begin{zsection}
  \SECTION\ buffer\_refinement\_multienv \parents\ circus\_toolkit
\end{zsection}

Z paragraphs can be defined at top level
\begin{axdef}
 maxbuff, maxring : \nat
\end{axdef}

CSP paragraphs can be defined at top-level 
\begin{circus}
 \circchannel\ input, output : \nat
\end{circus}

\begin{circus}
 \circchannel\ write, read : (1 \upto maxring) \cross \nat
\end{circus}

\begin{circus}
 \circchannel\ read\_1 : (1 \upto maxring)
\end{circus}

\begin{circus}
 \circchannel\ read\_2 : \nat
\end{circus}

With \verb'\CircusDeclSummary' you can create a summary of declarations
%
\CircusDeclSummary

\section{Buffer}

Unfortunately boxed processes (those spread across multiple
begin/end Circus) are not yet available. You need to define your
processes within one environment only. The only real problem is for
axiomatic definitions (that I am looking into now). Schemas can be
given horizontally.

A \textsf{circus} environment starts the process context
%
\begin{circus}
    \circprocess\ BufferMultiEnv \circdef \circbegin
\end{circus}

Z paragraphs within a process context are local to that process. They can be used
within the process' actions but are unknown outside the process. That means names
might be repeated, so long as they are globally unique, as well as locally unique
(i.e.~ variable $x$ can be declared globally, and within a process $P$ and another $Q$
with different types. You cannot declare a process named $x$, though, as this would
duplicate the name $x$ globally).
%
\begin{schema}{ControllerState}
    cache : \nat
\\%
   size: 0 \upto maxbuff
    \\%
    ringsize : 0 \upto maxring
    \\%
    top, bot : 1 \upto maxring
\where
        (ringsize \mod maxring) = ((top - bot) \mod maxring)
        \\%
        ringsize = size-1
\end{schema}

Schemas  can be given in multiple horizontal paragraph within a single Z environment.
\begin{zed}
    RingState ~~== ~~ [~ ring : \seq \nat | \#~ring = maxring ~] \\
    CBufferState ~~==~~ (ControllerState \lor RingState)
\end{zed}

\Circus\ state is marked accordingly within a \textsf{circusaction} environment
\begin{circusaction}
    \circstate\  CBufferState
\end{circusaction}

\begin{schema}{ControllerInit}
    ControllerState~'; RingState~'
\where
    size'=0 \\
    bot'=1 \\
    top'=1
\end{schema}

\begin{schema}{CacheInput}
   \Delta ControllerState
       \\%
       \Xi RingState
        \\%
        x?:\nat
\where
        (size = 0) \land (size' = 1)
        \\%
        (cache' = x?) \land (bot' = bot) \land (top' = top)
\end{schema}

\begin{schema}{StoreInput}
    \Delta CBufferState
        \\%
        x?: \nat
\where
        (0 < size) \land (size < maxbuff)
        \\%
        (size' = size+1) \land (cache'=cache)
        \\%
        (bot'=bot) \land (top'=(top \mod maxring)+1)
        \\%
        ring' = ring \oplus \{ top \mapsto x? \}
\end{schema}

\begin{schema}{StoreInputController}
    \Delta ControllerState
        \\%
        \Xi RingState
\where
        (0 < size) \land (size < maxbuff)
        \\%
        (size' = size+1) \land (cache'=cache)
        \\%
        (bot'=bot) \land (top'=(top \mod maxring)+1)
\end{schema}

Actions, and other \Circus\ declarations \textbf{must} use
\verb'\circdef' instead of \verb'\defs' or \verb'==' from Z. This
is needed to avoid parsing ambiguities with the Z grammar. 
One can use tabulation and blocks (see .tex file).

Because guards are Z predicates, and predicates can be
parenthesised, we need a different token for disambiguation as well.
So, every guard \textbf{requires} a \verb'\lcircguard pred \rcircguard \circguard A'
and it typesets as $\lcircguard pred \rcircguard \circguard A$. Other 
similar marker-tokens are used for disambiguation.

Like for variable names, channel names can accept $?$ or $!$ or $'$.
So, hard space is needed to indicate this is input prefix: you must type \verb'input~?x'
instead of simply \verb'input?x', and similarly for other strokes. Restricting channel
names not to have strokes (i.e. \verb'input~?x?' for the input on channel named $x?$)
is not straightforward (i.e. such names are in context because of the Z type rules for inputs), 
and it does not solve the problem anyway.

Moreover, also notice the extra parenthesis within the input prefixing. This is important 
in this context because of potential precedence confusion to the parser. They keep the
variable $x$ in scope for the $StoreInput$ schema, say.
%
\begin{circusaction}
    InputController ~~\circdef~~ \\
        \t1 \lcircguard size < maxbuff \rcircguard \circguard (input~?x \then \\
           \t2 \circblockbegin
            ((\lcircguard size = 0 \rcircguard \circguard \lschexpract CacheInput \rschexpract) \\
            \extchoice \\
             (\lcircguard size > 0 \rcircguard \circguard write.top~!x \then \lschexpract StoreInputController \rschexpract)
            ) 
            \circblockend) \\
    CInput \circdef \\
        \t1 \lcircguard size < maxbuff \rcircguard \circguard (input~?x \then \\
        	\t6 \circblockbegin
            ((\lcircguard size = 0 \rcircguard \circguard \lschexpract CacheInput \rschexpract) \\
            \extchoice \\
            (\lcircguard size > 0 \rcircguard \circguard \lschexpract StoreInput \rschexpract) )
           \circblockend)
\end{circusaction}

\begin{schema}{NoNewCache}
    \Delta ControllerState
    \\%
    \Xi RingState
\where
           size = 1
            \\%
            size' = 0 \land cache' = cache
            \\%
            bot' = bot \land top' = top
\end{schema}


Function application within Z schemas (as in Standard Z) do require hard spaces
or parenthesis (i.e. \verb'f~x' or \verb'f(x)'), otherwise the soft space is eaten during
lexing (i.e. \verb'f x' becomes \verb'fx', which leads to a type error).
%
\begin{schema}{StoreNewCache}
    \Delta CBufferState
\where
            size > 1
            \\
            % function application DO REQUIRE hard spaces.
            % otherwise, ring~bot is treated as ringbot
            size' = size-1 \land cache' = ring~bot
            \\
            bot' = (bot \mod maxring)+1 \land top' = top
            \\
            ring' = ring
\end{schema}

\begin{schema}{StoreNewCacheController}
    \Delta ControllerState
            \\%
            \Xi RingState
            \\%
            x? : \nat
\where
            size > 1
            \\%
            size' = size-1 \land cache' = x?
            \\%
            bot' = (bot \mod maxring)+1 \land top' = top
\end{schema}

New hard lines (\verb'\\' and \verb'\also') are option after \verb'\circdef'.
%
\begin{circusaction}
    OutputController ~~\circdef~~ \\
            \t1 \lcircguard size > 0 \rcircguard \circguard output~!cache \then \\
            \t2 \circblockbegin
            (\lcircguard size > 1 \rcircguard \circguard read.bot~?x \then \lschexpract StoreNewCacheController \rschexpract) \\ 
            \extchoice \\
            (\lcircguard size = 1 \rcircguard \circguard \lschexpract NoNewCache \rschexpract)
            \circblockend
\end{circusaction}

\begin{circusaction}
    COutput ~~\circdef~~ \\
            \t1 \lcircguard size > 0 \rcircguard \circguard output~!cache \then \\
            \t2 \circblockbegin
            (\lcircguard size > 1 \rcircguard \circguard \lschexpract StoreNewCache \rschexpract) \\
            \extchoice \\
            (\lcircguard size = 1 \rcircguard \circguard \lschexpract NoNewCache \rschexpract)
            \circblockend
\end{circusaction}


One should be careful with the precedences. See the file \textsf{process.tex} in the type checker tests directory for this, 
and then properly include the parenthesis. See the Z standard precedence table for Z,  and FDR manual for CSP precedence.
Sequential composition \textsf{is not} just normal semicolon ($;$). This creates too many conflicts with Z and we used a different
unicode/\LaTeX\ symbol ($\circseq$). It typesets just like $;$, however.  Missing-parenthesis errors are the harder to find and the 
worst in error generation!

\begin{circusaction}
    ControllerAction \circdef \lschexpract ControllerInit \rschexpract \circseq (\circmu\ X \circspot ((InputController \extchoice OutputController) \circseq X)) 
\end{circusaction}


\begin{schema}{StoreRingCmd}
    \Xi ControllerState
    \\%
    \Delta RingState
    \\%
    i? : 1 \upto maxring
    \\%
    x? : \nat
\where
    ring' = ring \oplus \{ i? \mapsto x? \}
\end{schema}

\begin{circusaction}
   StoreRing \circdef write~?i~?x \then \lschexpract StoreRingCmd \rschexpract \\
   NewCacheRing \circdef read~?i~!(ring~i) \then \Skip \\
   RingAction \circdef \circmu\ X \circspot ((StoreRing \extchoice NewCacheRing) \circseq X)
\end{circusaction}

\begin{circusaction}
    \t1 \circspot

   % The production is : circusAction:cal LPAR:lp nameSet:nsl BAR channelSet:cs BAR nameSet:nsr RPAR:rp circusAction:car
   % Why did you use two (identical) channelSets as if it was an Alphabetised parallel action (which does not exist)?
   \circblockbegin
   ( \t1 ControllerAction \\ \lpar 
        \{ size, ringsize, cache, top, bot \} |
        \lchanset write, read \rchanset |
        \{ ring \} \rpar \\
     \t1 RingAction)
    \circblockend    \\
     \t4 \circhide \lchanset write, read \rchanset
\end{circusaction}

\begin{circus}
    \circend
\end{circus}

\CircusDeclSummary

\end{document}
