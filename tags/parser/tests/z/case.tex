%\documentstyle[12pt,coz,times]{report}
\documentclass{llncs}
\usepackage{pt}
\usepackage{units}
\usepackage{coz}
\usepackage{epsf}
\usepackage{lncsexample}
\conttrue
\def\kph{\km \mulnuit{\hour}{-1}}

\begin{document}
Basically, the specification is for a golf tournament. It has a set of players with their corresponding 
scores. New players can be assigned while it can delete a player in the list. Also, it can calculate 
the score for each player based on their own shots.


Using Name as a set to present players.
\begin{zed}
[Name]
\end{zed}
Using an axiomatic definition to limit the number of players in the tournament.
\begin{axdef}
maxPlayer : \nat
\where maxPlayer \geq 100
\end{axdef}
This is the state schema for the tournament. It contains a set of players, with a full function 
named "score" to calculate the grade of a specific player.
\begin{schema}{Tournament}
players : \power Name \\
score : Name \pfun \num
\where \# players \leq maxPlayer \\
\dom score = players
\end{schema}

Intially, the set of players is empty.
\begin{schema}{TournamentInit}
Tournament
\where players = \emptyset
\end{schema}

The "Newplayer" schema can add a new player to the set of players.
\begin{schema}{NewPlayer}
\Delta Tournament \\
p? : Name
\where p? \notin players \\
players' = players \cup \{p?\} \\
score' = score \cup \{ p? \mapsto 0 \}
\end{schema}

The "PlayerResigns" schema can resign a player from the set of players.
\begin{schema}{PlayerResigns}
\Delta Tournament \\
p? : Name
\where p? \in players\\
players' = players - \{p?\} \\
score' = \{p?\} \ndres score
\end{schema}

The operation schema "PlayHole" can update the player's score based on the shots he/she has
\begin{schema}{PlayHole}
\Delta Tournament \\
p? : Name \\
shots? : \num
\where p? \in players \\
players' = players \\
score' = score \oplus \{ p? \mapsto score(p?) + shots? \}
\end{schema}
\end{document}
