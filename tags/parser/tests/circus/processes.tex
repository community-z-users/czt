\documentclass{article}

\usepackage{vmargin}
\usepackage{circus}

\setpapersize{A4}
%\setmarginsrb{leftmargin}{topmargin}{rightmargin}{bottommargin}{headheight}{headsep}{footheight}{footskip}
%\setmarginsrb{20mm}{10mm}{20mm}{10mm}{12pt}{11mm}{0pt}{11mm}
%\setmarginsrb{25mm}{20mm}{25mm}{20mm}{12pt}{11mm}{0pt}{10mm}
\setmarginsrb{40mm}{20mm}{40mm}{20mm}{12pt}{11mm}{0pt}{10mm}

\begin{document}

\title{\Circus\ Grammar Explained --- Processes}
\author{Leonardo Freitas}
\date{March 2006}

\maketitle

\begin{abstract}
    \noindent This article documents the various ways one can declare \Circus\ processes using \LaTeX\ markup.
    It also documents the open issues of the language related to this syntactic category.
\end{abstract}

\section{Introduction}

In this file, we explain the various aspect of \Circus\ related to
process declarations. At each section relevant to the grammar, we
include its corresponding CUP rule, as ``Section name ---
\texttt{grammarRule}''.

\Circus\ productions must be included within the \textsf{circus}
\LaTeX\ environment in order to be recognised by the parser.
Therefore, to use \Circus\ one needs to have the current version of
the \textsf{circus.sty} \LaTeX\ style file.

Amongst other commands, it contains all \Circus\ keywords, and some
useful environments.

\begin{issue}[creating \texttt{circus.sty} and \texttt{circus.dtx}]
   Make a suitable \texttt{circus.dtx} with nice and up-to-date \Circus\
   typesetting commands, as well as embedded documentation.

   Join version of \texttt{circus.sty} and add literate documentation.
   Also, gather the information for solving the spacing problem of the
   \code{circus} environment.
\end{issue}

\section{\Circus\ process paragraphs --- \grammar{processPara}}

In \Circus\ processes are global paragraphs at section level. Other
global paragraphs are channel, channel sets, and all other Z
paragraphs. Firstly, we need to create a new section containing the
\textsf{circus\_toolkit} as a parent.

\begin{zsection}
  \SECTION\ circus\_processes \parents\ circus\_toolkit
\end{zsection}
%
\begin{verbatim}
   begin{zsection}
      \SECTION\ circus\_channels \parents\ circus\_toolkit
   end{zsection}
\end{verbatim}
%
We include with each typeset text the corresponded \LaTeX\ code.

\Circus\ processes enables encapsulated specification of both state and
behavioural aspects of systems. They are defined by a process paragraph that
introduces a name for a process description. Like schemas and axiomatic
definitions, process paragraphs can also have formal parameters.
%
\[
   \circprocess [X, Y] P \circdef \cdots
\]
%
For the processes below, we define some channels.
%
\begin{circus}
   \circchannel [X] a, b: X \\
   \circchannel c,d: \nat \\
   \circchannel s1, s2
\end{circus}
%
More information on the syntax of channels can be found in the \textsf{channels.pdf} file.

In \Circus\ every process paragraph contains a name attached to a process description,
where formal parameters are optional.
%
\[
    \circprocess\ [N^+] N \circdef ProcessDesc
\]
%
Process names must be unique across the specification and the available process descriptions
are defined below. They usually come between \verb'begin{circus}...end{circus}' environments.

%\section{Process description --- \grammar{processDesc}}
%
%Processes can be described in four different ways:~(i)~basic, (ii)~``normal'', (iii)~parameterised, or (iv)~indexed.
%Basic processes encapsulate state information defined using Z and \Circus\ actions defined using CSP operators.
%Normal processes are those that use the usual CSP operators.
%Parameterised processes are process descriptions preceded by a qualified list parameters.
%Indexed processes like parameterised processes but with the special semantics, which includes
%change in channel type and name to accommodate the communication of indexes.
%Each of these categories is detailed below.
%
%\subsection{Basic process --- \grammar{basicProcess}}
%
%\begin{issue}[\LaTeX\ environments!]
%\end{issue}
%
%\begin{circusprocess}
%    \begin{procheader}
%        [X, Y] P \circdef x, y: X @ Q~(x) \extchoice R~(y)
%    \end{procheader}
%\end{circusprocess}
%
%\begin{circusprocess}[-10pt]
%    Starting the definition of process $P$
%    \begin{procheader}
%        [X, Y] P \circdef \circbegin
%    \end{procheader}
%    Now defining an action for $P$
%    \begin{circusaction}
%        A \circdef c \then A
%    \end{circusaction}
%    Now defining the main action of $P$
%    \begin{mainaction}
%        A
%    \end{mainaction}
%\end{circusprocess}
%%... define $BASIC$ later on...
%
%\begin{circus}
%  \circprocess [X] BASIC \circdef BASIC[X]
%\end{circus}

We need the hard space on $FibState$ for it to be a $DecorExpr$.
\begin{circus}
   \circprocess\ UnboxedFibonacci \circdef \circbegin
    \also %
    \circstate\ FibState ~~==~~ [~ x, y: \nat ~] \\
    InitFibState ~~==~~ [~ FibState~' | x' = y' = 1 ~] \\
    InitFib ~~\circdef~~ out~!1 \then out~!1 \then InitFibState \\
    OutFibState ~~==~~ [~ \Delta FibState; next!: \nat | next! = y' = x + y \land x' = y ~]\\
    OutFib ~~\circdef~~ \circmu\ X @ (\circvar\ next: \nat @ OutFibState \circseq\ out~!next \then X) \\
    \t1 @ (InitFib \circseq\ OutFib) \\
   \circend
\end{circus}


\subsection{Process definition --- \grammar{process}}

Normal process definition uses the various CSP operators to combine process.
They are detailed below.

\subsubsection{Hiding}

Process hiding conceals the events from the given channel set from the given process.
As mentioned before in the \textsf{channels.tex} file, channel set expressions cannot
mix Z expressions with \code{BasicChannelSetExpr}, hence the need for $HCS0$ and $HCS1$.
The typechecker will infer the type types of channels from $HCS$ in the definition of process $H_1$.
%
\begin{circus}
    \circchannelset HCS0 == \lchanset a, b \rchanset \\ % \cup \lchanset c \rchanset \\ !! Not yet valid!!
    \circchannelset HCS1 == \lchanset c \rchanset \\
    \circchannelset HCS == HCS0 \cup HCS1 \\
    \circprocess\ H \circdef BASIC[\nat] \circhide \lchanset a[\nat], b[\nat] \rchanset \\
    \circprocess\ [X, Y] H_1 \circdef BASIC[X \setminus Y] \circhide HCS
\end{circus}
%
\begin{verbatim}
     begin{circus}
         \circchannelset HCS0 == \lchanset a, b \rchanset \\
         \circchannelset HCS1 == \lchanset c \rchanset \\
         \circchannelset HCS == HCS0 \cup HCS1 \\
         \circprocess\ H \circdef BASIC[\nat] \circhide
                               \lchanset a[\nat], b[\nat] \rchanset \\
         \circprocess\ [X, Y] H_1 \circdef BASIC[X \setminus Y]
                                 \circhide HCS
     end{circus}
\end{verbatim}
%
Another interesting point is about typesetting. Although, the hiding ($\circhide$) keyword operator
does look like the set difference ($\setminus$) operator, their Unicode characters \textbf{*must*}
be different in order to avoid inaccurate scanning. This rule is applied everything. For instance,
the prefixing arrow ($\then$) is slightly bigger than the else guarded command arrow ($\circthen$),
and slightly smaller than the function symbol ($\fun$) from $relation\_toolkit$.

\subsubsection{Interleaving}

If $IL$ does not have generic formals, or $H_1$ does not receive adequate generic actual parameters,
the type checker will issue an error since calls to generic process must contain enough information
to infer the needed types.
%
\begin{circus}
    \circprocess\ [X, Y]\ IL \circdef H \interleave H_1[X, Y]
\end{circus}
%
\begin{verbatim}
     begin{circus}
        \circprocess\ [X, Y]\ IL \circdef H \interleave H_1[X, Y]
     end{circus}
\end{verbatim}

\subsubsection{Parallelism}

+ $HCS$ generic actuals
\begin{circus}
    \circprocess\ P \circdef BASIC[\nat] \lpar HCS[\nat] \rpar H
\end{circus}
%
\begin{verbatim}
     begin{circus}
        \circprocess\ P \circdef BASIC[\nat] \lpar HCS[\nat] \rpar H
     end{circus}
\end{verbatim}

\subsubsection{External choice}

\begin{circus}
    \circprocess\ EC \circdef IL \extchoice P
\end{circus}
%
\begin{verbatim}
     begin{circus}
        \circprocess\ EC \circdef IL \extchoice P
     end{circus}
\end{verbatim}


\subsubsection{Internal choice}

\begin{circus}
    \circprocess\ [X, Y]\ IC \circdef BASIC[X] \intchoice H_1[Y \cross X, X \fun Y]
\end{circus}
%
\begin{verbatim}
     begin{circus}
        \circprocess\ [X, Y]\ IC \circdef BASIC[X] \intchoice
                                          H_1[Y \cross X, X \fun Y]
     end{circus}
\end{verbatim}

\subsubsection{Sequential composition}

\begin{circus}
    \circprocess\ SC \circdef EC \circseq EC
\end{circus}
%
\begin{verbatim}
     begin{circus}
        \circprocess\ SC \circdef EC \circseq EC
     end{circus}
\end{verbatim}

\subsubsection{Prefixing}

For fields, something very tricky happens: strokes!
Channel names are just refName (hence DECORDWORD), which
means strokes are scanned accordingly.
So, \verb'c?x' is tokenised differently from \verb'c~?x' as:
\[
(DECORWORD, "c?"), (DECORDWORD, "x") \mbox{ and} \\
(DECORWORD, "c"), (INSTROKE, "?"), (DECORDWORD, "x")
\]
%
In this situation we run out of luck because all possible
solutions around it do create trouble somewhere else.

\begin{enumerate}
    \item New Unicode for "?" (input field)

    That means LaTeX typeset as: \verb'c\commIn x'.
    This also incurs change in the \code{KeywordScanner} (or \code{ContextFreeScanner})
    to recognise this new type of "word-glue".

    \item Channel names without strokes

    Doesn't change the fact the scanner will recognise $c?$ as
    one DECORWORD, which means no good will be done here.
    This in fact is the harder solution, as it demands various
    changes in both \code{ContextFree} and \code{Latex2Unicode} scanners.
    As a consequence, everywhere that channel appear, require
    new productions for refName/declName and corresponding lists.
    ABSOLUTE NIGHTMARE! Forget it.

    \item Using INSTROKE to represent input field

    That means disappearing with CIRCCOMMQUERY (easy) and using
    INSTROKE instead. This does not require any change anywhere else.
    The price to pay, is that the LaTeX must have a hard space as \verb'c~?x'.

    That is somehow similar to what happens with DOT. The difference
    is that "?" and "!" are more complicated because they can be lexed
    as either part of a DECORWORD or as STROKES.


\end{enumerate}

%\begin{circus}
%    \circprocess\ PR \circdef s1 \then c~?x \then a~!x \then PR  \\
%    \circprocess\ PR1 \circdef s1 \then c~?x \then a~!x \then PR1 \circseq \\
%                                s2 \then d~?y \prefixcolon y \leq 20 \then PR1
%\end{circus}

%\begin{circus}
%    \circprocess\ PR \circdef s1 \then c?x \then a!x \then PR \circseq \\
%                             \t3\ \ s2 \then d?y: y \geq 20 \then PR
%\end{circus}

\begin{issue}[Processes do not have prefixing]
    Discussed with Ana already: processes do not have prefixing, guards, or recursion.
\end{issue}

\subsubsection{Some examples on precedences}

The following processes are equivalent. The names are given as acronyms of
the operator used, where the subscripted version is the parenthesised equivalent.
Thus, for instance, $IC\_EC\_SC$ means a tree with $\intchoice$ as its root and
$\extchoice$ and $\circseq$ as its left and right branches respectively.
%
%\[
%\begin{array}{rcl}
%IC\_EC\_SC   & = & \intchoice(\extchoice(\circseq(SC, SC), EC), IC) \\
%IC\_EC\_SC_0 & = & \intchoice((\extchoice((\circseq(SC, SC)), EC)), IC)
%\end{array}
%\]
\begin{circus}
    \circprocess\ IC\_EC\_SC   \circdef SC \circseq SC \extchoice EC \intchoice IC[\nat, \nat] \\
    \circprocess\ IC\_EC\_SC_0 \circdef (~(SC \circseq SC) \extchoice EC) \intchoice IC[\nat, \nat]
\end{circus}
%
%\begin{verbatim}
%     begin{circus}
%        \circprocess\ IC\_EC\_SC   \circdef SC \circseq SC \extchoice EC \intchoice IC[\nat, \nat] \\
%        \circprocess\ IC\_EC\_SC_0 \circdef (~(SC \circseq SC) \extchoice EC) \intchoice IC[\nat, \nat]
%     end{circus}
%\end{verbatim}

%\[
%\begin{array}{rcl}
%IC\_SC\_EC   & = & \circseq(SC, \intchoice(\extchoice(SC, EC), IC)) \\
%IC\_SC\_EC_0 & = & \circseq(SC, (\intchoice((\extchoice(SC, EC), IC))))
%\end{array}
%\]
\begin{circus}
    \circprocess\ SC\_IC\_EC   \circdef SC \circseq (SC \extchoice EC \intchoice IC[\nat, \nat]) \\
    \circprocess\ SC\_IC\_EC_0 \circdef SC \circseq ((SC \extchoice EC) \intchoice IC[\nat, \nat])
\end{circus}

%\[
%\begin{array}{rcl}
%EC\_SC\_IC     &=& \extchoice(\circseq(SC, SC), \intchoice(EC, IC))\\
%EC\_SC\_IC_0   &=& \extchoice((\circseq(SC, SC)), (\intchoice(EC, IC)))
%\end{array}
%\]
\begin{circus}
    \circprocess\ EC\_SC\_IC   \circdef SC \circseq SC \extchoice (EC \intchoice IC[\nat, \nat]) \\
    \circprocess\ EC\_SC\_IC_0 \circdef (SC \circseq SC) \extchoice (EC \intchoice IC[\nat, \nat])
\end{circus}

\begin{circus}
    \circprocess\ IC\_SC\_EC   \circdef SC \circseq (SC \extchoice EC) \intchoice IC[\nat, \nat] \\
    \circprocess\ IC\_SC\_EC_0 \circdef (SC \circseq (SC \extchoice EC)) \intchoice IC[\nat, \nat]
\end{circus}

\begin{circus}
    \circprocess\ IL\_\_EC\_SC\_\_IC\_SC   \circdef SC \circseq SC \extchoice EC \interleave EC \intchoice EC \circseq SC \\
    \circprocess\ IL\_\_EC\_SC\_\_IC\_SC_0 \circdef ((SC \circseq SC) \extchoice EC) \interleave (EC \intchoice (EC \circseq SC))
\end{circus}

\subsubsection{Some examples with hard new lines and spaces}

Just like in Z, it is desirable to allow special scanning of hard new lines and spaces,
such as the \verb'\\' and \verb'\tN' symbols. This enables pretty printing and should
not affect parsing.
%
\begin{circus}
    \circprocess\ [X, Y]\ IC_2 \circdef BASIC[X] \intchoice \\ \t5 H_1[Y \cross X, X \fun Y]
\end{circus}
%
\begin{verbatim}
     begin{circus}
        \circprocess\ [X, Y]\ IC \circdef BASIC[X] \intchoice \\
                                     \t5  H_1[Y \cross X, X \fun Y]
     end{circus}
\end{verbatim}
%
From the grammar point of view, such tokens are a nuisance: they introduce nasty conflicts,
and badly affect the size of the parsing tables. Fortunately,

CZT has a very elegant solution for this, namely a layered architecture involving around
$10$ different specialised scanners, one of which is responsible for handling new line
problems alone! That means, a simple configuration on the \code{NewLineScanner} terminals
telling which tokens allows new lines before and/or after them to be treated specially.
This directly (and specifically) implements the specialised handling of new line lexis,
as defined in the Z standard Section $7.5$.

%\begin{circus}
%    \circprocess\ PR \circdef s1 \then c~?x \then a~!x \then PR  \\
%    \circprocess\ PR1 \circdef s1 \then c~?x \then a~!x \then PR1 \circseq \\
%                                s2 \then d~?y \prefixcolon y \leq 20 \then PR1
%\end{circus}

%\begin{circus}
%    \circprocess\ PR \circdef s1 \then c?x \then a!x \then PR \circseq \\
%                             \t3\ \ s2 \then d?y: y \geq 20 \then PR
%\end{circus}


\subsection{Parameterised process --- \grammar{paramProcess}}

Parameteri

\subsection{Indexed process --- \grammar{indexedProcess}}

%\begin{circus}
%    \circprocess\ [X, Y] S \circdef \circindex A \circseq B
%\end{circus}
%
%\begin{circus}
%    \circprocess\ R \circdef x, y: \nat @ A \circseq B \\
%    \circprocess\ [X, Y] S \circdef a, b: X \rel Y \circindex A \circseq B
%\end{circus}
%
%%\hspace{10pt}\begin{minipage}{0.8\linewidth}
%%  Vejamos....
%%  \begin{zed}\circprocess\ Oxente? \end{zed}\ignorespacesafterend
%%\end{minipage}\ignorespacesafterend
%
%% via (possibly empty generic formals
%%and) a given name
%%
%%
%%
%%There are two types of processes:~(i)~compound processes, where one can define
%%different processes combined using CSP operators;~(ii)~basic processes, where
%%one can define state and behaviour through Z paragraphs, action paragraphs
%%using CSP operators, or name set paragraphs.
%%
%%
%%Basic process
%%
%%
%%
%%
%%
%%
%%Untyped channels are also allowed. They represent synchronisation points with
%%the given type $Synch$. Thus, declaring a untyped channel has the same effect
%%as declaring the channel with type $Synch$, as defined in the
%%\textsf{circus\_toolkit.tex} file.
%%
%%Channels are declared within the \textsf{circus} \LaTeX\ environment, which are
%%typeset as Z unboxed items like in the \textsf{zed} \LaTeX\ environment. As the
%%Z standard, various unboxed \Circus\ items can be declared within the same
%%environment, as long as they are separated by a \grammar{NL} (new-line) token.
%%
%%\subsubsection{Typed channels}
%%
%%Let us define some typed channels,
%%%
%%\begin{circus}
%%  \circchannel\ a, b: \nat
%%\end{circus}%
%%%
%%\begin{verbatim}
%%     \begin{circus}
%%        \circchannel\ a, b: \nat
%%     \end{circus}%
%%\end{verbatim}
%%%
%%as well as typed channels with strokes.
%%%
%%\begin{circus}
%%  \circchannel\ x_1~, r, q'~, y_2: \num
%%\end{circus}%
%%%
%%\begin{verbatim}
%%     \begin{circus}
%%        \circchannel\ x_1~, r, q'~, y_2: \num
%%     \end{circus}%
%%\end{verbatim}
%%%
%%Note that due to the Z standard idiosyncrasy regarding name lists with strokes
%%and commas (\textit{i.e.}, \grammar{DECORWORD} or \grammar{DECLWORD} in CZT),
%%we \textbf{*must*} include a hard space to separate these names (see Z Standard
%%section 7.3). This is needed, for instance, to allow distinction between set
%%extensions and a list of variable names in a declaration. The soft space
%%through \verb'\ ' is optional, though.
%%
%%\subsubsection{Generic channels}
%%
%%Generic channels can also be declared, where generic formal parameters given
%%are right after the $\circchannel$ keyword.
%%%
%%\begin{circus}
%%   \circchannel\ [X, Y]\ g, h_1~, i: (X \rel Y)
%%\end{circus}%
%%%
%%\begin{verbatim}
%%     \begin{circus}
%%        \circchannel\ [X, Y]\ g, h_1~, i: (X \rel Y)
%%     \end{circus}%
%%\end{verbatim}
%%%
%%As before, names with decorations must have hard spaces before hitting a
%%\texttt{COMMA} token, as shown above.
%%
%%\begin{issue}[type inference on generic channels usage]
%%    In practice, how should generic actuals be instantiated for channels?
%%    The implications are clear at parallel composition and implementation of
%%    ``generically defined'' communications.
%%
%%    For example, in:
%%
%%    \[
%%        \circchannel\ [X] c, d: X \\
%%
%%        \vdots \\
%%
%%        c?x \then d!x \then Skip [|c, d|] c.v \then d?x \then Skip
%%    \]
%%
%%    Could the typechecker infer the types of both $c$ and $d$ to be the
%%    type of $v$? In a multisychronisation scenario, however, this would
%%    become a typechecking nightmare!
%%
%%    The suggestion is then to enforce generic actuals instantiation on
%%    generically defined channels, so that no type inference is needed,
%%    as in:
%%
%%    \[
%%       c[\nat]?x \then d[\nat]!x \then Skip [|c[\nat], d[\nat]|] c[\nat].v \then d[\nat]?x \then Skip
%%    \]
%%
%%    This involves a clumsy and too verbose specification, but simpler
%%    typechecking. What would be the compromise? Just leave it out for the
%%    moment?!
%%\end{issue}
%%
%%\subsubsection{Synchronisation channels}
%%
%%Synchronisation channels are defined without a type expression. The parser
%%introduces a $Synch$ \code{RefExpr} for it automatically, as it is necessary
%%for translation to and from ZML. This name is defined as a given set in the
%%\textsf{circus\_toolkit.tex} file.
%%%
%%\begin{circus}
%%   \circchannel\ c, d \\
%%   \circchannel\ z_2~, w_3
%%\end{circus}%
%%%
%%\begin{verbatim}
%%     \begin{circus}
%%        \circchannel\ c, d \\
%%        \circchannel\ z_2~, w_3
%%     \end{circus}%
%%\end{verbatim}
%%%
%%This time, we left two declarations within the same Z box, but separated by a
%%\grammar{NL} (new-line) token.
%%
%%% SOLVED: Discussion in an e-mail with Tim and Mark (CZT), 20,21/03/2006
%%%         (subject: Exception thrown while marshiling ZML with null Expr in VarDecl)
%%%
%%\begin{issue}[technicality about representation of $Synch$]
%%    Perhaps we shall remove the $Synch$ type from \texttt{circus\_toolkit} and
%%    set it as a reserved word that the typechecker does not allow anyone else to
%%    use.
%%
%%    If that is the path to choose, we can to do this by extending the
%%    directives of the parser to have something like \code{\%\%Zreserved}.
%%\end{issue}
%%
%%\subsection{Channels through schemas --- \grammar{channelFromDecl}}
%%
%%Another way of declaring typed channels is through schema references. Let us
%%define a schema named $S$ with two channels $a$ and $b$
%%%
%%\begin{schema}{S}
%%   a, b: \nat
%%\where
%%   a > b \land b \neq 0
%%\end{schema}
%%%
%%\begin{verbatim}
%%     \begin{schema}{S}
%%         a, b: \nat
%%     \where
%%         a > b \land b \neq 0
%%     \end{schema}
%%\end{verbatim}
%%%
%%With the $\circchannelfrom$ keyword, we can declare these channels in various
%%ways, depending on the chosen decoration.
%%%
%%\begin{circus}
%%   \circchannelfrom\ S \\
%%   \circchannelfrom\ S_1
%%\end{circus}%
%%%
%%\begin{verbatim}
%%     \begin{circus}
%%        \circchannelfrom\ S \\
%%        \circchannelfrom\ S_1
%%     \end{circus}%
%%\end{verbatim}
%%%
%%These declarations include four channels of type $\nat$, namely:~$a$, $b$,
%%$a_1$, and $b_1$.
%%
%%% SOLVED: Invariants of schemas are ignored on channelFrom - Jim
%%%
%%\begin{issue}[invariant of schemas in channelFrom]
%%  What happen with the invariant of $S$?
%%  Should it be enforced on the communication patterns or just be ignored?
%%\end{issue}
%%
%%% SOLVED: Restricted to RefExpr only - Jim
%%%
%%\begin{issue}[channelFrom with any expression]
%%What about this kind of declaration?
%%%
%%\begin{circus}
%%   \circchannelfrom\ S \cross S
%%\end{circus}%
%%%
%%\begin{verbatim}
%%     \begin{circus}
%%        \circchannelfrom\ S \cross S
%%     \end{circus}%
%%\end{verbatim}
%%
%%At the moment the grammar does allow it. The option is to restrict it to
%%\code{RefExpr} only (\textit{i.e.}, reference expressions that could carry
%%generic actuals).
%%\end{issue}
%%
%%Another option is to have schemas with generic types.
%%%
%%\begin{schema}{T}[X]
%%   x, y: \power~X
%%\where
%%   x \subseteq y
%%\end{schema}
%%%
%%\begin{verbatim}
%%     \begin{schema}{T}[X]
%%         x, y: \power~X
%%     \where
%%         x \subseteq y
%%     \end{schema}
%%\end{verbatim}
%%%
%%In this case, one can define the channels from the schema via
%%either:~(i)~declaring the schema name with corresponding number of generic
%%actuals;~(ii)~leaving the original generic formal parameters (possibly renaming
%%them);~(iii)~letting the typechecker decide;~or (iii)~redefining the generic
%%formal parameters of channel types in terms of a new pattern of formal
%%parameters for the schema generic actuals.
%%%
%%\begin{circus}
%%    \circchannelfrom\ T[\nat] \\
%%    \circchannelfrom\ [Y]\ T_1[Y] \\
%%    \circchannelfrom\ T_2 \\
%%    \circchannelfrom\ [A, B]\ T_3[A \cross B]
%%\end{circus}%
%%%
%%\begin{verbatim}
%%     \begin{circus}
%%         \circchannelfrom\ T[\nat] \\
%%         \circchannelfrom\ [Y]\ T_1[Y] \\
%%         \circchannelfrom\ T_2 \\
%%         \circchannelfrom\ [A, B]\ T_3[A \cross B]
%%     \end{circus}%
%%\end{verbatim}
%%%
%%It is the task of the typechecker to guarantee that either the instantiation or
%%(re)definitions are type-correct. For each case we would have:
%%%
%%\begin{itemize}
%%    \item[i] $\circchannel\ x, y: \power~\nat$
%%    \item[ii] $\circchannel\ [Y]\ x_1~, y_1: \power~Y$
%%    \item[iii] $\circchannel\ x_2~, y_2: \power~X$
%%    \item[iv] $\circchannel\ [A, B]\ x_3~, y_3: \power~(A \cross B)$
%%\end{itemize}
%%%
%%Thus, $\circchannelfrom$ is just syntactic sugar for a corresponding
%%$\circchannel$ declaration.
%%
%%% SOLVED: Allow ``layered'' generics. Typechecker needs to sort it out
%%%         and adjust the AST appropriately - Jim
%%%
%%\begin{issue}[desugaring $\circchannelfrom$ into $\circchannel$]
%%   The AST allows room for quite powerful use of generic types within channel
%%   declaration from generic schemas via the two set of generic types on each
%%   production (\textit{i.e.}, \code{AxPara} and \code{ChannelDecl}).
%%
%%   The burden is given to the typechecker to infer the appropriate generic
%%   formals/actuals, as well as to match/update the lists of generic names (and
%%   their corresponding type expressions).
%%
%%   Anyway, the possibly tricky business comes if we decide to restrict the
%%   communication patterns with the corresponding schema invariant. As it
%%   stands, it is left for the typechecker to solve.
%%\end{issue}
%%
%%\section{Channel sets --- \grammar{channelSetPara}}
%%
%%Channel sets defines an expression to represent a particular set of channels.
%%They are normally used as reference names for other \Circus\ operators that
%%have channel references as parts of their AST, such as the parallel and the
%%hiding operators.
%%
%%Channel sets are given as a name and a corresponding expression.
%%%
%%%\begin{circus}
%%%    \circchanset\ C1 ==
%%%\end{circus}%
%%%
%%The channel set expressions mostly used are: set union, intersection and
%%difference expressions;~set extension expressions;~reference expressions;~and
%%application expressions.
%%
%%% SOLVED: Generalise channel set expressions to allow any Z expression - Jim
%%%         Should it simplify typechecking in a great extent?
%%%
%%\begin{issue}[generalising channel set expressions]
%%  The restricted form of channel sets complicate the grammar in a great
%%  extent because it demands filtering through the (very complex) expression
%%  subtree, which also mixes with the predicate subtree.
%%
%%  If application expressions were not to be considered, the limitation would be
%%  rather simple, as the other channel set expressions are easy to (hard) code. Nevertheless,
%%  as one might like (or even require) to define particular functions for
%%  channel sets to be used over replicated operators, application expressions may not (or cannot) be
%%  avoided.
%%
%%  For example, this scenario might appear as a requirement when one needs to perform
%%  synchronisation between disjoint set of channels amongst more than two processes,
%%  in order to avoid multisynchronisation complexities.
%%
%%  Therefore, because of application expression, we need to include channel sets with mostly
%%  all expressibility power of Z expressions. The best choice seems to reuse the
%%  Z grammar for abbreviations, and encoded as an AST of \code{ChannelSet}
%%  encapsulating a \code{ConstDecl}. This is useful for simplifying typechecking
%%  (and perhaps other purposes).
%%
%%  This choice will also normalise the language with respect to generically
%%  defined channels, as Z abbreviations already take care of generic formal and
%%  actual parameters. \textbf{This would simplify the parser grammar as well as the AST structure in a grater
%%  extend.} Furthermore, a nice side effect is the ability to have generic
%%  operators as possible channel set functions!
%%
%%  On the other hand, if one wants to keep the restrictions, one possible
%%  approach would be to leave the parser with Z abbreviations, but restrict
%%  its use through the type checking of channel sets.
%%\end{issue}
%%
%%\begin{issue}[show rob about text]
%%  Let us see how it goes.
%%\end{issue}
%
\end{document}

%* AST structure for Qualified parameters reviewed.
%
%* DeclPart for Circus changed to accomodate conflicts.
%
%   -> That is to say, avoid using Expr:e
%       but use RefName for SchemaRef instead
%
%* Various minor changes on KeywordScanner and CircusChar for accomodating overlapping keywords.

%* Added process call
%
%* Identified conflicts:
%
%-> guards x call, due to parenthesis
%-> declPart needs fixing
%
%* Added communication to processes
%
%* Added parenthesized processes
%
%* Various tests including precedence
%
%* Problem with communication x strokes

%All work
%\circprocess\ IRingCell \circdef i: RingIndex \circindex RingCell
%\circprocess\ IRingCell \circdef (i: RingIndex \circindex RingCell) \circlinst i \circrinst
%
%This fails, otherwise it would hard to distinguish between on-the-fly process call.
%That means, indexed processes when defined like this, cannot be parenthesised.
%And that makes sense since  (i: RingIndex \circindex RingCell) \extchoice P should be ruled out.
%\circprocess\ IRingCell \circdef (i: RingIndex \circindex RingCell)
