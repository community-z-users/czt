\documentclass{llncs}
\pagestyle{headings}   % turn on page numbers
\usepackage{epsf}
\usepackage{epsfig}

\usepackage{url}

\newcommand{\Circus}{{\sf\slshape Circus}}
\newcommand{\Class}[1]{\texttt{#1}}
\newcommand{\Element}[1]{\texttt{#1}}
\newcommand{\Interface}[1]{\texttt{#1}}
\newcommand{\Method}[1]{\texttt{#1}}

\begin{document}
\title{CZT Support for Z Extensions}
\author{Tim Miller$^1$ \and Leo Freitas$^2$ \and Petra Malik$^3$ \and Mark Utting$^3$}

\institute{%
   \begin{tabular}{cc}
      University of Liverpool, UK  $\quad\quad$ & University of York, UK
      \\ %
      $^1$\texttt{tim@csc.liv.ac.uk} $\quad\quad$ & $^2$\texttt{leo@cs.york.ac.uk}
   \end{tabular}
   \begin{tabular}{c}
      University of Waikato, New Zealand
      \\ %
       $^3$\{\texttt{petra, marku}\}\texttt{@cs.waikato.ac.nz}
   \end{tabular}
} %

\maketitle


\begin{abstract}
  Community Z Tools (CZT) is an integrated
  framework for the Z formal specification language.  In this
  paper, we show how it is also designed to support extensions
  of Z, in a way that minimises the work required to build a
  new Z extension.  The goals of the framework are to maximise
  extensibility and reuse, and minimise code duplication and
  maintenance effort.  To achieve these goals, CZT uses a variety of
  different reuse mechanisms, including generation of Java
  code from a hierarchy of XML schemas, XML templates for shared
  code, and several design patterns for maximising reuse of Java
  code.
  %
  The CZT framework is being used to implement several integrated
  formal methods, which add object-orientation, real-time features
  and process algebra extensions to Z.  The effort required to
  implement such extensions of Z has been dramatically reduced
  by using the CZT framework.

  \noindent
  \textbf{Keywords}: Standard Z, Object-Z, TCOZ, \Circus, parsing,
     typechecking, animation, design patterns, framework, AST.
\end{abstract}

\section{Introduction} \label{sec:intro}

  The Z language~\cite{isoz} is a formal specification notation that
  can be used to precisely specify the behaviour of systems, and
  analyse them via proof, animation, test generation, and so on.
  Z was approved as an ISO standard in 2002, but currently there are few
  tools that conform to the standard.\footnote{CADiZ
  (\url{http://www-users.cs.york.ac.uk/~ian/cadiz}) is the only Z tool
  that conforms closely to the Z standard.  It is freely available,
  but is not open-source and does not aim at supporting Z extensions.}
  The Community Z Tools (CZT) project~\cite{czt} is an open-source Java
  framework for building formal methods tools for standard Z and Z extensions.

  CZT\footnote{See~\url{http://czt.sourceforge.net}.} provides the basic tools
  expected in a Z environment, such as
  conversion between \LaTeX, Unicode and XML formats for Z, and
  parsing, unparsing, typechecking and animation tools, with a WYSIWYG
  Z editing environment integrated within the
  jEdit\footnote{See~\url{http://www.jedit.org}.} editor.
  There are also several more experimental tools under development,
  such as a Z-to-B translator and a semi-automated GUI-builder for
  Z specifications.  However, the main design goal of CZT is to
  provide a framework which makes it easy to develop new Z tools.
  This paper describes how the framework also makes it easy to develop
  tools for extensions of Z.

  In recent years, there has been an increasing interest in combining
  different programming paradigms within a uniform formal notation,
  where Z plays a central role. This has given rise to many Z
  extensions, which add features such as
  %
  process algebras~\cite{fischer-1998,fischer-2000,circus.sem:intro},
  object orientation~\cite{oz,ohcircus},
  time~\cite{tcoz,circus.sem:real.time2},
  mobility~\cite{circus.sem:mobility}, and so forth.

  Among these extensions, CZT supports Object-Z~\cite{oz}, a
  specification language that extends Z with modularity and reuse
  constructs that resemble the object-oriented programming
  paradigm. Such constructs include classes, inheritance, and
  polymorphism. CZT supports Object-Z in the form of parsing,
  typechecking, and other facilities.  CZT is also being used to
  develop extensions for Timed Communicating Object-Z (TCOZ)~\cite{tcoz},
  which is a blend of Object-Z and Timed-CSP~\cite{timed-csp}, as well
  as extensions for \Circus~\cite{circus.sem:intro}, a unified
  refinement language that combines Z, CSP~\cite{csp.books:roscoe},
  and the refinement calculus~\cite{fm.ref:morgan}, with Hoare and
  He's \textit{Unifying Theories of Programming} (UTP) as the semantic
  background~\cite{hoare.utp}\footnote{See~\url{http://www.cs.york.ac.uk/circus/}}.

  This paper describes the engineering techniques used in the CZT framework
  to maximise extensibility and reuse.
  Most of these techniques could also be applied to frameworks
  for other integrated formal methods, especially when the
  framework must support several different extensions of a common
  base language (like the role of Z in CZT).

  In Section~\ref{xml-schemas}, we present a method for specifying an
  XML interchange format that maximises extensibility.
  Section~\ref{java-ast-classes} describes the automatic generation
  and design of the \emph{Annotated Syntax Tree} (AST) classes.
  Section~\ref{parsers} presents a method for generating parsers,
  scanners, and other related tools for the different Z extensions,
  and Section~\ref{typecheckers} presents the design of the CZT
  typecheckers, which are tailored for extendibility and
  reuse. Section~\ref{animation} briefly presents the CZT animator,
  ZLive, and discusses the possibility of using this to animate
  extensions to Z. Section~\ref{section-manager} presents the design
  of the {\em specification manager}, an integral component of CZT
  that caches information about specifications to improve the
  efficiency of the tools.  Section~\ref{sec:related-work} gives an
  overview of related work.  Finally, Section~\ref{sec:conclusions}
  concludes the paper and discusses the future of the CZT project.


\section{XML Schemas}
\label{xml-schemas}

  The first step in designing the CZT tools and libraries was the
  development of an XML schema that describes an XML markup for Z
  specifications (ZML)~\cite{UttEA:03}.  This is an interchange format
  that can be used to exchange parsed Z specifications between
  sessions and tools written in different languages.

  Standard Z allows specifications to be exchanged using Unicode,
  \LaTeX\ or email markup.  However, implementing a parser for such
  specifications is a non-trivial task that can take several months.
  ZML, in contrast, can be parsed immediately since virtually all
  programming languages provide XML reading and writing libraries.

  The idea of using XML for Z has also been explored in the
  Z/EVES theorem prover~\cite{tp.tools:zeves.ref}.  It allows one to
  create a customised theorem prover with additional tactics tailored
  for a particular specification by modifying the XML representation
  of the Z specification in Z/EVES~\cite{tp.tools:zeves.api}.
  The main problem however, is the lack of a common standard.

  The XML schema for ZML was carefully designed, via consensus between
  several groups of interested people, by selecting the best features
  of the abstract syntaxes of CADiZ, Zeta and the Z standard.
  ZML supports several kinds of extensibility:
  \begin{description}
  \item[Extensible Annotations:] Each Z construct can be \emph{annotated}
    with arbitrary information, such as type information, comments,
    anticipated usage, and source-file location.
  \item[Extensible ASTs:] This allows Z extensions to add new kinds
    of expressions, predicates, paragraphs, \textit{etc}.
  \item[Extensible Schemas:] The standard XML schema features, such as
    name\-spa\-ces and importing, mean that Z extensions can be defined without
    modifying the original ZML schema.
  \end{description}
  The following strategies have been used to achieve these kinds
  of extensibility.

  The \textit{``any''} element can be used in an XML schema to enable
  instance XML documents to contain additional elements not specified
  by the schema.  This concept has been used to define annotations.
  That is, an annotation to a term can either be one of the
  annotations defined in the XML schema for Z, or any other kind of
  data.  New kinds of annotations can be added without changing the ZML
  schema.  This allows a tool builder to decide what data makes sense
  for a particular tool.  Tools that do not use a particular kind
  of annotation simply ignore those annotations.

  A typical style of defining XML schemas or DTDs is to explicitly
  list the possible alternatives for expressions, predicates, \textit{etc}.
  This makes it difficult to extend the syntax of ASTs to allow new
  kinds of expressions or
  predicates.  In contrast, ZML uses \emph{inheritance}
  (\emph{substitution groups} in XML schema terminology) extensively
  throughout the XML schema.  Abstract elements are used to provide
  placeholders for their derived elements.  For example, the abstract
  element \Element{Para} is the parent of all concrete Z paragraphs,
  such as axiomatic paragraphs (element \Element{AxPara}), and free
  types paragraph (element \Element{FreePara}).  Other elements that
  contain paragraphs, like Z section (element \Element{ZSect}), are
  defined to contain a \emph{reference} to the abstract \Element{Para}
  element.  This allows any subtype of \Element{Para} to be used
  instead.  This has the same extensibility advantages as subtyping
  in object-oriented languages.

  A Z extension can add new kinds of paragraphs, expressions and predicates,
  simply by extending these ZML inheritance hierarchies.  It is
  important to note that this can be done without modifying the
  ZML schema file.  Instead, the Z extension creates a new XML schema
  which \emph{imports} the original ZML schema file, then defines the
  additional constructs using a new namespace.
  This means that several separate extensions of Z can easily coexist.
  For example, the XML schema for Object-Z imports
  the ZML schema file, and defines a new paragraph for classes (element
  \Element{ClassPara}) that is derived from element \Element{Para}
  defined in the ZML schema.  Instance documents of the Object-Z
  schema can now contain class paragraphs in addition to the standard Z
  paragraphs wherever an element \Element{Para} is expected.  Thus,
  an Object-Z specification in XML format can contain a mixture of
  Z and Object-Z constructs, such as:
\begin{small}
\begin{verbatim}
    <Z:ZSect>
      <OZ:ClassPara> ...  <Z:True/> ... </OZ:ClassPara>
    </Z:ZSect>
\end{verbatim}
\end{small}

  This process of extending the XML schema can be done multiple times, so
  that even a Z extension can be extended.  For example, the
  additional elements provided by the Object-Z XML schema are further
  extended by the TCOZ XML schema.  Again, the definitions of the elements
  for TCOZ are encapsulated into a TCOZ XML schema file, and the ZML and
  Object-Z XML schemas do not need to be modified.
  Similarly, the \Circus\ extension for CZT is encapsulated into a
  \Circus\ XML schema file that extends the main standard Z schema.
  This approach of extension via inclusion is explored throughout the
  different layers of CZT tools.
  The resulting net effect is that once one package is finished, it
  can be directly extended through inheritance, hence simplifying the
  task of extending standard Z to a great extent.

  The use of XML in CZT has proved to be an efficient and extensible
  solution for representing a Z specification and its extensions.  The
  XML approach helps to clarify design decisions in a straightforward
  fashion.  This representation is the key for the integrated
  development and exchange of information among different Z tools.

\section{Java AST Classes}\label{java-ast-classes}

  To manipulate Z \emph{Annotated Syntax Trees} (AST) within Java (or
  any other programming language), we must convert ZML files into Java
  objects.  This could easily be done using one of the Java XML
  reader/writer libraries, such as DOM, but this would result in a
  very generic interface to the Java objects --- to the programmer they
  would appear to be an N-ary tree of \texttt{Element} and
  \texttt{Text} objects. This does not accurately reflect Z
  syntax or semantics, is not elegant, and is error-prone to use.

  Instead, we provide a customised Java interface for each Z
  construct, with appropriately named \emph{get} and \emph{set}
  methods for each subcomponent.  This makes programs more readable,
  and provides much stronger typechecking.  However, there are some
  situations where the generic N-ary tree view is more convenient
  (for example, writing a deep copy procedure), so our Java interfaces
  also provide a low-level generic view of each node, via the following
  two methods:
\begin{small}
\begin{verbatim}
    Object[] getChildren();  // return all children of this node
    Term create(Object[] args); // create a new version of this node,
                                // with the given children.
\end{verbatim}
\end{small}
  Having these two alternative views of each node of the AST gives
  the best of both worlds --- one can write generic tree traversal
  algorithms using the above two methods, as well as readable and
  type-safe Z-specific syntax manipulations using the node-specific
  get and set methods.

  In fact, these CZT Java AST interfaces and their implementation
  classes are automatically generated from the XML schemas described
  in the previous section using our code generator \emph{GnAST}
  (GeNerator for AST).  The generated code looks similar to the code
  produced by Java data binding tools like
  JAXB\footnote{See~\url{http://java.sun.com/xml/jaxb/}} or
  Castor\footnote{See~\url{http://www.castor.org/}}.
  While the main purpose of a Java binding tool is to
  provide the ability to convert from XML format to Java objects and
  vice versa, the main purpose of GnAST is to generate well-designed
  AST classes.  For example, the AST classes generated by GnAST
  support an extensible variant of the visitor design
  pattern~\cite{GamEA:95,MaiCha:01}.

  The automatic AST generation from the XML schemas dramatically
  reduces the time required to develop a new Z extension, ensures a
  common style of interface, and improves maintainability.  For
  instance, the complete AST folder representing standard Z contains
  around $420$ Java files.  GnAST has also been used to generate AST
  interfaces and classes for Object-Z, TCOZ, and \Circus.  In total,
  from the four XML schema files for standard Z and its extensions,
  GnAST automatically generates around $2300$ Java files.  This
  provides a very convenient and consistent way to obtain AST interfaces
  and classes for Z extensions that fit well into the AST for standard Z.

  The visitor design pattern~\cite{GamEA:95,MaiCha:01} makes it very
  easy to write tools like typecheckers and printers, which need to
  traverse an AST.  It allows new traversal operations to be defined
  without modifying the AST classes.  To define a new operation, all
  one needs to do is to implement a new visitor class.

  The visitor design pattern used in CZT has been described in detail
  in~\cite{czt}.  It is a variant of the \emph{acyclic
  visitor}~\cite{Mar:97} pattern and the \emph{default
  visitor}~\cite{Nor:97} pattern.  Its additional advantages over the
  standard visitor pattern are that it allows the AST interfaces and
  classes to be extended without affecting existing visitors, and that
  it allows a visitor to take advantage of the AST inheritance
  relationships.  For example, a copy visitor that copies an AST can
  provide a default behaviour for \Interface{Term}, the base of the
  AST inheritance hierarchy.  Since AST classes for extensions also
  derive from \Interface{Term}, this copy visitor works for any
  extension.  On the other hand, if the default copy behaviour is not
  wanted for a particular extension class, say \texttt{XYZ}, one can
  simply add a \texttt{visitXYZ} method to the copy visitor, and that
  method will be used instead of the default \texttt{visitTerm}
  method.

  This has a big impact on the applicability of visitors for
  extensions like Object-Z, TCOZ, and \Circus.  Firstly, it ensures
  that the Z AST classes can be extended without having to modify
  existing Z visitors like typechecker, printer, \textit{etc}.
  %
  Secondly, it makes it easy to extend existing visitors
  to handle Z extensions --- one can simply define a new visitor class
  which inherits behaviour from an existing Z visitor and adds a few
  methods for the new or changed language constructs.
  %
  Finally, by defining default behaviours for abstract classes such as
  \texttt{Expr} or \texttt{Decl}, it is possible to implement tools
  that are applicable to all Z extensions.

  In conclusion, the CZT AST classes provide:
  \begin{description}
    \item[A Choice of Coding Style:] One for \textbf{generic low-level} algorithms
      and the other for \textbf{node-specific high-level} algorithms.
    \item[Automation:] the AST classes are generated automatically by
      \textbf{GnAST}.
    \item[Reuse of Algorithms:] The CZT \textbf{visitor pattern} allows
      AST traversal algorithms to be reused and extended in flexible ways.
    \item[Extensibility:] the standard Z AST can easily be extended by
      defining new \textbf{XML schemas}.
  \end{description}


\section{Parsers, Scanners, and Related Tools}
\label{parsers}

  CZT includes a suite of important tools for operations such as
  parsing, typechecking, and markup conversion. In addition to a
  parser and typechecker for Z, an Object-Z parser is provided, and
  \Circus\ and TCOZ parsers, as well as an Object-Z typechecker, are
  under development.  The Object-Z, TCOZ, and \Circus\ tools extend
  the Z tools by adding support for the additional constructs these
  languages provide.  As each language is an
  extension of Z, it is tempting to just keep adding to the tools
  for each extension, and use the largest superset of all
  extensions. For example, use the TCOZ tools to parse and typecheck
  Z. However, this has two distinct problems. Firstly, one aim of the
  CZT project is to create tools that strongly conform to the Z
  standard. However, allowing extra constructs to be parsed and using
  different type-rules will break the strong conformance. Secondly, the
  extensions of Z are not linear. For example, Object-Z extends Z with
  class paragraphs, and TCOZ extends Object-Z with concurrency
  operators, but \Circus\ extends neither of these --- only
  Z. Therefore, CZT requires an approach that produces separate tools
  for each Z extension, maximises the commonality between the parsers,
  and minimises versioning and maintenance problems via reuse.

\subsection{Parsers and Scanners}

  CZT includes parsers for standard Z specifications given either in
  Unicode or \LaTeX\ markup.  Support for email markup is planned.  Java
  Cup\footnote{See~\url{http://www.cs.princeton.edu/~appel/modern/java/CUP/}}
  is used to generate the CZT parsers from an LALR grammar, and
  JFlex\footnote{See~\url{http://jflex.de/}} is used to generate the
  scanners.

  Unfortunately, it is quite difficult to reuse code from an
  automatically generated scanner or parser, and neither Java Cup nor
  JFlex explicitly supports inheritance for parser or scanners
  respectively.  To avoid duplicated code, XML templates that contain
  the different parser and scanner variants are used. From this, the
  different source files for each Z extension are generated using
  XSLT\footnote{See~\url{http://www.w3.org/TR/xslt}}, a language for
  transforming XML documents.  This maximises the commonality between
  the parsers and minimises versioning and maintenance problems.

  All parser and scanner variants are maintained in master XML files.
  Each master file contains several XML tags that are used for
  substituting text for each Z extension. For example, the {\tt
  <package/>} tag is placed wherever one would normally write the Java
  package name, so that each parser and scanner can be contained in
  their own package. The tags {\tt <add:}{\em extension}{\tt >} and
  {\tt </add:}{\em extension}{\tt >} are used to wrap around code that
  are specific to particular Z extensions. Thus, to add a new type of
  expression to the Object-Z parser, one would add a new production to
  the appropriate grammar rule in the master file, and place it
  between the {\tt <add:oz>} and {\tt </add:oz>} tags. In other
  programming languages, conditional compilation could be used to
  achieve the same result. However, as Java does not support
  conditional compilation, we use the XML template translation
  approach.

To generate the individual Java Cup files for each extension of Z,
XSLT is used to include the necessary code, and to substitute in
values for tags. For example, to generate the Object-Z parser, XSLT is
applied to the master file, and supplied with the three arguments
below:
\begin{enumerate}
  \item {\tt "class"} substituted with {\tt "Parser"}.
  \item {\tt "package"} substituted with {\tt "net.sourceforge.czt.parser.oz"}.
  \item code in {\tt "oz"} tags to be included.
\end{enumerate}

Similar rules are specified for each parser and scanner variants. The
result is a series of Java Cup and JFlex files, one for each language,
which can then be used to generate the parser and scanner code.

The use of XML templates enables parsing code to be reused and easily
maintained.  Extending the parser and scanner for a new language can
be done by just adding the respective grammar and lexer rules together
with few modifications such as those parameters above.  For example,
we are experimenting the incorporation of the available \Circus\
parser~\cite{circus.other:parser} rules within the flexible CZT
framework. The obvious advantages are the widely tested and supported
standard Z classes, \LaTeX\ markup and Unicode, visiting and other
facilities.

\subsection{Multiple Markups}\label{multiple-markups}

 CZT supports multiple markups for each Z extension.  The different
 markup languages suit different communities.  For example, \LaTeX\ is
 preferred by researchers, while Unicode WYSIWYG editing might be more
 attractive for students or industrial users. At present, Unicode,
 \LaTeX, and the XML format are supported.  Adding additional markups
 is straightforward, as this section will present.  XML markup is not
 considered any further because it can be parsed immediately using
 existing XML parsers.  CZT uses
 JAXB\footnote{See~\url{http://java.sun.com/xml/jaxb/}} to unmarshal an
 XML document into a tree of Java objects, and then uses the visitor
 design pattern to convert this tree into an AST.

In order to avoid having to provide a parser for each markup language,
all specifications are first translated into Unicode and subsequently
parsed by a Unicode parser\footnote{See~\cite{czt} for a more detailed
description of the parser architecture.}.  This also makes sure that
names in the AST are markup independent:~they are represented in
Unicode independently on the actual markup used in the source
document.  This is a necessary precondition of allowing different
sections of a specification to be written in different markups.  If a
parser for a new markup is required, only a translator to Unicode
needs to be implemented.

A consequence of this architecture is that extensions of Z need
to support at least Unicode.  CZT provides a Z Unicode scanner, which
performs lexical analysis on a Unicode stream and breaks it into the
necessary tokens.  A scanner for a Z extension can be derived by
adding additional scanner rules to the CZT scanner template as
described above.  In order to support \LaTeX\ markup, it is convenient
to provide a \LaTeX\ toolkit section for a given extension that
defines new operators for that language.  In
addition to defining new operators, these \LaTeX\ markup documents
contain \LaTeX\ markup directives~\cite{isoz,czt} used to specify how
certain \LaTeX\ commands are to be converted into Unicode.  The
\LaTeX\ to Unicode translator parses these definitions and converts
each \LaTeX\ command into the corresponding Unicode sequence.
However, \LaTeX\ \verb+\begin{xxx}+ and \verb+\end{xxx}+ environments
cannot be defined using \LaTeX\ markup directives.  If a Z extension
needs to provide new \LaTeX\ environments, the \LaTeX\ to Unicode converter
needs to be adapted directly.  Again, this is possible by adding new
rules to the converter template file.

An additional benefit of this approach is that it reduces the number
of converters needed between languages. That is, CZT currently
implements \LaTeX\ to Unicode and Unicode to \LaTeX\ converters. In
the future, we plan to implement an email to Unicode converter to
allow parsing of specifications written in email. Using this and the
Unicode to \LaTeX\ converter, we could convert email to \LaTeX. So,
using an intermediate format reduces the number of converter tools
that need to be implemented from $M*(M-1)$ to $2*(M-1)$, in which $M$
is the number of markup languages supported.

In conclusion, CZT supports extensions to parser and scanners using:
\begin{description}
  \item[XML Templates for Code Sharing:] XML templates are used
  to maximise code reuse for the parser and scanner scripts.
  \item[Unicode as an Intermediate Format:] Unicode is used as an
    intermediate format to simplify the process of writing scanners
    and reduce the number of converters needed between markups.
\end{description}


\section{Typecheckers}
\label{typecheckers}

Typecheckers in CZT are written in a different way from the parsers
and scanners. Each Z extension has its own typechecker, and while reuse
is of high importance, using XML templates is unnecessary because
unlike the parsers, Java interfaces and inheritance can be used to
extend the typecheckers.

The Z typechecker is the base implementation.  When a Z specification
AST is passed to this typechecker, it applies all the typechecking
rules and, if the specification is type-correct, it returns TRUE and
annotates the original AST with type information as defined in the ISO
standard~\cite[Section~10]{isoz}.  If the specification contains type errors,
the result is FALSE, the AST is unchanged and a list of error messages
describing the type errors (including their line and column position)
is made available.

Most of the typechecker is written using visitors, which can be
extended as discussed in Section~\ref{java-ast-classes}.  While it is
tempting to write the typechecker as one large visitor, this would
create maintenance problems as this visitor would be quite large and
monolithic.  So we use a more sophisticated and extensible design,
shown in Fig.~\ref{fig:ztypechecker}.

\def\epsfsize#1#2{0.50#1}
\begin{figure}[t]
\begin{center}
\fbox{\epsfbox{ZTypeChecker.eps}}
\caption{UML class diagram for Z Typechecker}\label{fig:ztypechecker}
\end{center}
\end{figure}
\def\epsfsize#1#2{\epsfxsize}

This breaks up the overall task of typechecking into several
(currently six) smaller \texttt{Checker} visitors --- each subclass of
\texttt{Checker} typechecks a different kind of syntactic construct such
as paragraphs, predicates, expressions, \textit{etc}.  The \texttt{Checker}
class itself defines some shared resources, such as typing
environments and the type unification facilities, as well as common
``helper'' methods used throughout the implementation such as error
reporting.  In addition, each checker subclass object has a reference
back to the top-level \texttt{TypeChecker} object, which has links to
all the checkers --- this allows one checker to call another via the
\texttt{TypeChecker} object.

For example, for typechecking a schema text of an {\tt AxPara}, the
{\tt ParaChecker} class, which typechecks Z paragraphs, needs to
typecheck both the declarations and the predicate parts of the schema
text.  Although visiting through the given AST is the general
solution, the typechecking of the declarations part is delegated to
the {\tt DeclChecker} class, whereas the typechecking of the predicate
part is delegated to the {\tt PredChecker} class.  The {\tt
DeclChecker} in turn uses the {\tt ExprChecker} to ensure that
expressions defining the declaring variables type are
well-formed. Because each of these visitors share the same {\tt
TypeChecker} reference, and hence the same references to type
environments, the declarations added to the type environment by the
{\tt DeclChecker} will be accessible by the other checkers.

There are a few additional classes that are used in the typechecker,
but not shown in Fig.~\ref{fig:ztypechecker}, such as the
\texttt{UnificationEnv} class that performs
the unification of two types for type inference and for checking type
consistency.

The advantages of this typechecker design include:
\begin{itemize}

\item Methods that are common to all the \texttt{Checker} subclasses can be put in
  the \texttt{Checker} superclass.  Data that is shared between the checkers
  can be managed by the \texttt{Typechecker} class and made accessible
  to the checkers in a controlled way via access methods.

\item Splitting the overall typechecking task into several parts increases
  modularity and maintainability, and provides better encapsulation
  for the different checkers.  This aids debugging and allows development
  of the checkers to be somewhat independent (for example, assigned to
  different teams or to different iterations of an agile lifecycle).

\item Each \texttt{Checker} subclass is typechecking similar kinds of
  nodes (\textit{e.g.}, all expressions), so can have a uniform visiting
  protocol, which increases regularity and helps to reduce errors.
  For example, all the visitor methods of the
  \texttt{ParaChecker} class, which typechecks Z paragraphs, return a
  \texttt{Signature} of the name and type pairs declared in that
  \texttt{Para}.  In contrast, the \texttt{ExprChecker} class
  typechecks expression nodes and all its visitor methods return a
  \texttt{Type} with resolved reference parameters in which type
  unification has already been performed.

\item By defining several \texttt{Checker} subclasses over the same kinds of
  AST nodes, it becomes easy to have multiple algorithms over the same
  syntax nodes.  For example, post-checking for unresolved set and
  reference expressions, which may introduce an unresolved type, is
  implemented as a second kind of \texttt{ExprChecker}. This post-typechecking
  pass ensures that all implicit parameters such as generics actuals
  have been completely determined.  This would not be possible with a
  single monolithic visitor design, because one could not have two
  \texttt{visitRefExpr} methods in the same visitor.

\end{itemize}


\def\epsfsize#1#2{0.50#1}
\begin{figure}[t]
\begin{center}
\fbox{\epsfbox{OZTypeChecker.eps}}
\caption{UML class diagram for Object-Z Typechecker}\label{fig:oztypechecker}
\end{center}
\end{figure}
\def\epsfsize#1#2{\epsfxsize}

Fig.~\ref{fig:oztypechecker} shows how this design is extended to
define a typechecker for a Z extension --- Object-Z in this case.  A
new package (oz) is created for the Object-Z typechecker.  In this
package, a new {\tt oz.Checker} class is implemented, which inherits
the base {\tt z.Checker} class.  In this new class, any common methods
that are to be used by the Object-Z typechecker are implemented, and
existing methods are overridden or overloaded if additional
functionality is needed.  Then, new \texttt{Checker} subclasses are created,
one for each kind of language entity that requires Object-Z-specific
typechecking.  Each of these checkers (the
\texttt{oz.\emph{XXXX}Checker} subclasses in
Fig.~\ref{fig:oztypechecker}) implement the visitor methods only for
Object-Z constructs and for any Z constructs that require additional
Object-Z-specific checking.  The remaining standard Z constructs are
handled by delegation to the original
\texttt{z.\emph{XXXX}Checker} object.

It is interesting to see how this delegation is achieved, given that
Java does not support multiple inheritance.  We rely on the general
visiting protocol described in Section~\ref{java-ast-classes} and
in~\cite{czt}.  For example, the \texttt{oz.ExprChecker} class catches
all Object-Z-specific expressions.  It also implements an additional
\texttt{visitExpr} method which ``catches'' all remaining
\texttt{Expr} AST nodes and uses the visitor from
\texttt{z.ExprChecker} to check those nodes.
\begin{verbatim}
  private z.ExprChecker zExprChecker_;
  ...
  public Object visitExpr(Expr expr) {
    return expr.accept(zExprChecker_);
  }
\end{verbatim}

The Z typechecker has a reference to a \texttt{z.ExprChecker} object,
but in the Object-Z typechecker, this points to an \texttt{oz.ExprChecker}
instead.  When an Object-Z expression is typechecked, it is handled
directly by the \texttt{oz.ExprChecker} instance.  When a standard Z
expression is typechecked, the above \texttt{visitExpr} method
is called, delegating the typechecking to an instance of
\texttt{z.ExprChecker}.  Any subexpressions of the Z expression are
passed back to the top-level type\-chec\-ker, which uses the
\texttt{oz.Expr\-Chec\-ker} instance, to ensure that Ob\-ject-Z subexpressions
are checked correctly.

This also allows type-rules to be overridden. For example, a
selection expression, $a.b$, in standard Z requires that $a$ is a
schema binding, whereas in Object-Z, $a$ can also be an object. The
{\tt ExprChecker} in the Object-Z implements the visit method for such
expressions, and this method first checks if $a$ is an object, and if
not, delegates the call to the Z typechecker.


Although this is an unusual design, it has proven to provide good
and elegant support for extension.  An alternative approach that we
considered was for the Object-Z checkers to directly subclass the Z
checker subclasses (\textit{e.g.}, \texttt{oz.ParaChecker} to inherit
\texttt{z.ParaChecker}).  However, this would have meant that the
common code implemented in the current
\texttt{oz.Checker} class would have had to have been
implemented in the base {\tt Checker} class, which would have resulted
in an undesirable strong coupling between all of the typecheckers.

Other components are extended using inheritance.  For example, the
class \texttt{UnificationEnv}, which is responsible for type
unification, is extended by overriding its {\tt unify} method to
handle the new Object-Z types, while using the superclass's {\tt
unify} method for standard Z types.

Our experience is that the above extensible typechecker design makes
it much easier to build multi-lingual typecheckers.  That is, a family
of typechecker objects for Z and various extensions of Z.
%
For example, a static checker for Circus that checks some
context-sensitive rules such as variable and action declaration scope
has been developed following the guidelines for Z and Object-Z
typecheckers. This took only three to four days to develop and the task was
made significantly easier because of the code reuse and elegant
object-oriented design of the CZT typechecker.  The information
collected by this static checker is being used as an initial
environment for the \Circus\ operational
semantics~\cite{circus.mc:opsem}.  In the future, this static checker
can also be used as the basis for a full \Circus\ typechecker;~the
type-rules for \Circus\ are under development in~\cite{circus.other:typechecker}.
An obvious advantage of reusing the base Z typechecker is that the \Circus\ typechecker
will already enforce standard Z typechecking conformance.  Therefore, one can
concentrate on the implementation of new type-rules for \Circus\ in
this available skeleton for a \Circus\ typechecker.

In conclusion, CZT supports extendibility in its typecheckers by:
\begin{description}
 \item[Using Multiple Visitors]: A separate visitor is used for each
  group of type-rules;~this provides a straightforward way to implement
  type-rules for new constructs (by adding new visitors), or override
  existing type-rules (by subclassing existing visitors).
 \item[Sharing Common Code via Inheritance and Delegation:]
   Methods used throughout the typechecker
   are shared in several abstract super classes that are reused
   via both inheritance and delegation.
 \item[Sharing Resources:] The {\tt TypeChecker} class is used by
   visitors to provide access to common resources and to other visitors.
\end{description}

\section{Animation}\label{animation}

    Further to parsing and typechecking standard Z and its extensions,
    CZT also provides animation facilities with its ZLive tool.  Z
    animation is particularly useful for testing, rapid prototyping,
    and experimenting with specifications.  In addition, given
    suitable restrictions to finite state models, an animator can be
    used for finite theorem proving (or theorem testing), and model
    checking.  An extensive discussion and comparison of Z animation
    tools available is given in~\cite{utting-jaza}.

\subsection{Extending ZLive}

    ZLive is an animator capable of evaluating predicates and
    expressions using mode analysis~\cite{winikooff98}.
    Mode analysis consists of including additional (type and formulae
    ordering) information not present in specifications, which enable
    evaluation and animation.
    The architecture of ZLive is an evolution of a previous Z animator
    implemented in
    Haskell\footnote{See~\url{http://www.cs.waikato.ac.nz/~marku/jaza/}}.

    The ZLive architecture is divided into six tasks.  Firstly, a
    target expression is given. Secondly, the definitions are unfolded
    so that schema inclusions are grounded to base terms. Next, the
    unfolded definitions are flattened into a normal form of atomic predicates.
    After that, possible evaluation modes are calculated for each flatten
    predicate.  These moded-predicates are then reordered according to the
    cheapest solution order in terms of number of solutions.  Finally,
    all solutions are lazily enumerated as requested.

    ZLive currently supports basic logic and arithmetic operators
    (\textit{e.g.}, $\forall$, $\exists$, $\lnot$, $\land$, $-$,
    $+$, $*$, $\leq$, $<$, $\mathtt{div}$, $\mathtt{mod}$, $\mathtt{succ}$),
    set representations (comprehension, ranges, and displays), unfolding
    of simple definitions, tuples, and schema bindings.
    For efficient execution, the main issue is to find a good reordering
    of atomic predicates which minimises the expected enumeration time.
    Currently ZLive uses a naive algorithm for this, but in the future we
    expect to implement a best-first or $A^{*}$ path-finding algorithm.

    It is desirable to provide animation facilities for Z extensions
    as well as for standard Z.  To extend ZLive to animate a new Z
    extension, there are three possible approaches:
    \begin{description}

      \item[Explicit Inclusion:] Animation support for each new
      language construct, including any new evaluation algorithms, is
      directly added to ZLive by add\-ing new Java
      classes and methods.  This would use interfaces, inheritance
      and visitors to achieve an extensible architecture, similar to
      the CZT typechecker.

      \item[Transformation to Standard Z:] If each new construct of
      the Z extension can be transformed back into standard Z
      using rewriting rules, then ZLive can be used directly
      on the result of that translation.
      This approach is being used to develop an Object-Z
      animator, with Object-Z objects being
      transformed into Z bindings, \textit{etc}.  This approach of
      rewriting specifications is similar to the Z refinement
      calculus~\cite{z.others:ana.phd,fm.ref:morgan}.

      \item[Meta-Level Animation:] If the operational semantics of
      the new language can be given in standard Z, one can use ZLive
      directly to animate the new Z extension by animating its
      operational semantics.  Although this is a meta-level
      approach to execution, which usually results in very slow
      performance, the performance impact should be less in this case,
      because any standard Z constructs within the Z extension can
      animated efficiently and directly by ZLive.  That is, only the
      new constructs have to be animated by the slower, meta-level
      approach.  This approach is taken for animating the operational
      semantics of \Circus~\cite{circus.mc:opsem}.

    \end{description}

    Depending on the new language constructs to be animated, these
    possibilities can be combined.

\subsection{Extension Example: Animating \Circus}

    We are currently experimenting using ZLive within the development
    of a model checker for \Circus~\cite{circus.mc:leo}. Among other aspects,
    we are particularly interested in integration of model checking and
    theorem proving facilities for \Circus. In this direction, animation plays
    an important part in the evaluation of Z terms used to describe state
    aspects of dependable and distributed systems.

    The \Circus\ model checker architecture is divided into four main
    tasks as shown in Fig.~\ref{mc-stages}.  The first two involve
    parsing a \Circus\ specification in \LaTeX\ to produce an CZT AST,
    and typechecking to produce an annotated AST$_{+}$.  They use the
    CZT parser and typechecker described in earlier sections.  The
    last two stages involve compilation and refinement search.  From
    the annotated AST$_{+}$ the compiler builds a \textit{Predicate
    Transition System} (PTS) that finitely represent (possibly
    infinite specifications) base on the operational semantics of
    \Circus~\cite{circus.mc:opsem}. Both the PTS and the AST$_{+}$ are
    given to the model checker engine that integrates refinement model
    checking algorithms~\cite{csp.mc:main,csp.mc:bisequiv} together
    with theorem proving and debugging
    functionalities\footnote{See
    \url{http://www.cs.york.ac.uk/circus/model-checker}}.
    The result is a (possibly empty) set of witnesses representing
    failed refinement conditions. More details of this architecture
    can be found in~\cite{circus.mc:leo}.  \begin{figure}[t]
    \begin{center} %\epsfig{clip=, scale=0.9, file="mc.arch.stages.eps"}
    \end{center} \caption{\Circus\ Model Checking
    Stages}\label{mc-stages} \end{figure}

    In this architecture, ZLive is used from two different
    perspectives:~(i)~to animate the Z part of \Circus\ specifications, and
    (ii)~to evaluate the operational semantics of \Circus\ given in Z,
    while performing the model checking search.

    To implement the first perspective, we are extending ZLive via
    \emph{direct inclusion} of several Z constructs (like $\theta$ and
    some schema operators) that are frequently used in \Circus\
    specifications but not yet implemented by ZLive.  To implement the
    second perspective, we are using the \emph{meta-level animation}
    approach to animate the operational semantics of the CSP parts of
    \Circus.

    The \emph{transformation to standard Z} approach could also be
    used to animate the CSP parts of \Circus.  To have confidence in
    the correctness of this approach, it would be desirable to have
    correctness proofs for the rewriting laws.  As \Circus\ is heavily
    based on the notion of stepwise refinement, this transformation
    approach would fit nicely with the philosophy of \Circus.  Work in
    this direction of a refinement calculus for \Circus\ is under
    development~\cite{circus.ref:marcel}. It also includes the basis
    for a \Circus\ theorem prover~\cite{circus.sem:pp}.

    The theorem proving module in the \Circus\ model checker (which is
    used both in the compiler and refinement engine), dispatches
    requests for evaluation of Z expressions and predicates. These are
    either verification conditions over the state operations defined
    in Z, or possible enabling paths available for investigation from
    the behavioural actions given using CSP.  They are both given as
    standard Z statements from the operational semantics of \Circus.
    At this point, theorem proving is usually necessary to discharge
    proof obligations, and transform expressions or
    predicates. Nevertheless, for specifications with simple state
    operations, animation is also a good idea that could improve the
    automation levels of the model checking process.

    The role ZLive plays in this scenario is to tackle the requests to
    evaluate Z expressions and predicates from the theorem proving module
    within the compiler and refinement engine.  As the operational semantics of
    \Circus\ is given in Z itself, we can use ZLive as a meta-level
    animator for simple specifications, hence enabling automatic model
    checking of state-rich \Circus\ specifications.

    With a few improvements and extensions to the current implementation of
    the schema calculus in ZLive, it should be possible to automatically
    model check simple-state \Circus\ specifications within ZLive.
    Furthermore, as the theorem proving integration architecture of the
    \Circus\ model checker allows pluggable solutions suitable for individual
    contexts, if ZLive cannot handle some complex \Circus\ specifications,
    we can still resort to some alternative solution such as SAT solvers,
    and general-purpose theorem provers.

    These \Circus\ tools, some of which are currently under development,
    give some good examples of how to integrate different CZT
    tools across different notations and tool boundaries, from standard
    Z parsing through to extended typechecking and animation for \Circus.


\section{Specification Manager}
\label{section-manager}

  One of the core components of the CZT framework is the
  \emph{specification manager}, which is an extensible repository for
  formal methods objects.  Most of the tools mentioned in the previous
  sections use the specification manager to enquire about specific
  aspects of a specification.  For example, to be able to parse a Z
  section, the Z parser needs the operator definitions of the parent
  sections.  In order to typecheck a Z section, the section must be
  parsed and the parents of that section typechecked.  To print a Z
  section in \LaTeX\ markup, the operator definitions and \LaTeX\
  markup directives of the parent sections are needed.

  While it would be possible to hard-code these dependencies and let,
  for example, the \LaTeX\ markup printer call the parser for the
  parent sections directly, it is more convenient, extensible, and
  efficient to have a central repository that is responsible for this
  task.  The CZT specification manager caches information about all
  the specifications and Z sections that are being processed and
  automatically runs tools such as markup converters, parsers and
  typecheckers when necessary.  The caching of the parsed form of
  commonly used objects, such as standard toolkit sections, avoids
  repeated parsing and analysis of these objects and can give
  significant performance improvements.

  Abstractly, the cache is a mapping from a key to the actual data.
  The key is a \texttt{(String, Class)} pair, where the \texttt{String} is usually
  the name of the section, and the \texttt{Class} is the Java class of the
  type of data associated with this key.  This allows several
  different kinds of objects to be associated with one section, and
  provides some dynamic type security.  For example, the Z parser adds
  the AST of a specification it has parsed to the specification
  manager.  The type of a Z section in Java is
  \Interface{ZSect.class}. Thus the AST for a section called
  \texttt{foo} is cached under the key
  \texttt{(``foo'',~ZSect.class)}.

  The CZT specification manager supports two important kinds of
  extensibility:
  \begin{description}
  \item[Type Extensibility:] Z extensions can easily use the
    specification manager to store new types of information, since the
    flexible \texttt{(String, Class)} key system allows arbitrary Java objects
    to be stored and retrieved.  That is, the kinds of objects managed
    by the specification manager are open-ended, rather than being a
    fixed set of Z-related objects.
  \item[Command Extensibility:] A Z extension can easily add or
    override the \emph{default commands} of the specification manager.
    The default commands of the specification manager are responsible
    for automatically computing requested objects;~they are
    implemented using the command design pattern~\cite{GamEA:95}. For
    example, if the AST for section \texttt{foo} (\textit{i.e.}, data
    of type \Interface{ZSect.class}) is required and has not already been
    cached, the Z parser is called by the specification manager in
    order to parse the specification file containing section
    \texttt{foo}.  Here, the Z parser is the default command to
    compute data of type \Interface{ZSect.class}.  A Z extension that
    needs to use a different parser can simply override the default
    command associated with the type \Interface{ZSect.class}.  For
    example, the specification manager can be configured to always use
    the Object-Z parser.
    \end{description}

  A major advantage of this default command approach is that it
  simplifies tool development and makes tools more flexible, because a
  particular tool does not have to know which other tools to use in
  order to find information about a section --- it simply requests the
  key that it wants and the specification manager will locate the
  information if it is able.  This gives a more flexible,
  \emph{plugin} style of tool development.

\section{Related Work}\label{sec:related-work}

  Integrated formal methods frameworks have been investigated in the
  past. Anderson \textit{et al.}\ \cite{uitp-anderson} discuss a framework for
  integrating different formal methods tools. However, their aim is to
  specify generic interfaces to support integration of formal methods
  tools. Three types of interfaces are used: between the engineer and
  the tools; between cooperating tools; and between the tools and the
  project environment. They achieve this by using an {\em
  Encapsulation Toolkit} to allow a formal methods tool to communicate
  with other components in an intermediate format, and an {\em Active
  Document Toolkit} to allow communication between tools and their
  human users. The goals of this project are different to CZT, which
  aims to provide components for Z tools that can be extended and
  integrated into the project or other tools.

  Brillant\footnote{See \url{https://gna.org/projects/brillant}.}  is
  an open-source project with similar aims to CZT, but for the B
  method. It aims to integrate several existing projects (BCaml,
  jBTools and ABTools), which all contain parsers and typecheckers for
  various dialects of B.  Brillant is an approach to integrating these
  tools in a loosely-coupled style, with tools being written in
  several different languages (OCaml, Java and XSLT) and communicating
  via a common XML format for B machines.  Brillant includes a
  translator from UML to B, plus some experimental B extensions (Event
  B and a real-time extension of B based on the duration calculus),
  but the extensions seem to be designed on an individual basis,
  rather than being tightly integrated extensions of a core
  architecture like in CZT.  The extensible architecture of CZT, and
  of course, the consistent use of Java for writing the
  tools, enables a higher degree of reuse.

  Other formal methods toolkits exist, such as the RODIN
  project\footnote{See \url{http://rodin-b-sharp.sourceforge.net/}.}
  for the B specification language, and the Overture
  toolset\footnote{See \url{http://www.overturetool.org/}.} for VDM,
  but they focus on providing specific tool support for their
  respective languages, whereas CZT aims to provide extensible
  components that can also be used by other tools.

  Projects such as Eclipse\footnote{See
  \url{http://www.eclipse.org/}.}  and UQ* \footnote{See
  \url{http://www.itee.uq.edu.au/~uqstar/}.} are projects aimed at
  providing generic language-based environments for software
  development. However, these projects are not tailored towards formal
  methods, and provide support for generic languages, leaving the
  development of parsers, typecheckers, and other language-specific tools
  up to users who want such support. CZT is exactly the opposite of this, in
  that it focuses only Z and various Z extensions, allowing specific
  components, such as parsers and typecheckers, to be included within
  the framework. Therefore, CZT could be integrated into the Eclipse
  or UQ* environments.

\section{Conclusions and Future Work} \label{sec:conclusions}

  In this paper, we have presented a variety of reuse and extensibility
  mechanisms that makes the CZT framework an ideal starting point to
  develop new integrated formal methods tools for Z and its
  extensions.  We have shown how the XML schemas for Z, and for
  extensions of Z, support reuse and extension of the Z language.
  They also enable automatic generation of Java AST classes with two
  levels of interface, and a consistent implementation of the CZT visitor
  pattern.

  Using examples from Object-Z, TCOZ, and \Circus, we have discussed
  several practical strategies and techniques that allow the CZT tools
  like parsers, typecheckers, and animators developed for standard Z
  to be reused within these Z extensions in a way that minimises code
  duplication and maintenance.  The strategies and techniques
  presented can also help developers of integrated formal methods
  tools not based on Z to make their framework as extensible as
  possible.

  We plan to develop additional tools for Z and its extensions,
  as well as extending the CZT framework itself.
  For instance, extensions of ZLive providing Object-Z constructs,
  schema unfolding, predicate reordering, rewriting rules, and
  a tactic language in the spirit of ANGEL~\cite{z.others:angel} are
  on our agenda. These improvements would enable a basis for an extensible
  theorem prover for standard Z and its extensions that is open-source and
  cross-platform.

  Z parsing and typechecking is neither a novel idea, nor a
  unavailable resource.  Nevertheless, flexible and integrated
  open-source support for ISO standard Z heavily focused
  on strong conformance and extensibility has not previously been available.
  The \textbf{philosophy CZT advocates is simple: provide an
  open-source framework with a set of tools for editing, parsing,
  typechecking and animating formal specifications written in Z, with
  support for Z extensions}.  As new extensions are included and the
  framework matures, we expect it to become the common base for a
  growing number of strongly conforming tools for Z and its extensions.

\bibliographystyle{plain}
\bibliography{ifm}

\end{document}
