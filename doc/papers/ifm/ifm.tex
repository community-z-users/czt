\documentclass{llncs}
\pagestyle{headings}   % turn on page numbers

\begin{document}
\title{CZT support for Z extensions}
\author{Leo Freitas \and Tim Miller \and Petra Malik \and Mark Utting}
\maketitle

\begin{abstract}
  An integrated framework for supporting multiple dialects of a formal
  notation.
\end{abstract}

\section{Introduction} \label{sec:intro}

  Z\cite{isoz} is ...

\section{XML Schemas and AST classes}

  \begin{enumerate}
    \item Gnast generation of ASTs from hierarchy of XML schemas.
      (similar to other Jaxb projects?)

    \item Benefits of having XML representation of each language
      (AST interchange with other tools, in other languages).

      \begin{itemize}
        \item[LEO]
        The old Circus parser has a series of problems in the Z part that the CZT XML classes and parser have solved.
        Z/Eves files are also defined in XML and I've got the internal documentation and I am used with it (good for comparison).
        It may be an interesting example on how the idea of XML seems to be the way forward.
      \end{itemize}

    \item Visitor pattern that supports multiple hierarchies.
      \begin{itemize}
        \item[LEO] Here I can include the architecture of the Circus Compiler that uses
                   two Visitors, one for each semantic function.

                   The multiple hierarchies are really helpful as it allows me to implement
                   the compilation rules piecewise.
      \end{itemize}
  \end{enumerate}

\section{Parsing}

  \begin{enumerate}
    \item Multiple parsers and scanners (xml templates)
        \begin{itemize}
            \item[LEO] Here Tim or I could include the basics for setting up and extending parsers.
        \end{itemize}

    \item Multiple markups for each language.
      (suits different communities:  Latex for researchers,
      Unicode WYSIWYG editing for students and industry etc.)
      \begin{itemize}
         \item[LEO] Worthwhile mentioning performance penalties as the section manager,
                OpTable, and DefinitionTable, are quite slow.
      \end{itemize}
  \end{enumerate}

\subsection{Section Manager}

  (is independent of Z/Object-Z/Circus etc.
  Can easily be extended with new kinds of objects, commands)

    \begin{itemize}
        \item[LEO] The SectionManager commands archiecture is a clever solution I believe worthwhile explaining.
    \end{itemize}

\section{Type Checking}

      The (future!) modular design, using interfaces and visitors,
      allows the core Z typechecker to be extended separately
      to handle Object-Z and Circus.  (TODO!)

    \begin{itemize}
        \item[LEO] I can include my experiences in extending the Z:TypeChecker for Circus.
                    I should also discuss with Tim some of the ideas for decoupling the type checker.
    \end{itemize}

\section{Animating}

    Animation of Circus using Z animator???
      (Operational semantics of Circus are in Z itself, so this makes a meta-level animation possible...)
    This could set the scenario for the possibilities CZT gives towards integration.

    \begin{itemize}
        \item[LEO] I can write here about the Circus refinement engine that uses ZLive, as well as the compiler.
    \end{itemize}


\section{Other Tools}

    \begin{itemize}
        \item[LEO]
        There is another tool under development that translates Circus (Latex) specifications into Java code.
        It uses JCSP (a java csp-occam library) that allows execution of Circus.
        In the Z part, due to the lack of tools, extreme simplifications (via data refinement in the original Circus)
        are needed.
        This tool could also use ZLive to be able to animate/execute (run!) more expressive Circus specifications quite
        automatically, or at least minimise the data refinement effort required for the translation
    \end{itemize}


\section{Conclusions and Future Work} \label{sec:conclusions}

\section*{Acknowledgements}

\bibliographystyle{splncs}
\bibliography{ifm}

\end{document}
