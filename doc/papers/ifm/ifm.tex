\documentclass{llncs}
\pagestyle{headings}   % turn on page numbers

\begin{document}
\title{CZT support for Z extensions}
\author{Leo Freitas \and Tim Miller \and Petra Malik \and Mark Utting}
\maketitle

\begin{abstract}
  An integrated framework for supporting multiple dialects of a formal
  notation.
\end{abstract}


\section{Introduction} \label{sec:intro}

  Z\cite{isoz} is ...

\section{XML Schemas and AST classes}

  \begin{enumerate}
    \item Gnast generation of ASTs from hierarchy of XML schemas.
      (similar to other Jaxb projects?)

    \item Benefits of having XML representation of each language
      (AST interchange with other tools, in other languages).

    \item Visitor pattern that supports multiple hierarchies.
  \end{enumerate}

\section{Parser, Typechecker and other Tools}

  \begin{enumerate}
    \item Multiple parsers and scanners (xml templates)
    \item Multiple markups for each language.
      (suits different communities:  Latex for researchers,
      Unicode WYSIWYG editing for students and industry etc.)
    \item Example: how a Circus visitor can reuse/extend a Z visitor
      by overriding/adding a few methods.
    \item Typechecker(s).
      The (future!) modular design, using interfaces and visitors,
      allows the core Z typechecker to be extended separately
      to handle Object-Z and Circus.  (TODO!)
    \item Animation of Circus using Z animator???
      (Operational semantics of Circus are in Z itself,
      so this makes a meta-level animation possible...) 
  \end{enumerate}

\section{Section Manager}

  (is independent of Z/Object-Z/Circus etc.
  Can easily be extended with new kinds of objects, commands)

\section{Conclusions and Future Work} \label{sec:conclusions}

\section*{Acknowledgements}

\bibliographystyle{splncs}
\bibliography{ifm}

\end{document}
