\documentclass{llncs} %from http://www.springer.de/comp/lncs/authors.html
\pagestyle{headings}   % turn on page numbers for the reviewers
\parskip 1ex plus 3pt
\parindent 0pt

%\usepackage{newcent}
\usepackage[dvips]{graphicx}

% poor man's version of Standard Z macros
\usepackage{oz}
\newcommand{\relation}{\mbox{relation }}
\newcommand{\varg}{~\_~}

\usepackage{alltt}
\usepackage{url}
\newcommand{\Zeta}{Zeta}
\newenvironment{OOWEZ}{\textbf{OOWEZ Note: }}{\textbf{End Note.}}
\newenvironment{IMPL}{\textbf{Implementation Note: }}{\textbf{End Note.}}

\newcommand{\TODO}[1]{\textbf{TODO: #1}}   % For draft version
% \newcommand{\TODO}[1]{}    % For final version

\title{A Z Specification of the CZT Section Manager}
\author{Mark Utting  % \inst{1}
        \and Petra Malik  % \inst{1}
}
\institute{The University of Waikato, Hamilton, NZ\\
        \email{\{marku,petra\}@cs.waikato.ac.nz}
\\[2ex] $ID: Sat 16 Oct 2004$ % Just while working on the paper...
}

\begin{document}
\maketitle

% abstract should be 70-150 words.  This is 260+.
\begin{abstract}
  This paper gives a specification of the Community Z Tools
  (CZT) \emph{section manager}, which is the main control system of the
  CZT system.  The section manager manages all the specifications and Z
  sections that are being processed; records dependencies between them;
  finds imported sections and toolkits; automatically runs tools such as
  markup converters, parsers and typecheckers when necessary; reruns them
  after input files have been updated; and is extensible to support
  user-defined commands etc.  

  The paper is important for several reasons.  
  %
  Firstly, because it describes the design goals of the section manager and
  the rationale for those goals.  This will be useful for CZT users.
  %
  Secondly, because it develops the proposed design in stages, starting
  from the well-known \emph{Chain of Responsibility} design pattern.  This
  clear description of the design and its rationale will help developers of
  CZT tools, who will be using the section manager.
  %
  Thirdly, many other formal methods tools and software engineering tools
  need similar functionality to this section manager, so the designs and
  alternatives described here will be of interest to the designers of those
  tools.
  %
  Fourthly, the specification is written in standard Z, using the
  \emph{`Object Orientation without Extending Z'} (OOWEZ) style, so the
  specification is an interesting case-study of how this style can be used
  to specify complex systems and design patterns.
\end{abstract}

\section{Introduction}

TODO: explain CZT.

TODO: explain the role and need for a section manager.

TODO: give outline of the rest of the paper.


\section{Goals for the Section Manager}

\begin{description}
\item[LoadSpec:] Given a filename (URL) of a specification, it should be 
  able to read and parse the specification, so that the whole
  specification is available as an annotated syntax tree (AST).
\item[MultiSect:] If a specification contains multiple (named) sections,
  the section manager should make all the sections globally available,
  since they may be used by other specifications.
\item[MultiData:] Given a section name, it should be able to provide a
  variety of different kinds of information about that section, including
  the AST, the \LaTeX\ markup table, the operator table, the type
  environment etc. 
\item[FindToolkit:] It should be able to find the standard toolkit sections
  automatically, but also allow new or variant toolkit sections to be used.
\item[FindUserSection:] If a section specifies parent sections that are not in
  the file being loaded and are not standard toolkits, the section manager
  should be able to search several locations to find and load the desired
  sections.  The search algorithm should be customisable to a reasonable
  degree (at least by specifying a list of directories to search). 
\item[AnonSections:] It should allow several anonymous sections
  (specifications that contain Z paragraphs, but with no named sections)
  to be handled at the same time.  Rationale: it should be possible
  to write tools that compare several specifications (which may contain
  just anonymous sections).
\item[Markups:] It should handle all the available markup formats for Z,
  such as Unicode (UTF8 and UTF16), \LaTeX\ and Email markup.
\item[Dialects:] It should support all the available dialects of standard Z,
  such as Object-Z and TCOZ.
\item[Caching:] It should cache commonly used sections, such as the
  standard Z toolkit sections, so that they are not repeatedly
  parsed and analysed.  This is important for efficiency.
\item[Dependencies:] It should provide dependency tracking, so that when a
  source file is updated, the objects derived from that source file are
  also updated.  This implies that each object managed by the section
  manager must have a timestamp and a set of dependencies. 
\item[GUI independent:] It should be usable by both command-line and graphical
  user interfaces. 
\item[Extensible:] It should offer extensible facilities, so that it can
  easily support future Z tools that manipulate new kinds of data.
\end{description}

The \emph{criteria} that we used to evaluate proposed
designs was, firstly, the simplicity and elegance of the design,
and secondly, how many of the above goals the design satisfied.
We rejected several designs that did not satisfy key goals.

\section{Notes}

\begin{enumerate}
\item Overwriting a section after transforming it is bad, for toolkits.
\item Reparsing is quite different to transforming.  Reparsing MUST
  overwrite the old versions somehow.  (eg. when section "foo" is
  moved from one file to another).
\item ChainAll is difficult because it relies on naming conventions
  and is difficult to make output of one command be the input of the
  next (eg. write then read an XML file).  More flexible to write
  sequences of commands in Java (or Javascript).
\item IDEA: associate a section manager with each file and
  with the output of each transformation command.  When re-parsing
  a file, start with a fresh section manager!
\item JEdit interface could maintain mapping from buffer/file names 
  to the associated sectman.
\end{enumerate}

\section{Presentation of Design}

Here is the proposed design, developed as a series of extensions of a 
standard design pattern, and specified in Z (of course!).
In fact, it is specified using the \emph{`Object Orientation without
  Extending Z'} style (OOWEZ)~\cite{utting:oo-z03} (pronounced \emph{How
  Easy}, but with a silent `H').
In this style, objects are modeled as black boxes (members of a given set),
subclasses are modeled as subsets, an attribute of type $T$ defined
in a class $C$ is simply a global total function ($C \fun T$) and
methods are modeled as functions or relations (for example $meth : C \cross
\num \rel C$ corresponds to the update method \texttt{void meth(int arg)}
within class $C$).

\subsection{Chain of Responsibility}

We start with a variation of the \emph{chain of responsibility} design
pattern from the \emph{`Gang of Four'} book~\cite{gamma:design-patts95},
which is also one of the Apache Jakarta Commons
projects\footnote{See \url{http://jakarta.apache.org/commons/chain}.}. 
The key aspects of this design pattern are that \emph{Commands} are
first-class objects, which operate on \emph{Context} objects (which are
Key to Value mappings), and that commands can be composed into
\emph{Chains} of commands.  Also, a chain of commands is itself
a command (this is the \emph{composite} design pattern), which allows
chains to be nested and composed in interesting ways.

In our application, we split the Context object into \emph{volatile}
and \emph{preserved} entries.  The volatile entries are purely for
passing arguments to commands and between commands, so are discarded after
the execution of each top-level command.  The preserved entries persist
after the command execution, so are managed by the section manager.
% TODO: rationale?  obvious?

We start the Z specification by introducing most of the basic
sets of objects we need.  $Command$ is the set of all possible
commands (including chains), $Key$ is the set of keys, $VValue$ is the set
of volatile values, $PValue$ is the set of preserved values and $SectMan$
represents section managers).  These sets are defined as Z given sets, which
means that nothing is known about the internal structure of these
objects--they are black boxes.  As the specification develops,
structure will be introduced incrementally to these objects, by adding
attributes (access functions) and methods (functions and relations, defined
axiomatically).
\begin{zed}
  [Command, Key, VValue, PValue, SectMan]
\end{zed}

The context is passed to commands as two separate arguments, one
is the section manager (which contains the preserved entries) and the other
is a map (called $Args$) containing the volatile entries.  The $Args$
is just a simple map, so we define it via equality in Z.  On the other
hand, the section manager will contain other things in addition to the
preserved entries, so we leave it as a black box, and simply define an
attribute of $SectMan$ to give access to the preserved entries map.
\begin{zed}
  Args == Key \ffun VValue
\end{zed}

\begin{axdef}
  content : SectMan \fun (Key \ffun PValue)
\end{axdef}

We also specify $put$ and $remove$ operations for adding and removing
content.  The $\finset Key$ argument to add will be explained in
Section~\ref{sec:dependency}.
\begin{axdef}
  put : SectMan \cross Key \cross PValue \cross \finset Key \fun SectMan\\
  remove : SectMan \cross Key \fun SectMan
\where
  (\forall sm,sm':SectMan; k:Key; val:PValue; deps:\finset Key \\
  | put~(sm,key,val,deps) = sm' @ \\
  \t1 content~sm' = content~sm \oplus \{key \mapsto val\} ) 
\also
  (\forall sm,sm':SectMan; k:Key; val:PValue; deps:\finset Key \\
  | remove~(sm,key) = sm' @ \\
  \t1 content~sm' = \{key\} \ndres content~sm ) \\
\end{axdef}
Note that these are total functions, so are deterministic and
have $true$ preconditions.  The only result they return is the
modified $SectMan$, so the corresponding Java method will be void
and cannot throw exceptions.

Each command provides a boolean \emph{execute(SectMan,Args)} method,
which may modify the section manager and/or the $Args$.
For example, a parser might get a URL from $Args$,
open that URL and parse its contents, then add 
the resulting AST to the section manager.  Each command
has three possible kinds of results.
\begin{zed}
  Result ::= trueResult | falseResult | exceptionResult
\end{zed}
\begin{itemize}
\item a \texttt{true} result means that the command has
  completed the whole task it was given.
\item a \texttt{false} result indicates that the command
  could not perform the desired task, so some other (perhaps following)
  command should be tried.  Nevertheless, the command may have
  made some changes to the section manager or $Args$, in order
  to help following commands in some way.
\item an exception result means that the command has detected
  some serious error and the whole execution should be aborted.
  Note that $exceptionResult$ is an abstraction of the set of all Java
  exceptions.
\end{itemize}

Here is the specification of the main \emph{execute} operation
offered by all commands.  It may be non-deterministic, so we declare
it as a relation. 

\begin{axdef}
  execute : Command \cross SectMan \cross Args 
       \rel (SectMan \cross Args \cross Result)
\where
  (\forall cmd:Command; sm:SectMan; args: Args @ \\
  \t1 (cmd,sm,args) \in \dom execute) \\
  (\forall cmd:Command; sm,sm':SectMan; args,args' : Args; res : Result\\
  | (cmd,sm,args) \mapsto (sm', args', res) \in execute @ \\
  \t1 args \subseteq args')
\end{axdef}

The first axiom specifies the precondition of $execute$, which is
always true (because commands throw exceptions if something goes wrong
or they are given bad arguments).  The second axiom specifies the
postcondition of $execute$, which says that commands are allowed to
add entries to $Args$, but are not allowed to delete or modify
existing entries.  This restriction makes it easier/possible to rerun
commands (see Section~\ref{sec:dependency}).  For similar reasons,
note that the declaration of the execute method does not allow the
$Command$ to change itself (that is, the \emph{frame} of execute is
just the three output values $(SectMan \cross Args \cross Result)$).

This relates to the comments in the Jakarta Commons documentation for
Command: \emph{``Command implementations should be designed in a
  thread-safe manner, suitable for inclusion in multiple Chains that might
  be processed by different threads simultaneously. In general, this
  implies that Command classes should not maintain state information in
  instance variables. Instead, state information should be maintained via
  suitable modifications to the attributes of the Context that is passed to
  the execute() command.''}

% TODO: also provide a $precondition$ method which gives a 
% quick check to see if the command is likely to work???

Next we lift this $execute$ method up to the $SectMan$ level,
to specify exactly how commands can update the section manager,
and what happens when a command throws an exception halfway
through its execution.  In this case, it is important not to
leave the section manager in an inconsistent state.  For example,
consider a command that transforms all the sections within a
Z specification.  If it throws an exception while transforming the
last section, we do not want to leave the earlier sections transformed
while the last section is not transformed.  To avoid such problems, we
require the section manager to keep a log of all changes made to it,
and to undo partial changes when a command throws an exception.
This makes command execution atomic -- either the whole command
succeeds and updates the section manager, or it throws an exception,
which leaves the section manager unchanged.

\begin{axdef}
  update : SectMan \cross Command \cross Args 
       \fun (SectMan \cross Args \cross Result)
\where
  (\forall cmd:Command; sm,sm':SectMan; args,args':Args; res':Result\\
  | (sm, cmd, args) \mapsto (sm', args', res') \in update @ \\
  \t1 (\exists tmp:SectMan 
           | (cmd,sm,args) \mapsto (tmp,args',res') \in execute @\\
  \t2   (res' = exceptionResult \land sm' = sm \lor \\
  \t2   \: res' \neq exceptionResult \land sm' = tmp)))
\end{axdef}


\subsection{Building More Complex Commands (Chains)}

In this section we introduce Chains (sequences of commands),
with two subclasses: $ChainAll$ always tries to execute
all its commands, whereas $ChainFirst$ stops after the first
command that succeeds (returns $trueResult$).  These are similar to the
$\forall$ and $\exists$ quantifiers in Z.

Supporting both kinds of Chain allows very flexible commands
to be constructed, such as:  TODO better example
\begin{verbatim}
   ChainAll( findFile, 
             ChainFirst(parseZ, 
                        parseOZ,
                        parseTCOZ), 
             unfold )
\end{verbatim}

$Chain$ is a subclass of $Command$, and contains a sequence of $Command$.
\begin{axdef}
  Chain : \power Command
\end{axdef}

\begin{axdef}
  commands : Chain \fun \seq Command
\end{axdef}

$ChainAll$ and $ChainFirst$ are disjoint subclasses of $Chain$.
The axioms define the behaviour of executing each kind of chain.
\begin{axdef}
  ChainAll, ChainFirst : \power Command
\where
  ChainAll \cap ChainFirst = \{\}
\end{axdef}

To help us define the behaviour of the $execute$ method of
these chains, we first define an auxillary function $executeSeq~cmds~res$,
which executes a sequence of commands, $cmds$, while they return the
expected result $res$.
%% TODO make executeSeq and execute infix operators for readability.
%% or   method * Inputs 'gives' Outputs
\begin{axdef}
  executeSeq : \seq Command \cross Result \\
  \t1  \fun SectMan \cross Args \rel (SectMan \cross Args \cross Result)
\where
  (\forall cmds:\seq Command; expect:Result; \\
  \t1   sm,sm':SectMan; args,args':Args; res':Result \\
  | (sm,args) \mapsto (sm',args',res') \in execute~(cmds,expect) @ \\
  \t1  (cmds = \langle~\rangle \implies 
          sm'=sm \land args'=args \land res'=expect) \\
  \t1  (cmds \neq \langle~\rangle \implies \\
  \t2    (\exists sm_2:SectMan; args_2:Args; res_2:Result @ \\
  \t3      (head~cmds,sm,args) \mapsto (sm_2,args_2,res_2) \in execute \\
  \t3      ( res_2 = expect \land \\
  \t3      \ \ (sm_2,args_2) \mapsto (sm',args',res') 
                 \in executeSeq~(tail~cmds,expect) \\
  \t3      \lor res_2 \neq expect \land 
               (sm',args',res') = (sm_2,args_2,res_2)))))
\end{axdef}

Note that both $exceptionResult$ and a non-expected result
cause the execution of the chain to terminate early.  
Using this function, we can easily define the $execute$ method
for both kinds of chains.
A $ChainAll$ expects all commands to return $trueResult$, so stops after
the first command that returns $falseResult$ or throws an exception.
A $ChainFirst$ is searching for the first command to return $trueResult$,
so executes all the initial commands that return $falseResult$, then stops
after the first command that returns $trueResult$ or throws an exception.
\begin{axdef}
\where
  (\forall chAll:ChainAll; sm,sm':SectMan; args,args':Args; res':Result \\
  | (chAll,sm,args) \mapsto (sm',args',res') \in execute @ \\
  \t1  (sm,args) \mapsto (sm',args',res') 
          \in executeSeq~(commands~chAll,trueResult) )
  \also
  (\forall chFirst:ChainFirst; sm,sm':SectMan; args,args':Args; res':Result \\
  | (chFirst,sm,args) \mapsto (sm',args',res') \in execute @ \\
  \t1  (sm,args) \mapsto (sm',args',res') 
          \in executeSeq~(commands~chFirst,falseResult) )
\end{axdef}

\begin{OOWEZ}
  Notice how the power of our specification language has allowed
  us to specify the behaviour of $execute$ on chains very precisely.
  In fact, equivalent to the following Java code in class $ChainAll$:
\begin{verbatim}
  public boolean execute(SectMan sm, Map args) throws Exception
  {
    boolean result = true;
    Iterator<Command> i = commands.iterator();
    while (result && i.hasNext())
        result = i.next().execute(sm,args);
    return result;
  }
\end{verbatim}
  That is, we have stated that the commands will be executed left-to-right, 
  that the side-effects of the commands will accumulate and 
  that false results and exceptions will be returned immediately.
\end{OOWEZ}

% The Jakarta Commons documentation says: \emph{``Chain implementations
%   should be designed in a thread-safe manner, suitable for execution on
%   multiple threads simultaneously. In general, this implies that the state
%   information identifying which Command is currently being executed should
%   be maintained in a local variable inside the execute() method, rather
%   than in an instance variable. The Commands in a Chain may be configured
%   (via calls to addCommand()) at any time before the execute() method of
%   the Chain is first called. After that, the configuration of the Chain is
%   frozen.''}


\subsection{Multiple Types of Data per Section}

Next we extend keys to be $(String,Type)$ pairs, so that
one key name can have many different types of data associated with it.
For example, a section name might have an AST, a \LaTeX\ markup
table, an operator table etc.  We use Java class objects to
represent the type of information that is stored with a given key, 
and require the value associated with that key to be an \emph{instance}
of that type.

\begin{zed}
  [String, Class, Object]
\end{zed}

\begin{axdef}
  instanceOf : Object \rel Class
\end{axdef}

\begin{axdef}
  keyName : Key \fun String \\
  keyType : Key \fun Class
\where
  (\forall k,k':Key @ k=k' \\
  \t1   \iff keyName~k = keyName~k' \land keyType~k = keyType~k') \\
\end{axdef}

The next axiom ensures that, after looking up a key $(N,C)$,
the resulting value can always be cast into a class $C$ object.
\begin{axdef}
  value : PValue \fun Object  
\where
  (\forall sm:SectMan; k:Key | k \in \dom(content~sm) @ \\
  \t1 (value(content~sm~k), keyType~k) \in instanceOf)
\end{axdef}

Note that $Key$ is also used by $Args$.
We impose the same type restrictions on the $VValue$ entries
of $Args$, but do not show those here.
TODO: make the Args keys just ordinary strings.
% Because we cannot take a subset without changing the original defn
% of Args!


\subsection{Support for Anonymous Sections}

Typically, a key string will be some user-defined name for
a specification.  A typical convention might be to use the 
base part of the file name, say \texttt{"foo"}.  
When the parser parses this specification (perhaps from the file
\texttt{"foo.tex"}), it adds the entire AST of the specification
to the context under the key \texttt{("foo",Spec.class)}.
If this specification contains one or more explicitly named sections
(for example, \texttt{"foo"} and \texttt{"bar"}),
it also \emph{explodes} the specification by adding these
individual sections to the context (under the keys
\texttt{("foo",ZSect.class)} and \texttt{("bar",ZSect.class)}).
This reflects the fact that section names are in a single global 
namespace in standard Z, and allows dependent specifications to
find these sections by name.

An anonymous specification \texttt{"foo2"}, does not contain named
sections, just a sequence of paragraphs, which is translated into
a single default section called \texttt{"Specification"}.
In this case, the \emph{explode} operation does not take place,
so the only key entry that is added is \texttt{("foo2",Spec.class)}.  
Thus it is possible for the section manager to contain many
anonymous specifications at the same time (each under a different
user-defined name).

Note that most of the Z tools should be capable of operating
on a whole specification (given a key such as \texttt{("foo",Spec.class)})
or on a single section (given a key such as \texttt{("foo",ZSect.class)}).


\subsection{Default Commands}

It is useful for Section Managers to know how to perform some basic 
operations (such as finding, parsing and typechecking a toolkit section),
without having to supply an explicit command.  This allows user-defined
commands to be simpler, because they can simply request a certain key and
the section manager will compute the desired result itself.  This also
makes user-defined commands independent of the basic commands used to
perform parsing and typechecking etc., which means that those commands
can more easily be changed.  For example, an Object-Z user might wish
to replace the default Z parsing and typechecking commands by commands
that handle Object-Z.  

The Apache Jakarta Commons implementation of chains supports this
kind of flexibility with a special lookup table called a \emph{Catalog},
which maps command names to commands.  We provide this same
functionality, but incorporate this default command catalogue within 
the section manager.  Each Java class or interface can be associated
with a default command for creating objects of that type.
The semantics is that whenever a key $(Name,Class)$
is requested, but is not already present within the section manager, then
$Class$ is looked up in the default command catalogue, and if it returns
a command $C$, then that command is tried with an argument map containing
just $\{\, \texttt{"name"} \mapsto Name)\, \}$.

\begin{axdef}
  default : SectMan \fun (Class \ffun Command)
\end{axdef}

TODO: add operations for adding and removing these default commands. 

TODO: add isAvailable(Class) to SectMan

NOTE: An alternative possibility is some kind of `services manager', like
  Apache Excalibur or the Apache Jakarta Commons Discovery 
  pattern.\footnote{See \url{http://jakarta.apache.org/commons/discovery}.}


Next we give a partial specification of a $lookup$ operation, which
extracts an entry from within the section manager or computes it
using a default command if possible.  We specify the precondition
of $lookup$ here, but defer the postcondition to the next section, because
$lookup$ interacts with dependency management. 

\begin{axdef}
  lookup : SectMan \cross Key \rel SectMan \cross PValue
\where
  (\forall sm,sm':SectMan; k:Key; res':PValue @
  \t1  (sm,k) \in \dom lookup) \\
\end{axdef}


\subsection{Dependency Management} \label{sec:dependency}

To support dependency management, we extend the section
manager to record for each entry the command and arguments
that created that entry, a timestamp for when it was created
and the set of other entries (keys) that it depends upon.

\emph{Note:} This is similar to the Units in Zeta.

\begin{zed}
  [TimeStamp]
\end{zed}

\newcommand{\before}{\ll}
%%Zinchar \before U+226A
\begin{zed}
  \relation (\varg \before \varg)
\end{zed}

\begin{axdef}
  \_ \before \_ : TimeStamp \rel TimeStamp
\end{axdef} 

\begin{axdef}
  createTime : SectMan \fun (Key \ffun TimeStamp) \\
  createCommand  : SectMan \fun (Key \ffun Command) \\
  createArgs  : SectMan \fun (Key \ffun Args) \\
  dependencies : SectMan \fun (Key \ffun (Key \ffun TimeStamp))
\where
  (\forall sm:SectMan @ \\
  \t1  \dom content~sm \\
  \t1   = \dom createTime~sm \\
  \t1   = \dom createCommand~sm \\
  \t1   = \dom createArgs~sm \\
  \t1   = \dom dependencies~sm \\
  )
\end{axdef}

These data structures allow us to check whether or not a given entry
in the context is up-to-date.

\newcommand{\upToDate}{\mathrel{\mbox{upToDate}}}
%%Zinword \upToDate upToDate
\begin{zed}
  \relation (\varg \upToDate \varg)
\end{zed}
\begin{axdef}
  \_ upToDate \_ : SectMan \rel Key
\where
  \forall sm:SectMan; k:Key @ sm \upToDate k \iff \\
  \t1 k \in \dom content~sm \land \\
  \t1 \exists ktime == createTime~sm~k @ \\
  \t2   \forall dep: \dom dependencies~sm~k @ \\
  \t3     ktime \before dependencies~sm~k~dep \land \\
  \t3     sm \upToDate dep
\end{axdef} 

Note that our dependency information points backwards,
from the newly created entry back to the entries that it
depends upon.  This is useful when we want to check that a
given entity is up to date -- we simply recurse through
the dependencies checking that they are older, as shown
in the definition of $\upToDate$.  But what if we notice
that an input file has changed and we want to invalidate
all the entries that depend upon it?  This is not currently
possible, unless we add forward-pointing dependency information.
However, we do not do this, because it is impractical. 
It is impossible for performance reasons to continually monitor all inputs
to see if any have changed.  If dependencies point backwards, then when a
specific section is requested, we can check that all the things it depends
upon are up-to-date, and recalculate it if necessary.  This `lazy' approach
is more efficient.  However, it can still be expensive to check the
entire tree of dependencies every time an entry is requested -- to give
good performance, a section manager should \emph{cache} the up-to-date
flags, by recording a \emph{up-to-date} timestamp on an entry each time
that is verified as being up to date.  Then subsequent $\upToDate$
checks can immediately return $true$ if the timestamp is within a
small delta (say, 1 second) of the current system time.  This avoids
excessive $\upToDate$ checking, but means that changes in input files
are noticed within one second.

TODO: prove that this delay cannot lead to inconsistencies.

TODO: discuss rerunning of commands.  


\section{Unresolved Issues}

TODO: remove this section after issues are all resolved.

\begin{description}

\item[How to name transformed sections?]  Must be the same as the
  original if other child sections want to use it, but must
  be new to avoid confusion with the original version and
  to keep dependencies working.

  \textbf{I think if someone wants to transform a whole set of
    sections (like one file that contains multiple sections), then
    they must systematically rename ALL those sections. Commands
    (perhaps like the typechecker) that only add annotations to
    existing sections will not be affected by this, however, such
    commands still have to be able to remove those annotations
    when some following command fails!}

\item[Concurrency?]  Do we want to allow different threads to
  make concurrent updates to a common section manager?  If so
  we would need to provide a transaction manager with rollback
  facilities, similar to the Apache Jakarta Commons (Sandbox) `Transaction'
  package.  
  
  Supporting concurrency seems unnecessary, since the section manager is
  being used by just one user, who is presumably performing one task at a
  time.  Perhaps if we wanted a GUI to calculate extra information (why
  would it need to do this?) while a long parse is happening, then
  concurrency might occur.  However, I think that our initial goal should
  be to make the individual operations fast enough (and with good progress
  messages) that the user does not mind waiting a few seconds for a parse
  to finish.  Also, note that the Jakarta Commons Chain package says
  that users should NOT assume that Contexts are thread safe (whereas
  Command and Chain should be).  Conclusion: we can ignore concurrency.
\end{description}


\section{Java Interfaces}

This section gives an outline of the Java interface of each of the classes
discussed above, ignoring dependency management for the moment.
Note that the $ChainAll$ and $ChainFirst$ classes both implement the
same $Chain$ interface, but their $execute$ methods behave differently.
They need lots more documentation and pre/post conditions, but these are
given informally above. 

\begin{figure}[htbp]
  \centering
\begin{small}
\begin{verbatim}
interface Command {
  boolean execute(Context sectMan, Map args);
}

interface Chain extends Command {
  /** Adds a command to the end of the chain. */
  void add(Command cmd);
}
\end{verbatim}
\end{small}
  \caption{Command and Chain Interfaces}
  \label{fig:command}
\end{figure}

\begin{figure}[htbp]
  \centering
\begin{small}
\begin{verbatim}
interface SectMan {
  /** Find/compute an entry within the section manager.
      Returns null if the section manager cannot find
      the requested object, and cannot compute it. */ 
  //@ ensures \result != null ==> \result instanceof type;
  Object lookup(String name, Class type);

  /** Only for use by Commands, during update().
      This can be used to overwrite existing entries.
  */
  //@ requires value instanceof key.type;
  void put(Key key, Object value, Set<Key> dependencies);

  /** Only for use by Commands, during update().
   */
  void remove(Key key);

  /** The main entry point for updating the section manager.
      This executes cmd, which may update the section manager.
      If it returns an exception, it must undo any partial changes it made.
  */
  boolean update(Command cmd, Map args);

  /** Set the default command for computing objects of class 'kind'.
   *  @return Returns the old command associated with 'kind' (null if none).
  */
  Command setDefault(Class kind, Command defaultCommand);
}
\end{verbatim}
\end{small}
  \caption{SectMan Java Interface}
  \label{fig:sectman}
\end{figure}


\section{Example}

[This needs updating, to use separate args Map etc.]

We now go through a small example, showing how the following
chain of commands would modify the context.
\begin{verbatim}
  FindFile; ParseLatex; TypeCheck; ZLive
\end{verbatim}

Initially, the context might contain just the entry
\begin{small}
\begin{verbatim}
  ("input",String) |--> ("foo", 100, {}, null)
\end{verbatim}
\end{small}
(To make things more concise, we write \texttt{C.class} as just \texttt{C}.
We show the range of each context entry as a (Value,Time,Dependencies,Command)
tuple). 

The FindFile command searches the filesystem for files called
"foo" with known extensions, and finds "foo.zed", so adds the
following entry to the context.
\begin{small}
\begin{verbatim}
  ("foo",URL)  |--> ("foo.zed", 101, {}, null)
\end{verbatim}
\end{small}

The Parser command uses these two \texttt{("input",String.class}
and \texttt{("foo",URL.class)} keys to determine which file
to parse, which markup to use (\LaTeX) and perhaps even which Z dialect
parser to use (standard Z).  Assuming that \texttt{"foo.zed"}
contains two named sections, \texttt{"foo"} then \texttt{"bar"}
(where \texttt{"bar"} has \texttt{"foo"} as a parent)
it adds the following entries to the context:
\begin{small}
\begin{verbatim}
  ("foo",Spec)    |--> (..., 103, {("foo",URL),...}, Parser)
  ("foo",ZSect)   |--> (..., 104, {("foo",Spec)}, Parser)
  ("foo",OpTable) |--> (..., 104, {("foo",ZSect)}, Parser)
  ("bar",ZSect)   |--> (..., 104, {("foo",Spec),("foo",OpTable)}, Parser)
  ("bar",OpTable) |--> (..., 104, {("bar",ZSect),("foo",OpTable)}, Parser)
\end{verbatim}
\end{small}
Note 1. The details of these dependencies depend upon how the
parser works internally.
Note 2, the $\ldots$ here would actually be a list of dependencies
upon the standard Z toolkit sections.

The Typechecker adds annotations to these three ASTs,
but also adds type environments for each section:
\begin{small}
\begin{verbatim}
  ("foo",SectTypeEnv) |--> (..., 105, {("foo",ZSect),...}, Parser)
  ("bar",SectTypeEnv) |--> (..., 105, {("bar",ZSect),("foo",ZSect)}, Parser)
\end{verbatim}
\end{small}

Finally, the ZLive animator creates some definition tables, then goes into
interactive mode reading commands and displaying the answers. 
\begin{small}
\begin{verbatim}
  ("foo",DefnTable) |--> (..., 106, {("foo",ZSect)}, DefnVisitor)
  ("bar",DefnTable) |--> (..., 106, {("bar",ZSect)}, DefnVisitor)
\end{verbatim}
\end{small}

Note that if one of the values associated with these keys is
required later, it is possible to follow the dependency links
back to the URL \texttt{"foo.zed"} and check its timestamp to
ensure that all the derived data structures are up to date.

If the \texttt{("curr",String)} key is changed (because the
user now wants to work on a different specification), the
various \texttt{"foo"} keys remain valid.

\bibliographystyle{plain}
\bibliography{utting,oop,proceedings}
\end{document}
