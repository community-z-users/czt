\documentclass{article}
\usepackage{oz}
\usepackage{czt}

% Extra general macros for this paper.
\newenvironment{Rationale}{\\ \textbf{Rationale:}\it}{}

% Extra math-mode macros for this paper.
% Some of these may eventually go into czt.sty
\newcommand{\V}{\mathcal{V}}
\newcommand{\Zc}{Z_C}


\title{Transforming Z with Rules}
\author{Mark Utting, Petra Malik and Ian Toyn}
\begin{document}
\maketitle

\begin{abstract}
This paper describes a simple extension of Z that allows transformation and reasoning rules to be written in a natural, Z-like notation.  We show that these rules can be used to define schema unfolding of the schema expressions in Z, as well as being useful for a wide variety of other Z transformations.  
\end{abstract}

\section{Introduction}

The main goal of the Z rule system described in this paper is
to provide a high-level notation for Z transformations, by
allowing rules to be written in Z syntax as much as possible.  Each
rule has a single conclusion and possibly several antecedents, so is a
horn clause.  Thus we have a high-level Prolog-like notation (but with
Z syntax) for defining Z transformations and automatic proofs.  This
should be powerful enough to define all the schema unfolding (so we do
not have to implement it by hand in Java!) as well as have many other
uses...
  \begin{Rationale}
    A high-level notation is more readable, much quicker
    to write and easier to maintain than writing transformations in
    a low level notation like Java.  It also allows better reuse of the
    knowledge implicit in the rules~\cite{DenningACM}.
  \end{Rationale}

To provide a Java \texttt{Deduction} interface for applying a rule or
set of rules to a given Z construct, and to provide several
implementations of this interface that can be parameterised by a set
of restricted rules (for example, equality rewrites).
  \begin{Rationale} 
  We want to use some sets of rewrite rules
  (example, schema unfolding) as important components of CZT, so we
  must have reasonably efficient implementations of those rule sets,
  and a uniform way of applying them.  It is useful to allow many
  different rewrite engines (implementations of the \texttt{Deduction}
  interface) because each engine may place different restrictions on
  the rules that it allows, may apply rules using a different
  semantics (apply once, apply exhaustively, apply to all subterms
  bottom-up or top-down etc.), and may use a different implementation
  technology (for example, an interpreter that applies the rules or a
  compiler that transforms the rules into Java code).
  \end{Rationale}

Other goals for the future are
to be able to support a variety of Z logics, such as
$Z_C$~\cite{henson:revising-z-1-99,henson:revising-z-2-99}, 
$\mathcal{V}$~\cite{brien:calculus-schemas-z00} (a successor of
the $\mathcal{W}$ logic that appeared in early drafts of the Z standard).
It is desirable to 
    support both $Z_C$ and $\mathcal{V}$ (and any future proposals for Z
    logics), since they are rather different and it would be interesting
    to compare them in a common framework.  For example, substitution
    is a meta-level operation in $Z_C$, but is defined within the object
    logic of $\mathcal{V}$.

Build a tactic layer on top of this basic theorem prover, using a
tactic language such as ANGEL~\cite{martin:tactics}.  This has been
used as the basis of several other Z provers (CadiZ, Jigsaw1, Jigsaw2,
Ergo).
  \begin{Rationale}
    An explicit proof tree allows proofs to be recorded, replayed,
    displayed, checked by independent tools etc.
    The non-determinism of ANGEL could easily be implemented as
    Java iterators returning alternative solutions.
  \end{Rationale}

In the future, we would also like to be able to translate the rules
into a form that Z theorem provers will accept, so that we can prove
rules correct -- to allow this, the more declarative the rules are the
better.

\section{Related Work}

\section{Z Rules}

This section describes the syntax and semantics of the Z rules.


\section{Example Rules: Schema Unfolding}

\section{Example Rules: Simplification}

\section{Conclusions}

\end{document}