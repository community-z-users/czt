\documentclass{article}
\usepackage{oz}
\newenvironment{Rationale}{\\ \textbf{Rationale:}\it}{}
\newcommand{\V}{\mathcal{V}}

% Rule syntax: \begin{zedrule}{Name} Ant1 \\ Ant2 \derive Concl \end{zedrule}
\newenvironment{zedrule}[1]{\par\textbf{rule }#1\vspace{-2ex}\infrule}{\endinfrule}
\newcommand{\derives}{\derive{}}

\newenvironment{zedjoker}[1]{\par$\bigstar$ #1\ }{}

\newcommand{\Context}{\Gamma}
\newcommand{\notfreein}{\mathbin{\backslash}}
\newcommand{\HasType}{\mathbin{:}} % \small\rlap{$\circ$}{\raisebox{1ex}{$\circ$}}}}
\parindent 0pt
\parskip 1ex plus 3pt

%% Define some extra macros for the Z standard.
%% These will eventually be defined in a czt.sty file.
\newcommand{\negate}{\mathop{-}}
\newcommand{\arithmos}{{A\!\!\!\!A}}   %% TODO: get a proper symbol
\newcommand{\substitute}{\mathrel{\curvearrowleft}}  % U+21B6
\newcommand{\proviso}{\raisebox{0.5ex}{${}_{\blacktriangleright}\ $}}%%or\star
%       %%Zprechar \proviso U+22C6    % small star, or
%       %%Zprechar \proviso U+25B8    % small solid triangle  |> )


\title{Proof/Rewrite Rules in CZT}
\author{Mark Utting}
\begin{document}
\maketitle

This document describes a proposal for writing proof rules
and rewrite rules within CZT, in a way that should allow both
the $Z_C$ and $\V$ logics to be supported, and also
allow Z tools to use sets of rewrite rules to transform Z 
specifications. 

\section{Goals}

The goals of this proposal are:
\begin{enumerate}
\item to provide a high-level notation for Z transformations,
  by allowing rules to be written in Z syntax as much as possible.
  Each rule has a single conclusion and possibly several antecedents,
  so is a horn clause.  Thus we have a high-level Prolog-like notation
  (but with Z syntax) for defining Z transformations and automatic proofs.
  This should be powerful enough to define all the schema
  unfolding  (so we do not have to implement
  it by hand in Java!) as well as have many other uses...
  \begin{Rationale}
    A high-level notation is more readable, much quicker
    to write and easier to maintain than writing transformations in
    a low level notation like Java.  It also allows better reuse of the
    knowledge implicit in the rules~\cite{DenningACM}.
  \end{Rationale}

\item to be able to support a variety of Z logics, such as
  $Z_C$~\cite{henson:revising-z-1-99,henson:revising-z-2-99}, 
  $\mathcal{V}$~\cite{brien:calculus-schemas-z00} (a successor of
  the $\mathcal{W}$ logic that appeared in early drafts of the Z standard),
  as well as being useful for defining sets of rewrite rules
  for custom purposes (such as unfolding schema operators
  or transforming predicates into DNF form).
  \begin{Rationale}
    CZT needs to be able to define sets of rewrite rules.
    By generalising this slightly, it seems possible to support more
    general logics like $Z_C$ and $\mathcal{V}$ and thus support simple
    reasoning about concrete Z specifications (but not schematic
    reasoning about arbitrary schemas -- this would be useful for proving
    derived inference rules etc., but is quite a lot more complex
    to support).  It is desirable to 
    support both $Z_C$ and $\mathcal{V}$ (and any future proposals for Z
    logics), since they are rather different and it would be interesting
    to compare them in a common framework.  For example, substitution
    is a meta-level operation in $Z_C$, but is defined within the object
    logic of $\mathcal{V}$.
  \end{Rationale}

\item to be able to translate the rules into a form that Z theorem 
  provers will accept, so that we can prove rules correct -- to allow this,
  the more declarative the rules are the better.

\item to provide a Java \texttt{Deduction} interface for applying a rule or
  set of rules to a given Z construct, and to provide several
  implementations of this interface that can be parameterised by a set
  of restricted rules (for example, equality rewrites).
  \begin{Rationale} 
  We want to use some sets of rewrite rules
  (example, schema unfolding) as important components of CZT, so we
  must have reasonably efficient implementations of those rule sets,
  and a uniform way of applying them.  It is useful to allow many
  different rewrite engines (implementations of the \texttt{Deduction}
  interface) because each engine may place different restrictions on
  the rules that it allows, may apply rules using a different
  semantics (apply once, apply exhaustively, apply to all subterms
  bottom-up or top-down etc.), and may use a different implementation
  technology (for example, an interpreter that applies the rules or a
  compiler that transforms the rules into Java code).
  \end{Rationale}

\item Build a tactic layer on top of this basic theorem prover,
  using a tactic language such as
  ANGEL~\cite{martin:tactics}.  This has been used as the basis
  of several other Z provers (CadiZ, Jigsaw1, Jigsaw2, Ergo).
  \begin{Rationale}
    An explicit proof tree allows proofs to be recorded, replayed,
    displayed, checked by independent tools etc.
    The non-determinism of ANGEL could easily be implemented as
    Java iterators returning alternative solutions.
    (Assuming that Andrew can find a student who is interested 
    in implementing ANGEL in Java!)
  \end{Rationale}
\end{enumerate}


\section{Rule Syntax}

My (Mark's) original proposal was to write rewrite rules within the
existing standard Z syntax, as conjectures:
\begin{zed}
  \vdash? E1 - E2 = E1 + (\negate E2)
\end{zed}
or as axioms:
\begin{axdef}
\where
  (\forall E1,E2:\arithmos @ E1 - E2 = E1 + (\negate E2))
\end{axdef}
Special syntax, such as the substitution operator $\odot$ of $\mathcal{V}$
can be declared as infix operators using the standard Z features.

However, this approach does not extend nicely to sequents and
inference rules, so I now think we need a special \emph{rule} construct
which is an extension of standard Z.  Allowing this to appear within
Z specifications means that the context of each rule is clear
(for example, which sections it relies on) and we can use the usual
mechanisms of sections and files to group rules into rulesets.
An additional (important!) advantage is that we can give names
to the rules.  These would probably not be used by the automatic rewrite
engines, but could be used by interactive theorem provers.

For example, the \emph{Funct} rule from $\mathcal{V}$ might be
written in \LaTeX\ as (we omit the $\Gamma \vdash$ prefixes
since they are the same in all sequents):
\begin{verbatim}
\begin{zedrule}{Funct}
  \exists_1 x:f @ x.1 = e \\
  x \notfreein e
\derives
  y=f(e) \iff (e,y) \in f
\end{zedrule}
\end{verbatim}

which might be displayed as:
\begin{zedrule}{Funct}
  \exists_1 x:f @ x.1 = e \\
  x \notfreein e
\derives
  y=f(e) \iff (e,y) \in f
\end{zedrule}

It is not completely clear to me exactly what we need to
allow in the left hand side of sequents for the $Z_C$ and $\V$
logics.  I propose that we initially leave the conclusion context implicit
in all rules and simply list any predicates and typing facts that
are \emph{added} in the antecedent sequents.  This means that the
left hand side of a conclusion sequent is always empty, while the
left hand side of each antecedent sequent can be either empty or
a comma-separated sequence of predicates and typing facts (Var:Type).


\subsection{Metavariables}

It is also necessary to declare which names are metavariables
(jokers) and which syntactic roles they play.
\begin{Rationale}
  Otherwise the parser/typechecker cannot easily know whether
  a name is meant to be a metavariable, or a reference to some
  global constant or toolkit definition.

  CADiZ has declarations similar to these, but places them
  within tactic definitions rather than globally.  Globally
  seems more appropriate when we want to write a file full
  of rules.
\end{Rationale}

I propose an additional Z construct for this (cf. the operator
declarations of standard Z), which declares the role of each metavariable
explicitly.  Here is an example (the first two lines are sufficient for
the \emph{Funct} rule):

\begin{verbatim}
\begin{zedjoker}{Name} x \end{zedjoker}
\end{verbatim}

\begin{zedjoker}{Name} x \end{zedjoker}
\begin{zedjoker}{Expr} e,f,y \end{zedjoker}
\begin{zedjoker}{Expr} e1,e2,E1,E2 \end{zedjoker}
\begin{zedjoker}{Type} T1, T2 \end{zedjoker}
\begin{zedjoker}{Pred} P,P1,P2,P3,Q \end{zedjoker}
\begin{zedjoker}{DeclList} D0,D1,D2,D3 \end{zedjoker}

It is allowable to declare the same name several times, provided
that all declarations of that name are for the same kind of joker.
All metavariable names must be declared without decorations.  

Implementation Note: the parser will record these declarations,
and use them to transform those names into the appropriate kind
of metavariable during parsing.  If we want to typecheck rules
(it is not clear to me yet if this will be possible!) the typechecker will
presumably also have to assign a type variable to each of these names
during typechecking.


\subsection{Side Conditions}

The tricky question about inference rules is what primitive
side conditions they use (for checking not-free-in conditions,
signature disjointness conditions etc.).  
For CZT, we want to provide sufficient side conditions
to support both the $Z_C$ and $\V$ logics, plus enough to
allow us to write inference rules for easily unfolding schema
expressions etc.  The following kinds of side-conditions seem
to be necessary:

\begin{enumerate}
\item Invoking the typechecker to check types and to calculate
  the base type of a given expression or predicate.
\item Calculating sets of names and declaration lists (see below).
\item Checking properties of those sets of names, such as membership,
  non-membership, subset and equality.
\item Manipulating declaration lists by filtering, hiding
  or renaming various fields (see below).
\item Substitution (replacing variables by expressions)
  within expressions and predicates.  In CZT, as in CADiZ,
  each variable has a unique id, so variable capture is not an 
  issue (except that when printing specifications, sometimes
  variables must be renamed to avoid ambiguity).  This means that
  substitution is just a simple syntactic replacement of names
  by expressions.  The $\V$ logic uses $b \odot e$ for substitution,
  where $b$ is a binding.  However, this is part of the $\V$ object
  logic -- for other logics and rules we want a meta-level substitution.
  I propose that we use $e \substitute b$, where b is a binding.
\item Checking not-free-in conditions (this is done by explicit
  inference rules in the $\V$ logic, but at the meta level by $Z_C$.
  It is probably useful/necessary in some of the CZT transformation
  rules too.
\item Searching for various kinds of paragraphs in the context
  (which includes the complete specification that is in scope).
\end{enumerate}

The following operations are useful for sets of names:
\begin{itemize}
\item A literal set of names, like $\{a,b,c\}$.

\item $\alpha~Spec$, $\alpha~Schema$.  These return the \emph{alphabet}
  (the set of names) declared by a sequence of paragraphs, or by
  a sequence of declarations.

\item $\cup, \cap, \setminus$ of sets of names.

\item Functions which extract the subset of a set of names
  that end with a given decoration.  These all have type $\power Names
  \fun \power Names$ (or maybe $NameList \fun NameList$?).  
  There are five basic functions, $inputvars$
  (for names that end with $?$), $outputvars$ (for $!$),
  $numberedvars$ (for $\_n$ where $n$ is any digit), $primevars$ (for
  $'$) and $undecorvars$ (for no decorations at all), which partition
  all the possible kinds of decorations.  In addition, for
  convenience, we provide the $initialvars$ function which returns the
  union of $undecorvars$ and $numberedvars$ -- this is typically used
  for finding the input state variables of schema compositions.

\item Some way of adding or removing a given decoration (like $'$) to all
  names in a set of names.  For adding a decoration we can just use
  the usual decorated expression syntax of Z.  For example, $s~'$ will
  return a new set of names containing all the names in the set $s$,
  with an extra prime decoration added to the name.  For removing
  decorations, I propose a $removedecor$ function, which strips
  exactly one decoration off each name (this will usually be applied
  to a set of names with a known, common, decoration).

\item A way of creating fresh variants of a name, $fresh~x$.
  This returns a new name with a similar textual name, but
  with a new internal id number.

\item Predicates over sets of names: $\in, \notin, \subseteq, =$.
\end{itemize}

The following operations are useful on declaration lists (schemas).

\begin{itemize}
\item $[Decls1 ; Decls2]$ gives the union of two declaration lists.
  This is undefined (ie. throws an exception, which causes the whole
  sidecondition to be false) if $Decls1$ and $Decls2$ are not signature
  compatible (that is, if some common names have different base types).
\item $[Decls] \filter (Names)$ restricts the declarations in $Decls$
  to those that declare names in the set/list $Names$.  Note that
  $Names$ can be a comma separated list of names, name jokers and
  sets of names expressions -- this is a little more general than the
  standard Z syntax of $\filter$ (but if we require the sets of names
  to be calculated in separate provisos, then the syntax can still
  be just a list of names -- some of which will be jokers).
\item $[Decls] \hide (Names)$.  Same as the above filter, but
  \emph{removes} the declarations whose names are in $Names$.
\item Substitution or renaming, to rename the fields of a schema
  (including occurences of those field names in the predicates).
\end{itemize}

I propose the following four kinds of side conditions.
These will be written in the same way as ordinary sequents
(this allows some logics to specify the context for each
side-condition), but their conclusions will be prefixed by
a special `keyword' \verb!\proviso! (displayed as $\proviso$,
or U+25B8 in Unicode) to improve readability and avoid syntactic ambiguity.
\begin{description}
\item[Typecheck:] $\proviso E:BT$, where $E$ is an expression and $BT$
  is a base type.  Note that $BT$ may contain jokers or be just one 
  joker, in which case, the typechecker must calculate the type rather
  than just checking it.  We also need some way of type checking
  a predicate -- $\proviso \{|P\}:\power []$ would work, or we
  could do something prettier like $\proviso P:Bool$?
\item[Calculate:] $\proviso Joker == Expr$, or $\proviso PredJoker==Pred$.
  This calculates the right hand side, then assigns it to the left hand
  side joker (which must be an Expr joker or a NameList joker
  in the first case, or a Pred joker in the second case).
  $Expr$ is either an expression containing any of the functions mentioned
  above for calculating sets of names or declaration lists (schemas),
  or a substitution of the form $e \substitute b$, where $b$ is an
  extended kind of binding (it may contain a pair $s==e$, where $s$
  is a NameList joker and $e$ is an ExprList joker and the two
  jokers have the same length lists -- this means many substitutions
  in parallel).
  $Pred$ is a substitution of the form $p \substitute b$, where
  $b$ is also an extended binding.
\item[Check:] $\proviso simplepred$, where $simplepred$ is an
  equality, membership, non-membership or subset predicate, as
  described above, or a not-free-in condition $x \notfreein term$.
  These simple decidable predicates give us a
  way of checking properties of set of names.
\item[Lookup:] $\proviso? name==expr$, or $\proviso? pred$, or
  $\proviso? name:expr$.  These search for a matching ConstDecl,
  Pred, or VarDecl in the context (which usually includes the ZSect
  that we are in the scope of).  If there are multiple matches
  (e.g., of $pred$), then they are returned one at a time by backtracking.

  \textbf{Note: this is the one that I am a little undecided about.
  This `search for assumptions' style is the same style of sequent
  calculus that the Ergo prover used, and means that the conclusion
  sequent is always implicit and simply extended by the antecedents.
  But a more traditional alternative (used partially by $\V$)
  is to allow the conclusion
  sequent to match predicates or paragraphs in the context,
  like $\Context_1, x==E, \Context_2 \vdash P$.
  This is more similar to how we plan to pattern match declarations
  (e.g., $[D_1;x==E;D_2]$) and also more flexible,
  since can be used to \emph{remove} the $x==E$ paragraph.
  However, it is also rather harder to implement in general,
  since the context includes the complete specification -- we
  do not want proof rules to start removing paragraphs from the middle of
  a Z section!  
  Also, one must decide if order is significant in conclusions like
  $\Context_1, x==E, \Context_2, y==F \vdash P$.
  }
\end{description}

For example, this rule unfolds a schema name $v$, by looking
up its definition.  It then uses the typechecker to check that 
the schema types are compatible.

\begin{zedrule}{UnfoldSchema2}
   \proviso? v == [D2|P2] \\
   \proviso [D1; D2; D3] : T
\derives
   [D1; v; D3 | P]  =  [D1; D2; D3 | P \land P2]
\end{zedrule}

Note: If we had used the other (more traditional) style of searching
for paragraphs in contexts, this would be written as
\begin{zedrule}{UnfoldSchema2}
   \Context_1; v==[D2|P2]; \Context_2 \vdash
      \proviso [D1; D2; D3] : T
\derives
   \Context_1; v==[D2|P2]; \Context_2 \vdash
      [D1; v; D3 | P]  =  [D1; D2; D3 | P \land P2]
\end{zedrule}



\subsection{Notes on the Rule Syntax}

Here are some notes and discussion points about the syntax of rules.

\begin{itemize}
\item 
  %The $\Context \vdash$ prefixes of all the sequents can be omitted 
  %in rules where it is the same in all antecedents as in the conclusion.
  If there are no antecedents, the horizontal line (\verb!\derives!)
  must be omitted.
\item Contexts can only appear in antecedents and are a comma-separated
  list of Z paragraphs.  They are usually empty, in which case the
  turnstile must be omitted too.  This means that the context is
  unchanged (i.e., the same as the context of the conclusion).
\item Commas are used to separate items within the context ($\Context$).
  ($\V$ uses $|$ to suggest that the left hand side adds variables
  and the right hand side is in the scope of those variables.  This
  is nice, but comma is more traditional).
\item Antecedents are separated by either semicolons or newlines
  -- using semicolons allows one to put several antecedents on
  one line, whereas using newlines puts them on separate lines.
  Note that semicolons and newlines are normally allowed \emph{within}
  predicates in Z, but this will not be allowed within rules 
  -- explicit conjunction $\land$ must be used instead.  
  The predicate conclusion of each antecedent 
  will be parsed using the `term' production of our Z grammar, which
  has a tighter precedence than semicolons or newlines.
\end{itemize}

Compare this with CADiZ syntax for a single sequent
(CADiZ does not have syntax for inference rules, since
it uses a fixed set of built-in inference rules):
\[ name == [Formals] Hyp \vdash? Concs \]
\begin{verbatim}
\begin{theorem}{name}
[Formals] Hyp \thrm Concs
\end{theorem}
\end{verbatim}


\subsection{\LaTeX\ markup of Joker and Rule definitions}

Here are the rules and definitions for translating \LaTeX\ markup
of joker and rule definitions into Unicode.  The easiest way
of making these markup definitions available to the parser is 
probably to put them all into a Z section called 'zedrule'
and require all sections that contain Z rules to include that
section as a parent.  This would also be a nice way of defining
the special proviso functions and relations, 
like $primevars$, $\substitute$, $\notfreein$ etc.

The \verb!\begin{zedjoker}{Kind}! macro is translated into
the Unicode character JOKER (U+2605, $\bigstar$ BLACK STAR)
followed by Kind, followed by space.
The \verb!\end{zedjoker}! macro is translated into the usual
Z ENDCHAR character (U+2029).

The \verb!\begin{rule}{Name}! is translated into
the Unicode character RULE (U+25A0, $\blacksquare$ BLACK SQUARE)
followed by Name, then followed by a space.
The \verb!\end{rule}! macro is translated into the usual
Z ENDCHAR character (U+2029).

The \verb!\vdash! within sequents is expanded into U+22A2 ($\vdash$ RIGHT TACK)
as defined in the Z standard (I've called this VDASH in the grammar,
as there does not seem to be a separate name for it in the CZT lexer 
at the moment).

The following \LaTeX\ markup commands define the translation
of the \verb!\derives! and \verb!\proviso! macros.
The \verb!\derives! macro, which is displayed as a horizontal line in \LaTeX,
is translated into U+2500 (BOX DRAWINGS LIGHT HORIZONTAL) -- I call this
character RULELINE in the following grammar.  
The \verb!\proviso! macro is translated into U+25B8 (BLACK RIGHT-POINTING
SMALL TRIANGLE), which I call PROVISO in the grammar.


\begin{verbatim}
%%Zinchar \proviso U+25B8
%%Zinchar \derives U+2500
\end{verbatim}

\subsection{Grammar of Joker and Rule definitions}

Here is a LALR grammar for the Unicode form of these two
new kinds of paragraphs, which both extend \verb!boxedParagraph!
in the CZT grammar.
\begin{verbatim}
/* CZT joker declarations */
jokerPara ::= JOKER name nameList END ;
\end{verbatim}

\begin{verbatim}
/* CZT proof rules */
rulePara ::=
        RULE name
            optAntecedents
            conclusion
            END
        ;

optAntecedents ::=
        antecedents RULELINE
        | /*empty*/
        ;

antecedents ::=
        antecedents sep anySequent
        | anySequent
        ;

conclusion ::=
        predicate
        ;

anySequent ::=
        provisoSequent
        | predSequent
        ;

predSequent ::=
        sequentLHS VDASH predicate
        | predicate
        ;

provisoSequent ::=
        sequentLHS VDASH PROVISO proviso
        | PROVISO proviso
        ;

/* Left hand side of a sequent is a comma-separated list
   of predicates and typing facts.
*/
sequentLHS ::= sequentLHS COMMA antecedent
        | antecedent
        ;

antecedent ::= predicate
        | name COLON baseType
        ;


/* The various proviso side conditions of CZT rules */
proviso ::= typeProviso
        | calcProviso
        | checkProviso
        | lookupProviso
        ;

/* Pred : Type, or Expr : Type */
typeProviso ::= term COLON baseType ;

/* Base types are an extremely restricted subset of expressions */
baseType ::= expression ;


/* Joker == Expr, or Joker == Pred, where Expr/Pred must contain a
   restricted set of operators as described above.
*/
calcProviso ::= name DEFEQUALS term ;


/* Simple predicates such as equality, membership, subset of sets of names.*/
testProviso ::= predicate ;


/* This searches the context (usually the specification)
   for a given definition or predicate.
   QUESTION is the '?' character.
*/
lookupProviso ::= QUESTION paragraph ;

Or we could restrict this to a few simple kinds of lookup:
lookupProviso ::=
        QUESTION name == expr
        | QUESTION name : expr
        | QUESTION predicate
        ;
\end{verbatim}



\section{Examples}

\subsection{Simple Rewrite Rules}

Here is an example of a simple rewrite rule
for rewriting subtraction into addition of a negated number.
\begin{zedjoker}{Expr} e1, e2 \end{zedjoker} \\
\begin{zedrule}{ElimMinus}
  \proviso e1 - e2 \HasType \arithmos \\
\derives
  e1 - e2 = e1 + (\negate e2)
\end{zedrule}

A typical well-formedness condition on rewrite rules like this
is that all jokers appearing in the right hand side must also
appear in the left hand side and the conclusion of the rule
must be an equality.  Such conditions will be checked by the
Java methods that process and apply the rewrite rules.


\subsection{Schema Normalisation Rules}

Here is a set of rules for normalising and unfolding schemas
(that is, moving all the fancy type information out of the
declarations into the predicate part, leaving just base types
in the decl part).   This is a very common step before doing
schema expansion in Z.  Most of the goals in these rules
have the form $SExpr = [D|P]$, but the intention of the rules
is that the right hand side will be in normalised form
(just base types in $D$).  We could introduce a special
equality relation, $SExpr \mathrel{norm} [D|P]$, but it seems
okay to use equality directly.

\begin{zedjoker}{DeclList} D, D0, D1, D2, D3, D1r,D2r,D1h,D2h\end{zedjoker} \\
\begin{zedjoker}{Pred} P, P0, P1, P2, P1b, P2b \end{zedjoker} \\
\begin{zedjoker}{Expr} S, S1, S2, E, E1, E2 \end{zedjoker} \\
\begin{zedjoker}{Type} T, T1, T2 \end{zedjoker} \\
\begin{zedjoker}{Name} v, v2, vnew \end{zedjoker} \\
\begin{zedjoker}{NameList} hide, hidden \end{zedjoker} \\

We process the declarations one by one, from left to right.
The first step in the process is to move the predicates into
the normalised result.
\begin{zedrule}{startNormalise}
   [D1 | true] = [D2 | P2]
\derives
   [D1 | P] = [D2 | P2 \land P]
\end{zedrule}

Now we handle each possible kind of declaration, from left to right.
This rule matches when the first declaration is a VarDecl.
Note that if the first declaration declares multiple variables,
like $a,b,c:\nat$, then the matcher automatically splits this
so that the $v:S$ pattern matches $a:\nat$ and the $b,c:\nat$
becomes the first declaration within $D1$.
\begin{zedrule}{normaliseVarDecl}
   \proviso S : \power T \\
   [D1 | true] = [D2 | P2]
\derives
   [v:S; D1 | true] = [v:T; D2 | v \in S \land P2]
\end{zedrule}

The rule for constant declarations is similar.
\begin{zedrule}{normaliseConstDecl}
   \proviso S : \power T \\
   [D1 | true] = [D2 | P2]
\derives
   [v==S; D1 | true] = [v:\power T; D2 | v = S \land P2]
\end{zedrule}

A schema inclusion can be an arbitrary schema expression,
so we need to have a whole set of rules for expanding
the various kinds of schema operators (see below for a 
few examples of these).
\begin{zedrule}{normalizeIncludeDecl}
   S = [D1 | P1] \\
   [D | true] = [D2 | P2]
\derives
   [S; D | true] = [D1; D2 | P1 \land P2]
\end{zedrule}

The base case is an empty declaration list, $[|true]$.
For this we only need the reflexivity axiom $E=E$ which
is probably available as part of whatever logic we are
using.  However, for completeness, here it is as a special rule:
\begin{zedrule}{normalizeEmpty}
   [~ | true] = [~ | true]
\end{zedrule}


\subsection{Schema Expression Expansion Rules}

This rule expands a simple undecorated schema name.
\begin{zedrule}{SchemaName}
  \proviso? v == S \\
  \proviso S : \power [\_] \\
  S = [D|P]
\derives
  v = [D|P]
\end{zedrule}

Expanding a generic (but undecorated) schema name
could be done as follows.  Note that the $==$ proviso
must instantiate $S$ with the type parameter list $EL$.
\begin{zedrule}{SchemaName}
  \proviso? v[EL] == S \\
  \proviso S : \power [\_] \\
  S = [D|P]
\derives
  v[EL] = [D|P]
\end{zedrule}

To expand a decorated schema name or expression is more complex.
We need a joker that ranges over decorations 
(or even better, over decoration sequences)!
Without this we would have to have at least 13 separate rules,
one for each possible decoration.  The next rule shows
how we would use a joker over decoration lists -- it is
the non-generic version, but the generic version would
be used instead in practice.

\begin{zedjoker}{DecorList} d \end{zedjoker}

\begin{zedrule}{SchemaNameDecor}
  \proviso? v == S \\
  \proviso S : \power [\_] \\
  S = [D|P]
\derives
  v_d = [D_d|P_d]
\end{zedrule}

The Z convention of how to expand $\Delta N$ and $\Xi N$
does not seem possible to encode as a rule, since the $\Delta$
and $\Xi$ are part of the name (unless we added a
special proviso for recognising such names and extracting the $N$
part).  But it is probably easier to build these naming conventions
into our CZT section manager, so that it automatically constructs
the result of $\Delta N$ when we look up that name.

Here is a rule for expanding a schema conjunction.
Note that the proviso that calculates $[D1;D2]$ will fail
if $D1$ and $D2$ are not signature compatible, as described above.
This illustrates the difference between calculating $D$ in
a proviso (which runs a Java algorithm), rather than simply
putting $D1;D2$ directly into the right hand side of the
conclusion (which might create a badly typed schema).

\begin{zedrule}{SchemaNameDecor}
  S1 = [D1|P1] \\
  S2 = [D2|P2] \\
  \proviso D == [D1;D2]
\derives
  S1 \land S2 = [D|P1 \land P2]
\end{zedrule}


Expanding a schema disjunction is a little different,
since it is necessary for the semantics that the two
disjuncts are normalised.  We check this by using
the typechecker to calculate the base types and check
that they are the same as $D1$ and $D2$.

\begin{zedrule}{SchemaNameDecor}
  S1 = [D1|P1] \\
  S2 = [D2|P2] \\
  \proviso [D1] : \power D1 \\
  \proviso [D2] : \power D2 \\
  \proviso D == [D1;D2]
\derives
  S1 \lor S2 = [D|P1 \lor P2]
\end{zedrule}

The other boolean schema operators are similar.


Here are some quite complex rules for unfolding schema compositions
($S1 \semi S2$) one step at a time.  We assume here that $S1$
and $S2$ have already been normalised.  Rather than going from
left to right through every declaration of $S1$, we just match
the \emph{primed} names and remove them one by one.

TODO: we want a way of saying that $xnew$ must be fresh here
-- a special proviso, $\proviso xnew == fresh~x$, for choosing a
fresh name $xnew$ by uniquely decorating $x$ would be a nice solution.
(Internally, this would just name the new variable $x$, but would 
choose a fresh id for it so that it gets renamed or decorated on output).
These rules cope with repeated occurences of $x$ or $x'$
in the argument schemas.

\begin{zedrule}{Semi1}
  \proviso xnew == fresh~x \\
  \proviso S1 == [D0;D1|P1] \substitute \lblot x' == xnew \rblot \\
  \proviso S2 == [D2;D3|P2] \substitute \lblot x == xnew \rblot
  xnew:T \vdash S1 \semi S2 = \exists D @ [D4|P]
\derives
  [D0; x'~: T; D1 | P1] \semi [D2; x:T; D3 | P2]\\
      = \exists [xnew:T; D] @ [D4|P]
\end{zedrule}

The next rule handles names that do \emph{not} match. 
(Note that we need membership and non-membership provisos for signatures).
\begin{zedrule}{Semi2}
  \proviso v' \not\in D1 \\
  v:T \vdash [D1 | P1] \semi [D2 | P2] = [D3 | P3]
\derives
  [D1 | P1] \semi [v:T ; D2 | P2]\\
    = (\exists [] @ [ v:T; D3 | P3 ])
\end{zedrule}

And once we have removed \emph{all} names of the second
schema, we can finally remove the $\semi$.
\begin{zedrule}{Semi3}
  [D1 | P1] \semi [~|P2] = (\exists [] @ [ D1 | P1 \land P2])
\end{zedrule}

A nicer alternative to the above set of recursive rules for
schema sequential composition is this single rule, which is
made possible by the more powerful calculational provisos that
are proposed above:

\begin{zedrule}{SemiInOneStep}
  \proviso hide == removedecor(primevars(\alpha[D1])) 
                   \cap undecorvars(\alpha[D2]) \\
  \proviso hidden == fresh~hide \\
  \proviso P1b == P1 \substitute \lblot hide' == hidden \rblot \\
  \proviso P2b == P2 \substitute \lblot hide' == hidden \rblot \\
  \proviso D1r == D1 \hide (hide') \\
  \proviso D2r == D2 \hide (hide) \\
  \proviso D == [D1r;D2r] \\
  \proviso D1h == (D1 \filter(hide'))\substitute\lblot hide' == hidden \rblot\\
  \proviso D2h == (D2 \filter(hide))\substitute\lblot hide == hidden \rblot\\
  \proviso D3 == [D1h;D2h]
  \derives
  [D1|P1] \semi [D2|P2] = [ D | (\exists [D3] @ P1 \land P2)] 
\end{zedrule}



\section{Implementation Notes}

To implement this proposal, the following steps will be necessary:
\begin{enumerate}
\item extend the XML schema of Z to support the metavariable
  declarations, the corresponding joker expressions/predicates etc.
  and the rule paragraphs.  DONE.

  An interesting design issue here
  will be how to make these rewrite rule facilities available
  to the Object-Z and TCOZ schemas, which are independent extensions
  of the standard Z XML schema.

\item Generate Java AST classes for this extended XML schema
  
  DONE: thanks to Petra's GnAST tool, this is automatic.

\item Modify the parser to handle the extensions.
  It must parse the two new kinds of paragraphs (MetaVar and ZedRule), and
  replace metavariable names by joker terms (within ZedRules only).
  This gives us a way of defining rulesets (in \LaTeX\ or Unicode)
  and parsing them into XML files and into Java trees.

  Note: This extended parser could be a separate variant of the
  parser (like we already do for the Object-Z and TCOZ parsers),
  or a direct addition to the Z parser (note that any Z section 
  that does not contain MetaVar or ZedRule paragraphs is a standard
  Z section).

\item Implement the expressions and predicates that are used
  in the provisos.  This amounts to writing subclasses of
  those Expr classes that implement an 'evaluate()' method,
  which has the semantics described above (for each kind of expression).
  When a proviso is executed (with its \verb!check()! method),
  its evaluate method will be called to compute the result.

  When parsing provisos, the parser could either use a factory that 
  returns these special Expr classes, or it could use the normal factory,
  then have a post-processing visitor that checks that only the allowable
  expressions appear in the provisos and replaces the standard expression
  objects by the special proviso subclasses (with the `evaluate' method).

\item Define the \texttt{Deduction} interface and implement a
  subclass of it that allows a given rule to be applied to
  a proof tree.

  Here is the proposed \texttt{Deduction} interface.
\begin{verbatim}
interface Deduction
{
    /** The name of the rule or deduction engine. */
    String getName();

    /** Defines which sequent this rule will be applied to. */
    void setConclusion(PredSequent concl);

    /** Searches for the next solution.
        isValid can be called to see if a solution was found.
    */
    void next();

    /** This rule/engine is correctly applied to the conclusion. */
    boolean isValid();

    /** The number of child sequents (antecedents) of this rule.
        Note that this may change after each call to next().
      */
    //@ requires isValid();
    int nChildren();

    /** The i'th child sequent. */
    //@ requires isValid() && 0 <= i && i < nChildren();
    Sequent child(int i);

    /** Undoes this rule application. */
    //@ ensures ! isValid();
    void reset();
}
\end{verbatim}

  A proof tree consists of alternating layers of Sequent objects
  and Deduction objects.  Each Deduction object may have zero or more
  children (this creates the tree shape), while each Sequent object
  may point to a Deduction object (if a rule has been applied to that Sequent),
  or not.  Note that (as defined in the ZPattern XML schema) there
  are two different kinds of Sequents:
  \begin{itemize}
  \item PredSequent, that contains a predicate.  This kind of sequent
    can be 'proved' by applying a Deduction object to it.
    
  \item Primitive sequents which implement the built-in CZT side conditions,
    such as calling the typechecker, checking membership or non-membership
    of a signature, or looking up a definition.  These primitive
    sequents cannot be proved using a Deduction object, so never have
    children.  Instead, they provide a 'check()' method that checks
    the correctness of the sidecondition.  Before this check() method
    is called, we say that the sequent has \emph{Unknown} status.
    After this check() method
    has been called, the sequent can be in one of three states:
    \emph{Pass} (the side-condition was correctly discharged),
    \emph{Fail} (the side-condition is false),
    \emph{TooEarly} (the side-condition needs to be checked later,
    when the sequent is instantiated more fully). 
  \end{itemize}

  To support traversal of the proof tree, and to distinguish
  the two kinds of Sequent and allow its status to be queried,
  the Sequent interface has these methods:
\begin{verbatim}
  /** False for PredSequent objects, True for all other Sequent subclasses. */
  boolean isPrimitive();

  /** Returns null (if unproved) or a deduction object. */
  //@ requires ! isPrimitive();
  Deduction getDeduction();

  /** Check the side-condition of a primitive Sequent */
  //@ requires isPrimitive();
  //@ ensures getStatus() != Unknown;
  void check();

  /** The status of a primitive Sequent. */
  //@ requires isPrimitive();
  Status getStatus();
\end{verbatim}


\item Implement a more sophisticated subclasses of Deduction
  whose constructors take a \emph{set} of rules (for example, all the rules
  in a given Z section) and applies one of those rules to the given
  sequent.  If several rules match, it should apply each matching rule
  in turn.  (Petra did a prototype of such a `rule set interpreter'
  last year, but it will need a few changes to bring it up to date).

  Secondly, write a higher-level Deduction subclass that takes
  another Deduction class/factory as a parameter, and applies it
  to a given sequent, then applies the same Deduction subclass to the
  children of that sequent, and so on, recursively.  This is effectively
  a hard-coded recursive tactic, similar to those found in most
  theorem prover tactic languages.

\item [Later] Possibly implement ANGEL (using Java iterators)
  so that more flexible and higher-level tactics can be written.
\end{enumerate}

\section{Acknowledgements}

Thanks to Andrew Martin for suggesting that the original
CZT rewrite-rule design should be generalised to inference
rules and support both $Z_C$ and $\V$.

\bibliography{spec,logic,thmprove}
\bibliographystyle{alpha}
\end{document}

%%% Local Variables: 
%%% mode: latex
%%% TeX-master: t
%%% End: 
