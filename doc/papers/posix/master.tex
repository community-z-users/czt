% Adapted from LLNCS.DEM Springer-Verlag for LNCS, version 2.3 for LaTeX2e
\documentclass{llncs}
\usepackage{czt}  %% for the Standard Z specification macros
%
\begin{document}
\pagestyle{headings}  % switches on printing of running heads
%
\title{Testing a Z Specification of Posix}
\titlerunning{Testing a Z Spec of Posix}  % abbreviated title (for running head)
%                                     also used for the TOC unless
%                                     \toctitle is used
%
\author{Mark Utting\inst{1} \and Petra Malik\inst{2}}
%
\authorrunning{Utting and Malik}   % abbreviated author list (for running head)
%
%%%% list of authors for the TOC (use if author list has to be modified)
%\tocauthor{Mark Utting, Petra Malik}
%
\institute{Department of Computer Science, The University of Waikato, NZ\\
\email{marku@cs.waikato.ac.nz},\\
% WWW home page: \texttt{http://www.cs.waikato.ac.nz/\homedir marku}
\and
Faculty of Engineering,
Victoria University of Wellington, NZ
\email{petra.malik@mcs.vuw.ac.nz}
}

\maketitle              % typeset the title of the contribution

\begin{abstract}
We refactor Morgan and Suffrin's original Z specification of the
Posix file system, to break it into Z sections and make it conform
to the Z Standard.  We develop a simple framework for testing Z
specifications, and use this framework to test the first few levels
of the specification.  The tests are written in standard Z, and are
executable by the CZT animator, ZLive.
\end{abstract}

%%%%%%%%%%%%%%%%%%%%%%%%%%%%%%%%%%%%%%%%%%%%%%%%%%%%%%%%%%%%%%%%%%%%%%
\section{Introduction}
%%%%%%%%%%%%%%%%%%%%%%%%%%%%%%%%%%%%%%%%%%%%%%%%%%%%%%%%%%%%%%%%%%%%%%

TODO: Give a brief overview of the grand challenge and introduce the
Posix example. \cite{Hoa03}


%%%%%%%%%%%%%%%%%%%%%%%%%%%%%%%%%%%%%%%%%%%%%%%%%%%%%%%%%%%%%%%%%%%%%%
\section{Posix Standardized}
%%%%%%%%%%%%%%%%%%%%%%%%%%%%%%%%%%%%%%%%%%%%%%%%%%%%%%%%%%%%%%%%%%%%%%

In this section, we describe the refactored Posix specification.
The main change was to break up the original specification into
sections.  Figure~\ref{fig:sects} shows the structure of the
resulting Z sections, using a notation similar to a UML class diagram.
Each box represents a Z section, and the three parts within each box
show the name of the section, the main variables within the state
schema of that definition, and the names of its operation schemas
(we omit Init schemas and auxiliary schemas).

As well as breaking the specification into sections, we made several other
changes.  We decided to explicitly specify the allowable values of $BYTE$
and $ZERO$, to make it easier to express test values.  TODO: describe the
other changes we made.


\begin{figure}[htbp]
\newcommand{\DIVIDER}{\vspace{0.5ex} \hrule \vspace{1ex}}
\newcommand{\FATDIVIDER}{\vspace{1ex} \hrule \vspace{1.5ex}}
\newcommand{\SECTNAME}[2]{\hfil\hfil\hfil\textbf{#1}\hfil\emph{(#2)}}
  \centering
  \setlength{\unitlength}{1cm}
  \begin{picture}(10,20)
%
  \put(1,18){\framebox(8,1){\parbox{6cm}{
        \hfil \textbf{standard\_toolkit} \hfil
    }}}
  \put(5,17){\vector(0,1){1}}
  \put(1,15){\framebox(8,2){\parbox{8cm}{
        \SECTNAME{DS}{Data System}
        \DIVIDER
        ~$FILE == \seq BYTE$ \\ \hbox{}
        ~$file : FILE$
        \DIVIDER
        $~ReadFile, WriteFile$
    }}}
  \put(5,14){\vector(0,1){1}}
  \put(1,12){\framebox(8,2){\parbox{8cm}{
        \SECTNAME{SS}{Storage System}
        \FATDIVIDER
        ~$fstore : FID \pfun FILE$
        \FATDIVIDER
        $~createSS, destroySS, ReadSS, WriteSS$
    }}}
  \put(5,11){\vector(0,1){1}}
  \put(1,9){\framebox(8,2){\parbox{8cm}{
        \SECTNAME{CS}{Channel System}
        \FATDIVIDER
        ~$cstore : CID \pfun CHAN$
        \FATDIVIDER
        $~openCS, closeCS$
    }}}
  \put(5,8){\vector(0,1){1}}
  \put(1,6){\framebox(8,2){\parbox{8cm}{
        \SECTNAME{AS}{Access System}
        \FATDIVIDER
        ~$SS; CS$
        \FATDIVIDER
        $~ReadAS, WriteAS, SeekAS$
    }}}
  \put(5,5){\vector(0,1){1}}
  \put(1,3){\framebox(8,2){\parbox{8cm}{
        \SECTNAME{NS}{Name System}
        \FATDIVIDER
        ~$nstore : NAME \pfun FID$
        \FATDIVIDER
        $~createNS, lookupNS, destroyNS, lsNS$
    }}}
  \put(5,2){\vector(0,1){1}}
  \put(1,0){\framebox(8,2){\parbox{8cm}{
        \SECTNAME{FS}{File System}
        \DIVIDER
        ~$SS; CS; NS; usedfids : \power FID$
        \DIVIDER
        $~createFS, openFS, readFS, writeFS, closeFS,$ \\
        \hbox{}$~unlinkFS0, destroyFS, linkFS, moveFS$
    }}}

  \end{picture}
  \caption{Overview of Z Sections in the Posix Specification}
  \label{fig:sects}
\end{figure}

\input{ds.zed}


%%%%%%%%%%%%%%%%%%%%%%%%%%%%%%%%%%%%%%%%%%%%%%%%%%%%%%%%%%%%%%%%%%%%%%
\section{How to Test a Z Specification}
%%%%%%%%%%%%%%%%%%%%%%%%%%%%%%%%%%%%%%%%%%%%%%%%%%%%%%%%%%%%%%%%%%%%%%

This section introduces a simple framework for expressing tests
of a Z specification that is written in the usual Z state-based style.

TODO: introduce some testing theory in Z syntax and show how we can
use membership, subset and schema conjunction and negation to 
express positive tests and negative tests.


%%%%%%%%%%%%%%%%%%%%%%%%%%%%%%%%%%%%%%%%%%%%%%%%%%%%%%%%%%%%%%%%%%%%%%
\section{The ZLive Animator}
%%%%%%%%%%%%%%%%%%%%%%%%%%%%%%%%%%%%%%%%%%%%%%%%%%%%%%%%%%%%%%%%%%%%%%

One of the tools available in the CZT system is the ZLive animator.
It is the successor to the Jaza animator for Z~\cite{utting:jaza}.

TODO: brief description of how ZLive works, and a trivial example.


%%%%%%%%%%%%%%%%%%%%%%%%%%%%%%%%%%%%%%%%%%%%%%%%%%%%%%%%%%%%%%%%%%%%%%
\section{Testing the DS Specification}
%%%%%%%%%%%%%%%%%%%%%%%%%%%%%%%%%%%%%%%%%%%%%%%%%%%%%%%%%%%%%%%%%%%%%%

\input{dstest.zed}


%%%%%%%%%%%%%%%%%%%%%%%%%%%%%%%%%%%%%%%%%%%%%%%%%%%%%%%%%%%%%%%%%%%%%%
\section{Testing the SS Specification}
%%%%%%%%%%%%%%%%%%%%%%%%%%%%%%%%%%%%%%%%%%%%%%%%%%%%%%%%%%%%%%%%%%%%%%

TODO: describe how we can use a framing schema to `promote' the DS tests
up to this level.  Give examples.


%%%%%%%%%%%%%%%%%%%%%%%%%%%%%%%%%%%%%%%%%%%%%%%%%%%%%%%%%%%%%%%%%%%%%%
\section{Conclusions}
%%%%%%%%%%%%%%%%%%%%%%%%%%%%%%%%%%%%%%%%%%%%%%%%%%%%%%%%%%%%%%%%%%%%%%

Testing is useful, especially when the execution of the tests can
be automated.

The refactored Z specification of Posix may be a useful starting point
for other researchers who want to work on refinement or proofs
about the Posix case study.  It is available from the CZT sourceforge
website...


%
% ---- Bibliography ----
%
\bibliographystyle{alpha}
\bibliography{posix}
%\begin{thebibliography}{5}
%
%\bibitem {clarke}
%TODO...
%\end{thebibliography}
\end{document}
