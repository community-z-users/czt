% Adapted from LLNCS.DEM Springer-Verlag for LNCS, version 2.3 for LaTeX2e
\documentclass{llncs}
\usepackage{czt}  %% for the Standard Z specification macros
\begin{document}
\pagestyle{headings}  % switches on printing of running heads
%
\title{Testing a Z Specification of Posix}
\titlerunning{Testing a Z Spec of Posix}  % abbreviated title (for running head)
%                                     also used for the TOC unless
%                                     \toctitle is used
%
\author{Mark Utting\inst{1} \and Petra Malik\inst{2}}
%
\authorrunning{Utting and Malik}   % abbreviated author list (for running head)
%
%%%% list of authors for the TOC (use if author list has to be modified)
%\tocauthor{Mark Utting, Petra Malik}
%
\institute{Department of Computer Science, The University of Waikato, NZ\\
\email{marku@cs.waikato.ac.nz},\\
% WWW home page: \texttt{http://www.cs.waikato.ac.nz/\homedir marku}
\and
Faculty of Engineering,
Victoria University of Wellington, NZ
\email{petra.malik@mcs.vuw.ac.nz}
}

\maketitle              % typeset the title of the contribution

\begin{abstract}
We refactor Morgan and Suffrin's original Z specification of the
Posix file system, to break it into Z sections and make it conform
to the Z Standard.  We develop a simple framework for testing Z
specifications, and use this framework to test the first few levels
of the specification.  The tests are written in standard Z, and are
executable by the CZT animator, ZLive.
\end{abstract}

%%%%%%%%%%%%%%%%%%%%%%%%%%%%%%%%%%%%%%%%%%%%%%%%%%%%%%%%%%%%%%%%%%%%%%
\section{Introduction}
%%%%%%%%%%%%%%%%%%%%%%%%%%%%%%%%%%%%%%%%%%%%%%%%%%%%%%%%%%%%%%%%%%%%%%

TODO: Give a brief overview of the grand challenge and introduce the Posix example.


%%%%%%%%%%%%%%%%%%%%%%%%%%%%%%%%%%%%%%%%%%%%%%%%%%%%%%%%%%%%%%%%%%%%%%
\section{Posix Standardized}
%%%%%%%%%%%%%%%%%%%%%%%%%%%%%%%%%%%%%%%%%%%%%%%%%%%%%%%%%%%%%%%%%%%%%%

In this section, we describe the refactored Posix specification.
The main change was to break up the original specification into
sections.  Figure~\ref{fig:sects} shows the structure of the
resulting Z sections, using a notation similar to a UML class diagram.

We also decided to explicitly specify the allowable values of $BYTE$ and
$ZERO$, to make it easier to express test values.

\input{ds.zed}


%%%%%%%%%%%%%%%%%%%%%%%%%%%%%%%%%%%%%%%%%%%%%%%%%%%%%%%%%%%%%%%%%%%%%%
\section{How to Test a Z Specification}
%%%%%%%%%%%%%%%%%%%%%%%%%%%%%%%%%%%%%%%%%%%%%%%%%%%%%%%%%%%%%%%%%%%%%%

This section introduces a simple framework for expressing tests
of a Z specification that is written in the usual Z state-based style.

TODO: introduce some testing theory in Z syntax and show how we can
use membership, subset and schema conjunction and negation to 
express positive tests and negative tests.


%%%%%%%%%%%%%%%%%%%%%%%%%%%%%%%%%%%%%%%%%%%%%%%%%%%%%%%%%%%%%%%%%%%%%%
\section{The ZLive Animator}
%%%%%%%%%%%%%%%%%%%%%%%%%%%%%%%%%%%%%%%%%%%%%%%%%%%%%%%%%%%%%%%%%%%%%%

One of the tools available in the CZT system is the ZLive animator.
It is the successor to the Jaza animator for Z~\ref{utting:jaza}.

TODO: brief description of how ZLive works, and a trivial example.


%%%%%%%%%%%%%%%%%%%%%%%%%%%%%%%%%%%%%%%%%%%%%%%%%%%%%%%%%%%%%%%%%%%%%%
\section{Testing the DS Specification}
%%%%%%%%%%%%%%%%%%%%%%%%%%%%%%%%%%%%%%%%%%%%%%%%%%%%%%%%%%%%%%%%%%%%%%

\input{dstest.zed}


%%%%%%%%%%%%%%%%%%%%%%%%%%%%%%%%%%%%%%%%%%%%%%%%%%%%%%%%%%%%%%%%%%%%%%
\section{Testing the SS Specification}
%%%%%%%%%%%%%%%%%%%%%%%%%%%%%%%%%%%%%%%%%%%%%%%%%%%%%%%%%%%%%%%%%%%%%%

TODO: describe how we can use a framing schema to `promote' the DS tests
up to this level.  Give examples.


%%%%%%%%%%%%%%%%%%%%%%%%%%%%%%%%%%%%%%%%%%%%%%%%%%%%%%%%%%%%%%%%%%%%%%
\section{Conclusions}
%%%%%%%%%%%%%%%%%%%%%%%%%%%%%%%%%%%%%%%%%%%%%%%%%%%%%%%%%%%%%%%%%%%%%%

Testing is useful, especially when the execution of the tests can
be automated.

The refactored Z specification of Posix may be a useful starting point
for other researchers who want to work on refinement or proofs
about the Posix case study.  It is available from the CZT sourceforge
website...


%
% ---- Bibliography ----
%
\begin{thebibliography}{5}
%
\bibitem {clarke}
TODO...

\end{thebibliography}
\end{document}
