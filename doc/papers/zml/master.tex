\documentstyle{llncs}  % from http://www.springer.de/comp/lncs/authors.html

\newcommand{\TODO}[1]{\textbf{TODO: #1}}   % For draft version
% \newcommand{\TODO}[1]{}    % For final version

\title{ZML: XML Support for Standard Z}
\author{Mark Utting\inst{1} 
        \and Ian Toyn\inst{2}
        \and Jing SUN\inst{4}
        \and Andrew Martin\inst{3}   % and perhaps David Curry?
        \and Jin Song DONG\inst{4}
        \and Nicholas Daley\inst{1}
}
\institute{The University of Waikato, Hamilton, NZ\\
        Email: \texttt{\{marku,ntd1\}@cs.waikato.ac.nz}
  \and  The University of York\\
        Email: \texttt{ian@cs.york.ac.uk}
  \and  Oxford University\\
        Email: \texttt{Andrew.Martin@comlab.ox.ac.uk}
  \and  The National University of Singapore \\
        Email: \texttt{\{sunjing,dongjs\}@comp.nus.edu.sg}
}

\begin{document}
\maketitle

\begin{abstract}
  This paper proposes an XML format for standard Z.
  We describe several earlier versions of DTDs for Z,
  the problems and issues that arose, and the reasons
  for our new proposal.
\end{abstract}

\section{Why an XML format for Z?}

  \TODO{Andrew: give an overview of the CZT project and describe the benefits
        of an XML format.  Also, why annotations are interesting/important.
        And why it is useful to allow some sections to be unparsed.}

  \TODO{Mark: Describe the advantages of an open-source Z AST in Java.}

  \TODO{Ian: Describe why we want to STANDARDIZE an XML format eventually.}

\section{Previous Work}

  Before ZB2002, Ian Toyn wrote a DTD for Z, based on the abstract
  syntax of the Z standard, plus ideas from CadiZ and Zeta.
  Since then, David Curry has validated that DTD and Utting and Daley
  have derived a Java library from it.

  \TODO{Give examples of the structure of this DTD.}

  \TODO{Mark: Describe conclusions from this experience.   Like:}
\begin{enumerate}
\item The presence of an `unparsed' alternative for each element
      complicates the Java class hierarchy.
\item The combination of `unparsed' alternatives and attributes
      gives some strange effects:
\begin{verbatim}
<!ELEMENT %Z-LogPred; ((%Z-Pred;, %Z-Pred;, %Z-LogPredAnns;?) | #PCDATA)>
<!ATTLIST %Z-LogPred; %Z-Log; (And|Or|Imp|Iff|Nl|Semi|Chain) #REQUIRED>
\end{verbatim}
      This means that the type of operator must be given,
      even for the unparsed version, which is not sensible. 
\item There are several different approaches to annotations,
        and it is not clear which is best.
\item The Java class hierarchy, with inheritance, is more flexible
        than the DTD structure.  This means that the ideal structure
        of the Java classes cannot exactly match the DTD
        structure. \TODO{Example...}
\end{enumerate}

  At the National University of Singapore, Dong and Sun developed an XML
  schema from Toyn's DTD, simplifying it by omitting all unparsed
  alternatives and all annotations, but extending it to support Object-Z
  and TCOZ.   
  \TODO{Sun: describe the differences from the DTD and interesting 
        features of the XML schema version?} 

  \TODO{Sun: give a short overview of the other tools you using with
        this XML Schema.  Translation to UML, using XSL etc.}

\section{Our Proposal}

\subsection{The Challenge of Name Scoping}
  \TODO{Describe the problems of nested identical names.  Neither the DTD
  or XML Schema proposals above solve this problem.  The best solution
  seems to be to use the ID and IDREF features of XML.  Should this be
  compulsory (every variable reference must have an IDREF) or optional
  (but those that would incorrect without it, must have it)?
}

  \TODO{Simplify the Predicate defs to remove pre/post/infix distinction,
    if possible (because it is related to the concrete syntax not the
    abstract), while preserving source invertibility.}

  \TODO{Discuss how the parsed/unparsed problem will be solved and how
        annotations will be handled.  Annotations will be loosely typed,
        like in Zeta?}

  \TODO{Give examples.} 

\section{A Java Library for Z}

  \TODO{Mark: Describe how we propose to generate the Java class hierarchy
        from the XML Schema definition.  Using XML Spy or Schema2Java
        hopefully...  Briefly describe the features that the Java library
        will offer (to/from XML, parsing from standard Z representations,
        visitor classes, transformation rules etc.), and how it will fit
        into CZT, or form the basis of a Java-based CZT.} 

\section{Support for Z Special Symbols}

% Adapted from http://nt-appn.comp.nus.edu.sg/fm/zml
Z/Object-Z languages consist of a rich set of mathematical symbols. Those
symbols can be presented directly in Unicode that is supported by XML,
but this is somewhat unreadable for human readers.  To make the ZML
files more readable, we have defined all the entities in the DTD so that
users do not have to memorize all the Unicode numbers when authoring their
XML documents. Part of the entity declaration DTD is defined as follows. 

\begin{verbatim} 
<?xml version="1.0" encoding="UTF-8"?>
<!-- some definition omitted -->

<!ENTITY emptyset "&#x2205;">
<!ENTITY mem "&#x2208;">
<!ENTITY nem "&#x2209;">
<!ENTITY pset "&#x2119;">
<!ENTITY fset "F">
<!ENTITY psetone "&#x2119;&#x2081;">
<!ENTITY fsetone "F&#x2081;">
<!ENTITY uni "&#x222a;">
<!-- some definitions omitted -->
\end{verbatim}

As most existing Z specifications were constructed in LaTeX, translating
them to our format can be a trivial task due to that each entity is given a
Z LaTeX compatible name.  We use a DTD declare these constants 
because XML Schema does not support entity declarations at the moment.

 
\section{Conclusions}

  \TODO{A promising interchange format, and a good basis for
  a Java library.  Advantages...}

  Maybe copy some conclusions from http://nt-appn.comp.nus.edu.sg/fm/zml/ !
\end{document}
