\section{Type Inference and Unification}
\label{inference}

Using unification to detect type inconsistences between two class
types is straightforward.  Abstractly, the unification of the types is
simply checking that the sets of object identities in the two types
intersect, meanwhile unifying the types of the variables contained in
those identities. For example, if we have the object $a \in \poly
A[\nat]$ and the object $b \in B[\nat]$, in which $B$ is a subclass of
$A$, the the predicate $a = b$ is type correct. This may seem strange
because it is possible that $a \in A[\nat]$, but from a specification
point of view, typechecking is used to eliminate type consistencies to
prevent meaningless predicates that can never be true. However, the
predicate $a = b$ can be true if $a \in B[\nat]$.

In this section, we present two unification algorithms. The first has
a notion of {\em weak} type consistency between class types. That is,
it unifies two class types if any of the class references in those
types unify. The second has a notion of {\em strong} type consistency
between class types. It enforces that the sets of class references of
two class types are equal. That is, the predicate $a = b$ from above
is inconsistent. 
%In the CZT Object-Z typechecker, a switch is provided
%for the user to tell the typechecker which one to use.

\subsection{Weak Type Consistency}

Spivey \cite{spivey90} presents an algorithm for unification of Z
types. This relies on variable types, which are introduced when the
type of an expression is as yet unknown, and must be
inferred. These variables are assigned values by via unification. For
example, if we unify ``$\power$ {\tt GIVEN g}'' with ``$\power \alpha_1$'', in
which $\alpha_1$ is a type variable, the unification assigns the value
``{\tt GIVEN g}'' to $\alpha_1$. This variable assignment propogates
througout the specification, and other types that refer to $\alpha_1$
are also result updated.

We extend Spivey's algorithm for unification by including class
types. This extended algorithm is presented in
Figure~\ref{weakUnify}. In this figure $\alpha$ represents variable
types, $\tau$ represents types, $\omega$ represents class signatures,
and $\psi$ represents sets of class references. Number subscripts are
appended to these names to indicate different variables. That is
$\alpha_1$ and $\alpha_2$ are distinct variables. We use $...$ to
indicate that no major changes are made to Spivey's original
algorithm.

\begin{figure}[!t]
\begin{tabular}{llll}
  {\bf fun} & \multicolumn{3}{l}{$weakUnify(\tau_1 :Type, ~\tau_2:Type) : \bool =$}\\
%  {\bf let} & \multicolumn{3}{l}{{\bf val} $\tau_1 = resolve(\tau_1) \land \tau_2 = resolve(\tau_2)$ {\bf in}}\\
  & {\bf case} $(\tau_1,\tau_2)$ {\bf of} & $(GivenType~i, GivenType~j)$ &
  $\dot i=j$\\
      & & ... & \\
      & & $(TypeVar~\_, ClassType~\_)$ & $\dot true$\\
      & & $(ClassRefType~\omega_1, ClassRefType~\omega_2$) & $\dot$\\
      & & \multicolumn{2}{l}{\quad\quad{\bf if}
             $intersect(\{\tau_1.thisClass\}, \{\tau_2.thisClass)\})$}\\
      & & \multicolumn{2}{l}{\quad\quad{\bf then} $weakUnifyClassRefSig(\omega_1, \omega_2)$}\\
      & & \multicolumn{2}{l}{\quad\quad{\bf else} $false$}\\
      & & $(ClassType~\omega_1, ClassType~\omega_2$) & $\dot weakUnifyClassSig(\omega_1, \omega_2) $\\
      & & $(TypeVar~\alpha, \_)$ & ...\\
      & & $(\_,\_)$ & $\dot false$
\end{tabular}
\vspace{2mm}\\
\begin{tabular}{llll}
  {\bf fun} & \multicolumn{3}{l}{$weakUnifyClassSig(\omega_1:ClassSignature,~\omega_2:ClassSignature) : \bool =$}\\
%  {\bf let} & \multicolumn{3}{l}{{\bf val} $\omega_1 = resolve(\omega_1) \land \omega_2 = resolve(\omega_2)$ {\bf in}}\\
  & {\bf case} $(\omega_1,\omega_2)$ {\bf of} & $(VarSig~\_, \_ )$ & $\dot true$\\
      & & $(\_, VarSig~\_)$ & $\dot weakUnifyClassSig(\omega_2,\omega_1)$\\
      & & $(\_,\_)$ & $\dot intersect(\omega_1.classRef, \omega_2.classRef)$
\end{tabular}
\vspace{2mm}\\
\begin{tabular}{llll}
  {\bf fun} &
  \multicolumn{3}{l}{$weakUnifyClassRefSig(\omega_1:ClassSignature,~\omega_2:ClassSignature) : \bool =$}\\
%  {\bf let} & \multicolumn{3}{l}{{\bf val} $\omega_1 = resolve(\omega_1) \land \omega_2 = resolve(\omega_2)$ {\bf in}}\\
  & {\bf case} $(\omega_1,\omega_2)$ {\bf of} & $(VarSig \_, \_ )$
  & $\dot \omega_1 := \omega_2; true$\\
      & & $(\_, VarSig \_)$ & $\dot weakUnifyClassRefSig(\omega_2,\omega_1)$\\
      & & $(\_, \_)$ & $\dot true$\\
\end{tabular}
\vspace{2mm}\\
\begin{tabular}{ll}
  {\bf fun} & $intersect(\psi_1:\power ClassRef, \psi_2:\power ClassRef) : \bool =$\\
   & \quad\quad$\exists r_1 \in \psi_1; r_2 \in \psi_2 @$\\
   & \quad\quad\quad\quad $r_1.refName = r_2.refName \land \#r_1.type = \#r_2.type \land$\\
   & \quad\quad\quad\quad $r_1.rename = r_2.rename \land$\\
   & \quad\quad\quad\quad $\forall n \in \dom r_1.type @
       weakUnify(r_1.type(n), r_2.type(n))$
\end{tabular}
\caption{Weak Unification of Object-Z Types}
\label{weakUnify}
\end{figure}

There are several differences between the algorithm in
Figure~\ref{weakUnify} and the original algorithm. Firstly, let's
consider type inference in Object-Z. If we have the predicate $a = b$,
in which $a$ is an instance of the class $A$, and $b$'s type is
unknown, we cannot infer anything about the type of $B$ except that it
is a class type, because $b$'s type could be the same as $a$'s, or any
class union or polymorphic type of which $A$ is a part. So, unlike
Spivey's unification algorithm, when we unify a variable type $\alpha$
with a non-variable type $\tau$, we do not assign the value $\tau$ to
$\alpha$ each time --- we do so only when the type is not a class
type. Instead, the algorithm just reports that the unification was
successful.
  
Similarly, if a class signature of one class type is variable, we
cannot assign the value of the second class signature to this
variable, except in one case: when both of our types are instances of
$ClassRefType$. This is because a $ClassRefType$'s class reference
list will always have only one element: its own class reference
($thisClass$ in Figure~\ref{ClassType.fig}). So, if the class names
and instantiation of the generic parameters unify, then the signatures
must match.

If any of generic parameters have not been explicitly instantiated,
that is, their types are variables, then the unification algorithm
will unify them not only in the class reference set, but also in any
other features from the class signature, such as the state signature,
because any reference to a generic parameter will be replaced with the
same variable type as any other reference to that
parameter. Therefore, the types of other class features such as state
variables need not be explicitly unified in our algorithm.


\subsection{Strong Type Consistency}

Enforcing strongly typed specifications implies changes in our
unification algorithm. Other than trivial
changes such as renaming $weakUnify$ with $strongUnify$, several
non-trivial changes are made. Firstly, when a variable type is unified
with a class type, we assign the value of the type to the variable,
because the types being unified must be equal, unlike with weak
typing. As a result, we need not explicitly handle $ClassRefType$s,
which removes the need for $weakUnifyClassRefSig$. Similarly, we
assign class signature variables the type with which they are being
unified. The major change is that we no longer use the function
$intersect$, but a new function, $equal$, which ensures that each
class reference in one class signature has a corresponding class
reference in the other. The $equal$ function is shown in
Figure~\ref{equal-classref}.

\begin{figure}[t]
\begin{tabular}{ll}
  {\bf fun} & $equal(\psi_1:\power ClassRef, \psi_2:\power ClassRef) : \bool =$\\
   & \quad\quad$\#\psi_1 = \#\psi_2 \land$\\
   & \quad\quad$\forall r_1 \in \psi_1 @ \exists r_2 \in \psi_2 | r_1.refName = r_2.refName @$\\
   & \quad\quad\quad\quad $\#r_1.type = \#r_2.type \land r_1.rename = r_2.rename \land$\\
   & \quad\quad\quad\quad $\forall n \in \dom r_1.type @
       weakUnify(r_1.type(n), r_2.type(n))$
\end{tabular}
\caption{Function $equal$ used for Strong Type Unification}
\label{equal-classref}
\end{figure}

These unification algorithms use an extension of the definition of
Object-Z classes specified in \cite{griffiths94}. That is, a set of
object identities represented by a class is calculated using the {\em
carrier set} of its type. A carrier set of a type is the set of
possible values that satisfy that type. In Object-Z, the carrier set
of an instantiated generic class is dependent upon the carrier sets of
the instantiated types (\cite{griffiths94} considers only non-generic
classes). Thus, if we have a class $A$ with a generic parameter $X$,
then the set of object identities comprising $A[\nat]$ is the same as
the set of object identities comprising $A[1..2]$. The possible values
of the state of the object $a_1 \in A[\nat]$ (assuming there is a
reference to $X$ in the state declaration) is far greater than the
possible values of $a_2 \in A[1..2]$, but the two sets contain the
same object identities. Using carrier sets, one can see that the set
of object identities $A[\nat]$ is disjoint from $A[\power
\nat]$. Similarly, a class reference with renamed variables is
disjoint from the same class reference without renamed variables.
