\section{Type Inference and Unification}
\label{inference}

In this section, we present two algorithms for type unification in
Object-Z. These algorithms are used in our typechecker to detect
inconsistencies between types, and to infer types of expressions and
declarations. Abstractly, the unification of class types is checking
that the sets of object identities in the two types are compatible,
meanwhile unifying the types contained in those identities. The first
unification algorithm is based around the notion of {\em weak
unification}, while the second is based around the notion of {\em
strong unification}.

\subsection{Weak Type Unification}

The first unification algorithm is based around the notion of {\em
  weak unification}. To explain this, we use some  examples. Take the
  following declarations.

\begin{axdef}
  a : \poly A; b : B
\where
  a = b
\end{axdef}

In this example, if $B$ is neither a direct nor indirect a subclass of
$A$, then the predicate $a=b$ is ill-typed, because the types of $a$
and $b$ come from different subtrees of the inheritance
hierarchy. However, if $B$ is a subclass of $A$, such that the
expression $\poly A$ is well-typed , then the predicate $a=b$ is
well-typed, even though it is possible that $a \in A$, and therefore
the identities of $a$ and $b$ come from different classes. Checking
the compatible of two class types is therefore checking whether the
two sets of object identities in the types intersect.

One non-trivial aspect of the Z (and therefore Object-Z) type systems
is that all generic parameters must be instantiated with a type,
either explicitly by the specifier, or implicitly by the
typechecker. If there is not enough information for the typechecker to
infer a type for the parameters, the specification is ill-typed. Toyn
et al.\ \cite{toyn00} discuss type inferencing for Z specifications
via several carefully crafted examples. However, Object-Z presents
problems with instantiation of generic types that are not found in
Z. First, take the following definition.

\begin{zed}
 G[X] == X
\end{zed}

This introduces, into the typing environment, the declaration $G : [X]
\power$ {\tt GENTYPE} $X$.
Now, we refer to $G$ in a predicate without explicitly instantiating
its parameters.

\begin{axdef}
  g_1 : G
\where
  g_1 = 1
\end{axdef}


Thus, at the point of referencing $G$, the typechecker cannot
determine a type with which to instantiate the generic parameter type
{\tt GENTYPE} $X$, so it will instantiate it with a {\em variable
type}, say $\alpha_1$. A variable type is not a type itself, but a
placeholder for a type. The value of a variable type can be another
variable type. After evaluating the declaration $g : G$, $g$ will be
introduced into the typing environment with the type $\alpha_1$.  The
typechecker then evaluates the predicate $g = 1$, it will unify the
type of $g$ with the type of $1$. Therefore, it will unify $\alpha_1$
with {\tt GIVEN} $\arithmos$, which will give $g$ the type {\tt GIVEN}
$\arithmos$, and will in turn instantiate the parameters of the
reference to $G$ with {\tt GIVEN} $\arithmos$\footnote{We assume the
use of a pointer-like system in which $\alpha_1$ in the type of $g$ is
the same as $\alpha_1$ in the type of $G$, so an update in one will be
reflected in the other.}.

For class types, however, the inference is not so straightforward.
Consider another example using $G$ without explicitly instantiating
its parameters.

\begin{axdef}
  g_2 : G; a : A; b : B
\where
  g_2 \neq a \land g_2 \neq b
\end{axdef}

In this example, $A$ and $B$ are introduced by two unrelated class
paragraphs (by unrelated, we mean neither inherits the other).  The
typechecker attempts to find the minimal type that satisfies the
predicate $ g_2 \neq a \land g_2 \neq b$. That type is $ClassUnionType
\{A, B\}$. Therefore, when the typechecker evaluates $g_2 \neq a$, the
variable type $\alpha_1$ cannot be unified with the type of $A$,
otherwise the predicate $g_2 \neq b$ will fail due to the type of
$g_2$ being $A$ and the type of $b$ being $B$, which are
incompatible. Therefore, our unification algorithm needs to be more
clever than standard unification algorithms.

Figure~\ref{weak-unification-algorithm} presents our algorithm for
weak type unification. In this figure $\alpha$ represents variable
types, $s$ and $t$ represent types, $p$ and $q$ represent schema
signatures, $m$ and $n$ represent names, and $c$ and $d$ represent
$ClassRef$s. Number subscripts are appended to these names to indicate
different variables. That is, $\alpha_1$ and $\alpha_2$ are distinct
variable types. We use the format $s = Type(s_1, \ldots, s_m)$ to
indicate that $s$ is of type $Type$ containing children $s_1, \ldots,
s_m$. For example, $s = PowerType~s_1$ says that $s$ is a power type
of the type $s_1$. $s = ClassType \{c_1, \ldots, c_m\}$ indicates that
$s$ is a class type with class references $c_1, \ldots, c_m$. By class
references, we mean the state variable $classRef$ in the class
$ClassSignature$ in Figure~\ref{ClassType.fig}. $TypeVar$ and
$ClassTypeVar$ refer to variable types and {\em variable class types}
respectively. A variable class type is similar to a variable type,
except it is a placeholder for a class type only.

\begin{figure}
\begin{algorithmic}[1]
\STATE $Unify(s : Type,~t : Type)$\\
\STATE \nexti\TBEGIN\\
\STATE \nexti\nexti\TIF $s = TypeVar~\alpha$ \TTHEN\label{alg:start-variable-types}\\
%\STATE \nexti\nexti\nexti\TIF $t = ClassType \mathrel{\_}$~  \TTHEN\label{alg:if-empty-class-type}\\
%\STATE \nexti\nexti\nexti\nexti $\alpha :=
%ClassTypeVar~t$\label{alg:then-empty-class-type}\\
\STATE \nexti\nexti\nexti\TIF $t = ClassType \mathrel{\_}$~  \TTHEN $\alpha :=
ClassTypeVar~t$\label{alg:empty-class-type}\\
\STATE \nexti\nexti\nexti\TELSE\TIF $\alpha$ occurs in $t$ \TTHEN $FAIL$\\
%\STATE \nexti\nexti\nexti\TELSE\TIF $t = TypeVar \_$~\TTHEN $PARTIAL$\\
\STATE \nexti\nexti\nexti\TELSE\TIF $\alpha \neq t$ \TTHEN $\alpha := t$\\
\STATE \nexti\nexti\TELSE\TIF $t = TypeVar~\alpha$ \TTHEN $Unify(\alpha, s)$\label{alg:end-variable-types}\\
\STATE \nexti\nexti\TELSE\TIF $s = GivenType~ m \land t = GivenType~n$ \TTHEN\\
\STATE \nexti\nexti\nexti\TIF $m \neq n$ \TTHEN $FAIL$\\
\STATE \nexti\nexti\TELSE\TIF $s = GenType~ m \land t = GenType~n$ \TTHEN\\
\STATE \nexti\nexti\nexti\TIF $m \neq n$ \TTHEN $FAIL$\\
%\STATE \nexti\nexti\TELSE\TIF $s = PowerType~s_1 \land t = PowerType~t_1$ \TTHEN\\
%\STATE \nexti\nexti\nexti $Unify(s_1, t_1)$\\
\STATE \nexti\nexti\TELSE\TIF $s = PowerType~s_1 \land t = PowerType~t_1$ \TTHEN $Unify(s_1, t_1)$\\
\STATE \nexti\nexti\TELSE\TIF $s = ProdType(s_1, \ldots, s_m) \land t = ProdType(t_1, \ldots, t_n)$ \TTHEN\\
\STATE \nexti\nexti\nexti\TIF $m = n$ \TTHEN\\
%\STATE \nexti\nexti\nexti\nexti\TFOR $i \in 1 \upto n$ \TDO\label{myline}\\
%\STATE \nexti\nexti\nexti\nexti\nexti$Unify(s_i, t_i)$\\
%\STATE \nexti\nexti\nexti\nexti\TEND\\
\STATE \nexti\nexti\nexti\nexti\TFOR $i \in 1 \upto n$ \TDO
$Unify(s_i, t_i)$ \TEND\\
\STATE \nexti\nexti\nexti\TELSE $FAIL$\\
\STATE \nexti\nexti\TELSE\TIF $s = SchemaType~p \land t = SchemaType~q$ \TTHEN\label{myline}\\
\STATE \nexti\nexti\nexti $UnifySignature(p, q)$\label{alg:unify-signature}\\
\STATE \nexti\nexti\TELSE\TIF $s=ClassTypeVar~s_1 \land t=ClassTypeVar~t_1$ \TTHEN
      \label{alg:start-var-class-types}\\
\STATE \nexti\nexti\nexti \TIF $s_1 \approx_{c} t_1$ \TTHEN%\\
 $s_1 := s_1 \cup_{c} t_1;~ t_1 := s_1$\\
\STATE \nexti\nexti\nexti \TELSE $FAIL$\\
\STATE \nexti\nexti\TELSE\TIF $s = ClassTypeVar~s_1 \land t =
ClassType\mathrel{\_}$~~\TTHEN\\
\STATE \nexti\nexti\nexti\TIF $s_1 \approx_{c} t$ \TTHEN $s_1 := s_1 \cup_{c} t$\\
\STATE \nexti\nexti\nexti\TELSE $FAIL$\label{alg:end-var-class-types}\\
\STATE \nexti\nexti\TELSE\TIF $s = ClassType \{c_1, \ldots, c_m\} \land t =
   ClassType\{d_1, \ldots, d_n\}$ \TTHEN\label{alg:state-ground-class-types}\\
\STATE \nexti\nexti\nexti{\bf boolean} $matched := false$\\
\STATE \nexti\nexti\nexti\TFOR $i \in 1 \upto m$ \TDO\\
\STATE \nexti\nexti\nexti\nexti\TFOR $j \in 1 \upto n$ \TDO\\
\STATE \nexti\nexti\nexti\nexti\nexti\TIF $c_i.refName = d_j.refName$ \TTHEN\\
\STATE \nexti\nexti\nexti\nexti\nexti\nexti$matched := true;$\\
%\STATE \nexti\nexti\nexti\nexti\nexti\nexti\TFOR $k \in 1 \upto \#c_i.type$ \TDO\\
%\STATE \nexti\nexti\nexti\nexti\nexti\nexti\nexti $Unify(c_i.type(k), d_j.type(k))$\\
%\STATE \nexti\nexti\nexti\nexti\nexti\nexti\TEND\\
\STATE \nexti\nexti\nexti\nexti\nexti\nexti\TFOR  $k \in 1 \upto
\#c_i.type$ \TDO $Unify(c_i.type(k), d_j.type(k))$ \TEND\\
\STATE \nexti\nexti\nexti\nexti\TEND\\
\STATE \nexti\nexti\nexti\TEND\\
\STATE \nexti\nexti\nexti\TIF $\lnot matched$ \TTHEN $FAIL$\label{alg:end-ground-class-types}\\
\STATE \nexti\nexti\TELSE $FAIL$\\
\STATE \nexti\TEND
\end{algorithmic}

\caption{Weak Unification Algorithm}
\label{weak-unification-algorithm}
\end{figure}

%\begin{figure}
%\begin{algorithmic}[1]
\STATE $UnifySignature(p : Signature,~ q : Signature)$\\
\STATE \nexti\TBEGIN\\
\STATE \nexti\nexti\TIF $p = SignatureVar~\beta$ \TTHEN\\
\STATE \nexti\nexti\nexti\TIF $\beta$ occurs in $q$ \TTHEN $FAIL$\\
\STATE \nexti\nexti\nexti\TELSE\TIF $\beta \neq q$ \TTHEN $\beta := q$\\
\STATE \nexti\nexti\TELSE\TIF $q = SignatureVar~\beta$ \TTHEN $UnifySignature(\beta, p)$\\
\STATE \nexti\nexti\TELSE\TIF $p = (u_1 : s_1, \ldots, u_m : s_m) \land q =
(v_1 : t_1, \ldots, v_n : t_n)$ \TTHEN\\
\STATE \nexti\nexti\nexti\TIF $m = n$ \TTHEN\\
\STATE \nexti\nexti\nexti\nexti\TFOR $i \in 1 \upto n$ \TDO\\
\STATE \nexti\nexti\nexti\nexti\nexti\TIF $u_i = v_i$  \TTHEN\\
\STATE \nexti\nexti\nexti\nexti\nexti\nexti $Unify(s_i, t_i)$\\
\STATE \nexti\nexti\nexti\nexti\nexti\TELSE $FAIL$\\
\STATE \nexti\nexti\nexti\nexti\TEND\\
\STATE \nexti\nexti\nexti\TELSE $FAIL$\\
\STATE \nexti\TEND
\end{algorithmic}

%\caption{Unification Algorithm for Schema Signatures}
%\label{schema-unification-algorithm}
%\end{figure}

This algorithm is quite different to other type unification
algorithms, such as Spivey and Sufrin's algorithm for unifying Z
types \cite{spivey90}. Unification of the Z types resembles other
unification algorithms, in which unification of atoms is successful if
the atoms are equal, unification of terms is successful if the
function symbol, for example $PowerType$, is the same, and the
parameters can be unified, and unification of a variable with a term
succeeds provides that variable does not occur in the term. The
function $UnifySignature$, referenced on
line~\ref{alg:unify-signature}, is not shown, but this is a
straightforward comparison of two signatures to check that both
contain the same names with unifiable types. We assume the use of a
pointer-like system in the assignment of a type to a variable will be
reflected in other types that use that variable. Therefore, if
unification succeeds, parameters $s$ and $t$ are both the most general
unifier.

Class type unification is quite different. For starters, it is not
strictly unification, as we unify only parts of each of the types. For
example, calling $Unify$ with arguments $ClassUnionType \{A,
B[\alpha_1]\}$ and $ClassUnionType \{B[${\tt GIVEN} $\arithmos], C\}$
will succeed, with $\alpha_1$ assigned the value {\tt GIVEN}
$\arithmos$. However, the unification is only performed in the class
references with common names.

Stepping through the algorithm, we first examine the case of two
ground (containing no variable types) class types, which is handled in
between lines \ref{alg:state-ground-class-types} and
\ref{alg:end-ground-class-types} of
Figure~\ref{weak-unification-algorithm}. Recall that compatible class
types are those who object identities intersect. This part of the
algorithm checks this property. It does so by comparing the names of
the class references in each type. If there is no common names, the
types do not match, and unification fails (line
\ref{alg:end-ground-class-types}). Otherwise, the algorithm unifies
the types of the generic parameters of each class reference, which
will fail if any are incompatible. If any of generic parameters have
not been explicitly instantiated, that is, their types are variables,
then the unification algorithm will unify them not only in the class
reference set, but also in any other features from the class
signature, such as the state signature, because any reference to a
generic parameter will be replaced with the same variable type as any
other reference to that parameter. Therefore, the types of other class
features such as state variables need not be explicitly unified.

Now, we look at the handling of variable types --- lines
\ref{alg:start-variable-types} to
\ref{alg:end-variable-types}. Without line \ref{alg:empty-class-type},
this unification is much like others. If $\alpha$ occurs in type $t$,
which means that we have encountered a cyclic type, then the algorithm
fails. Otherwise, the type $t$ is assigned to the $\alpha$, unless
they are the same variable type ($\alpha = t$), then nothing
happens. Line \ref{alg:empty-class-type} is introduced to handle the
case when a class type is unified with a variable type. Recall the
example earlier in this section in which a reference to the generic
definition $G$ required implicit instantiation, and the resulting type
is $ClassUnionType \{A, B\}$. In this case, when the predicate $g_2
\neq a$ is encountered, the typechecker will call $Unify(\alpha_1,
ClassType\{A\})$. Assigning $ClassType\{A\}$ to $\alpha_1$ is not
possible, because when $g_2 \neq b$ is encountered, the unification
will fail (the class references to not intersect). Instead, we assign
to $\alpha_1$ the variable class type $ClassTypeVar~t$, in which
$t$ is the {\em candidate type}.

Now, we move to handling of variable class types --- lines
\ref{alg:start-var-class-types} to \ref{alg:end-var-class-types}. The
binary relation $\approx_{c}$ over two class types holds if and
only if the signatures of those two types are compatible. That is, the
types of commonly named features unify. The binary function
$\cup_{c}$ represents the union of two class types. The first
branch in this block deals with the case that two class variable types
are unified.  If the signatures of the two candidate types are
compatible, then the candidate types, $s_1$ and $t_1$, are unioned, and
both class variable types are the same; otherwise the unification
fails. The second branch in this block deals with the case that a
variable class type is unified with a ground class type. If the
signatures of the candidate type in the variable class type and the
ground class type are compatible, then the candidate type is unioned
with the class type; otherwise the unification fails.

From this algorithm, one can see that the minimal class type that
satisfies a specification such as our example above is found by
gathering all possible candidate types to which an object can belong,
and creating a class union from them. Until one candidate type is
found, the type is a variable type. When a candidate type is found,
any subsequent candidates must be compatible with the existing
candidate type. The compatibility of two ground class types is checked
using the names and parameter instantiations of their class
references, which represents a check that their object identities
intersect.

Once a paragraph has been typechecked, the typechecker returns to
evaluate expressions whose type was initially a variable type to
ensure that the type is fully determined, raising an error if it is
not. Similarly, it returns to expressions whose type is a variable
class type, and uses the candidate type as the actual type.

\subsection{Strong Type Consistency}

The second unification algorithm is based around the notion of {\em
  strong unification}. To explain this, we again turn to an
  example. Z type rules specify that for a declaration, $v : E$, $E$
  must be a set; therefore, the type of $E$ must be power type. To
  typecheck this, the CZT typechecker will get the type of $E$, and
  check that it unifies with the type $\power \alpha_1$, in which
  $\alpha_1$ is a fresh variable type. If the unification fails, $E$
  is not a set, and the typechecker raises an error. If the
  unification succeeds, the type assigned to $\alpha_1$ (which may
  still be $\alpha_1$ or another variable type) becomes the type of
  $v$ in the typing environment. Now, consider the paragraph.

\begin{axdef}
 a : A\\
 b : B
\where
 a = b
\end{axdef}

In this example, $A$ and $B$ are unrelated classes. If we use the weak
 unification algorithm, after evaluating the two declarations, the
 types of $a$ and $b$ will be variable class types, because the
 assignment statement on line \ref{alg:empty-class-type} will be
 executed. When the predicate $a=b$ is evaluated, the candidate types
 in these, $A$ and $B$ respectively, will be unioned, and the final
 type of both $a$ and $b$ will be $ClassUnionType \{A, B\}$. However,
 this is not the correct inference. In fact, the typechecker should
 raise an error for the predicate $a=b$, because these classes are
 incompatible. Therefore, declarations and several other types of
 Z/Object-Z structures must be handled differently. For these cases,
 we use strong unification.

 For strong unification, there are three changes to the algorithm in
 Figure~\ref{weak-unification-algorithm}. Firstly, when we attempt to
 unify a variable type with a class type, the variable is assigned the
 value of the class type. Therefore, line \ref{alg:empty-class-type}
 is removed from the algorithm, so that the declaration $a:A$ from
 above will infer the type of $a$ as $ClassRefType\{A\}$.  Secondly,
 there are no variable class types (these essentially become just
 variable types), so lines \ref{alg:start-var-class-types} to
 \ref{alg:end-var-class-types} are removed. Thirdly, the unification
 of two ground class types is different. Rather than checking that the
 class references intersect, they must be equal; that is, the same
 class reference names with unifiable types.

 Strong unification can also be used to give a tighter check on
 specifications. For example, if we have $a \in \poly A$ and $b \in
 B$, in which $B$ inherits $A$, then the predicate $a=b$ can be a
 possible downcast. That is, it is possible that $a \in A$ and $b \in
 B$, so we are comparing two incompatible types. The CZT typechecker
 provides a switch which allows strong unification to be used at all
 times, not just on declarations and other expressions that require
 strong unification, therefore reporting the $a=b$ is ill-typed. Many
 Object-Z specifications that we have used seen in literature and run
 through the CZT typechecker have successfully typechecked using
 strong unification.

\subsection{Termination, Correctness, and Completeness}

We have done proofs that our algorithms terminate, are complete
and are correct. Termination is straightforward to prove. Types are
represented using finite, directed tree structure, with the {\em
occurs} check in the unification algorithm preventing cycles. Each
case in the algorithm either halts immediately, either failing or
succeeding, or recursively calls $Unify$ on the child nodes of the
current types, for which the depth of the tree must be less than the
parent.

Proving correctness and completeness is done using induction on any
recursive calls to $Unify$.  Proofs for the Z types and for the ground
class types are straightforward, as those parts of the algorithm
resemble generic unification algorithms. For these, we prove each case
(each top-level branch block in the algorithm) separately. Proofs for
the cases involving variable class types require more discussion than
other cases, but are still straightforward to prove.
