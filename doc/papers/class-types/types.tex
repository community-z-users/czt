\section{Types in ISO Z and Object-Z}

Z and Object-Z are statically-typed languages
\footnote{Strictly speaking, dynamic typing is possible in Object-Z
since $\oid$ can be used for any and all object types, but this is not
considered good practice} whose type systems enforce consistency
between operators and expressions.

For example, the membership predicate $\{1\} \in \{1\}$ is not
permitted by the Z type system because the declaration of the
membership operator specifies that the type of the expression on the
left-hand side must be the same as the type of the elements inside the
right-hand side set, and the right-hand side expression must be a set
construct.  In this section, we present an overview of the Z type
system, and how it is represented in \theStandard. We then present
our extension of this representation to include Object-Z class types.

\subsection{Overview of Z Types}

Z has several primitive types, from which more complex types can be
derived. {\em Given types} are the types of {\em given sets}, which are
defined using the following syntax:
\begin{zed}
  [Resource]
\end{zed}

This adds ``$Resource : \power$ {\tt GIVEN} $Resource$'' into the type
environment.

{\em Power types} are the types of sets. A set can be declared using
$\{\ldots\}$ notation, or by using an operator such as $\power$. To
declare a set of resources, we can use the following:
\begin{axdef}
  resources : \power Resource
\end{axdef}

This adds ``$resources : \power (\power$ {\tt GIVEN}
$Resource)$'' into the type environment.

{\em Product types} are the type of tuple expressions, which are
declared using $(a, \ldots, b)$ syntax, of cross product expressions. To
declare a triple of resources, we can use the following:
\begin{axdef}
   triple : Resource \cross Resource \cross Resource
\end{axdef}

This adds ``$triple : $ {\tt GIVEN}
$Resource \cross$ {\tt GIVEN} $Resource \cross$ {\tt GIVEN} $Resource$''
into the type environment.

{\em Schema types} are the type of schema expressions, which are sets
of variable bindings. The most common way of declaring a schema
expression is using a schema box:
\begin{schema}{System}
  resource_1 : Resource\\
  resource_2 : Resource
\where
  resource_1 \neq resource_2
\end{schema}

This adds ``$System
: \power [resource_1 : $ {\tt GIVEN} $Resource; resource_2 : $ {\tt GIVEN}
$Resource]$'' into the type environment.

{\em Generic parameter types} and {\em generic types} are the final
types in the Z type system. They are related in that one is needed for
the other. Generic definitions in Z permit definitions whose type is
polymorphic, in that the definition can be instantiated. For example,
the following generic paragraph introduces the prefix relation
$singleton$, which holds if the supplied set is a singleton set.
\begin{gendef}[X]
  singleton~\_ : \power X
\where
  \forall xs : \power X @ singleton~xs \iff \#xs = 1
\end{gendef}
% I made the <= an =
%
% This is a really interesting point.
% emptyset subset emptyset is true, but emptyset in emptyset
% is false (this is because the definition of subset relies upon 
% a universal quantifier over the LHS emptyset which is trivially true).
%
% From MathWorld:
% A nonempty set is a set containing one or more elements. Any set other
% than the empty set is therefore a nonempty set. Nonempty sets are
% sometimes also called nonvoid sets (Grätzer 1971, p. 6). A nonempty
% set containing a single element is called a singleton set.


This adds ``$singleton~\_ : [X] \power $ {\tt GENTYPE} $X$ into the
type environment''. When this name is referenced, the generic
parameter $X$ is instantiated to be the type of the set parameter. For
example, in the paragraph:
\begin{axdef}
  resources : \power Resource
\where
  singleton~resources
\end{axdef}

$X$ is instantiated to ``{\tt GIVEN} $Resource$''.

The Z type hierarchy is defined in \ISOZ{10.2}. It has, at its root,
the abstract type {\em Type}, which is the supertype of all Z
types. {\em Type} is partitioned into: {\em GenericType}, which is
used for definitions that are made in a generic environment, and which
must be instantiated when used; and {\em Type2}, which are all
non-generic types: {\em GivenType}, {\em GenParamType}, {\em
PowerType}, {\em ProdType}, and {\em SchemaType}. A schema type has a
{\em Signature}, which is a possibly-empty function from names to
their types. The Z type hierarchy can be seen in Figure~\ref{z-types}.

\def\epsfsize#1#2{0.7#1}
\begin{figure}
\begin{center}
\epsfbox{fig/z-types.eps}
\end{center}
\caption{The Z Type Hierarchy}
\label{z-types}
\end{figure}
\def\epsfsize#1#2{\epsfxsize}

\subsection{Extending the Z Type System for Object-Z}

Extending this type system to include class types is straightforward:
add a new type {\em ClassType}, which is a subtype of {\em
Type2}. This {\em ClassType} is used represent the set of object
identities and the polymorphic core of those identities. For a class
$A$, the type of the name $A$ introduced into the type environment is
a power set of the class type representing the objects in
$A$. Figure~\ref{ClassType.fig} specifies in Object-Z
the structure of the class type.

\begin{figure}[t]
 \begin{sidebyside}
\begin{class}{ClassType}
  Type2\\
\begin{state}
  classSignature : ClassSignature
\end{state}
\end{class}
\nextside
\begin{class}{ClassRef}
\begin{state}
  refName : RefName\\
  type : \seq Type2 \\
  rename : DeclName \pfun DeclName
\end{state}
\end{class}
\end{sidebyside}
\begin{sidebyside}
\begin{class}{ClassSignature}
\begin{state}
  classRef : \finset ClassRef\\
  state : Signature\\
  attribute : DeclName \pfun Type\\
  operation : DeclName \pfun Signature
\where
  \#classRef \geq 1
\end{state}
\end{class}
\nextside
\begin{class}{ClassRefType}
  ClassType\\
\begin{state}
  thisClass : ClassRef\\
  superClass : \finset ClassRef\\
  visiblity : \finset RefName
\where
  classRef = \{thisClass\}
\end{state}  
\end{class}
\end{sidebyside}


\caption{$ClassType$ and related classes}
\label{ClassType.fig}
\end{figure}

\vspace{2mm} {\em ClassType} is made up only of a signature called
the {\em ClassSignature}. Each class signature has 4 features:
\begin{itemize}
  \item {\em classRef}: The list of class names, their variable renames, and
    their generic parameter instantiations (if
    they are generic classes), that make up the core of
    this class. For example, if we have a class $A$, then for an
    instance of $A$, $classRef = \{ A \}$. If we have a generic class
    $B[X]$, then for an instance from the class union $A \classuni
    B[Resource]$, $classRef = \{ A, B[${\tt GIVEN} $Resource]\}$.
  \item {\em attribute}: A function of variable names to types that records
    both the local and inherited variable definitions.
  \item {\em state}: A signature contains the declarations made in a
    class's state (after being conjoined with any superclasses).
  \item {\em operation}: A function of names to signatures that
    records the operations declared in a class, their signatures, and
    the inherited operations and their respective signatures.
\end{itemize}
% I've updated those definitions to reflect the fact that we are 
% collapsing the inheritance hierarchy in the class type definitions.
% I'm not sure about this. You mentioned in the related work that
% Chen's model is being extended to cache the definitions (to 
% improve the implementation). This does have its drawbacks when it 
% comes to giving meaningful output from the typechecker. If you
% don't cache the definitions, and are forced to lookup the class
% type records of the superclasses, then you can tell the user where
% the original definition is if it is being misused.
% I could be way off ;)

The attribute $classRef$ is the basis of Chen's model of class types
as sets of classes. The class $ClassRef$ represents a set of object
identities. Crudely, the set of object identities produced by two
$ClassRef$s are equal if the class name, feature renames, and
instantiations are all equal.

In addition to the class references, the attributes, operations, and
state variables are modelled in the class type. These three features
together model the polymorphic core of the type. Therefore, our model
is flexible enough to handle class references, polymorphic
expressions, and class unions.

However, there is other information that needs to be recorded when a
class is declared using a class paragraph: the generic parameters, the
class name, the superclasses, and the visibility list, a list of names
features that are visible from outside the class. Generic parameters
are recorded by declaring the class as a generic type, just as
$singleton$ was declared in the previous section, so this information
does not need to be modelled explicitly in the class type. The other
features are declared by creating a type class $ClassRefType$, which
inherits $ClassType$. This is shown in
Figure~\ref{ClassType.fig}. Similarly, two other type classes
$PolyType$ and $ClassUnionType$ are declared that extend $ClassType$,
but they add no extra features, so are not shown in the figure.

\subsection{Design Rationale}

This class type model uses an extension of the definition of Object-Z
classes specified in \cite{griffiths94}. That is, a set of object
identities represented by a class is calculated using the {\em carrier
set} of its type. A carrier set of a type is the set of possible
values that satisfy that type. In Object-Z, the carrier set of an
instantiated generic class is dependent upon the carrier sets of the
instantiated types (\cite{griffiths94} considers only non-generic
classes). Thus, if we have a class $A$ with a generic parameter $X$,
then the set of object identities comprising $A[\nat]$ is the same as
the set of object identities comprising $A[1..2]$. The possible values
of the state of the object $a_1 \in A[\nat]$ (assuming there is a
reference to $X$ in the state declaration) is far greater than the
possible values of $a_2 \in A[1..2]$, but the two sets contain the
same object identities. Using carrier sets, one can see that the set
of object identities $A[\nat]$ is disjoint from $A[\power
\nat]$. Similarly, a class reference with renamed variables is
disjoint from the same class reference without renamed variables.

Subclassing $ClassType$ for the three distinct flavours of class types
is useful because the $ClassRefType$ needs to model information that
was not relevant to the other two types. Also, modelling the
polymorphic core and class references in $ClassType$ aids analysis in
the case that the type rules require only that the type is a class
type.Some semantic restrictions in Object-Z also reinforce this
approach. In a class union $A \classuni B$, both $A$ and $B$ need to
be class references or class unions --- polymorphic expressions are
not permitted. Using our subclasses makes both cases straightforward.

Having a $ClassSignature$ is important, because the type system must
be developed in such a way that unification can be partially
performed. That is, we can infer from the context that the type of a
particular name reference is a class type, but cannot infer the class
signature until the class is later declared.  To this point, {\em
variable class signatures} can be used, which are similar to {\em
variable types} and {\em variable signatures} in Z, as discussed in
\cite{toyn00}. 

As an example, take the declaration $b : \poly B$. From the reference
to $\poly B$, the typechecker can infer that $B$ is a class reference
type even if $B$ is declared after this declaration. Using this
information, any constraints over $b$ that require an object instance
can be solved even without the signature of the class being known.

However, unlike type unification in standard Z, if we have
a predicate $b = c$, in which $c$ is an instance of another class $C$
(possibly a subclass of $B$), we cannot unify the
variable class signature with the class signature of $C$ type, because
there may be features declared in $C$ that are not declared in
$B$. We elaborate on this problem, and propose a solution, 
in the following section.
