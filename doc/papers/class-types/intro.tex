\section{Introduction}

Object-Z \cite{smith00} is an object-oriented extension of the ISO
standardised Z specification language \cite{isoz}. Since both Z
and Object-Z mandate explicit type declarations over variables,
checking for type consistency over expressions (static-semantic
analysis) can be automated. Unlike Z, however, typechecking Object-Z
has been given little attention in the research community.  At
present, Wizard \cite{johnston96} is the only implemented typechecker
of which we are aware that even partially supports Object-Z.

The Community Z Tools (CZT) project\footnote{See
\url{http://czt.sourceforge.net}} is an open-source project to build a
suite of tools implemented in Java for the ISO standardised version of
Z \cite{isoz}, and extensions to Z, such as Object-Z. Work so far has
seen the development of core classes and libraries for generating
abstract syntax trees, and parsers for Z, Object-Z, and other Z
extensions, as well as a typechecker for Z. Current development
includes an editing environment, a Z animator, and of course, an
Object-Z typechecker.

Typechecking Object-Z is a difficult task, and presents many more
problems than typechecking standard Z. Object-Z introduces such
notions as inheritance and polymorphism, which create types containing
many different (possibly unrelated) subtypes; implicit downcasting, in
which an object can be implicitly downcast to one of many subtypes
that cannot be determined statically, thus making the detection of
type errors difficult; generic classes, which specify a set of classes
instead of a single class; and feature renaming, which, in allowing
the specifier to rename operations or attributes in a class, creates a
new interface for a class, but does not introduce a new class type.

In this paper, we present our model of {\em class types}, which are
used to annotate expressions in an abstract syntax tree of a parsed
Object-Z specification, and to allow automatic application of type
rules to find type inconsistencies in specifications. This class type
is flexible enough to solve the problems discussed above. We also
present two unification algorithms for type inference and determining
type consistency between two class types. The first enforces a {\em
weak} form of consistency between two class types, and the second
enforces a {\em strong} form. Our class type model and unification
algorithm should be flexible enough to work for other formal
object-oriented specification languages with type systems similar to
Object-Z.
