\section{Background and Motivation}
\label{background}

In this section, we discuss some background and the motivation for the
work discussed in this paper.

\subsection{Object-Z}

The Object-Z specification language extends Z with modularity and reuse
constructs that resemble the object-oriented programming paradigm. Such
constructs include classes, inheritance, and polymorphism. Late-binding
of types is typical in such languages, and allows for a high degree of
flexibility when specifying or implementing a system. However, the
Object-Z language provides no specific mechanism for explicit
type-casting of objects to aid the static analysis of type consistency
in such circumstances. Thus, an automated approach to type-checking
Object-Z must make inferences of object types taking into account the
inheritance hierarchy and object identity sets (which will be discussed
in further detail in the next section). 

Because casting of types is a dynamic property of the system (it is
essentially a test for set membership), and must be inferred
statically by the type-checker, a {\em completely} automated analysis
of type-consistency is an intractable problem; it requires proofs to
be discharged. As a consequence, the type-analysis algorithm discussed
in this paper can only provide a warning when an object is being
down-cast and the onus is upon the user to determine whether it is
possibly an illegal downcast (in Object-Z such a situation would
result in an operation being blocked). Detecting when such warnings
must be issued, and when it is not necessary, relies upon a thorough
understanding of how object identity is treated in Object-Z.

\subsection{Object Identity and Classes}

An object's identity distinguishes it from all other objects.  Two
different objects of the same type, $o_{1}$ and $o_{2}$, can have the
same values for their state, yet still be distinct objects; that is
$o_{1} \neq o_{2}$.

Object-Z's semantics implicitly include the concept of object identity
in the language, therefore giving it a {\em reference semantics}
similar to the treatment of object equality and assignment in
languages such as Smalltalk \cite{goldberg83} and
Java\footnote{\url{http://java.sun.com}}. This is opposed to a {\em
value-based semantics}, in which objects are denoted by the value of
the state.
% Tim, I had to make the distinction in that paragraph about assignment
% because Java actually has a pass-by-value semantics, and someone could
% pick on that.

The {\em dot notation} is commonly used in object-oriented languages
to dereference attributes or operations\footnote{From here on
the collective terms for attributes and operations is {\em features}.}
of an object. For example, the feature $x$ of an object $a$ is
accessed using the syntax $a.x$.

Using these semantics, an Object-Z class is a countably infinite set
of object identities, defined syntactically using a class
paragraph. The set $\oid$ represents the set of all object identities
in a system, such that for any class $A$, $A \subseteq \oid$ is true,
and for the set $\{ A_{1}, ..., A_{n} \}$ containing all classes in a
system, $\duni \{ A_{1}, ..., A_{n} \} = \oid$ is true.

In Object-Z, classes can be unioned to create polymorphic types. That
is, two classes $A$ and $B$ that share no common superclass, can be
unioned together to create a new set of object identities: $A
\classuni B$. The Object-Z type rules specify that any commonly named
attributes or operations in $A$ and $B$ must share the same types or
signatures respectively. These commonly named features are known as
the {\em polymorphic core} of the class union \cite{dong93}.

Object-Z also provides a {\em polymorphic operator}. That is, for a
class $A$, the expression $\poly A$ refers to the set of object
identities in $A$ and any subclasses of $A$. The polymorphic core of
this expression is the features of $A$.

\subsection{Motivation}

Chen \cite{chen94} discusses a model in which a class type is a set of
classes. That is, if we have a class $A$ with subclasses $B$ and $C$,
then the expression $\poly A$ is the set of classes $\{A, B,
C\}$. This idea is used as the basis for the class type in our
typechecker, but we make two significant changes.

Firstly, Chen does not discuss how to represent classes. In Object-Z,
classes are defined as countably infinite sets of object
identities. However, we need a way to represent these sets in the
typechecer, which is difficult because programming languages do not
support infinite sets, and Object-Z provides us with no idea of what
the value object identities look like. Chen's examples use the class
names to represent the classes, but this does not provide complete
support for generic classes. For example, take the Object-Z class
defined in Figure~\ref{generic-class-A}.  In this example, we have the
generic class $A$, with parameter $X$ and no features.  Two objects
are declared using $A$: one with $X$ instantiated to $\nat$ and the
other to $\power \nat$. The equality $a_{1} = a_{2}$ is type incorrect
because $A[\nat]$ and $A[\power \nat]$ are disjoint sets of object
identities. If only the class name is used, we cannot not detect this,
because the types of both instances is $\{A\}$. 

\begin{figure}[t]
\begin{class}{A[X]}
\begin{state}
x : X
\end{state}
\end{class}

\begin{axdef}
  a_{1} : A[\nat]\\
  a_{2} : A[\power \nat]
\where
  a_{1} = a_{2}
\end{axdef}

\caption{A Generic Reference in Error}
\label{generic-class-A}
\end{figure}


The second difference between our approach and Chen's approach is on
an implementation level. If we reference a feature of an instance (for
example, $a.x$), then we have to lookup the type of $x$ in $a$ each
time we encounter such a reference, because the class type itself does
not maintain features and their signatures or types.  Similarly, if we
have a reference to a type $\{ A, B, C\}$ constructed via a class
union, then for a reference $a.x$, we need to calculate the
polymorphic core of $\{A,B,C\}$ each time it is referenced. We believe
a class type that records features, including those inherited from any
superclasses, simplifies and optimises the implementation of the
typechecker.

In addition, Object-Z supports the renaming of features in a class,
for example, renaming an operation. The class $A$ with operation $op$
is not disjoint from the class $A[newOp/op]$, which is the class $A$ with
the operation $op$ renamed to $newOp$. The Object-Z reference manual
\cite{smith00} defines the set of object identites of a renamed class
to be equal to that of its base class. Using only the object
identities is not flexible enough to detect whether a reference to a
feature is valid.

The purpose of the work in this paper is to extend Chen's model with a
way of modelling classes to provide a suitable class type that
supports class union, polymorphic expressions, and appropriately
manages dereferencing on objects, and to specify a unification
algorithm for these types to support type inference.
% ``class reference expressions'' == ``dereferencing of object types''??
