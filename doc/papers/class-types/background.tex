\section{Background and Motivation}
\label{background}

In this section, we discuss some background and the motivation for the
work in this paper.

\subsection{Object-Z}


{\bf TJM}:

Brief discussion of class paragraphs, objects, and maybe operations?

\subsection{Object Identity and Classes}

An object's identity distinguishes it from all other objects.  Two
different objects of the same type, $o_{1}$ and $o_{2}$, can have the
same values for their state, yet still be distinct objects; that is
$o_{1} \neq o_{2}$.

Object-Z's semantics implicitly include the concept of object identity
in the language, therefore making them a {\em reference semantics},
similar to languages such as Smalltalk \cite{goldberg83} and
Java\footnote{\url{http://java.sun.com}}, as opposed to a {\em
value-based semantics}, in which objects are denoted by the value of
the state. As commonly used in object-oriented languages, the {\em dot
notation} is used to dereference attributes or
operations\footnote{From here on the collective terms for attributes
and operations is {\em features}.} of an object. For example, the
feature $x$ of an object $a$ is accessed using the syntax $a.x$.

Using these semantics, an Object-Z class is a countably infinite set
of object identities, defined using a class paragraph. The set $\oid$
represents the set of all object identities in a system, such that for
any class $A$, $A \subseteq \oid$ is true, and for the set $\{ A_{1},
..., A_{n} \}$ containing all classes in a system, $\duni \{ A_{1},
..., A_{n} \} = \oid$ is true.

In Object-Z, classes can be unioned to create polymorphic types. That
is, two classes $A$ and $B$ that share no common superclass, can be
unioned together to create a new set of object identities: $AB == A
\classuni B$. The Object-Z type rules specify that any commonly named
features in $A$ and $B$ must share the same types or
signatures respectively. These commonly named features are known as
the {\em polymorphic core} of the class union \cite{dong93}.

Object-Z also provides a {\em polymorphic operator}. That is, for a
class $A$, the expression $\poly A$ refers to the set of object
identities in $A$ and any subclasses of $A$. The polymorphic core of
this expression is the features of $A$.

\subsection{Motivation}

Chen \cite{chen94} discusses a model in which a class type is a set of
classes. That is, if we have a class $A$ with subclasses $B$ and $C$,
then the expression $\poly A$ is the set of classes $\{A, B,
C\}$. This idea is used as the basis for the class type in our
typechecker, but we make two significant changes.

Firstly, on a theoretical level, this approach does not provide good
support for typechecking a language that supports generic classes. For
example, take the Object-Z class defined in
Figure~\ref{generic-class-A}.  In this example, we have the generic
class $A$, with parameter $X$ and no features.  Two
instances of $A$ are declared: one with $X$ instantiated to $\nat$ and
the other to $\power \nat$. The equality $a_{1} = a_{2}$ is type
incorrect because $A[\nat]$ and $A[\power \nat]$ are disjoint sets of
object identities. However, if the class type model used in
\cite{chen94} is used, we cannot not detect this, because the types of
both instances is $\{A\}$.

\begin{figure}[t]
\begin{class}{A[X]}
\begin{state}
x : X
\end{state}
\end{class}

\begin{axdef}
  a_{1} : A[\nat]\\
  a_{2} : A[\power \nat]
\where
  a_{1} = a_{2}
\end{axdef}

\caption{A Generic Reference in Error}
\label{generic-class-A}
\end{figure}


The second difference between our approach and Chen's approach is on
an implementation level. If we reference a feature of an instance (for
example, $a.x$), then we have to lookup the type of $x$ in $a$ each
time we encounter such a reference, because the class type itself does
not maintain features and their types.  Similarly, if we have a
reference to a type $\{ A, B, C\}$ constructed via a class union, then
for a reference $a.x$, we need to calculate the polymorphic core of
$\{A,B,C\}$ each time it is referenced. We believe a class type that
records its features, including those from any superclasses,
simplifies the typechecker code.

The purpose of the work in this paper is to extend the model in
\cite{chen94} to allow the typechecking of generic classes, and to
provide a suitable class type for our typechecker that can handle
class unions, polymorphic expressions, and class reference
expressions.
