\begin{zsection}
   \SECTION typechecker\_test \parents zstate\_toolkit
\end{zsection}

%%%%%%%%%%%%%%%%%%%%%%%%%%%%%%%%%%%%%%%%%%%%%%%%%%%%%%%%%%%%%%%%%%%%%%%%%%%%%%%%%%%%%%%%%%
%%%%%%%%%%%%%%%%%%%%%%%%%%%%%%%%%%%%%%%%%%%%%%%%%%%%%%%%%%%%%%%%%%%%%%%%%%%%%%%%%%%%%%%%%%
\section{Operators and auxiliary types}

\newcommand{\readby}{\zrelop{readBy}}
\newcommand{\writes}{\zrelop{writes}}
\newcommand{\readableTo}{\zrelop{readableTo}}
\newcommand{\canWrite}{\zrelop{canWrite}}
\syndef{\readby}{inrel}{"readBy"}
\syndef{\writes}{inrel}{"writes"}
\syndef{\readableTo}{inrel}{"readableTo"}
\syndef{\canWrite}{inrel}{"canWrite"}

%%Zinword \readBy readBy
%%Zinword \writes writes
%%Zinword \readableTo readableTo
%%Zinword \canWrite canWrite

\begin{zed}
\relation (\varg \readBy \varg)
\\
\relation (\varg \writes \varg)
\\
\relation (\varg \readableTo \varg)
\\
\relation (\varg \canWrite \varg)
\end{zed}

\begin{zed}
   [Service, Data, Cloud]
\end{zed}

%%%%%%%%%%%%%%%%%%%%%%%%%%%%%%%%%%%%%%%%%%%%%%%%%%%%%%%%%%%%%%%%%%%%%%%%%%%%%%%%%%%%%%%%%%
%%%%%%%%%%%%%%%%%%%%%%%%%%%%%%%%%%%%%%%%%%%%%%%%%%%%%%%%%%%%%%%%%%%%%%%%%%%%%%%%%%%%%%%%%%
\section{Workflow}

\begin[disabled]{schema}{Workflow}
   \_ \readBy \_ : Data \rel Service
\\
   \_ \writes \_ : Service \rel Data
\\
   \_ \readableTo \_ : Data \rel Service
\\
   \_ \canWrite \_ : Service \rel Data
\\
   wf, graph: Data \rel Data
\\
   knownServices: \power~Service
\\
   knownData: \power~Data
\where
   % A workflow is the transitive closure of the composition between
   % data read by services that write data subsequentially to it.
   wf = ((\_ \readBy \_) \comp (\_ \writes \_))\plus
\\
   graph = ((\_ \readableTo \_) \comp (\_ \canWrite \_))\plus
\\
   knownServices = \dom~(\_ \writes \_) \cup \ran~(\_ \readBy \_)
\\
   knownData = \dom~(\_ \readBy \_) \cup \ran~(\_ \writes \_)
\end{schema}

\begin{schema}{WFInit}
   Workflow~'
\where
   (\_ \readBy \_) = (\_ \writes \_) = \emptyset
\end{schema}